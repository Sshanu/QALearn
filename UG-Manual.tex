%% ================================================================================
%% This LaTeX file was created by AbiWord.                                         
%% AbiWord is a free, Open Source word processor.                                  
%% More information about AbiWord is available at http://www.abisource.com/        
%% ================================================================================

\documentclass[a4paper,portrait,12pt]{article}
\usepackage[latin1]{inputenc}
\usepackage{calc}
\usepackage{setspace}
\usepackage{fixltx2e}
\usepackage{graphicx}
\usepackage{multicol}
\usepackage[normalem]{ulem}
%% Please revise the following command, if your babel
%% package does not support en-IN
\usepackage[en]{babel}
\usepackage{color}
\usepackage{hyperref}
 
\begin{document}


\begin{flushleft}
UNDERGRADUATE
\end{flushleft}


\begin{flushleft}
PROGRAMMES
\end{flushleft}


\begin{flushleft}
B.TECH.
\end{flushleft}


\begin{flushleft}
B.S.
\end{flushleft}


\begin{flushleft}
Bachelors-Masters Dual Degree
\end{flushleft}


\begin{flushleft}
M.SC. Two-Year
\end{flushleft}


\begin{flushleft}
M.Sc.-Ph.D. (MSPD) Dual Degree
\end{flushleft}


\begin{flushleft}
Manual of
\end{flushleft}





\begin{flushleft}
PROCEDURES \& REQUIREMENTS
\end{flushleft}





\begin{flushleft}
INDIAN INSTITUTE OF TECHNOLOGY KANPUR
\end{flushleft}


\begin{flushleft}
UG Manual Version: Sept. 13, 2017
\end{flushleft}





\begin{flushleft}
\newpage
Table of Contents
\end{flushleft}


\begin{flushleft}
1 Introduction............................................................................................................................ 05
\end{flushleft}


\begin{flushleft}
2 Programmes of Study \ldots{}\ldots{}\ldots{}\ldots{}\ldots{}\ldots{}\ldots{}\ldots{}\ldots{}\ldots{}\ldots{}\ldots{}\ldots{}\ldots{}\ldots{}\ldots{}\ldots{}\ldots{}\ldots{}\ldots{}\ldots{}\ldots{}\ldots{}\ldots{}\ldots{}\ldots{}\ldots{}\ldots{}\ldots{}\ldots{}\ldots{}\ldots{}\ldots{}\ldots{}\ldots{}\ldots{}\ldots{}\ldots{}\ldots{}\ldots{}\ldots{}\ldots{} 06
\end{flushleft}


\begin{flushleft}
2.1 Programmes for New Students \ldots{}\ldots{}\ldots{}\ldots{}\ldots{}\ldots{}\ldots{}\ldots{}\ldots{}\ldots{}\ldots{}\ldots{}\ldots{}\ldots{}\ldots{}\ldots{}\ldots{}\ldots{}\ldots{}\ldots{}\ldots{}\ldots{}\ldots{}\ldots{}\ldots{}\ldots{}\ldots{}\ldots{}\ldots{}\ldots{}\ldots{}\ldots{}\ldots{}\ldots{}. 06
\end{flushleft}


\begin{flushleft}
2.1.1 Admission through JEE \ldots{}\ldots{}\ldots{}\ldots{}\ldots{}\ldots{}\ldots{}\ldots{}\ldots{}\ldots{}\ldots{}\ldots{}\ldots{}\ldots{}\ldots{}\ldots{}\ldots{}\ldots{}\ldots{}\ldots{}\ldots{}\ldots{}\ldots{}\ldots{}\ldots{}\ldots{}\ldots{}\ldots{}\ldots{}\ldots{}\ldots{}\ldots{}\ldots{}\ldots{}.. 06
\end{flushleft}


\begin{flushleft}
2.1.2 Admission through JAM \ldots{}\ldots{}\ldots{}\ldots{}\ldots{}\ldots{}\ldots{}\ldots{}\ldots{}\ldots{}\ldots{}\ldots{}\ldots{}\ldots{}\ldots{}\ldots{}\ldots{}\ldots{}\ldots{}\ldots{}\ldots{}\ldots{}\ldots{}\ldots{}\ldots{}\ldots{}\ldots{}\ldots{}\ldots{}\ldots{}\ldots{}\ldots{}\ldots{}\ldots{} 06
\end{flushleft}


\begin{flushleft}
2.2 Options for Already Enrolled Students \ldots{}\ldots{}\ldots{}\ldots{}\ldots{}\ldots{}\ldots{}\ldots{}\ldots{}\ldots{}\ldots{}\ldots{}\ldots{}\ldots{}\ldots{}\ldots{}\ldots{}\ldots{}\ldots{}\ldots{}\ldots{}\ldots{}\ldots{}\ldots{}\ldots{}\ldots{}\ldots{}\ldots{}\ldots{}\ldots{}. 06
\end{flushleft}


\begin{flushleft}
2.2.1 Branch Change \ldots{}\ldots{}\ldots{}\ldots{}\ldots{}\ldots{}\ldots{}\ldots{}\ldots{}\ldots{}\ldots{}\ldots{}\ldots{}\ldots{}\ldots{}\ldots{}\ldots{}\ldots{}\ldots{}\ldots{}\ldots{}\ldots{}\ldots{}\ldots{}\ldots{}\ldots{}\ldots{}\ldots{}\ldots{}\ldots{}\ldots{}\ldots{}\ldots{}\ldots{}\ldots{}\ldots{}\ldots{}\ldots{}\ldots{}. 06
\end{flushleft}


\begin{flushleft}
2.2.2 Double-Major \ldots{}\ldots{}\ldots{}\ldots{}\ldots{}\ldots{}\ldots{}\ldots{}\ldots{}\ldots{}\ldots{}\ldots{}\ldots{}\ldots{}\ldots{}\ldots{}\ldots{}\ldots{}\ldots{}\ldots{}\ldots{}\ldots{}\ldots{}\ldots{}\ldots{}\ldots{}\ldots{}\ldots{}\ldots{}\ldots{}\ldots{}\ldots{}\ldots{}\ldots{}\ldots{}\ldots{}\ldots{}\ldots{}\ldots{}\ldots{} 06
\end{flushleft}


\begin{flushleft}
2.2.3 Dual-Degree \ldots{}\ldots{}\ldots{}\ldots{}\ldots{}\ldots{}\ldots{}\ldots{}\ldots{}\ldots{}\ldots{}\ldots{}\ldots{}\ldots{}\ldots{}\ldots{}\ldots{}\ldots{}\ldots{}\ldots{}\ldots{}\ldots{}\ldots{}\ldots{}\ldots{}\ldots{}\ldots{}\ldots{}\ldots{}\ldots{}\ldots{}\ldots{}\ldots{}\ldots{}\ldots{}\ldots{}\ldots{}\ldots{}\ldots{}\ldots{}.. 06
\end{flushleft}


\begin{flushleft}
2.2.4 Minors \ldots{}\ldots{}\ldots{}\ldots{}\ldots{}\ldots{}\ldots{}\ldots{}\ldots{}\ldots{}\ldots{}\ldots{}\ldots{}\ldots{}\ldots{}\ldots{}\ldots{}\ldots{}\ldots{}\ldots{}\ldots{}\ldots{}\ldots{}\ldots{}\ldots{}\ldots{}\ldots{}\ldots{}\ldots{}\ldots{}\ldots{}\ldots{}\ldots{}\ldots{}\ldots{}\ldots{}\ldots{}\ldots{}\ldots{}\ldots{}\ldots{}\ldots{}\ldots{}\ldots{} 07
\end{flushleft}


\begin{flushleft}
3 Admission Procedure and Rules \ldots{}\ldots{}\ldots{}\ldots{}\ldots{}\ldots{}\ldots{}\ldots{}\ldots{}\ldots{}\ldots{}\ldots{}\ldots{}\ldots{}\ldots{}\ldots{}\ldots{}\ldots{}\ldots{}\ldots{}\ldots{}\ldots{}\ldots{}\ldots{}\ldots{}\ldots{}\ldots{}\ldots{}\ldots{}\ldots{}\ldots{}\ldots{}\ldots{}\ldots{}\ldots{}\ldots{} 08
\end{flushleft}


\begin{flushleft}
3.1 For New Students \ldots{}\ldots{}\ldots{}\ldots{}\ldots{}\ldots{}\ldots{}\ldots{}\ldots{}\ldots{}\ldots{}\ldots{}\ldots{}\ldots{}\ldots{}\ldots{}\ldots{}\ldots{}\ldots{}\ldots{}\ldots{}\ldots{}\ldots{}\ldots{}\ldots{}\ldots{}\ldots{}\ldots{}\ldots{}\ldots{}\ldots{}\ldots{}\ldots{}\ldots{}\ldots{}\ldots{}\ldots{}\ldots{}\ldots{}\ldots{}\ldots{}\ldots{} 08
\end{flushleft}


\begin{flushleft}
3.1.1 B Tech and BS Programmes \ldots{}\ldots{}\ldots{}\ldots{}\ldots{}\ldots{}\ldots{}\ldots{}\ldots{}\ldots{}\ldots{}\ldots{}\ldots{}\ldots{}\ldots{}\ldots{}\ldots{}\ldots{}\ldots{}\ldots{}\ldots{}\ldots{}\ldots{}\ldots{}\ldots{}\ldots{}\ldots{}\ldots{}\ldots{}\ldots{}\ldots{}\ldots{}. 08
\end{flushleft}


\begin{flushleft}
3.1.2 MSc and MSPD Programmes \ldots{}\ldots{}\ldots{}\ldots{}\ldots{}\ldots{}\ldots{}\ldots{}\ldots{}\ldots{}\ldots{}\ldots{}\ldots{}\ldots{}\ldots{}\ldots{}\ldots{}\ldots{}\ldots{}\ldots{}\ldots{}\ldots{}\ldots{}\ldots{}\ldots{}\ldots{}\ldots{}\ldots{}\ldots{}\ldots{}\ldots{}. 08
\end{flushleft}


\begin{flushleft}
3.2 Non-Degree Students \ldots{}\ldots{}\ldots{}\ldots{}\ldots{}\ldots{}\ldots{}\ldots{}\ldots{}\ldots{}\ldots{}\ldots{}\ldots{}\ldots{}\ldots{}\ldots{}\ldots{}\ldots{}\ldots{}\ldots{}\ldots{}\ldots{}\ldots{}\ldots{}\ldots{}\ldots{}\ldots{}\ldots{}\ldots{}\ldots{}\ldots{}\ldots{}\ldots{}\ldots{}\ldots{}\ldots{}\ldots{}\ldots{}\ldots{}. 09
\end{flushleft}


\begin{flushleft}
3.3 Validity of Admission and Its Cancellation \ldots{}\ldots{}\ldots{}\ldots{}\ldots{}\ldots{}\ldots{}\ldots{}\ldots{}\ldots{}\ldots{}\ldots{}\ldots{}\ldots{}\ldots{}\ldots{}\ldots{}\ldots{}\ldots{}\ldots{}\ldots{}\ldots{}\ldots{}\ldots{}\ldots{}\ldots{}\ldots{}.. 09
\end{flushleft}


\begin{flushleft}
4 Academic Session \ldots{}\ldots{}\ldots{}\ldots{}\ldots{}\ldots{}\ldots{}\ldots{}\ldots{}\ldots{}\ldots{}\ldots{}\ldots{}\ldots{}\ldots{}\ldots{}\ldots{}\ldots{}\ldots{}\ldots{}\ldots{}\ldots{}\ldots{}\ldots{}\ldots{}\ldots{}\ldots{}\ldots{}\ldots{}\ldots{}\ldots{}\ldots{}\ldots{}\ldots{}\ldots{}\ldots{}\ldots{}\ldots{}\ldots{}\ldots{}\ldots{}\ldots{}\ldots{}\ldots{} 10
\end{flushleft}


\begin{flushleft}
4.1 Dates \ldots{}\ldots{}\ldots{}\ldots{}\ldots{}\ldots{}\ldots{}\ldots{}.\ldots{}\ldots{}\ldots{}\ldots{}\ldots{}\ldots{}\ldots{}\ldots{}\ldots{}\ldots{}\ldots{}\ldots{}\ldots{}\ldots{}\ldots{}\ldots{}\ldots{}\ldots{}\ldots{}\ldots{}\ldots{}\ldots{}\ldots{}\ldots{}\ldots{}\ldots{}\ldots{}\ldots{}\ldots{}\ldots{}\ldots{}\ldots{}\ldots{}\ldots{}\ldots{}\ldots{}\ldots{}\ldots{}\ldots{}\ldots{}.. 10
\end{flushleft}


\begin{flushleft}
4.2 Duration \ldots{}\ldots{}\ldots{}\ldots{}\ldots{}\ldots{}\ldots{}\ldots{}\ldots{}\ldots{}\ldots{}\ldots{}\ldots{}\ldots{}\ldots{}\ldots{}\ldots{}\ldots{}\ldots{}\ldots{}\ldots{}\ldots{}\ldots{}\ldots{}\ldots{}\ldots{}\ldots{}\ldots{}\ldots{}\ldots{}\ldots{}\ldots{}\ldots{}\ldots{}\ldots{}\ldots{}\ldots{}\ldots{}\ldots{}\ldots{}\ldots{}\ldots{}\ldots{}\ldots{}\ldots{}\ldots{}\ldots{} 10
\end{flushleft}


\begin{flushleft}
4.3 Academic Calendar \ldots{}\ldots{}\ldots{}\ldots{}\ldots{}\ldots{}\ldots{}\ldots{}\ldots{}\ldots{}\ldots{}\ldots{}\ldots{}\ldots{}\ldots{}\ldots{}\ldots{}\ldots{}\ldots{}\ldots{}\ldots{}\ldots{}\ldots{}\ldots{}\ldots{}\ldots{}\ldots{}\ldots{}\ldots{}\ldots{}\ldots{}\ldots{}\ldots{}\ldots{}\ldots{}\ldots{}\ldots{}\ldots{}\ldots{}\ldots{}\ldots{} 10
\end{flushleft}


\begin{flushleft}
5 Curriculum \ldots{}\ldots{}\ldots{}\ldots{}\ldots{}\ldots{}\ldots{}\ldots{}\ldots{}\ldots{}\ldots{}\ldots{}\ldots{}\ldots{}\ldots{}\ldots{}\ldots{}\ldots{}\ldots{}\ldots{}\ldots{}\ldots{}\ldots{}\ldots{}\ldots{}\ldots{}\ldots{}\ldots{}\ldots{}\ldots{}\ldots{}\ldots{}\ldots{}\ldots{}\ldots{}\ldots{}\ldots{}\ldots{}\ldots{}\ldots{}\ldots{}\ldots{}\ldots{}\ldots{}\ldots{}\ldots{}\ldots{}\ldots{} 11
\end{flushleft}


\begin{flushleft}
5.1 B Tech and BS Programmes \ldots{}\ldots{}\ldots{}\ldots{}\ldots{}\ldots{}\ldots{}\ldots{}\ldots{}\ldots{}\ldots{}\ldots{}\ldots{}\ldots{}\ldots{}\ldots{}\ldots{}\ldots{}\ldots{}\ldots{}\ldots{}\ldots{}\ldots{}\ldots{}\ldots{}\ldots{}\ldots{}\ldots{}\ldots{}\ldots{}\ldots{}\ldots{}\ldots{}\ldots{}\ldots{}... 11
\end{flushleft}


\begin{flushleft}
5.1.1 Double Major \ldots{}\ldots{}\ldots{}\ldots{}\ldots{}\ldots{}\ldots{}\ldots{}\ldots{}\ldots{}\ldots{}\ldots{}\ldots{}\ldots{}\ldots{}\ldots{}\ldots{}\ldots{}\ldots{}\ldots{}\ldots{}\ldots{}\ldots{}\ldots{}\ldots{}\ldots{}\ldots{}\ldots{}\ldots{}\ldots{}\ldots{}\ldots{}\ldots{}\ldots{}\ldots{}\ldots{}\ldots{}\ldots{}\ldots{}\ldots{} 11
\end{flushleft}


\begin{flushleft}
5.2 Bachelors-Masters Dual Degree Programme \ldots{}\ldots{}\ldots{}\ldots{}\ldots{}\ldots{}\ldots{}\ldots{}\ldots{}\ldots{}\ldots{}\ldots{}\ldots{}\ldots{}\ldots{}\ldots{}\ldots{}\ldots{}\ldots{}\ldots{}\ldots{}\ldots{}\ldots{}\ldots{}\ldots{}\ldots{} 11
\end{flushleft}


\begin{flushleft}
5.3 MSc Programme \ldots{}\ldots{}\ldots{}\ldots{}\ldots{}\ldots{}\ldots{}\ldots{}\ldots{}\ldots{}\ldots{}\ldots{}\ldots{}\ldots{}\ldots{}\ldots{}\ldots{}\ldots{}\ldots{}\ldots{}\ldots{}\ldots{}\ldots{}\ldots{}\ldots{}\ldots{}\ldots{}\ldots{}\ldots{}\ldots{}\ldots{}\ldots{}\ldots{}\ldots{}\ldots{}\ldots{}\ldots{}\ldots{}\ldots{}\ldots{}\ldots{}\ldots{} 11
\end{flushleft}


\begin{flushleft}
5.4 MSPD Dual Degree Programme \ldots{}\ldots{}\ldots{}\ldots{}\ldots{}\ldots{}\ldots{}\ldots{}\ldots{}\ldots{}\ldots{}\ldots{}\ldots{}\ldots{}\ldots{}\ldots{}\ldots{}\ldots{}\ldots{}\ldots{}\ldots{}\ldots{}\ldots{}\ldots{}\ldots{}\ldots{}\ldots{}\ldots{}\ldots{}\ldots{}\ldots{}\ldots{}\ldots{}\ldots{} 12
\end{flushleft}


\begin{flushleft}
5.5 List of Courses \ldots{}\ldots{}\ldots{}\ldots{}\ldots{}\ldots{}\ldots{}\ldots{}\ldots{}\ldots{}\ldots{}\ldots{}\ldots{}\ldots{}\ldots{}\ldots{}\ldots{}\ldots{}\ldots{}\ldots{}\ldots{}\ldots{}\ldots{}\ldots{}\ldots{}\ldots{}\ldots{}\ldots{}\ldots{}\ldots{}\ldots{}\ldots{}\ldots{}\ldots{}\ldots{}\ldots{}\ldots{}\ldots{}\ldots{}\ldots{}\ldots{}\ldots{}\ldots{}\ldots{} 12
\end{flushleft}


\begin{flushleft}
6 Registration \ldots{}\ldots{}\ldots{}\ldots{}\ldots{}\ldots{}\ldots{}\ldots{}\ldots{}\ldots{}\ldots{}\ldots{}\ldots{}\ldots{}\ldots{}\ldots{}\ldots{}\ldots{}\ldots{}\ldots{}\ldots{}\ldots{}\ldots{}\ldots{}\ldots{}\ldots{}\ldots{}\ldots{}\ldots{}\ldots{}\ldots{}\ldots{}\ldots{}\ldots{}\ldots{}\ldots{}\ldots{}\ldots{}\ldots{}\ldots{}\ldots{}\ldots{}\ldots{}\ldots{}\ldots{}\ldots{}\ldots{}... 13
\end{flushleft}


\begin{flushleft}
6.1 Academic Registration \ldots{}\ldots{}\ldots{}\ldots{}\ldots{}\ldots{}\ldots{}\ldots{}\ldots{}\ldots{}\ldots{}\ldots{}\ldots{}\ldots{}\ldots{}\ldots{}\ldots{}\ldots{}\ldots{}\ldots{}\ldots{}\ldots{}\ldots{}\ldots{}\ldots{}\ldots{}\ldots{}\ldots{}\ldots{}\ldots{}\ldots{}\ldots{}\ldots{}\ldots{}\ldots{}\ldots{}\ldots{}\ldots{}\ldots{}. 13
\end{flushleft}


\begin{flushleft}
6.1.1 Pre-Registration \ldots{}\ldots{}\ldots{}\ldots{}\ldots{}\ldots{}\ldots{}\ldots{}\ldots{}\ldots{}\ldots{}\ldots{}\ldots{}\ldots{}\ldots{}\ldots{}\ldots{}\ldots{}\ldots{}\ldots{}\ldots{}\ldots{}\ldots{}\ldots{}\ldots{}\ldots{}\ldots{}\ldots{}\ldots{}\ldots{}\ldots{}\ldots{}\ldots{}\ldots{}\ldots{}\ldots{}\ldots{}\ldots{}.. 13
\end{flushleft}


\begin{flushleft}
6.1.2 Final Registration \ldots{}\ldots{}\ldots{}\ldots{}\ldots{}\ldots{}\ldots{}\ldots{}\ldots{}\ldots{}\ldots{}\ldots{}\ldots{}\ldots{}\ldots{}\ldots{}\ldots{}\ldots{}\ldots{}\ldots{}\ldots{}\ldots{}\ldots{}\ldots{}\ldots{}\ldots{}\ldots{}\ldots{}\ldots{}\ldots{}\ldots{}\ldots{}\ldots{}\ldots{}\ldots{}\ldots{}\ldots{}\ldots{} 13
\end{flushleft}


\begin{flushleft}
6.1.3 Add-Drop of Courses \ldots{}\ldots{}\ldots{}\ldots{}\ldots{}\ldots{}\ldots{}\ldots{}\ldots{}\ldots{}\ldots{}\ldots{}\ldots{}\ldots{}\ldots{}\ldots{}\ldots{}\ldots{}\ldots{}\ldots{}\ldots{}\ldots{}\ldots{}\ldots{}\ldots{}\ldots{}\ldots{}\ldots{}\ldots{}\ldots{}\ldots{}\ldots{}\ldots{}\ldots{}\ldots{}\ldots{} 14
\end{flushleft}


\begin{flushleft}
6.1.4 Cancellation of Registration in a Course \ldots{}\ldots{}\ldots{}\ldots{}\ldots{}\ldots{}\ldots{}\ldots{}\ldots{}\ldots{}\ldots{}\ldots{}\ldots{}\ldots{}\ldots{}\ldots{}\ldots{}\ldots{}\ldots{}\ldots{}\ldots{}\ldots{}\ldots{}\ldots{}\ldots{}. 14
\end{flushleft}


\begin{flushleft}
6.1.5 Academic Load In Regular Semesters \ldots{}\ldots{}\ldots{}\ldots{}\ldots{}\ldots{}\ldots{}\ldots{}\ldots{}\ldots{}\ldots{}\ldots{}\ldots{}\ldots{}\ldots{}\ldots{}\ldots{}\ldots{}\ldots{}\ldots{}\ldots{}\ldots{}\ldots{}\ldots{}\ldots{}\ldots{}.. 14
\end{flushleft}


\begin{flushleft}
6.1.5.1 Exceptions to Regular Rules regarding Academic Load\ldots{}\ldots{}\ldots{}\ldots{}\ldots{}\ldots{}\ldots{}\ldots{}\ldots{}\ldots{}.. 14
\end{flushleft}


\begin{flushleft}
6.1.6 Academic Load in Summer Term \ldots{}\ldots{}\ldots{}\ldots{}\ldots{}\ldots{}\ldots{}\ldots{}\ldots{}\ldots{}\ldots{}\ldots{}\ldots{}\ldots{}\ldots{}\ldots{}\ldots{}\ldots{}\ldots{}\ldots{}\ldots{}\ldots{}\ldots{}\ldots{}\ldots{}\ldots{}\ldots{}\ldots{}\ldots{}. 15
\end{flushleft}


\begin{flushleft}
6.1.7 Cancellation of Registration \ldots{}\ldots{}\ldots{}\ldots{}\ldots{}\ldots{}\ldots{}\ldots{}\ldots{}\ldots{}\ldots{}\ldots{}\ldots{}\ldots{}\ldots{}\ldots{}\ldots{}\ldots{}\ldots{}\ldots{}\ldots{}\ldots{}\ldots{}\ldots{}\ldots{}\ldots{}\ldots{}\ldots{}\ldots{}\ldots{}\ldots{}\ldots{} 15
\end{flushleft}


\begin{flushleft}
6.2 Administrative Registration \ldots{}\ldots{}\ldots{}\ldots{}\ldots{}\ldots{}\ldots{}\ldots{}\ldots{}\ldots{}\ldots{}\ldots{}\ldots{}\ldots{}\ldots{}\ldots{}\ldots{}\ldots{}\ldots{}\ldots{}\ldots{}\ldots{}\ldots{}\ldots{}\ldots{}\ldots{}\ldots{}\ldots{}\ldots{}\ldots{}\ldots{}\ldots{}\ldots{}\ldots{}\ldots{}\ldots{}. 15
\end{flushleft}


\begin{flushleft}
6.3 Late Registration \ldots{}\ldots{}\ldots{}\ldots{}\ldots{}\ldots{}\ldots{}\ldots{}\ldots{}\ldots{}\ldots{}\ldots{}\ldots{}\ldots{}\ldots{}\ldots{}\ldots{}\ldots{}\ldots{}\ldots{}\ldots{}\ldots{}\ldots{}\ldots{}\ldots{}\ldots{}\ldots{}\ldots{}\ldots{}\ldots{}\ldots{}\ldots{}\ldots{}\ldots{}\ldots{}\ldots{}\ldots{}\ldots{}\ldots{}\ldots{}\ldots{}\ldots{}.. 15
\end{flushleft}


\begin{flushleft}
7 Teaching and Evaluation \ldots{}\ldots{}\ldots{}\ldots{}\ldots{}\ldots{}\ldots{}\ldots{}\ldots{}\ldots{}\ldots{}\ldots{}\ldots{}\ldots{}\ldots{}\ldots{}\ldots{}\ldots{}\ldots{}\ldots{}\ldots{}\ldots{}\ldots{}\ldots{}\ldots{}\ldots{}\ldots{}\ldots{}\ldots{}\ldots{}\ldots{}\ldots{}\ldots{}\ldots{}\ldots{}\ldots{}\ldots{}\ldots{}\ldots{}\ldots{}.. 16
\end{flushleft}


\begin{flushleft}
7.1 Teaching \ldots{}\ldots{}\ldots{}\ldots{}\ldots{}\ldots{}\ldots{}\ldots{}\ldots{}\ldots{}\ldots{}\ldots{}\ldots{}\ldots{}\ldots{}\ldots{}\ldots{}\ldots{}\ldots{}\ldots{}\ldots{}\ldots{}\ldots{}\ldots{}\ldots{}\ldots{}\ldots{}\ldots{}\ldots{}\ldots{}\ldots{}\ldots{}\ldots{}\ldots{}\ldots{}\ldots{}\ldots{}\ldots{}\ldots{}\ldots{}\ldots{}\ldots{}\ldots{}\ldots{}\ldots{}\ldots{}\ldots{}. 16
\end{flushleft}


\begin{flushleft}
7.1.1 Medium of Instruction \ldots{}\ldots{}\ldots{}\ldots{}\ldots{}..\ldots{}\ldots{}\ldots{}\ldots{}\ldots{}\ldots{}\ldots{}\ldots{}\ldots{}\ldots{}\ldots{}\ldots{}\ldots{}\ldots{}\ldots{}\ldots{}\ldots{}\ldots{}\ldots{}\ldots{}\ldots{}\ldots{}\ldots{}\ldots{}\ldots{}\ldots{}\ldots{}\ldots{}\ldots{}. 16
\end{flushleft}


\begin{flushleft}
7.1.2 Offering a New Course \ldots{}\ldots{}\ldots{}\ldots{}\ldots{}\ldots{}\ldots{}\ldots{}\ldots{}\ldots{}\ldots{}\ldots{}\ldots{}\ldots{}\ldots{}\ldots{}\ldots{}\ldots{}\ldots{}\ldots{}\ldots{}\ldots{}\ldots{}\ldots{}\ldots{}\ldots{}\ldots{}\ldots{}\ldots{}\ldots{}\ldots{}\ldots{}\ldots{}\ldots{}\ldots{} 16
\end{flushleft}


\begin{flushleft}
7.1.3 Courses Offerings for a Given Semester\ldots{}\ldots{}\ldots{}\ldots{}\ldots{}\ldots{}\ldots{}\ldots{}\ldots{}\ldots{}\ldots{}\ldots{}\ldots{}\ldots{}\ldots{}\ldots{}\ldots{}\ldots{}\ldots{}\ldots{}\ldots{}\ldots{}\ldots{}\ldots{}.. 16
\end{flushleft}


\begin{flushleft}
7.1.4 Duration of Courses \ldots{}\ldots{}\ldots{}\ldots{}\ldots{}\ldots{}\ldots{}\ldots{}\ldots{}\ldots{}\ldots{}\ldots{}\ldots{}\ldots{}\ldots{}\ldots{}\ldots{}\ldots{}\ldots{}\ldots{}\ldots{}\ldots{}\ldots{}\ldots{}\ldots{}\ldots{}\ldots{}\ldots{}\ldots{}\ldots{}\ldots{}\ldots{}\ldots{}\ldots{}\ldots{}\ldots{}.. 16
\end{flushleft}


\begin{flushleft}
7.1.5 Conduct of Courses \ldots{}\ldots{}\ldots{}\ldots{}\ldots{}\ldots{}\ldots{}\ldots{}\ldots{}\ldots{}\ldots{}\ldots{}\ldots{}\ldots{}\ldots{}\ldots{}\ldots{}\ldots{}\ldots{}\ldots{}\ldots{}\ldots{}\ldots{}\ldots{}\ldots{}\ldots{}\ldots{}\ldots{}\ldots{}\ldots{}\ldots{}\ldots{}\ldots{}\ldots{}\ldots{}\ldots{}\ldots{} 16
\end{flushleft}


\begin{flushleft}
7.1.6 Attendance in Class \ldots{}\ldots{}\ldots{}\ldots{}\ldots{}\ldots{}\ldots{}\ldots{}\ldots{}\ldots{}\ldots{}\ldots{}\ldots{}\ldots{}\ldots{}\ldots{}\ldots{}\ldots{}\ldots{}\ldots{}\ldots{}\ldots{}\ldots{}\ldots{}\ldots{}\ldots{}\ldots{}\ldots{}\ldots{}\ldots{}\ldots{}\ldots{}\ldots{}\ldots{}\ldots{}\ldots{}.. 16
\end{flushleft}


\begin{flushleft}
7.1.7 Work-Week and Class Timings \ldots{}\ldots{}\ldots{}\ldots{}\ldots{}\ldots{}\ldots{}\ldots{}\ldots{}\ldots{}\ldots{}\ldots{}\ldots{}\ldots{}\ldots{}\ldots{}\ldots{}\ldots{}\ldots{}\ldots{}\ldots{}\ldots{}\ldots{}\ldots{}\ldots{}\ldots{}\ldots{}\ldots{}\ldots{}\ldots{}. 17
\end{flushleft}


\begin{flushleft}
7.2 Evaluation and Performance Feedback \ldots{}\ldots{}\ldots{}\ldots{}\ldots{}\ldots{}\ldots{}\ldots{}\ldots{}\ldots{}\ldots{}\ldots{}\ldots{}\ldots{}\ldots{}\ldots{}\ldots{}\ldots{}\ldots{}\ldots{}\ldots{}\ldots{}\ldots{}\ldots{}\ldots{}\ldots{}\ldots{}\ldots{}\ldots{}.. 17
\end{flushleft}


\begin{flushleft}
7.2.1 Examinations \ldots{}\ldots{}\ldots{}\ldots{}\ldots{}\ldots{}\ldots{}\ldots{}\ldots{}\ldots{}\ldots{}\ldots{}\ldots{}\ldots{}\ldots{}\ldots{}\ldots{}\ldots{}\ldots{}\ldots{}\ldots{}\ldots{}\ldots{}\ldots{}\ldots{}\ldots{}\ldots{}\ldots{}\ldots{}\ldots{}\ldots{}\ldots{}\ldots{}\ldots{}\ldots{}\ldots{}\ldots{}\ldots{}\ldots{}\ldots{} 17
\end{flushleft}


\begin{flushleft}
7.2.2 Quizzes \ldots{}\ldots{}\ldots{}\ldots{}\ldots{}\ldots{}\ldots{}\ldots{}\ldots{}\ldots{}\ldots{}\ldots{}\ldots{}\ldots{}\ldots{}\ldots{}\ldots{}\ldots{}\ldots{}\ldots{}\ldots{}\ldots{}\ldots{}\ldots{}\ldots{}\ldots{}\ldots{}\ldots{}\ldots{}\ldots{}\ldots{}\ldots{}\ldots{}\ldots{}\ldots{}\ldots{}\ldots{}\ldots{}\ldots{}\ldots{}\ldots{}\ldots{}\ldots{}.. 17
\end{flushleft}


\begin{flushleft}
7.2.3 Make-up Examination \ldots{}\ldots{}\ldots{}\ldots{}\ldots{}\ldots{}\ldots{}\ldots{}\ldots{}\ldots{}\ldots{}\ldots{}\ldots{}\ldots{}\ldots{}\ldots{}\ldots{}\ldots{}\ldots{}\ldots{}\ldots{}\ldots{}\ldots{}\ldots{}\ldots{}\ldots{}\ldots{}\ldots{}\ldots{}\ldots{}\ldots{}\ldots{}\ldots{}\ldots{}\ldots{}. 17
\end{flushleft}


\begin{flushleft}
7.2.4 Results of Examinations and Quizzes \ldots{}\ldots{}\ldots{}\ldots{}\ldots{}\ldots{}\ldots{}\ldots{}\ldots{}\ldots{}\ldots{}\ldots{}\ldots{}\ldots{}\ldots{}\ldots{}\ldots{}\ldots{}\ldots{}\ldots{}\ldots{}\ldots{}\ldots{}\ldots{}\ldots{}\ldots{}\ldots{} 18
\end{flushleft}





2





\begin{flushleft}
\newpage
7.2.5 Letter Grades and Weights \ldots{}\ldots{}\ldots{}\ldots{}\ldots{}\ldots{}\ldots{}\ldots{}\ldots{}\ldots{}\ldots{}\ldots{}\ldots{}\ldots{}\ldots{}\ldots{}\ldots{}\ldots{}\ldots{}\ldots{}\ldots{}\ldots{}\ldots{}\ldots{}\ldots{}\ldots{}\ldots{}\ldots{}\ldots{}\ldots{}\ldots{}\ldots{}.. 18
\end{flushleft}


\begin{flushleft}
7.2.6 Semester Performance Index \ldots{}\ldots{}\ldots{}\ldots{}\ldots{}\ldots{}\ldots{}\ldots{}\ldots{}\ldots{}\ldots{}\ldots{}\ldots{}\ldots{}\ldots{}\ldots{}\ldots{}\ldots{}\ldots{}\ldots{}\ldots{}\ldots{}\ldots{}\ldots{}\ldots{}\ldots{}\ldots{}\ldots{}\ldots{}\ldots{}\ldots{}. 19
\end{flushleft}


\begin{flushleft}
7.2.7 Cumulative Performance Index \ldots{}\ldots{}\ldots{}\ldots{}\ldots{}\ldots{}\ldots{}\ldots{}\ldots{}\ldots{}\ldots{}\ldots{}\ldots{}\ldots{}\ldots{}\ldots{}\ldots{}\ldots{}\ldots{}\ldots{}\ldots{}\ldots{}\ldots{}\ldots{}\ldots{}\ldots{}\ldots{}\ldots{}\ldots{}\ldots{} 19
\end{flushleft}


\begin{flushleft}
7.2.8 Declaration of the Final Result \ldots{}\ldots{}\ldots{}\ldots{}\ldots{}\ldots{}\ldots{}\ldots{}\ldots{}\ldots{}\ldots{}\ldots{}\ldots{}\ldots{}\ldots{}\ldots{}\ldots{}\ldots{}\ldots{}\ldots{}\ldots{}\ldots{}\ldots{}\ldots{}\ldots{}\ldots{}\ldots{}\ldots{}\ldots{}.... 19
\end{flushleft}


\begin{flushleft}
7.2.9 Withholding of Grades \ldots{}\ldots{}\ldots{}\ldots{}\ldots{}\ldots{}\ldots{}\ldots{}\ldots{}\ldots{}\ldots{}\ldots{}\ldots{}\ldots{}\ldots{}\ldots{}\ldots{}\ldots{}\ldots{}\ldots{}\ldots{}\ldots{}\ldots{}\ldots{}\ldots{}\ldots{}\ldots{}\ldots{}\ldots{}\ldots{}\ldots{}\ldots{}\ldots{}\ldots{}\ldots{} 19
\end{flushleft}


\begin{flushleft}
7.2.10 Change of an already awarded grade \ldots{}\ldots{}\ldots{}\ldots{}\ldots{}\ldots{}\ldots{}\ldots{}\ldots{}\ldots{}\ldots{}\ldots{}\ldots{}\ldots{}\ldots{}\ldots{}\ldots{}\ldots{}\ldots{}\ldots{}\ldots{}\ldots{}\ldots{}\ldots{}\ldots{}\ldots{} 19
\end{flushleft}


\begin{flushleft}
8 Academic Requirements and Degree Eligibility \ldots{}\ldots{}\ldots{}\ldots{}\ldots{}\ldots{}\ldots{}\ldots{}\ldots{}\ldots{}\ldots{}\ldots{}\ldots{}\ldots{}\ldots{}\ldots{}\ldots{}\ldots{}\ldots{}\ldots{}\ldots{}\ldots{}\ldots{}\ldots{}\ldots{}\ldots{}\ldots{}.. 20
\end{flushleft}


\begin{flushleft}
8.1 Minimum and Maximum Duration \ldots{}\ldots{}\ldots{}\ldots{}\ldots{}\ldots{}\ldots{}\ldots{}\ldots{}\ldots{}\ldots{}\ldots{}\ldots{}\ldots{}\ldots{}\ldots{}\ldots{}\ldots{}\ldots{}\ldots{}\ldots{}\ldots{}\ldots{}\ldots{}\ldots{}\ldots{}\ldots{}\ldots{}\ldots{}\ldots{}\ldots{}\ldots{}. 20
\end{flushleft}


\begin{flushleft}
8.2 Minimum Academic Requirements \ldots{}\ldots{}\ldots{}\ldots{}\ldots{}\ldots{}\ldots{}\ldots{}\ldots{}\ldots{}\ldots{}\ldots{}\ldots{}\ldots{}\ldots{}\ldots{}\ldots{}\ldots{}\ldots{}\ldots{}\ldots{}\ldots{}\ldots{}\ldots{}\ldots{}\ldots{}\ldots{}\ldots{}\ldots{}\ldots{}\ldots{}\ldots{} 20
\end{flushleft}


\begin{flushleft}
8.3 Graduation \ldots{}\ldots{}\ldots{}\ldots{}\ldots{}\ldots{}\ldots{}\ldots{}\ldots{}\ldots{}\ldots{}\ldots{}\ldots{}\ldots{}\ldots{}\ldots{}\ldots{}\ldots{}\ldots{}\ldots{}\ldots{}\ldots{}\ldots{}\ldots{}\ldots{}\ldots{}\ldots{}\ldots{}\ldots{}\ldots{}\ldots{}\ldots{}\ldots{}\ldots{}\ldots{}\ldots{}\ldots{}\ldots{}\ldots{}\ldots{}\ldots{}\ldots{}\ldots{}\ldots{}\ldots{}.. 20
\end{flushleft}


\begin{flushleft}
8.3.1 Graduation with Distinction \ldots{}\ldots{}\ldots{}\ldots{}\ldots{}\ldots{}\ldots{}\ldots{}\ldots{}\ldots{}\ldots{}\ldots{}\ldots{}\ldots{}\ldots{}\ldots{}\ldots{}\ldots{}\ldots{}\ldots{}\ldots{}\ldots{}\ldots{}\ldots{}\ldots{}\ldots{}\ldots{}\ldots{}\ldots{}\ldots{}\ldots{}\ldots{} 20
\end{flushleft}


\begin{flushleft}
8.4 Award of Degrees \ldots{}\ldots{}\ldots{}\ldots{}\ldots{}\ldots{}\ldots{}\ldots{}\ldots{}\ldots{}\ldots{}\ldots{}\ldots{}\ldots{}\ldots{}\ldots{}\ldots{}\ldots{}\ldots{}\ldots{}\ldots{}\ldots{}\ldots{}\ldots{}\ldots{}\ldots{}\ldots{}\ldots{}\ldots{}\ldots{}\ldots{}\ldots{}\ldots{}\ldots{}\ldots{}\ldots{}\ldots{}\ldots{}\ldots{}\ldots{}\ldots{}\ldots{} 20
\end{flushleft}


\begin{flushleft}
8.5 Withdrawal of the Degree \ldots{}\ldots{}\ldots{}\ldots{}\ldots{}\ldots{}\ldots{}\ldots{}\ldots{}\ldots{}\ldots{}\ldots{}\ldots{}\ldots{}\ldots{}\ldots{}\ldots{}\ldots{}\ldots{}\ldots{}\ldots{}\ldots{}\ldots{}\ldots{}\ldots{}\ldots{}\ldots{}\ldots{}\ldots{}\ldots{}\ldots{}\ldots{}\ldots{}\ldots{}\ldots{}\ldots{}\ldots{}. 21
\end{flushleft}


\begin{flushleft}
9 Inadequate Academic Performance \ldots{}\ldots{}\ldots{}\ldots{}\ldots{}\ldots{}\ldots{}\ldots{}\ldots{}\ldots{}\ldots{}\ldots{}\ldots{}\ldots{}\ldots{}\ldots{}\ldots{}\ldots{}\ldots{}\ldots{}\ldots{}\ldots{}\ldots{}\ldots{}\ldots{}\ldots{}\ldots{}\ldots{}\ldots{}\ldots{}\ldots{}\ldots{}\ldots{}.. 22
\end{flushleft}


\begin{flushleft}
9.1 Mechanism to Address Inadequate Academic Performance \ldots{}\ldots{}\ldots{}\ldots{}\ldots{}\ldots{}\ldots{}\ldots{}\ldots{}\ldots{}\ldots{}\ldots{}\ldots{}\ldots{}\ldots{}\ldots{}\ldots{}\ldots{} 22
\end{flushleft}


\begin{flushleft}
9.2 Warning \ldots{}\ldots{}\ldots{}\ldots{}\ldots{}\ldots{}\ldots{}\ldots{}\ldots{}\ldots{}\ldots{}\ldots{}\ldots{}\ldots{}\ldots{}\ldots{}\ldots{}\ldots{}\ldots{}\ldots{}\ldots{}\ldots{}\ldots{}\ldots{}\ldots{}\ldots{}\ldots{}\ldots{}\ldots{}\ldots{}\ldots{}\ldots{}\ldots{}\ldots{}\ldots{}\ldots{}\ldots{}\ldots{}\ldots{}\ldots{}\ldots{}\ldots{}\ldots{}\ldots{}\ldots{}\ldots{}\ldots{}..22
\end{flushleft}


\begin{flushleft}
9.3 Academic Probation \ldots{}\ldots{}\ldots{}\ldots{}\ldots{}\ldots{}\ldots{}\ldots{}\ldots{}\ldots{}\ldots{}\ldots{}\ldots{}\ldots{}\ldots{}\ldots{}\ldots{}\ldots{}\ldots{}\ldots{}\ldots{}\ldots{}\ldots{}\ldots{}\ldots{}\ldots{}\ldots{}\ldots{}\ldots{}\ldots{}\ldots{}\ldots{}\ldots{}\ldots{}\ldots{}\ldots{}\ldots{}\ldots{}\ldots{}\ldots{}.. 22
\end{flushleft}


\begin{flushleft}
9.4 Programme Termination \ldots{}\ldots{}\ldots{}\ldots{}\ldots{}\ldots{}\ldots{}\ldots{}\ldots{}\ldots{}\ldots{}\ldots{}\ldots{}\ldots{}\ldots{}\ldots{}\ldots{}\ldots{}\ldots{}\ldots{}\ldots{}\ldots{}\ldots{}\ldots{}\ldots{}\ldots{}\ldots{}\ldots{}\ldots{}\ldots{}\ldots{}\ldots{}\ldots{}\ldots{}\ldots{}\ldots{}\ldots{}\ldots{} 23
\end{flushleft}


\begin{flushleft}
9.5 Appeal Against Termination \ldots{}\ldots{}\ldots{}\ldots{}\ldots{}\ldots{}\ldots{}\ldots{}\ldots{}\ldots{}\ldots{}\ldots{}\ldots{}\ldots{}\ldots{}\ldots{}\ldots{}\ldots{}\ldots{}\ldots{}\ldots{}\ldots{}\ldots{}\ldots{}\ldots{}\ldots{}\ldots{}\ldots{}\ldots{}\ldots{}\ldots{}\ldots{}\ldots{}\ldots{}\ldots{}\ldots{}. 23
\end{flushleft}


\begin{flushleft}
10 Rules Governing Change or Addition to the Programme \ldots{}\ldots{}\ldots{}\ldots{}\ldots{}\ldots{}\ldots{}\ldots{}\ldots{}\ldots{}\ldots{}\ldots{}\ldots{}\ldots{}\ldots{}\ldots{}\ldots{}\ldots{}\ldots{}\ldots{}\ldots{}. 24
\end{flushleft}


\begin{flushleft}
10.1 Branch Change \ldots{}\ldots{}\ldots{}\ldots{}\ldots{}\ldots{}\ldots{}\ldots{}\ldots{}\ldots{}\ldots{}\ldots{}\ldots{}\ldots{}\ldots{}\ldots{}\ldots{}\ldots{}\ldots{}\ldots{}\ldots{}\ldots{}\ldots{}\ldots{}\ldots{}\ldots{}\ldots{}\ldots{}\ldots{}\ldots{}\ldots{}\ldots{}\ldots{}\ldots{}\ldots{}\ldots{}\ldots{}\ldots{}\ldots{}\ldots{}\ldots{}\ldots{}\ldots{} 24
\end{flushleft}


\begin{flushleft}
10.1.1 Eligibility \ldots{}\ldots{}\ldots{}\ldots{}\ldots{}\ldots{}\ldots{}\ldots{}\ldots{}\ldots{}\ldots{}\ldots{}\ldots{}\ldots{}\ldots{}\ldots{}\ldots{}\ldots{}\ldots{}\ldots{}\ldots{}\ldots{}\ldots{}\ldots{}\ldots{}\ldots{}\ldots{}\ldots{}\ldots{}\ldots{}\ldots{}\ldots{}\ldots{}\ldots{}\ldots{}\ldots{}\ldots{}\ldots{}\ldots{}\ldots{}\ldots{}\ldots{}. 24
\end{flushleft}


\begin{flushleft}
10.1.2 Application Process \ldots{}\ldots{}\ldots{}\ldots{}\ldots{}\ldots{}\ldots{}\ldots{}\ldots{}\ldots{}\ldots{}\ldots{}\ldots{}\ldots{}\ldots{}\ldots{}\ldots{}\ldots{}\ldots{}\ldots{}\ldots{}\ldots{}\ldots{}\ldots{}\ldots{}\ldots{}\ldots{}\ldots{}\ldots{}\ldots{}\ldots{}\ldots{}\ldots{}\ldots{}\ldots{}\ldots{} 24
\end{flushleft}


\begin{flushleft}
10.1.3 Academic Road-Map \ldots{}\ldots{}\ldots{}\ldots{}\ldots{}\ldots{}\ldots{}\ldots{}\ldots{}\ldots{}\ldots{}\ldots{}\ldots{}\ldots{}\ldots{}\ldots{}\ldots{}\ldots{}\ldots{}\ldots{}\ldots{}\ldots{}\ldots{}\ldots{}\ldots{}\ldots{}\ldots{}\ldots{}\ldots{}\ldots{}\ldots{}\ldots{}\ldots{}\ldots{}\ldots{}. 24
\end{flushleft}


\begin{flushleft}
10.2 Bachelors-Masters Dual Degree Programme \ldots{}\ldots{}\ldots{}\ldots{}\ldots{}\ldots{}\ldots{}\ldots{}\ldots{}\ldots{}\ldots{}\ldots{}\ldots{}\ldots{}\ldots{}\ldots{}\ldots{}\ldots{}\ldots{}\ldots{}\ldots{}\ldots{}\ldots{}\ldots{}\ldots{}\ldots{} 25
\end{flushleft}


\begin{flushleft}
10.2.1 Eligibility \ldots{}\ldots{}\ldots{}\ldots{}\ldots{}\ldots{}\ldots{}\ldots{}\ldots{}\ldots{}\ldots{}\ldots{}\ldots{}\ldots{}\ldots{}\ldots{}\ldots{}\ldots{}\ldots{}\ldots{}\ldots{}\ldots{}\ldots{}\ldots{}\ldots{}\ldots{}\ldots{}\ldots{}\ldots{}\ldots{}\ldots{}\ldots{}\ldots{}\ldots{}\ldots{}\ldots{}\ldots{}\ldots{}\ldots{}\ldots{}\ldots{}\ldots{}. 25
\end{flushleft}


\begin{flushleft}
10.2.2 Application Process \ldots{}\ldots{}\ldots{}\ldots{}\ldots{}\ldots{}\ldots{}\ldots{}\ldots{}\ldots{}\ldots{}\ldots{}\ldots{}\ldots{}\ldots{}\ldots{}\ldots{}\ldots{}\ldots{}\ldots{}\ldots{}\ldots{}\ldots{}\ldots{}\ldots{}\ldots{}\ldots{}\ldots{}\ldots{}\ldots{}\ldots{}\ldots{}\ldots{}\ldots{}\ldots{}\ldots{} 25
\end{flushleft}


\begin{flushleft}
10.2.3 Academic Road-Map \ldots{}\ldots{}\ldots{}\ldots{}\ldots{}\ldots{}\ldots{}\ldots{}\ldots{}\ldots{}\ldots{}\ldots{}\ldots{}\ldots{}\ldots{}\ldots{}\ldots{}\ldots{}\ldots{}\ldots{}\ldots{}\ldots{}\ldots{}\ldots{}\ldots{}\ldots{}\ldots{}\ldots{}\ldots{}\ldots{}\ldots{}\ldots{}\ldots{}\ldots{}\ldots{}. 25
\end{flushleft}


\begin{flushleft}
10.2.4 Withdrawal from the Bachelors-Masters Dual Degree Programme \ldots{}\ldots{}\ldots{}\ldots{}\ldots{}\ldots{}\ldots{}\ldots{}... 26
\end{flushleft}


\begin{flushleft}
10.2.5 Termination of PG Part of the Bachelors-Masters Dual Degree Programme \ldots{}\ldots{}\ldots{}\ldots{} 26
\end{flushleft}


\begin{flushleft}
10.3 Double Major \ldots{}\ldots{}\ldots{}\ldots{}\ldots{}\ldots{}\ldots{}\ldots{}\ldots{}\ldots{}\ldots{}\ldots{}\ldots{}\ldots{}\ldots{}\ldots{}\ldots{}\ldots{}\ldots{}\ldots{}\ldots{}\ldots{}\ldots{}\ldots{}\ldots{}\ldots{}\ldots{}\ldots{}\ldots{}\ldots{}\ldots{}\ldots{}\ldots{}\ldots{}\ldots{}\ldots{}\ldots{}\ldots{}\ldots{}\ldots{}\ldots{}\ldots{}\ldots{}.. 26
\end{flushleft}


\begin{flushleft}
10.3.1 Eligibility \ldots{}\ldots{}\ldots{}\ldots{}\ldots{}\ldots{}\ldots{}\ldots{}\ldots{}\ldots{}\ldots{}\ldots{}\ldots{}\ldots{}\ldots{}\ldots{}\ldots{}\ldots{}\ldots{}\ldots{}\ldots{}\ldots{}\ldots{}\ldots{}\ldots{}\ldots{}\ldots{}\ldots{}\ldots{}\ldots{}\ldots{}\ldots{}\ldots{}\ldots{}\ldots{}\ldots{}\ldots{}\ldots{}\ldots{}\ldots{}\ldots{}\ldots{} 26
\end{flushleft}


\begin{flushleft}
10.3.2 Application Process \ldots{}\ldots{}\ldots{}\ldots{}\ldots{}\ldots{}\ldots{}\ldots{}\ldots{}\ldots{}\ldots{}\ldots{}\ldots{}\ldots{}\ldots{}\ldots{}\ldots{}\ldots{}\ldots{}\ldots{}\ldots{}\ldots{}\ldots{}\ldots{}\ldots{}\ldots{}\ldots{}\ldots{}\ldots{}\ldots{}\ldots{}\ldots{}\ldots{}\ldots{}\ldots{}.. 26
\end{flushleft}


\begin{flushleft}
10.3.3 Academic Road-Map \ldots{}\ldots{}\ldots{}\ldots{}\ldots{}\ldots{}\ldots{}\ldots{}\ldots{}\ldots{}\ldots{}\ldots{}\ldots{}\ldots{}\ldots{}\ldots{}\ldots{}\ldots{}\ldots{}\ldots{}\ldots{}\ldots{}\ldots{}\ldots{}\ldots{}\ldots{}\ldots{}\ldots{}\ldots{}\ldots{}\ldots{}\ldots{}\ldots{}\ldots{}... 26
\end{flushleft}


\begin{flushleft}
10.3.4 Withdrawal from the Double Major Programme \ldots{}\ldots{}\ldots{}\ldots{}\ldots{}\ldots{}\ldots{}\ldots{}\ldots{}\ldots{}\ldots{}\ldots{}\ldots{}\ldots{}\ldots{}\ldots{}\ldots{}\ldots{}\ldots{} 27
\end{flushleft}


\begin{flushleft}
10.3.5 Termination of the Double Major Programme \ldots{}\ldots{}\ldots{}\ldots{}\ldots{}\ldots{}\ldots{}\ldots{}\ldots{}\ldots{}\ldots{}\ldots{}\ldots{}\ldots{}\ldots{}\ldots{}\ldots{}\ldots{}\ldots{}\ldots{}. 27
\end{flushleft}


\begin{flushleft}
10.4 Minor \ldots{}\ldots{}\ldots{}\ldots{}\ldots{}\ldots{}\ldots{}\ldots{}\ldots{}\ldots{}\ldots{}\ldots{}\ldots{}\ldots{}\ldots{}\ldots{}\ldots{}\ldots{}\ldots{}\ldots{}\ldots{}\ldots{}\ldots{}\ldots{}\ldots{}\ldots{}\ldots{}\ldots{}\ldots{}\ldots{}\ldots{}\ldots{}\ldots{}\ldots{}\ldots{}\ldots{}\ldots{}\ldots{}\ldots{}\ldots{}\ldots{}\ldots{}\ldots{}\ldots{}\ldots{}\ldots{}\ldots{}\ldots{}. 27
\end{flushleft}


\begin{flushleft}
10.4.1 Eligibility \ldots{}\ldots{}\ldots{}\ldots{}\ldots{}\ldots{}\ldots{}\ldots{}\ldots{}\ldots{}\ldots{}\ldots{}\ldots{}\ldots{}\ldots{}\ldots{}\ldots{}\ldots{}\ldots{}\ldots{}\ldots{}\ldots{}\ldots{}\ldots{}\ldots{}\ldots{}\ldots{}\ldots{}\ldots{}\ldots{}\ldots{}\ldots{}\ldots{}\ldots{}\ldots{}\ldots{}\ldots{}\ldots{}\ldots{}\ldots{}\ldots{}\ldots{}. 27
\end{flushleft}


\begin{flushleft}
10.4.2 Application Process \ldots{}\ldots{}\ldots{}\ldots{}\ldots{}\ldots{}\ldots{}\ldots{}\ldots{}\ldots{}\ldots{}\ldots{}\ldots{}\ldots{}\ldots{}\ldots{}\ldots{}\ldots{}\ldots{}\ldots{}\ldots{}\ldots{}\ldots{}\ldots{}\ldots{}\ldots{}\ldots{}\ldots{}\ldots{}\ldots{}\ldots{}\ldots{}\ldots{}\ldots{}\ldots{}\ldots{} 27
\end{flushleft}


\begin{flushleft}
10.4.3 Retrospective Minor \ldots{}\ldots{}\ldots{}\ldots{}\ldots{}\ldots{}\ldots{}\ldots{}\ldots{}\ldots{}\ldots{}\ldots{}\ldots{}\ldots{}\ldots{}\ldots{}\ldots{}\ldots{}\ldots{}\ldots{}\ldots{}\ldots{}\ldots{}\ldots{}\ldots{}\ldots{}\ldots{}\ldots{}\ldots{}\ldots{}\ldots{}\ldots{}\ldots{}\ldots{}\ldots{}. 28
\end{flushleft}


\begin{flushleft}
10.4.4 Academic Road-Map \ldots{}\ldots{}\ldots{}\ldots{}\ldots{}\ldots{}\ldots{}\ldots{}\ldots{}\ldots{}\ldots{}\ldots{}\ldots{}\ldots{}\ldots{}\ldots{}\ldots{}\ldots{}\ldots{}\ldots{}\ldots{}\ldots{}\ldots{}\ldots{}\ldots{}\ldots{}\ldots{}\ldots{}\ldots{}\ldots{}\ldots{}\ldots{}\ldots{}\ldots{}\ldots{}. 28
\end{flushleft}


\begin{flushleft}
10.4.5 Withdrawal from a Minor \ldots{}\ldots{}\ldots{}\ldots{}\ldots{}\ldots{}\ldots{}\ldots{}\ldots{}\ldots{}\ldots{}\ldots{}\ldots{}\ldots{}\ldots{}\ldots{}\ldots{}\ldots{}\ldots{}\ldots{}\ldots{}\ldots{}\ldots{}\ldots{}\ldots{}\ldots{}\ldots{}\ldots{}\ldots{}\ldots{}\ldots{}\ldots{}.. 28
\end{flushleft}


\begin{flushleft}
10.5 MSPD Dual Degree Programme \ldots{}\ldots{}\ldots{}\ldots{}\ldots{}\ldots{}\ldots{}\ldots{}\ldots{}\ldots{}\ldots{}\ldots{}\ldots{}\ldots{}\ldots{}\ldots{}\ldots{}\ldots{}\ldots{}\ldots{}\ldots{}\ldots{}\ldots{}\ldots{}\ldots{}\ldots{}\ldots{}\ldots{}\ldots{}\ldots{}\ldots{}\ldots{}\ldots{}. 28
\end{flushleft}


\begin{flushleft}
10.5.1 Eligibility \ldots{}\ldots{}\ldots{}\ldots{}\ldots{}\ldots{}\ldots{}\ldots{}\ldots{}\ldots{}\ldots{}\ldots{}\ldots{}\ldots{}\ldots{}\ldots{}\ldots{}\ldots{}\ldots{}\ldots{}\ldots{}\ldots{}\ldots{}\ldots{}\ldots{}\ldots{}\ldots{}\ldots{}\ldots{}\ldots{}\ldots{}\ldots{}\ldots{}\ldots{}\ldots{}\ldots{}\ldots{}\ldots{}\ldots{}\ldots{}\ldots{}\ldots{}. 28
\end{flushleft}


\begin{flushleft}
10.5.2 Application Process \ldots{}\ldots{}\ldots{}\ldots{}\ldots{}\ldots{}\ldots{}\ldots{}\ldots{}\ldots{}\ldots{}\ldots{}\ldots{}\ldots{}\ldots{}\ldots{}\ldots{}\ldots{}\ldots{}\ldots{}\ldots{}\ldots{}\ldots{}\ldots{}\ldots{}\ldots{}\ldots{}\ldots{}\ldots{}\ldots{}\ldots{}\ldots{}\ldots{}\ldots{}\ldots{}\ldots{} 28
\end{flushleft}


\begin{flushleft}
10.5.3 Academic Roadmap \ldots{}\ldots{}\ldots{}\ldots{}\ldots{}\ldots{}\ldots{}\ldots{}\ldots{}\ldots{}\ldots{}\ldots{}\ldots{}\ldots{}\ldots{}\ldots{}\ldots{}\ldots{}\ldots{}\ldots{}\ldots{}\ldots{}\ldots{}\ldots{}\ldots{}\ldots{}\ldots{}\ldots{}\ldots{}\ldots{}\ldots{}\ldots{}\ldots{}\ldots{}\ldots{}\ldots{} 28
\end{flushleft}


\begin{flushleft}
10.5.4 Withdrawal from the M.Sc.-Ph.D. Dual Degree Programme \ldots{}\ldots{}\ldots{}\ldots{}\ldots{}\ldots{}\ldots{}\ldots{}\ldots{}\ldots{}\ldots{}\ldots{}\ldots{} 29
\end{flushleft}


\begin{flushleft}
10.6 Calculation of Seat Availability \ldots{}\ldots{}\ldots{}\ldots{}\ldots{}\ldots{}\ldots{}\ldots{}\ldots{}\ldots{}\ldots{}\ldots{}\ldots{}\ldots{}\ldots{}\ldots{}\ldots{}\ldots{}\ldots{}\ldots{}\ldots{}\ldots{}\ldots{}\ldots{}\ldots{}\ldots{}\ldots{}\ldots{}\ldots{}\ldots{}\ldots{}\ldots{}\ldots{}... 29
\end{flushleft}


\begin{flushleft}
10.6.1 Branch Change \ldots{}\ldots{}\ldots{}\ldots{}\ldots{}\ldots{}\ldots{}\ldots{}\ldots{}\ldots{}\ldots{}\ldots{}\ldots{}\ldots{}\ldots{}\ldots{}\ldots{}\ldots{}\ldots{}\ldots{}\ldots{}\ldots{}\ldots{}\ldots{}\ldots{}\ldots{}\ldots{}\ldots{}\ldots{}\ldots{}\ldots{}\ldots{}\ldots{}\ldots{}\ldots{}\ldots{}\ldots{}\ldots{}.. 29
\end{flushleft}


\begin{flushleft}
10.6.2 Bachelors-Masters Dual Degree \ldots{}\ldots{}\ldots{}\ldots{}\ldots{}\ldots{}\ldots{}\ldots{}\ldots{}\ldots{}\ldots{}\ldots{}\ldots{}\ldots{}\ldots{}\ldots{}\ldots{}\ldots{}\ldots{}\ldots{}\ldots{}\ldots{}\ldots{}\ldots{}\ldots{}\ldots{}\ldots{}\ldots{}.. 29
\end{flushleft}


\begin{flushleft}
10.6.3 Double Major \ldots{}\ldots{}\ldots{}\ldots{}\ldots{}\ldots{}\ldots{}\ldots{}\ldots{}\ldots{}\ldots{}\ldots{}\ldots{}\ldots{}\ldots{}\ldots{}\ldots{}\ldots{}\ldots{}\ldots{}\ldots{}\ldots{}\ldots{}\ldots{}\ldots{}\ldots{}\ldots{}\ldots{}\ldots{}\ldots{}\ldots{}\ldots{}\ldots{}\ldots{}\ldots{}\ldots{}\ldots{}\ldots{}\ldots{}. 30
\end{flushleft}


\begin{flushleft}
10.6.4 Minor\ldots{}\ldots{}\ldots{}\ldots{}\ldots{}\ldots{}\ldots{}\ldots{}\ldots{}\ldots{}\ldots{}\ldots{}\ldots{}\ldots{}\ldots{}\ldots{}\ldots{}\ldots{}\ldots{}\ldots{}\ldots{}\ldots{}\ldots{}\ldots{}\ldots{}\ldots{}\ldots{}\ldots{}\ldots{}\ldots{}\ldots{}\ldots{}\ldots{}\ldots{}\ldots{}\ldots{}\ldots{}\ldots{}\ldots{}\ldots{}\ldots{}\ldots{}\ldots{}\ldots{} 30
\end{flushleft}


\begin{flushleft}
11 Leave of Absence \ldots{}\ldots{}\ldots{}\ldots{}\ldots{}\ldots{}\ldots{}\ldots{}\ldots{}\ldots{}\ldots{}\ldots{}\ldots{}\ldots{}\ldots{}\ldots{}\ldots{}\ldots{}\ldots{}\ldots{}\ldots{}\ldots{}\ldots{}\ldots{}\ldots{}\ldots{}\ldots{}\ldots{}\ldots{}\ldots{}\ldots{}\ldots{}\ldots{}\ldots{}\ldots{}\ldots{}\ldots{}\ldots{}\ldots{}\ldots{}\ldots{}\ldots{}\ldots{}\ldots{}. 31
\end{flushleft}


\begin{flushleft}
11.1 Mid-Semester Recess and Vacation \ldots{}\ldots{}\ldots{}\ldots{}\ldots{}\ldots{}\ldots{}\ldots{}\ldots{}\ldots{}\ldots{}\ldots{}\ldots{}\ldots{}\ldots{}\ldots{}\ldots{}\ldots{}\ldots{}\ldots{}\ldots{}\ldots{}\ldots{}\ldots{}\ldots{}\ldots{}\ldots{}\ldots{}\ldots{}\ldots{}\ldots{}.31
\end{flushleft}


\begin{flushleft}
11.2 Short Leave \ldots{}\ldots{}\ldots{}\ldots{}\ldots{}\ldots{}\ldots{}\ldots{}\ldots{}\ldots{}\ldots{}\ldots{}\ldots{}\ldots{}\ldots{}\ldots{}\ldots{}\ldots{}\ldots{}\ldots{}\ldots{}\ldots{}\ldots{}\ldots{}\ldots{}\ldots{}\ldots{}\ldots{}\ldots{}\ldots{}\ldots{}\ldots{}\ldots{}\ldots{}\ldots{}\ldots{}\ldots{}\ldots{}\ldots{}\ldots{}\ldots{}\ldots{}\ldots{}\ldots{}\ldots{} 31
\end{flushleft}





3





\begin{flushleft}
\newpage
11.3 Temporary Withdrawal/ Semester Leave \ldots{}\ldots{}\ldots{}\ldots{}\ldots{}\ldots{}\ldots{}\ldots{}\ldots{}\ldots{}\ldots{}\ldots{}\ldots{}\ldots{}\ldots{}\ldots{}\ldots{}\ldots{}\ldots{}\ldots{}\ldots{}\ldots{}\ldots{}\ldots{}\ldots{}\ldots{}\ldots{}\ldots{} 31
\end{flushleft}


\begin{flushleft}
11.4 Penalty for Unsanctioned or Excessive Leaves \ldots{}\ldots{}\ldots{}\ldots{}\ldots{}\ldots{}\ldots{}\ldots{}\ldots{}\ldots{}\ldots{}\ldots{}\ldots{}\ldots{}\ldots{}\ldots{}\ldots{}\ldots{}\ldots{}\ldots{}\ldots{}\ldots{}\ldots{}\ldots{}\ldots{}. 31
\end{flushleft}


\begin{flushleft}
11.5 Permission to proceed to other Institutions \ldots{}\ldots{}\ldots{}\ldots{}\ldots{}\ldots{}\ldots{}\ldots{}\ldots{}\ldots{}\ldots{}\ldots{}\ldots{}\ldots{}\ldots{}\ldots{}\ldots{}\ldots{}\ldots{}\ldots{}\ldots{}\ldots{}\ldots{}\ldots{}\ldots{}\ldots{}. 32
\end{flushleft}


\begin{flushleft}
11.5.1 Eligibility \ldots{}\ldots{}\ldots{}\ldots{}\ldots{}\ldots{}\ldots{}\ldots{}\ldots{}\ldots{}\ldots{}\ldots{}\ldots{}\ldots{}\ldots{}\ldots{}\ldots{}\ldots{}\ldots{}\ldots{}\ldots{}\ldots{}\ldots{}\ldots{}\ldots{}\ldots{}\ldots{}\ldots{}\ldots{}\ldots{}\ldots{}\ldots{}\ldots{}\ldots{}\ldots{}\ldots{}\ldots{}\ldots{}\ldots{}\ldots{}\ldots{}\ldots{}. 32
\end{flushleft}


\begin{flushleft}
11.5.2 Application Procedure \ldots{}\ldots{}\ldots{}\ldots{}\ldots{}\ldots{}\ldots{}\ldots{}\ldots{}\ldots{}\ldots{}\ldots{}\ldots{}\ldots{}\ldots{}\ldots{}\ldots{}\ldots{}\ldots{}\ldots{}\ldots{}\ldots{}\ldots{}\ldots{}\ldots{}\ldots{}\ldots{}\ldots{}\ldots{}\ldots{}\ldots{}\ldots{}\ldots{}\ldots{}. 32
\end{flushleft}


\begin{flushleft}
11.5.3 Transfer of Credits and Waiver in-lieu thereof \ldots{}\ldots{}\ldots{}\ldots{}\ldots{}\ldots{}\ldots{}\ldots{}\ldots{}\ldots{}\ldots{}\ldots{}\ldots{}\ldots{}\ldots{}\ldots{}\ldots{}\ldots{}\ldots{}\ldots{}.. 32
\end{flushleft}


\begin{flushleft}
12 Scholarships, Awards and Medals \ldots{}\ldots{}\ldots{}\ldots{}\ldots{}\ldots{}\ldots{}\ldots{}\ldots{}\ldots{}\ldots{}\ldots{}\ldots{}\ldots{}\ldots{}\ldots{}\ldots{}\ldots{}\ldots{}\ldots{}\ldots{}\ldots{}\ldots{}\ldots{}\ldots{}\ldots{}\ldots{}\ldots{}\ldots{}\ldots{}\ldots{}\ldots{}\ldots{}\ldots{}. 33
\end{flushleft}


\begin{flushleft}
12.1 Scholarships \ldots{}\ldots{}\ldots{}\ldots{}\ldots{}\ldots{}\ldots{}\ldots{}\ldots{}\ldots{}\ldots{}\ldots{}\ldots{}\ldots{}\ldots{}\ldots{}\ldots{}\ldots{}\ldots{}\ldots{}\ldots{}\ldots{}\ldots{}\ldots{}\ldots{}\ldots{}\ldots{}\ldots{}\ldots{}\ldots{}\ldots{}\ldots{}\ldots{}\ldots{}\ldots{}\ldots{}\ldots{}\ldots{}\ldots{}\ldots{}\ldots{}\ldots{}\ldots{}\ldots{}. 33
\end{flushleft}


\begin{flushleft}
12.2 Withdrawal of Scholarships \ldots{}\ldots{}\ldots{}\ldots{}\ldots{}\ldots{}\ldots{}\ldots{}\ldots{}\ldots{}\ldots{}\ldots{}\ldots{}\ldots{}\ldots{}\ldots{}\ldots{}\ldots{}\ldots{}\ldots{}\ldots{}\ldots{}\ldots{}\ldots{}\ldots{}\ldots{}\ldots{}\ldots{}\ldots{}\ldots{}\ldots{}\ldots{}\ldots{}\ldots{}\ldots{}.. 33
\end{flushleft}


\begin{flushleft}
12.3 Awards and Medals \ldots{}\ldots{}\ldots{}\ldots{}\ldots{}\ldots{}\ldots{}\ldots{}\ldots{}\ldots{}\ldots{}\ldots{}\ldots{}\ldots{}\ldots{}\ldots{}\ldots{}\ldots{}\ldots{}\ldots{}\ldots{}\ldots{}\ldots{}\ldots{}\ldots{}\ldots{}\ldots{}\ldots{}\ldots{}\ldots{}\ldots{}\ldots{}\ldots{}\ldots{}\ldots{}\ldots{}\ldots{}\ldots{}\ldots{}\ldots{}. 33
\end{flushleft}


\begin{flushleft}
13 Conduct and Discipline \ldots{}\ldots{}\ldots{}\ldots{}\ldots{}\ldots{}\ldots{}\ldots{}\ldots{}\ldots{}\ldots{}\ldots{}\ldots{}\ldots{}\ldots{}\ldots{}\ldots{}\ldots{}\ldots{}\ldots{}\ldots{}\ldots{}\ldots{}\ldots{}\ldots{}\ldots{}\ldots{}\ldots{}\ldots{}\ldots{}\ldots{}\ldots{}\ldots{}\ldots{}\ldots{}\ldots{}\ldots{}\ldots{}\ldots{}\ldots{}.. 34
\end{flushleft}


\begin{flushleft}
13.1 Code of Conduct \ldots{}\ldots{}\ldots{}\ldots{}\ldots{}\ldots{}\ldots{}\ldots{}\ldots{}\ldots{}\ldots{}\ldots{}\ldots{}\ldots{}\ldots{}\ldots{}\ldots{}\ldots{}\ldots{}\ldots{}\ldots{}\ldots{}\ldots{}\ldots{}\ldots{}\ldots{}\ldots{}\ldots{}\ldots{}\ldots{}\ldots{}\ldots{}\ldots{}\ldots{}\ldots{}\ldots{}\ldots{}\ldots{}\ldots{}\ldots{}\ldots{}\ldots{} 34
\end{flushleft}


\begin{flushleft}
13.2 Disciplinary Actions and Related Matter \ldots{}\ldots{}\ldots{}\ldots{}\ldots{}\ldots{}\ldots{}\ldots{}\ldots{}\ldots{}\ldots{}\ldots{}\ldots{}\ldots{}\ldots{}\ldots{}\ldots{}\ldots{}\ldots{}\ldots{}\ldots{}\ldots{}\ldots{}\ldots{}\ldots{}\ldots{}\ldots{}\ldots{}.. 34
\end{flushleft}


\begin{flushleft}
14 A Quick Guide for Students \ldots{}\ldots{}\ldots{}\ldots{}\ldots{}\ldots{}\ldots{}\ldots{}\ldots{}\ldots{}\ldots{}\ldots{}\ldots{}\ldots{}\ldots{}\ldots{}\ldots{}\ldots{}\ldots{}\ldots{}\ldots{}\ldots{}\ldots{}\ldots{}\ldots{}\ldots{}\ldots{}\ldots{}\ldots{}\ldots{}\ldots{}\ldots{}\ldots{}\ldots{}\ldots{}\ldots{}\ldots{}\ldots{} 36
\end{flushleft}


\begin{flushleft}
16 Waiver and Amendments \ldots{}\ldots{}\ldots{}\ldots{}\ldots{}\ldots{}\ldots{}\ldots{}\ldots{}\ldots{}\ldots{}\ldots{}\ldots{}\ldots{}\ldots{}\ldots{}\ldots{}\ldots{}\ldots{}\ldots{}\ldots{}\ldots{}\ldots{}\ldots{}\ldots{}\ldots{}\ldots{}\ldots{}\ldots{}\ldots{}\ldots{}\ldots{}\ldots{}\ldots{}\ldots{}\ldots{}\ldots{}\ldots{}\ldots{} 38
\end{flushleft}


\begin{flushleft}
16.1 Waiver \ldots{}\ldots{}\ldots{}\ldots{}\ldots{}\ldots{}\ldots{}\ldots{}\ldots{}\ldots{}\ldots{}\ldots{}\ldots{}\ldots{}\ldots{}\ldots{}\ldots{}\ldots{}\ldots{}\ldots{}\ldots{}\ldots{}\ldots{}\ldots{}\ldots{}\ldots{}\ldots{}\ldots{}\ldots{}\ldots{}\ldots{}\ldots{}\ldots{}\ldots{}\ldots{}\ldots{}\ldots{}\ldots{}\ldots{}\ldots{}\ldots{}\ldots{}\ldots{}\ldots{}\ldots{}\ldots{}\ldots{}. 38
\end{flushleft}


\begin{flushleft}
16.2 Amendments \ldots{}\ldots{}\ldots{}\ldots{}\ldots{}\ldots{}\ldots{}\ldots{}\ldots{}\ldots{}\ldots{}\ldots{}\ldots{}\ldots{}\ldots{}\ldots{}\ldots{}\ldots{}\ldots{}\ldots{}\ldots{}\ldots{}\ldots{}\ldots{}\ldots{}\ldots{}\ldots{}\ldots{}\ldots{}\ldots{}\ldots{}\ldots{}\ldots{}\ldots{}\ldots{}\ldots{}\ldots{}\ldots{}\ldots{}\ldots{}\ldots{}\ldots{}\ldots{}.. 38
\end{flushleft}





4





\begin{flushleft}
\newpage
Chapter 1
\end{flushleft}





\begin{flushleft}
Introduction
\end{flushleft}


\begin{flushleft}
The objectives of the undergraduate (UG) programmes at IIT Kanpur are:
\end{flushleft}


$\bullet$


$\bullet$


$\bullet$


$\bullet$





\begin{flushleft}
To provide the highest level of education in technology and science, and to produce competent, creative, and
\end{flushleft}


\begin{flushleft}
imaginative engineers and scientists
\end{flushleft}


\begin{flushleft}
To promote a spirit of free and objective enquiry, and development of knowledge in different disciplines
\end{flushleft}


\begin{flushleft}
To produce highly skilled technologists and scientists with well-honed managerial and entrepreneurial skills having
\end{flushleft}


\begin{flushleft}
team spirit and leadership qualities
\end{flushleft}


\begin{flushleft}
To increase student participation in nation building through technology development that is sensitive to local needs
\end{flushleft}





\begin{flushleft}
This manual sets out the procedures and requirements of the undergraduate programmes of study that fall under the
\end{flushleft}


\begin{flushleft}
purview of the Senate Under-Graduate Committee (SUGC), which include B Tech, BS, MSc, Double Major, and Dual Degree
\end{flushleft}


\begin{flushleft}
programmes. Following are the committees and administrative units in the institute that are directly concerned with the
\end{flushleft}


\begin{flushleft}
above programmes:
\end{flushleft}


\begin{flushleft}
Departmental Undergraduate Committee (DUGC): Each academic department constitutes this committee which consists of
\end{flushleft}


\begin{flushleft}
a Convener nominated by the Head of the department (in consultation with the faculty of the department), Head of the
\end{flushleft}


\begin{flushleft}
department, four to eight members of the faculty, and two student representatives. The student representatives are
\end{flushleft}


\begin{flushleft}
nominated by the undergraduate students of the department for a one-year period. The tenure of faculty members is two
\end{flushleft}


\begin{flushleft}
years, with half of them retiring each year (as decided by the procedure adopted by the department). The DUGC:
\end{flushleft}


\begin{flushleft}
$\bullet$ Advises the students about their curriculum
\end{flushleft}


\begin{flushleft}
$\bullet$ Advises them about academic opportunities
\end{flushleft}


\begin{flushleft}
$\bullet$ Monitors the progress of academically weak students
\end{flushleft}


\begin{flushleft}
$\bullet$ Handles any problem faced by students in their academic programmes
\end{flushleft}


\begin{flushleft}
Senate Undergraduate Committee (SUGC): This is a standing committee formed by the Senate to look after all the issues
\end{flushleft}


\begin{flushleft}
regarding institute-wide UG programmes. It makes recommendations to the senate on all academic issues including policy
\end{flushleft}


\begin{flushleft}
matters and specific problem instances related to UG students and UG programmes. Its constituents are the conveners of
\end{flushleft}


\begin{flushleft}
various DUGCs, PUGCs (Programme Under Graduate Conveners) where applicable, last SUGC chairperson (ex-officio), one
\end{flushleft}


\begin{flushleft}
Senate nominee, and four student representatives nominated by Student Senate. The chairperson is elected by the
\end{flushleft}


\begin{flushleft}
constituent members. The SUGC constitutes two subcommittees, namely, Academic Performance Evaluation Committee
\end{flushleft}


\begin{flushleft}
(APEC) and Core Curriculum Committee (CCC). The chairpersons of these subcommittees are nominated by the SUGC
\end{flushleft}


\begin{flushleft}
chairperson, and they, in turn, constitute their five-member committees from the faculty members of the SUGC in
\end{flushleft}


\begin{flushleft}
consultation with SUGC chairperson. The CCC coordinates and oversees various facets of the core curriculum. The APEC
\end{flushleft}


\begin{flushleft}
evaluates the academic performance of undergraduate students at the end of each semester and makes recommendations
\end{flushleft}


\begin{flushleft}
regarding their further Programme of studies and action to be taken in the case of academically deficient students. Both
\end{flushleft}


\begin{flushleft}
these subcommittees make their recommendations to the SUGC.
\end{flushleft}





5





\begin{flushleft}
\newpage
Chapter 2
\end{flushleft}





\begin{flushleft}
Programmes of Study
\end{flushleft}


\begin{flushleft}
2.1 Programmes for New Students
\end{flushleft}


\begin{flushleft}
2.1.1 Admission through JEE: Currently students are admitted to the following programmes. New programmes may be
\end{flushleft}


\begin{flushleft}
added as and when approved by the Senate.
\end{flushleft}


\begin{flushleft}
i.
\end{flushleft}





\begin{flushleft}
ii.
\end{flushleft}





\begin{flushleft}
Bachelor of Technology (B Tech): A 4-year Programme in Aerospace Engineering (AE), Biological Sciences and
\end{flushleft}


\begin{flushleft}
Bio-Engineering (BSBE), Chemical Engineering (CHE), Civil Engineering (CE), Computer Science and Engineering (CSE),
\end{flushleft}


\begin{flushleft}
Electrical Engineering (EE), Mechanical Engineering (ME), Material Science and Engineering (MSE), Earth Sciences (ES)
\end{flushleft}


\begin{flushleft}
(from 2017)
\end{flushleft}


\begin{flushleft}
Bachelor in Science (BS): A 4-year Programme in Chemistry (CHM), Economics (ECO), Mathematics and Scientific
\end{flushleft}


\begin{flushleft}
Computing (MTH), Physics (PHY)
\end{flushleft}





\begin{flushleft}
2.1.2 Admission through JAM:
\end{flushleft}


\begin{flushleft}
i.
\end{flushleft}


\begin{flushleft}
ii.
\end{flushleft}





\begin{flushleft}
Master of Science (MSc): A 2-year programme in CHE, MTH, PHY, STAT
\end{flushleft}


\begin{flushleft}
Master of Science-Doctor of Philosophy (MSPD Dual Degree): This programme is currently offered only in PHY
\end{flushleft}





\begin{flushleft}
2.2 Options for Already Enrolled Students
\end{flushleft}


\begin{flushleft}
A student already admitted to a programme described in section 2.1 may switch to another department or to a multidisciplinary programme through branch change, enhance his/her programme to a double-major or a dual-degree
\end{flushleft}


\begin{flushleft}
programme, or include a minor in his/her existing programme. The rules governing these changes are given in Chapter 10.
\end{flushleft}


\begin{flushleft}
2.2.1 Branch Change
\end{flushleft}


\begin{flushleft}
A B Tech/BS student may choose to apply for a change to a B Tech/BS program in any other discipline, including Multidisciplinary programme (B Tech) in Engineering Sciences which is offered in two streams (i) Engineering Science (Mechanics)
\end{flushleft}


\begin{flushleft}
and (ii) Engineering Science (Energy, Environment and Climate).
\end{flushleft}


\begin{flushleft}
2.2.2 Double Major
\end{flushleft}


\begin{flushleft}
This is a five year programme for a Bachelor's degree with majors in two departments/disciplines that offer B Tech/BS
\end{flushleft}


\begin{flushleft}
degrees. The first major is in the parent department, while the second major is in the department to which the student is
\end{flushleft}


\begin{flushleft}
admitted for this purpose after the 4th semester.
\end{flushleft}


\begin{flushleft}
2.2.3 Dual-Degree
\end{flushleft}


\begin{flushleft}
Bachelors-Masters Dual Degree: This is a five-year programme in which a student earns a Bachelors and a Master's degree.
\end{flushleft}


\begin{flushleft}
The dual degree programme has three categories:
\end{flushleft}


\begin{flushleft}
Category A: In this case both degrees are in the same discipline/department. The two combinations in this category are B
\end{flushleft}


\begin{flushleft}
Tech-M Tech and BS-MS.
\end{flushleft}


\begin{flushleft}
Category B: This is a two discipline/department programme where the Bachelor's degree is in the parent
\end{flushleft}


\begin{flushleft}
discipline/department and the Master's degree is in a different discipline/department. Thus six degree
\end{flushleft}


\begin{flushleft}
combinations are possible in this category: B Tech-M Tech, B Tech-MS, B Tech-MDES, BS- M Tech, BS-MS, and BSMDES.
\end{flushleft}


\begin{flushleft}
Category C: In this case the Bachelor's degree is in the parent discipline/department and the Master's degree is MBA.
\end{flushleft}


\begin{flushleft}
Thus the two possible combinations are B Tech-MBA and BS-MBA.
\end{flushleft}





6





\begin{flushleft}
\newpage
MSPD Dual Degree: A 2-year MSc student may apply for a dual degree combining Masters with a PhD in departments where
\end{flushleft}


\begin{flushleft}
this is allowed.
\end{flushleft}


\begin{flushleft}
2.2.4 Minors
\end{flushleft}


\begin{flushleft}
Students may include a specialization within a discipline other than their parent discipline during the regular 4-year
\end{flushleft}


\begin{flushleft}
Bachelor's programme. This specialization is called a Minor, and is acknowledged as such on a student's grade sheet. A Minor
\end{flushleft}


\begin{flushleft}
can be completed within the time and credits required for a student's regular 4-year Bachelors' programme. A list of Minors
\end{flushleft}


\begin{flushleft}
available to students may be found in the templates.
\end{flushleft}





7





\begin{flushleft}
\newpage
Chapter 3
\end{flushleft}





\begin{flushleft}
Admission Procedure and Rules
\end{flushleft}


\begin{flushleft}
3.1 For New Students
\end{flushleft}


\begin{flushleft}
3.1.1 B Tech and BS Programmes: The rules and procedures for both programmes are the same.
\end{flushleft}


\begin{flushleft}
i.
\end{flushleft}





\begin{flushleft}
General Admission: Admission to the B Tech and BS programmes is done once a year in the month of July through the
\end{flushleft}


\begin{flushleft}
Joint Entrance Examination (JEE) conducted on an all-India level by the IITs. The procedures and other requirements for
\end{flushleft}


\begin{flushleft}
admission are specified in the JEE Information Brochure brought out every year.
\end{flushleft}





\begin{flushleft}
ii.
\end{flushleft}





\begin{flushleft}
Admission against Reserved Seats: Reservation of seats for various categories shall be as prescribed by the Board of
\end{flushleft}


\begin{flushleft}
Governors subject to the current policy formulated from time to time by the Government of India. Details about the
\end{flushleft}


\begin{flushleft}
breakup of reserved seats for various categories can be found with the JEE office. The admission process for reserved
\end{flushleft}


\begin{flushleft}
seats is as stated below:
\end{flushleft}


\begin{flushleft}
1) SC and ST Candidates: Reserved seats are filled on the basis of JEE qualifying norms specified for them. In case these
\end{flushleft}


\begin{flushleft}
reserved seats remain vacant, other SC and ST candidates (who appeared in JEE and satisfy certain relaxed
\end{flushleft}


\begin{flushleft}
conditions) are offered admission to the Preparatory Course of one year duration in Physics, Chemistry,
\end{flushleft}


\begin{flushleft}
Mathematics and English. On completion of the preparatory course and passing of the examination conducted by
\end{flushleft}


\begin{flushleft}
the Institute, the candidates are offered admission to the first year of B Tech /BS programmes against the vacant
\end{flushleft}


\begin{flushleft}
reserved seats of the year of their appearance in JEE.
\end{flushleft}


\begin{flushleft}
2) OBC Candidates (Not belonging to Creamy Layer): Reserved seats are filled on the basis of JEE qualifying norms
\end{flushleft}


\begin{flushleft}
specified for them. In case these reserved seats remain vacant, other candidates may be offered admission.
\end{flushleft}


\begin{flushleft}
3) PwD (Person with Disability) Candidates: Reserved seats are filled on the basis of JEE qualifying norms specified for
\end{flushleft}


\begin{flushleft}
them. In case these reserved seats remain vacant, other candidates in their respective categories may be offered
\end{flushleft}


\begin{flushleft}
admission.
\end{flushleft}





\begin{flushleft}
iii. Admission with Advanced Standing: Normally, admissions are made to the first year of the B Tech and BS programmes.
\end{flushleft}


\begin{flushleft}
However, under exceptional circumstances, the Senate may admit a student with advanced standing (up to a maximum
\end{flushleft}


\begin{flushleft}
of four semesters) on the basis of her/his partial completion of a similar programme elsewhere.
\end{flushleft}


\begin{flushleft}
3.1.2 M.Sc. and MSPD Programmes
\end{flushleft}


\begin{flushleft}
i.
\end{flushleft}





\begin{flushleft}
General Admission: Admission is done once a year in the month of July through the Joint Admission Test for MS (JAM)
\end{flushleft}


\begin{flushleft}
conducted on an all India level. The minimum academic qualification for admission is a B.Sc. degree (or equivalent) from
\end{flushleft}


\begin{flushleft}
a recognized university. The procedures and other requirements for admission are specified in the JAM Information
\end{flushleft}


\begin{flushleft}
Brochure brought out every year.
\end{flushleft}





\begin{flushleft}
ii.
\end{flushleft}





\begin{flushleft}
Admission against Reserved Seats: Details of the reservation of seats among various categories as approved by the Board
\end{flushleft}


\begin{flushleft}
of Governors subject to the current policy formulated from time to time by the Government of India may be found with
\end{flushleft}


\begin{flushleft}
the JAM office. The admission process for the reserved seats is as stated below:
\end{flushleft}


\begin{flushleft}
1) SC and ST Candidates' reserved seats are filled on the basis of JAM qualifying norms specified for them.
\end{flushleft}


\begin{flushleft}
2) Other Backward classes (OBC) candidates' (Not belonging to creamy layer) reserved seats are filled on the basis of
\end{flushleft}


\begin{flushleft}
JAM qualifying norms specified for them.
\end{flushleft}


\begin{flushleft}
3) PwD (Person with Disability) Candidates' reserved seats are filled on the basis of JAM qualifying norms specified for
\end{flushleft}


\begin{flushleft}
them. In case these reserved seats remain vacant, other candidates in their respective categories are offered
\end{flushleft}


\begin{flushleft}
admission.
\end{flushleft}





8





\begin{flushleft}
\newpage
3.2 Non-Degree Students
\end{flushleft}


\begin{flushleft}
A non-degree student is a student who is registered for a degree in a recognized Institute (other than IIT Kanpur) or a
\end{flushleft}


\begin{flushleft}
University in India or abroad, and who is officially sponsored by that Institute or University to complete a part of her/his
\end{flushleft}


\begin{flushleft}
academic programme at IIT Kanpur. For that purpose, the non-degree student may carry out research, take courses for credit
\end{flushleft}


\begin{flushleft}
or otherwise, or may use other academic facilities at IIT Kanpur. An official transcript of work done at IIT Kanpur, along with
\end{flushleft}


\begin{flushleft}
grades obtained, if any, would be given to the non-degree student for her/his use as s/he may deem appropriate. However,
\end{flushleft}


\begin{flushleft}
any credits earned at the Institute by a non-degree student cannot be counted towards any degree programme of IIT Kanpur
\end{flushleft}


\begin{flushleft}
at any time.
\end{flushleft}


\begin{flushleft}
A person will be admitted as a non-degree student on the basis of a sponsored application to the Dean of Academic Affairs,
\end{flushleft}


\begin{flushleft}
who will recommend for admission on the approval of the Chairperson Senate. The Chairperson's decision will be based on
\end{flushleft}


\begin{flushleft}
the advice of the concerned DUGC and the SUGC. A non-degree student may be admitted for a maximum period of one year
\end{flushleft}


\begin{flushleft}
only. The strength of non-degree students in any programme should not be more than 5\% of the programme strength.
\end{flushleft}


\begin{flushleft}
A non-degree student will be required to pay all applicable fees depending upon the status, programme and nationality.
\end{flushleft}


\begin{flushleft}
Students so admitted will be governed by all rules, regulations and discipline of IIT Kanpur.
\end{flushleft}





\begin{flushleft}
3.3 Validity of Admission and Its Cancellation
\end{flushleft}


\begin{flushleft}
Admission to any undergraduate programme requires that the applicant fulfill all three of the following conditions:
\end{flushleft}


\begin{flushleft}
i. Be eligible
\end{flushleft}


\begin{flushleft}
ii. Go through the laid down admission procedure
\end{flushleft}


\begin{flushleft}
iii. Pay the prescribed fees
\end{flushleft}


\begin{flushleft}
All admissions to undergraduate programmes should be formally approved by the Senate.
\end{flushleft}


\begin{flushleft}
All students admitted provisionally or otherwise to any programme shall submit copies of their mark sheets, provisional
\end{flushleft}


\begin{flushleft}
certificates, etc. of the qualifying examination and other documents by the last date specified for the purpose in the
\end{flushleft}


\begin{flushleft}
Academic Calendar. The Senate can cancel the admission of any student who fails to submit the prescribed documents by the
\end{flushleft}


\begin{flushleft}
specified date or to meet other stipulated requirement(s). The Senate may also cancel admission at any later time if it is
\end{flushleft}


\begin{flushleft}
found that the student had supplied false information or suppressed relevant information while seeking admission.
\end{flushleft}





9





\begin{flushleft}
\newpage
Chapter 4
\end{flushleft}





\begin{flushleft}
Academic Session
\end{flushleft}


\begin{flushleft}
4.1 Dates
\end{flushleft}


\begin{flushleft}
The academic session normally runs from the end of July in one year to the middle of July in the next year. It is divided into
\end{flushleft}


\begin{flushleft}
three parts:
\end{flushleft}


\begin{flushleft}
Semester I: From the fourth week of July to the last week of November
\end{flushleft}


\begin{flushleft}
Semester II: From the last week of December to the last week of April
\end{flushleft}


\begin{flushleft}
Summer Term (not a regular semester): From the middle of May to the middle of July
\end{flushleft}





\begin{flushleft}
4.2 Duration
\end{flushleft}


\begin{flushleft}
Each of the two regular semesters consists of about eighteen weeks including one week of mid-semester recess. About nine
\end{flushleft}


\begin{flushleft}
working days of each semester are used for the end-semester examination and one week period during the semester is
\end{flushleft}


\begin{flushleft}
utilized for the mid-semester examination. The first day of classes in a regular semester and the first day of the end-semester
\end{flushleft}


\begin{flushleft}
examination should normally be a Monday. The equivalent of fourteen weeks is devoted to teaching which excludes all
\end{flushleft}


\begin{flushleft}
holidays and days spent on both the examinations, in each semester. The Summer Term consists of eight teaching weeks, not
\end{flushleft}


\begin{flushleft}
including holidays and examinations days.
\end{flushleft}





\begin{flushleft}
4.3 Academic Calendar
\end{flushleft}


\begin{flushleft}
The dates of all academic activities including those of registration, late registration, last date of document submission, first
\end{flushleft}


\begin{flushleft}
and the last days of classes, add-drop of courses, examinations, make-up examination, deadline for final grade submission,
\end{flushleft}


\begin{flushleft}
conversion of I grade, mid-semester recess, and vacation are published in the Academic Calendar every year by the DoAA
\end{flushleft}


\begin{flushleft}
office. The Academic Calendar is available on the DoAA website.
\end{flushleft}





10





\begin{flushleft}
\newpage
Chapter 5
\end{flushleft}





\begin{flushleft}
List of Courses
\end{flushleft}


\begin{flushleft}
5.1 B Tech and BS Programmes
\end{flushleft}


\begin{flushleft}
The entire curriculum is divided into several parts.
\end{flushleft}


\begin{flushleft}
Institute Core (IC): This is a compulsory set of courses for all B Tech /BS student which includes basic courses in Physics,
\end{flushleft}


\begin{flushleft}
Chemistry, Mathematics, Biological Sciences, Computing, Electronics, Engineering Graphics, Manufacturing Processes,
\end{flushleft}


\begin{flushleft}
Communication Skills and Physical Education
\end{flushleft}


\begin{flushleft}
ESO/SO electives: These are elective courses from a basket of Science/Engineering-Science options.
\end{flushleft}


\begin{flushleft}
HSS electives: These are elective courses from a basket of courses in the Humanities and Social Sciences.
\end{flushleft}


\begin{flushleft}
Open electives (OE): These are elective courses that students may take from any department/programme in the Institute.
\end{flushleft}


\begin{flushleft}
The only condition is that they should not do more than one course on the same topic and at the same level of coverage.
\end{flushleft}


\begin{flushleft}
They are meant to widen the student's knowledge beyond the parent discipline.
\end{flushleft}


\begin{flushleft}
Departmental compulsory courses (DC): This is the compulsory set of courses for Bachelor's students in their parent
\end{flushleft}


\begin{flushleft}
discipline.
\end{flushleft}


\begin{flushleft}
Departmental elective (DE): These are elective courses that students have to take from within their parent discipline.
\end{flushleft}


\begin{flushleft}
The details of the programmes may be found in the course templates available from the DoAA website.
\end{flushleft}


\begin{flushleft}
5.1.1 Double Major
\end{flushleft}


\begin{flushleft}
Students who opt for a Double Major are required to complete Departmental Compulsory Courses and, in some cases,
\end{flushleft}


\begin{flushleft}
Departmental Electives in their second major discipline, in addition to completing all the requirements of their parent
\end{flushleft}


\begin{flushleft}
discipline. The courses taken in the second discipline will provide a knowledge-base in this discipline which is comparable to
\end{flushleft}


\begin{flushleft}
students' knowledge-base in their parent discipline. The Institute Core courses and the HSS electives will be done only once.
\end{flushleft}


\begin{flushleft}
The course templates for the double Major programme of each department may be found with the regular four year
\end{flushleft}


\begin{flushleft}
programme templates at the DoAA website.
\end{flushleft}





\begin{flushleft}
5.2 Bachelors-Masters Dual Degree Programme
\end{flushleft}


\begin{flushleft}
This programme requires students to complete all their Bachelors programme requirements in addition to completing
\end{flushleft}


\begin{flushleft}
required post-graduate courses to gain a strong foundation in their chosen Masters' discipline. These may include seminar
\end{flushleft}


\begin{flushleft}
courses in certain disciplines. In addition, the Masters' part of the programme also requires a thesis or project in certain
\end{flushleft}


\begin{flushleft}
disciplines to provide students with research experience. The course template of each department, available from the DoAA
\end{flushleft}


\begin{flushleft}
website, gives the details of the courses and credits to be completed by the Bachelors-Masters Dual Degree students.
\end{flushleft}





\begin{flushleft}
5.3 MSc Programme
\end{flushleft}


\begin{flushleft}
This programme requires students to take a set of compulsory courses designed to lay a strong foundation in the discipline. In
\end{flushleft}


\begin{flushleft}
addition, a few elective courses are to be credited to develop and pursue an area of specialization. In some departments
\end{flushleft}


\begin{flushleft}
there is a project or a thesis, while in other departments projects may be replaced by elective courses. In some programmes,
\end{flushleft}


\begin{flushleft}
seminars are also included in the curriculum. The templates of MSc programme are available from the DoAA website.
\end{flushleft}





11





\begin{flushleft}
\newpage
5.4 MSPD Dual Degree Programme
\end{flushleft}


\begin{flushleft}
Students are required to take a set of compulsory courses designed to lay a strong foundation in the discipline. In addition,
\end{flushleft}


\begin{flushleft}
some elective courses are to be credited to develop and pursue an area of specialization. The students are initiated into
\end{flushleft}


\begin{flushleft}
research methodology quite early. The programme is aimed to provide young motivated individuals with rigorous training,
\end{flushleft}


\begin{flushleft}
desired level of understanding and scientific maturity and solid base at an early stage to enable them to pursue a research
\end{flushleft}


\begin{flushleft}
career. The templates of MSPD Dual Degree programme are available at the DoAA website.
\end{flushleft}





\begin{flushleft}
5.5 List of Courses
\end{flushleft}


\begin{flushleft}
Details of various courses for undergraduate programmes being offered by various departments are contained in the
\end{flushleft}


\begin{flushleft}
COURSES OF STUDY bulletin, published periodically by the Institute, and available from the DoAA website.
\end{flushleft}





12





\begin{flushleft}
\newpage
Chapter 6
\end{flushleft}





\begin{flushleft}
Registration
\end{flushleft}


\begin{flushleft}
Each admitted student is required to register before the commencement of each semester/summer to study during that
\end{flushleft}


\begin{flushleft}
period in the Institute. New students who await the final result of the qualifying examination are allowed to register
\end{flushleft}


\begin{flushleft}
provisionally on submission of a certificate from their last institution stating that they have appeared in the final
\end{flushleft}


\begin{flushleft}
examinations (both theory and practical). Such students are required to submit documents of having passed the qualifying
\end{flushleft}


\begin{flushleft}
examination by the last date given in the Academic Calendar, failing which their admission shall be cancelled.
\end{flushleft}


\begin{flushleft}
There are two parts to the registration process: academic registration and administrative registration. The responsibility for
\end{flushleft}


\begin{flushleft}
completing both parts of this process rests with the students. If a student fails to complete the registration process within the
\end{flushleft}


\begin{flushleft}
specified time, then her/his programme shall be terminated by the Senate.
\end{flushleft}





\begin{flushleft}
6.1 Academic Registration
\end{flushleft}


\begin{flushleft}
This involves the selection of courses consistent with the specific credit requirements detailed in the programme template
\end{flushleft}


\begin{flushleft}
and subject to some rules described below. Within these rules, students have flexibility of choice regarding which courses to
\end{flushleft}


\begin{flushleft}
do within a specific semester. The academic registration process gets completed after the DUGC convener approves the
\end{flushleft}


\begin{flushleft}
registration form. A registration is considered valid only if there is no time-table conflict between the courses for which the
\end{flushleft}


\begin{flushleft}
student has registered.
\end{flushleft}


\begin{flushleft}
6.1.1 Pre-Registration
\end{flushleft}


\begin{flushleft}
Every student must pre-register for the next semester at the time specified in the Academic Calendar for this purpose within
\end{flushleft}


\begin{flushleft}
the current semester. Pre-registration is done entirely online. Students may directly register for compulsory core courses at
\end{flushleft}


\begin{flushleft}
the time specified in the template. For core courses in semesters other than those specified in the template as well as all
\end{flushleft}


\begin{flushleft}
other courses, students must make an online request to the concerned course instructor. If the instructor accepts the
\end{flushleft}


\begin{flushleft}
request, students may add the course to their registration form. Otherwise, they must make a request for other course(s).
\end{flushleft}


\begin{flushleft}
Students may select the number of courses permitted by course load rules, while ensuring that all pre-requisites have been
\end{flushleft}


\begin{flushleft}
completed and there is no timetable clash amongst the courses. Courses with a timetable clash will be dropped automatically
\end{flushleft}


\begin{flushleft}
from students' online registration form. When all the courses they plan to do in the next semester have been accepted by the
\end{flushleft}


\begin{flushleft}
concerned instructors, students need to submit the online registration form for the approval of the DUGC.
\end{flushleft}


\begin{flushleft}
If due to an unavoidable reason, a student is unable to do academic pre-registration during the specified period, the student
\end{flushleft}


\begin{flushleft}
should put in an application to the Chairperson SUGC for permission to do manual final registration at the specified time at
\end{flushleft}


\begin{flushleft}
the beginning of the semester for which the registration is to be done. This application should be submitted to the
\end{flushleft}


\begin{flushleft}
Chairperson SUGC within a month of the last date for academic pre-registration. If the application is approved, the student
\end{flushleft}


\begin{flushleft}
may do manual registration during the Final Registration period.
\end{flushleft}


\begin{flushleft}
6.1.2 Final Registration
\end{flushleft}


\begin{flushleft}
Academic pre-registration is compulsory for all students. However, if there is a problem with a student's pre-registration, the
\end{flushleft}


\begin{flushleft}
student may do manual academic registration at the beginning of the new semester on the specified day in the Academic
\end{flushleft}


\begin{flushleft}
Calendar during the Final Registration period. Students may use the Final Registration option only under the following
\end{flushleft}


\begin{flushleft}
circumstances:
\end{flushleft}


\begin{flushleft}
i.
\end{flushleft}


\begin{flushleft}
ii.
\end{flushleft}





\begin{flushleft}
Their pre-registration has been cancelled.
\end{flushleft}


\begin{flushleft}
They did not do online pre-registration at the specified time in the previous semester. In such cases, if they do not have
\end{flushleft}


\begin{flushleft}
an official prior permission from the Chairperson SUGC to do manual final registration, they will need to pay a fine to
\end{flushleft}


\begin{flushleft}
avail the Final Registration option.
\end{flushleft}





13





\begin{flushleft}
\newpage
6.1.3 Add-Drop of Courses
\end{flushleft}


\begin{flushleft}
Students may add or drop courses using the online registration system during the period specified for this purpose in the
\end{flushleft}


\begin{flushleft}
Academic Calendar. Each add/drop request needs to be accepted by the concerned course instructor. Once an add/drop
\end{flushleft}


\begin{flushleft}
request has been accepted, students need to change their online registration form accordingly and submit it to the DUGC for
\end{flushleft}


\begin{flushleft}
final approval.
\end{flushleft}


\begin{flushleft}
Students may also request to drop course(s) up to about four weeks prior to the last date of classes (exact date is specified in
\end{flushleft}


\begin{flushleft}
Academic Calendar) with the following conditions:
\end{flushleft}


\begin{flushleft}
a)
\end{flushleft}





\begin{flushleft}
Dropping of course(s) should not result in the net registration becoming less than the specified minimum number of
\end{flushleft}


\begin{flushleft}
credits for a semester.
\end{flushleft}


\begin{flushleft}
b) The request to drop course(s) must be approved by the Instructor-in-charge of the course and the Convener, DUGC.
\end{flushleft}


\begin{flushleft}
c) Total number of credits of courses dropped beyond the add-drop week should not exceed 44 during the entire
\end{flushleft}


\begin{flushleft}
programme. This applies to all UG programmes.
\end{flushleft}


\begin{flushleft}
Adding of courses is not permitted in the summer term. However, students may drop a course up to two weeks prior to the
\end{flushleft}


\begin{flushleft}
last day of classes.
\end{flushleft}


\begin{flushleft}
6.1.4 Cancellation of Registration in a Course
\end{flushleft}


\begin{flushleft}
Registration of a student in a course may be cancelled at any stage if it is found that s/he does not meet the pre-requisites of
\end{flushleft}


\begin{flushleft}
the course, or if there is a clash in the student's time table preventing her/him from attending the course or if it is found that
\end{flushleft}


\begin{flushleft}
s/he is not eligible to register for that course for any other reason.
\end{flushleft}


\begin{flushleft}
An instructor of a course may also recommend cancellation of registration of any student in that course for reasons such as
\end{flushleft}


\begin{flushleft}
absence from classes without proper authorization. The instructor may recommend such de-registration of students up to
\end{flushleft}


\begin{flushleft}
four weeks prior to the last day of classes. The instructor should send de-registration recommendations to the SUGC
\end{flushleft}


\begin{flushleft}
Chairperson. The same information should also be sent to the concerned DUGC. The SUGC Chairperson's decision in each
\end{flushleft}


\begin{flushleft}
case shall be conveyed to the instructor and the student at least two weeks prior to the last day of classes.
\end{flushleft}


\begin{flushleft}
6.1.5 Academic Load in Regular Semesters
\end{flushleft}


\begin{flushleft}
Each course carries a weightage in terms of credits indicating the approximate number of contact hours (lectures and
\end{flushleft}


\begin{flushleft}
tutorials and/or laboratory hours) as well as self-study hours per week required for the course. Credit calculation for a course
\end{flushleft}


\begin{flushleft}
is done by the following formula: C=3L+2T+P+A, where C is the number of credits, L is the number of lecture hours, T is the
\end{flushleft}


\begin{flushleft}
number of tutorial hours, P is the number of laboratory hours, and A is the additional number of hours needed for
\end{flushleft}


\begin{flushleft}
assignments and projects, as decided at the time of approval of the course.
\end{flushleft}


\begin{flushleft}
Normal academic load is fifty credits per semester. Students may register for up to 30 percent less or 30 percent more credits
\end{flushleft}


\begin{flushleft}
than the normal load. That is, students may register for 35-65 credits.
\end{flushleft}


\begin{flushleft}
6.1.5.1 Exceptions to Regular Rules regarding Academic Load
\end{flushleft}


\begin{flushleft}
Under-load: Students who are identified as academically deficient (on academic probation) may register for a minimum of 30
\end{flushleft}


\begin{flushleft}
credits.
\end{flushleft}


\begin{flushleft}
Over-load: In special cases students can register for courses beyond the graduation requirement.
\end{flushleft}


\begin{flushleft}
i.
\end{flushleft}





\begin{flushleft}
Students in the advanced stage of their programme may register for extra courses (within the 65 upper credit limit) that
\end{flushleft}


\begin{flushleft}
are not part of their graduation requirements. The rules regarding such registration are as follows:
\end{flushleft}


\begin{flushleft}
1) Students are allowed to take extra course(s) only when they need 100 or less additional credits to complete all the
\end{flushleft}


\begin{flushleft}
requirements of their programme.
\end{flushleft}


\begin{flushleft}
2) Such extra course(s) may only be taken with the consent of the course instructor(s).
\end{flushleft}


\begin{flushleft}
3) Students should submit the list of the extra course(s) to the DoAA office.
\end{flushleft}





14





\begin{flushleft}
\newpage
4) At the time of registration in the extra course(s), the student has to declare whether s/he would do the extra
\end{flushleft}


\begin{flushleft}
course(s) on the basis of a letter Grade (A-F) or pass/fail (S/X).
\end{flushleft}


\begin{flushleft}
5) All such extra courses will be shown on the student's transcript.
\end{flushleft}


\begin{flushleft}
6) If the student chooses to do the extra course(s) on the basis of a letter grade, then the letter grade received in such
\end{flushleft}


\begin{flushleft}
extra course(s) will be counted towards SPI/CPI.
\end{flushleft}


\begin{flushleft}
ii.
\end{flushleft}





\begin{flushleft}
Any student with a CPI of 8.5 or higher may request registration for more than 65 credits in a semester. The rules
\end{flushleft}


\begin{flushleft}
regarding such registration are as follows.
\end{flushleft}


\begin{flushleft}
1) A student has to declare which course s/he intends to register as extra load at the time of the registration.
\end{flushleft}


\begin{flushleft}
2) If a student registers for additional load, then the total credits for that semester may be up to 70 credits.
\end{flushleft}


\begin{flushleft}
3) These additional courses will not count towards satisfying their graduation requirement.
\end{flushleft}


\begin{flushleft}
4) The student may choose to take such courses on the basis of a letter grade (A-F) or pass/fail (S/X). The grades earned
\end{flushleft}


\begin{flushleft}
in the overload courses will be shown on the student's transcript. But these grades will not be included in the
\end{flushleft}


\begin{flushleft}
calculation of SPI/CPI.
\end{flushleft}





\begin{flushleft}
6.1.6 Academic Load in Summer Term
\end{flushleft}


\begin{flushleft}
Students may register for a maximum of 25 credits in a summer term when advised by the DUGC. The summer term is not a
\end{flushleft}


\begin{flushleft}
regular semester and only academically deficient students (who have low CPI and/or less credits than advised by the
\end{flushleft}


\begin{flushleft}
template till that point in the programme), Double Major students, Bachelors-Masters Dual Degree students (only after their
\end{flushleft}


\begin{flushleft}
eighth semester) and such students who are short of only one or two courses to complete all graduation requirements may
\end{flushleft}


\begin{flushleft}
register during this period. For core courses, only students with failed backlog may be considered for summer registration. In
\end{flushleft}


\begin{flushleft}
case any vacancies are left in the courses being offered during the summer term once the requirements of the above
\end{flushleft}


\begin{flushleft}
categories of students are fulfilled. Other students may register for these courses to fulfill some of their graduation
\end{flushleft}


\begin{flushleft}
requirements in advance.
\end{flushleft}


\begin{flushleft}
6.1.7 Cancellation of Registration
\end{flushleft}


\begin{flushleft}
If a student is found to be absent from all academic activities for more than 20 working days (not necessarily contiguous) in a
\end{flushleft}


\begin{flushleft}
semester with or without sanction, then his/her registration from all the courses in that semester will be cancelled. The
\end{flushleft}


\begin{flushleft}
corresponding number of days of absence for a summer term is 10. In such cases, the result is a forced semester drop.
\end{flushleft}





\begin{flushleft}
6.2 Administrative Registration
\end{flushleft}


\begin{flushleft}
This involves two steps:
\end{flushleft}


\begin{flushleft}
i. Payment of fees and clearance of outstanding dues (if any)
\end{flushleft}


\begin{flushleft}
ii. Signing the registration roll in the office of the Dean of Students' Affairs (DOSA)
\end{flushleft}





\begin{flushleft}
6.3 Late Registration
\end{flushleft}


\begin{flushleft}
Students are expected to complete the registration process (both academic and administrative) by the date specified in the
\end{flushleft}


\begin{flushleft}
Academic Calendar. In exceptional circumstances they may be allowed to complete the process by the date of late
\end{flushleft}


\begin{flushleft}
registration after paying the late registration fee. This fee may be waived if prior permission for late registration is obtained.
\end{flushleft}


\begin{flushleft}
Besides, it may also be waived in a case of unexpected events, such as illness or family emergency, when it may not be
\end{flushleft}


\begin{flushleft}
possible to take prior permission.
\end{flushleft}





15





\begin{flushleft}
\newpage
Chapter 7
\end{flushleft}





\begin{flushleft}
Teaching and Evaluation
\end{flushleft}


\begin{flushleft}
7.1 Teaching
\end{flushleft}


\begin{flushleft}
7.1.1 Medium of Instruction
\end{flushleft}


\begin{flushleft}
The medium of instruction is English. All students admitted to the first year of the B Tech and BS programmes are required to
\end{flushleft}


\begin{flushleft}
take a diagnostic test in English. Based on their performance, selected students are required to credit a course in English
\end{flushleft}


\begin{flushleft}
Language and Communication Skills as a part of their HSS course requirement.
\end{flushleft}


\begin{flushleft}
7.1.2 Offering a New Course
\end{flushleft}


\begin{flushleft}
Any faculty member can offer a new course by submitting a new-course proposal in the format described in the proposal for
\end{flushleft}


\begin{flushleft}
new courses form available from the DoAA website. The proposal must be submitted to the concerned DUGC convener. The
\end{flushleft}


\begin{flushleft}
convener must circulate the proposal among the entire academic staff of the institute through email, requesting them to
\end{flushleft}


\begin{flushleft}
send any feedback to the course proposer and/or the DUGC convener. Three weeks after the proposal is circulated, the
\end{flushleft}


\begin{flushleft}
course proposer should make suitable amendments based on the feedback received. Finally the modified proposal, details of
\end{flushleft}


\begin{flushleft}
the feedback, and details of how they were addressed must be resubmitted to the DUGC convener. The convener must
\end{flushleft}


\begin{flushleft}
forward the same with the DUGC comments, if any, to the SUGC chairperson. A new course can be offered only if it has been
\end{flushleft}


\begin{flushleft}
approved by the SUGC before the pre-registration for the concerned semester begins.
\end{flushleft}


\begin{flushleft}
7.1.3 Courses Offerings for a Given Semester
\end{flushleft}


\begin{flushleft}
The list of courses to be offered by a department / programme in the next semester is finalised before the pre-registration
\end{flushleft}


\begin{flushleft}
period in the current semester by the Head in consultation with the faculty. For the summer term, this list is finalised before
\end{flushleft}


\begin{flushleft}
the registration date for the summer term. The courses to be offered are decided by taking into consideration all the
\end{flushleft}


\begin{flushleft}
requirements of the programme templates.
\end{flushleft}


\begin{flushleft}
7.1.4 Duration of Courses
\end{flushleft}


\begin{flushleft}
Courses may be for a full semester or half a semester. A full semester course typically has 40 lectures of 50 minutes each or
\end{flushleft}


\begin{flushleft}
any other combination of equivalent time. Half semester courses, also called modular courses, are taught in half of the above
\end{flushleft}


\begin{flushleft}
time.
\end{flushleft}


\begin{flushleft}
7.1.5 Conduct of Courses
\end{flushleft}


\begin{flushleft}
Each course is conducted by the Instructor-in-charge with the assistance of the required number of instructors, tutors, and
\end{flushleft}


\begin{flushleft}
teaching assistants. The Instructor-in-charge is responsible for conducting the course, holding the examinations, evaluating
\end{flushleft}


\begin{flushleft}
the performance of the students, awarding and submitting the grades to the Undergraduate office.
\end{flushleft}


\begin{flushleft}
7.1.6 Attendance in Class
\end{flushleft}


\begin{flushleft}
If a student remains absent from a class without sanctioned leave, then the course instructor may recommend deregistration of the student from the course.
\end{flushleft}


\begin{flushleft}
If a student is found to be absent from all academic activities for more than 20 working days (not necessarily contiguous) in a
\end{flushleft}


\begin{flushleft}
semester, with or without sanction, then his/her registration for all the courses in that semester will be cancelled resulting in
\end{flushleft}


\begin{flushleft}
a forced semester drop.
\end{flushleft}





16





\begin{flushleft}
\newpage
If a student is found to be absent from all academic activities in a semester without authorization for more than 30 working
\end{flushleft}


\begin{flushleft}
days contiguous or does not appear, without a compelling reason, for the end-semester examinations in all the courses in
\end{flushleft}


\begin{flushleft}
which she/he is registered, then her/his programme will be terminated.
\end{flushleft}


\begin{flushleft}
7.1.7 Work-Week and Class Timings
\end{flushleft}


\begin{flushleft}
The institute operates on a 5-day per week schedule. Classes are held Monday through Friday from 8am to 6:30pm. No
\end{flushleft}


\begin{flushleft}
classes are scheduled on a regular basis outside this time period. Lecture / tutorial classes are usually scheduled in 50 minute
\end{flushleft}


\begin{flushleft}
slots, with some 75 minute slots for PG-level courses. Lab classes are usually scheduled in 180 minute slots. No classes are
\end{flushleft}


\begin{flushleft}
usually held on Saturdays and Sundays, unless announced by DoAA as make up for some holiday.
\end{flushleft}


\begin{flushleft}
Extra classes may be scheduled by an instructor in case the regular schedule does not allow for 40 hours of instruction (in a
\end{flushleft}


\begin{flushleft}
lecture course), and/or if an instructor has to miss a regularly scheduled class. In such cases, the instructor may schedule an
\end{flushleft}


\begin{flushleft}
extra class in consultation with the students registered in the course at a time mutually convenient to everyone. Extra classes
\end{flushleft}


\begin{flushleft}
to hold a quiz or a laboratory test should not be inconveniently scheduled for any concerned student.
\end{flushleft}





\begin{flushleft}
7.2 Evaluation and Performance Feedback
\end{flushleft}


\begin{flushleft}
The evaluation of students' performance in a course is a continuous process. Students' performance is evaluated through a
\end{flushleft}


\begin{flushleft}
mid-semester examination, an end-semester examination, quizzes (short-tests), assignments, laboratory work (if applicable),
\end{flushleft}


\begin{flushleft}
etc. The weightage of each component to determine the final grade in the course is decided by the course instructor who
\end{flushleft}


\begin{flushleft}
must inform the students about these weightages at the start of the semester.
\end{flushleft}


\begin{flushleft}
7.2.1 Examinations
\end{flushleft}


\begin{flushleft}
The mid-semester and end-semester examinations are scheduled by the Dean of Academic Affairs during the periods
\end{flushleft}


\begin{flushleft}
specified in the Academic Calendar.
\end{flushleft}


\begin{flushleft}
A modular course has only one examination. It is held during the mid-semester examination period if the course is taught
\end{flushleft}


\begin{flushleft}
during the first half of the semester. Otherwise this exam is held during the end-semester examination period.
\end{flushleft}





\begin{flushleft}
7.2.2 Quiz, evaluated assignments.
\end{flushleft}


\begin{flushleft}
To ensure the principle of continuous evaluation, it is recommended that core course instructors conduct at least two
\end{flushleft}


\begin{flushleft}
quizzes/tests, one before the mid-semester examination and other between the mid-semester and the end--semester
\end{flushleft}


\begin{flushleft}
examination. In a core modular course, it is recommended to have at least one quiz since it has only one examination.
\end{flushleft}


\begin{flushleft}
Schedule and number of quizzes for other course will be decided by the instructor.
\end{flushleft}


\begin{flushleft}
7.2.3 Make-up Examination
\end{flushleft}


\begin{flushleft}
If a student, for bona fide reasons such as illness, etc., fails to appear in the end-semester examinations in one or more
\end{flushleft}


\begin{flushleft}
course(s), s/he may make a request to the SUGC Chairperson for a make-up examination within a day of the last scheduled
\end{flushleft}


\begin{flushleft}
examination. Such a request must be made on the prescribed form available from the DoAA site, giving reasons for the failure
\end{flushleft}


\begin{flushleft}
to appear in the examination along with documents supporting the given reason. In case of illness a certificate from the Chief
\end{flushleft}


\begin{flushleft}
Medical officer of the Institute Health Center should be submitted.
\end{flushleft}


\begin{flushleft}
If a student fails to appear in a mid-semester examination or quiz, or to submit an assignment etc., it is entirely up to the
\end{flushleft}


\begin{flushleft}
Instructor to decide whether or not to provide a make-up opportunity. This rule applies even if the student's inability to do
\end{flushleft}


\begin{flushleft}
the work at the scheduled time was a result of illness and/or sanctioned leave.
\end{flushleft}


\begin{flushleft}
7.2.4 Results of Examinations and Quizzes
\end{flushleft}


\begin{flushleft}
The final grades of all the students in a course must be submitted to the DoAA within 72 hours, 96 hours, and 120 hours after
\end{flushleft}


\begin{flushleft}
the examination for courses with class strength up to 50, between 51 and 150, and above 150 respectively. Instructors are
\end{flushleft}





17





\begin{flushleft}
\newpage
required to show the graded answer books for all examinations/quizzes/assignments as soon as possible (within 14 days of
\end{flushleft}


\begin{flushleft}
the last date of exams for the mid-semester examination, and within the prescribed period as indicated above for the endsemester examination). It is the student's responsibility to be available at the time specified by the instructor for this
\end{flushleft}


\begin{flushleft}
purpose. Answer books of the final examination must be returned to the instructor after the students see them, and saved by
\end{flushleft}


\begin{flushleft}
the instructor for a minimum of six months.
\end{flushleft}


\begin{flushleft}
7.2.5 Letter Grades and Weightages
\end{flushleft}


\begin{flushleft}
At the end of the semester/summer term, students are awarded a letter grade in each course by the concerned Instructor-incharge taking into account their performance in various examinations, quizzes, assignments, laboratory work (if any), etc.,
\end{flushleft}


\begin{flushleft}
besides regularity of attendance in classes. In some courses such as projects, seminars, physical education etc. Satisfactory (S)
\end{flushleft}


\begin{flushleft}
/ Unsatisfactory (X) grade is awarded. Grade {`}X' implies that the student has failed the course. S/X grades are not used for
\end{flushleft}


\begin{flushleft}
the calculation of CPI/SPI.
\end{flushleft}


\begin{flushleft}
Each department has its own procedure for the award of grades in project courses.
\end{flushleft}


\begin{flushleft}
There are seven letter grades: A*, A, B, C, D, E and F. The letter grades, their descriptions, and the numerical equivalents on a
\end{flushleft}


\begin{flushleft}
10-point scale (called Grade Points) are as follows:
\end{flushleft}





\begin{flushleft}
Grade
\end{flushleft}


\begin{flushleft}
A*
\end{flushleft}


\begin{flushleft}
A
\end{flushleft}


\begin{flushleft}
B
\end{flushleft}


\begin{flushleft}
C
\end{flushleft}


\begin{flushleft}
D
\end{flushleft}


\begin{flushleft}
E
\end{flushleft}


\begin{flushleft}
F
\end{flushleft}





\begin{flushleft}
Grade Point
\end{flushleft}


10


10


8


6


4


2


0





\begin{flushleft}
Description
\end{flushleft}


\begin{flushleft}
Outstanding
\end{flushleft}


\begin{flushleft}
Excellent
\end{flushleft}


\begin{flushleft}
Good
\end{flushleft}


\begin{flushleft}
Fair
\end{flushleft}


\begin{flushleft}
Pass
\end{flushleft}


\begin{flushleft}
Fail/ Exposure
\end{flushleft}


\begin{flushleft}
Fail
\end{flushleft}





\begin{flushleft}
A* grade is intended to recognize and encourage outstanding performance in a class. This grade is to be awarded sparingly.
\end{flushleft}


\begin{flushleft}
E grade indicates that the student has failed the course but s/he may be allowed to register for a course for which this course
\end{flushleft}


\begin{flushleft}
is a pre-requisite, even before this course is repeated and passed. This facility of waiver of pre-requisite requirement is
\end{flushleft}


\begin{flushleft}
subject to the approval of the instructor of the course (of which this course is pre-requisite) and the concerned DUGC.
\end{flushleft}


\begin{flushleft}
Two additional letters, namely, `I' and `W', which stand for Incomplete and Waiver respectively, may be given for a course.
\end{flushleft}


\begin{flushleft}
These are not grades.
\end{flushleft}


\begin{flushleft}
Incomplete: A student may be awarded the letter I (Incomplete) in a course if s/he has missed, for a genuine reason, a minor
\end{flushleft}


\begin{flushleft}
part of the course requirement but has done satisfactorily in all other parts. An `I' must, however, be converted by
\end{flushleft}


\begin{flushleft}
the Instructor-in-charge into an appropriate letter grade and communicated to the undergraduate office by the last
\end{flushleft}


\begin{flushleft}
date specified in the Academic Calendar. Any outstanding `I' after this date shall be converted automatically into an
\end{flushleft}


\begin{flushleft}
F grade. In case of project courses `I' may not be awarded for mere non-completion of project due to lack of facility
\end{flushleft}


\begin{flushleft}
etc.
\end{flushleft}


\begin{flushleft}
Waiver: The letter `W' is awarded when a student earns credits at another institution and the SUGC decides to waive similar
\end{flushleft}


\begin{flushleft}
credits from her/his programme of study at IIT Kanpur. The grade earned in lieu of which the waiver is granted is not
\end{flushleft}


\begin{flushleft}
to be used for computation of SPI/CPI.
\end{flushleft}





\begin{flushleft}
7.2.6 Semester Performance Index
\end{flushleft}


\begin{flushleft}
The Semester Performance Index (SPI) is a weighted average of the grade points earned by a student in all the courses
\end{flushleft}


\begin{flushleft}
credited in a semester. If the grade points associated with the awarded grades to a student are g1, g2 ... in a given semester
\end{flushleft}


\begin{flushleft}
and the corresponding course credits are c1, c2,.... , then the SPI for that semester is calculated by multiplying the number of
\end{flushleft}





18





\begin{flushleft}
\newpage
credits for each course with the grade point for that course, adding these up for all the courses registered in the semester,
\end{flushleft}


\begin{flushleft}
and then dividing this sum by the total course credits for the semester:
\end{flushleft}





\begin{flushright}
 =   �   
\end{flushright}


\begin{flushleft}
$\in$
\end{flushleft}





\begin{flushleft}

\end{flushleft}





\begin{flushleft}
S and X grades shall not be considered in the computation of the SPI.
\end{flushleft}


\begin{flushleft}
7.2.7 Cumulative Performance Index
\end{flushleft}


\begin{flushleft}
The Cumulative Performance Index (CPI) indicates the overall academic performance of a student. It is computed in the same
\end{flushleft}


\begin{flushleft}
manner as the SPI, except that here we consider all the courses registered up to and including the latest completed
\end{flushleft}


\begin{flushleft}
semester/summer term.
\end{flushleft}





\begin{flushright}
 =   �   
\end{flushright}


\begin{flushleft}
$\in$
\end{flushleft}





\begin{flushleft}

\end{flushleft}





\begin{flushleft}
Whenever a student is permitted to repeat or substitute a course, the new letter grade replaces the old letter grade in the
\end{flushleft}


\begin{flushleft}
computation of the CPI, but both grades are mentioned in the Grade Report.
\end{flushleft}


\begin{flushleft}
7.2.8 Declaration of the Final Result
\end{flushleft}


\begin{flushleft}
The grades earned by a student in a semester/summer term shall be communicated to her/him after ten days of the last date
\end{flushleft}


\begin{flushleft}
for submission of grades. A printed copy of the Grade Report will be issued to each student after each semester/summer
\end{flushleft}


\begin{flushleft}
term. A duplicate copy, if required, can be obtained on payment of the prescribed fee.
\end{flushleft}


\begin{flushleft}
7.2.9 Withholding of Grades
\end{flushleft}


\begin{flushleft}
The grades of a student may be withheld if s/he has not paid her/his dues, or if there is a case of indiscipline pending against
\end{flushleft}


\begin{flushleft}
her/him, or for any other appropriate reason as per the directives of the Senate.
\end{flushleft}


\begin{flushleft}
7.2.10 Change of an already awarded grade
\end{flushleft}


\begin{flushleft}
A letter grade once awarded shall not be changed unless the request is made by either the Instructor-in-charge or another
\end{flushleft}


\begin{flushleft}
Instructor/tutor of the course, and is approved by the Chairperson, Senate. Any such request for a change of grade must be
\end{flushleft}


\begin{flushleft}
made within six weeks of the start of the next semester on the prescribed form available from the DoAA website, with all
\end{flushleft}


\begin{flushleft}
relevant records and justifications.
\end{flushleft}





19





\begin{flushleft}
\newpage
Chapter 8
\end{flushleft}





\begin{flushleft}
Academic Requirement and Degree Eligibility
\end{flushleft}


\begin{flushleft}
8.1 Minimum and Maximum Duration
\end{flushleft}


\begin{flushleft}
The minimum duration requirements and maximum duration allowed for various undergraduate programmes are as under:
\end{flushleft}


\begin{flushleft}
Academic Programme
\end{flushleft}


\begin{flushleft}
B Tech, BS
\end{flushleft}


\begin{flushleft}
Double Major
\end{flushleft}


\begin{flushleft}
Bachelors-Masters Dual Degree
\end{flushleft}


\begin{flushleft}
MSc
\end{flushleft}


\begin{flushleft}
MSPD Dual Degree (MSc Part)
\end{flushleft}





\begin{flushleft}
Minimum Duration (Semesters)
\end{flushleft}


7


9


9


4


8





\begin{flushleft}
Maximum Duration (Semesters)
\end{flushleft}


12


12


15


06


-





\begin{flushleft}
The minimum and maximum duration allowed will include any semester(s) in which a student is registered at IITK, but may
\end{flushleft}


\begin{flushleft}
spend as a non-degree student at some other Institution while still pursuing the said academic programme at IITK.
\end{flushleft}


\begin{flushleft}
The Senate may grant relaxation in the prescribed minimum residence to a student in view of the work done by her/him in
\end{flushleft}


\begin{flushleft}
the Institute or elsewhere, to the extent considered appropriate according to the merit of the case.
\end{flushleft}


\begin{flushleft}
A student failing to complete the programme even within the maximum duration specified may be allowed by the Senate to
\end{flushleft}


\begin{flushleft}
continue depending on the merits of the case.
\end{flushleft}





\begin{flushleft}
8.2 Minimum Academic Requirement
\end{flushleft}


\begin{flushleft}
In order to graduate, a student must clear all courses as per the respective programme template, satisfying the minimum
\end{flushleft}


\begin{flushleft}
credit requirement in each course category. In addition, a Bachelors-Masters dual-degree student must achieve at least 6.5
\end{flushleft}


\begin{flushleft}
CPI in the PG part of the programme, and MSc or MSPD dual degree student (in MSc part) must achieve at least 6.0 CPI. For
\end{flushleft}


\begin{flushleft}
students admitted to the MSc Integrated programme (discontinued in 2011), the minimum graduation CPI is 5.0.
\end{flushleft}


\begin{flushleft}
If a student is short of 1 credit in the SO/ESO category for the completion of the programme, then s/he may be granted
\end{flushleft}


\begin{flushleft}
relaxation for the same by the SUGC. In exceptional circumstances Senate may grant any other relaxation in minimum
\end{flushleft}


\begin{flushleft}
academic requirements.
\end{flushleft}





\begin{flushleft}
8.3 Graduation
\end{flushleft}


\begin{flushleft}
A student is deemed to have completed the requirements for graduation if s/he has:
\end{flushleft}


\begin{flushleft}
i.
\end{flushleft}


\begin{flushleft}
ii.
\end{flushleft}


\begin{flushleft}
iii.
\end{flushleft}


\begin{flushleft}
iv.
\end{flushleft}





\begin{flushleft}
Met the minimum duration and academic requirements outlined in Sections 8.1 and 8.2
\end{flushleft}


\begin{flushleft}
Satisfied additional requirements, if any, of the concerned department
\end{flushleft}


\begin{flushleft}
Paid all dues to the Institute and the Halls of Residence
\end{flushleft}


\begin{flushleft}
No case of indiscipline is pending against her/him
\end{flushleft}





\begin{flushleft}
8.3.1 Graduation with Distinction
\end{flushleft}


\begin{flushleft}
A student graduating with a CPI of 8.5 or above is said to graduate with distinction. This fact is noted on the student's final
\end{flushleft}


\begin{flushleft}
grade report.
\end{flushleft}





20





\begin{flushleft}
\newpage
8.4 Award of Degrees
\end{flushleft}


\begin{flushleft}
A student who completes all the graduation requirements specified in Section 8.3 is recommended by the Senate to the
\end{flushleft}


\begin{flushleft}
Board of Governors (BOG) for award of the appropriate degree in the ensuing convocation. The degree can be awarded only
\end{flushleft}


\begin{flushleft}
after the BOG accords its approval.
\end{flushleft}





\begin{flushleft}
8.5 Withdrawal of the Degree
\end{flushleft}


\begin{flushleft}
Under extremely exceptional circumstances, where gross violation of graduation requirements is detected at a later stage,
\end{flushleft}


\begin{flushleft}
the Senate may recommend to the Board of Governors withdrawal of a degree already awarded.
\end{flushleft}





21





\begin{flushleft}
\newpage
Chapter 9
\end{flushleft}





\begin{flushleft}
Inadequate Academic Performance
\end{flushleft}


\begin{flushleft}
9.1 Mechanism to Address Inadequate Performance
\end{flushleft}


\begin{flushleft}
A student is expected to maintain at least a minimum level of performance at all times. The academic performance of each
\end{flushleft}


\begin{flushleft}
UG student is reviewed by the Academic Performance Evaluation Committee (APEC) at the end of each regular semester. A
\end{flushleft}


\begin{flushleft}
deficient student may be placed on Warning or Academic Probation or his/her academic programme may be terminated as
\end{flushleft}


\begin{flushleft}
per rules applicable for that particular batch. A student on Warning or Academic Probation is required to sign an undertaking
\end{flushleft}


\begin{flushleft}
incorporating the following conditions:
\end{flushleft}


\begin{flushleft}
i. S/he shall register with higher priority for those courses (or their substitute) in which grade F/E/X is obtained
\end{flushleft}


\begin{flushleft}
ii. S/he shall not hold any office in the Hall of Residence, Students Gymkhana or any other organization/body
\end{flushleft}


\begin{flushleft}
iii. Any other terms and conditions laid down by the SUGC/Senate
\end{flushleft}


\begin{flushleft}
The parents/guardian of these students is required to countersign this undertaking. If a student is unable to meet these terms
\end{flushleft}


\begin{flushleft}
and conditions due to some genuine reasons, s/he must explain this to the DUGC/SUGC before the semester ends.
\end{flushleft}


\begin{flushleft}
The criteria for placing students on Warning, Academic Probation, and Programme Termination are described in the following
\end{flushleft}


\begin{flushleft}
sections. Here N denotes the number of semesters for which the student has registered, SC denotes the number of credits
\end{flushleft}


\begin{flushleft}
completed in the last regular semester, and TC denotes the number of credits completed in all the semesters till that point.
\end{flushleft}





\begin{flushleft}
9.2 Warning
\end{flushleft}


\begin{flushleft}
The following table shows the criteria for being placed on Warning:
\end{flushleft}


\begin{flushleft}
Batch
\end{flushleft}


\begin{flushleft}
2010 and earlier
\end{flushleft}


2011,2012


\begin{flushleft}
2013 and later
\end{flushleft}





\begin{flushleft}
B Tech /BS/Bachelors-Masters Dual-Degree/MS(I)
\end{flushleft}


\begin{flushleft}
SPI $\leq$ 4.5 and CPI $\geq$ 5.0
\end{flushleft}


\begin{flushleft}
OR SPI $>$ 4.5 and CPI $<$ 5.0
\end{flushleft}


\begin{flushleft}
None
\end{flushleft}


\begin{flushleft}
SC $\geq$ 30 and (24 + N)N $\leq$ TC$<$ 36N
\end{flushleft}


\begin{flushleft}
or SC$<$ 30 and TC $\geq$ 36N
\end{flushleft}


\begin{flushleft}
For PG part of Dual Degree:
\end{flushleft}


\begin{flushleft}
Refer to PG Manual
\end{flushleft}





\begin{flushleft}
MSc and MSc part of MSPD Dual Degree
\end{flushleft}


\begin{flushleft}
SPI $\leq$ 5.5 and CPI $\geq$ 6.0
\end{flushleft}


\begin{flushleft}
or SPI $>$ 5.5 and CPI $<$ 6.0
\end{flushleft}


\begin{flushleft}
Same as above
\end{flushleft}


\begin{flushleft}
Same as above
\end{flushleft}





\begin{flushleft}
Students in the PhD part of the MSPD dual degree programme will be governed by PG rules.
\end{flushleft}





\begin{flushleft}
9.3 Academic Probation
\end{flushleft}


\begin{flushleft}
The following table shows the criteria for being placed on Academic Probation:
\end{flushleft}


\begin{flushleft}
Batch
\end{flushleft}


\begin{flushleft}
2010 and earlier
\end{flushleft}


2011,2012





\begin{flushleft}
B Tech /BS/Bachelors-Masters Dual-Degree/MS(I)
\end{flushleft}


\begin{flushleft}
SPI $\leq$ 4.5 and CPI $<$ 5.0
\end{flushleft}


\begin{flushleft}
TC$<$ 35 for N = 1
\end{flushleft}


\begin{flushleft}
TC$<$ 37.5N for N $\geq$ 2, for N in UG
\end{flushleft}


\begin{flushleft}
PG part of Dual Degree: No provision for AP
\end{flushleft}





22





\begin{flushleft}
MSc and
\end{flushleft}


\begin{flushleft}
MSc part of MSPD Dual Degree
\end{flushleft}


\begin{flushleft}
SPI $\leq$ 5.5 and CPI $<$ 6.0
\end{flushleft}


\begin{flushleft}
Same as above
\end{flushleft}





\begin{flushleft}
\newpage
2013 and later
\end{flushleft}





\begin{flushleft}
SC $\geq$ 30 and TC$<$ (24 + N)N
\end{flushleft}


\begin{flushleft}
or SC$<$ 30 and (24 + N)N $\leq$ TC$<$ 36N, for N in UG
\end{flushleft}


\begin{flushleft}
or SC$<$ 30 and TC$<$ (24 + N)N and not on Probation
\end{flushleft}


\begin{flushleft}
in previous semester
\end{flushleft}


\begin{flushleft}
PG part of Dual Degree: Refer to PG Manual
\end{flushleft}





\begin{flushleft}
Same as above
\end{flushleft}





\begin{flushleft}
A student on academic probation may be allowed by the DUGC to register for a minimum of 30 credits in the subsequent
\end{flushleft}


\begin{flushleft}
semester.
\end{flushleft}





\begin{flushleft}
9.4 Programme Termination
\end{flushleft}


\begin{flushleft}
No student's programme may be terminated who is not already on academic probation (or Warning in case of the PG part of
\end{flushleft}


\begin{flushleft}
a programme).
\end{flushleft}


\begin{flushleft}
The following table shows the criteria for programme termination:
\end{flushleft}


\begin{flushleft}
Batch
\end{flushleft}





\begin{flushleft}
2010 and earlier
\end{flushleft}





2011,2012


\begin{flushleft}
2013 and later
\end{flushleft}





\begin{flushleft}
B Tech /BS/Bachelors-Masters Dual-Degree/MS(I)
\end{flushleft}





\begin{flushleft}
On Probation and SPI $<$ 4.5
\end{flushleft}


\begin{flushleft}
The termination of the PG part of Dual Degree programme is
\end{flushleft}


\begin{flushleft}
governed by the PG manual.
\end{flushleft}


\begin{flushleft}
On Probation and TC$<$ 25N, for N in UG
\end{flushleft}


\begin{flushleft}
PG part of Dual Degree: Same as above
\end{flushleft}


\begin{flushleft}
On Probation and SC$<$ 30 and TC$<$ (24 + N)N, for N in UG
\end{flushleft}


\begin{flushleft}
PG part of Dual Degree: Same as above
\end{flushleft}





\begin{flushleft}
MSc and
\end{flushleft}


\begin{flushleft}
MSc part of MSPD Dual Degree
\end{flushleft}


\begin{flushleft}
On Probation and SPI $<$ 5.5
\end{flushleft}





\begin{flushleft}
Same as above
\end{flushleft}


\begin{flushleft}
Same as above
\end{flushleft}





\begin{flushleft}
If a Bachelors-Masters student's programme is terminated due to CPI considerations in the PG part and if her/his CPI is above
\end{flushleft}


\begin{flushleft}
6.0, then s/he may be allowed to continue on the recommendation of the DUGC and the approval of the SUGC.
\end{flushleft}


\begin{flushleft}
Students in the PhD part of the MSPD dual degree programme will be governed by PG rules.
\end{flushleft}





\begin{flushleft}
9.5 Appeal Against Termination
\end{flushleft}


\begin{flushleft}
A student whose programme is terminated may appeal to the Chairperson, Senate, for re-reinstatement in the programme.
\end{flushleft}


\begin{flushleft}
In cases of termination due to inadequate academic performance, the student should clearly explain causes for the poor
\end{flushleft}


\begin{flushleft}
performance, including how those causes will not adversely affect her/his performance in the future. The Senate shall take a
\end{flushleft}


\begin{flushleft}
final decision after considering all available inputs. A student may re-appeal even after a previous appeal has been rejected.
\end{flushleft}


\begin{flushleft}
However, the Senate may not entertain any re-appeal for review unless substantial additional information is brought to its
\end{flushleft}


\begin{flushleft}
notice.
\end{flushleft}





23





\begin{flushleft}
\newpage
Chapter 10
\end{flushleft}





\begin{flushleft}
Rules Governing Change,Addition in the Programme
\end{flushleft}


\begin{flushleft}
10.1 Branch Change
\end{flushleft}


\begin{flushleft}
A student may be allowed a change of branch / programme, including change to the multi-disciplinary Engineering Science
\end{flushleft}


\begin{flushleft}
programme, on the basis of her/his academic performance, subject to strength constraints of the departments. Change of
\end{flushleft}


\begin{flushleft}
branch / programme for a student is a matter of privilege and not a right. Once a Branch Change has been granted, a student
\end{flushleft}


\begin{flushleft}
cannot revert to the original department.
\end{flushleft}





\begin{flushleft}
10.1.1 Eligibility
\end{flushleft}


\begin{flushleft}
A student may apply after her/his second semester if she/he has acquired credits (i.e., received a passing grade) for all
\end{flushleft}


\begin{flushleft}
the course prescribed in the template for the first year of the programme, including the courses where S/X grades are
\end{flushleft}


\begin{flushleft}
awarded.
\end{flushleft}


\begin{flushleft}
Any student may apply after their third or fourth semester even if they have NOT acquired credits (i.e., received a
\end{flushleft}


\begin{flushleft}
passing grade) for all the courses prescribed in the template for the first year of the programme, including the courses
\end{flushleft}


\begin{flushleft}
where S/X grades are awarded.
\end{flushleft}





\begin{flushleft}
10.1.2 Application Process
\end{flushleft}


\begin{flushleft}
i.
\end{flushleft}


\begin{flushleft}
ii.
\end{flushleft}


\begin{flushleft}
iii.
\end{flushleft}


\begin{flushleft}
iv.
\end{flushleft}





\begin{flushleft}
The DoAA office will call for branch change applications twice in an academic year in April and November.
\end{flushleft}


\begin{flushleft}
Eligible students may apply on the prescribed branch change form addressed to the Chairperson, SUGC.
\end{flushleft}


\begin{flushleft}
Completed application forms should be submitted in the UG section of the DoAA office by the given deadline.
\end{flushleft}


\begin{flushleft}
Students whose branch change applications have been accepted will be informed accordingly before the date for Final
\end{flushleft}


\begin{flushleft}
Registration in the following regular semester.
\end{flushleft}





\begin{flushleft}
10.1.3 Academic Road-Map
\end{flushleft}


\begin{flushleft}
i.
\end{flushleft}





\begin{flushleft}
Once a branch change has been granted, students are expected to follow the template for the new department to which
\end{flushleft}


\begin{flushleft}
they have been admitted.
\end{flushleft}


\begin{flushleft}
ii. Students granted a branch change will have their pre-registration cancelled. Such students will be expected to do manual
\end{flushleft}


\begin{flushleft}
academic registration during the final registration period.
\end{flushleft}


\begin{flushleft}
iii. These students should register for courses after consulting the DUGC convener of the department to which they have
\end{flushleft}


\begin{flushleft}
been admitted.
\end{flushleft}


\begin{flushleft}
iv. Each such student is responsible for ensuring that all academic requirements of the new department are fulfilled.
\end{flushleft}


\begin{flushleft}
v. For students who are granted branch change after their third or fourth semester, it is mandatory to complete any
\end{flushleft}


\begin{flushleft}
additional department-specific requirements for the third and/or fourth semester as listed in the template.
\end{flushleft}


\begin{flushleft}
vi. There is no provision for withdrawal from a branch change. Once a student's branch change application has been
\end{flushleft}


\begin{flushleft}
accepted, s/he will be considered a student of the new department for the entire period of her/his programme.
\end{flushleft}





\begin{flushleft}
10.2 Bachelors-Masters Dual Degree Programme
\end{flushleft}


\begin{flushleft}
The Bachelors-Masters Dual Degree programme is divided into three categories:
\end{flushleft}


\begin{flushleft}
Category A: Both degrees in the same department (available in all departments running a Bachelors programme)
\end{flushleft}


\begin{flushleft}
Category B: Bachelors and Master's Degrees in different departments (Masters under this category is NOT available in CSE
\end{flushleft}


\begin{flushleft}
and EE; it is available in all other departments and programmes, including Design, EEM, IME, NET, and Photonics)
\end{flushleft}





24





\begin{flushleft}
\newpage
Category C: Bachelors in any department combined with an MBA degree
\end{flushleft}


\begin{flushleft}
10.2.1 Eligibility
\end{flushleft}


\begin{flushleft}
i.
\end{flushleft}





\begin{flushleft}
Students should have a minimum CPI of 6.0 at the time of applying.
\end{flushleft}


\begin{flushleft}
(Senate approved a one-time relaxation of this criterion to 5.0 CPI for BS students of the 2011 and 2012 batches applying
\end{flushleft}


\begin{flushleft}
for MS in MTH, CHM, or ECO)
\end{flushleft}


\begin{flushleft}
ii. Students should have no backlogs in non-OE credits for their UG programme at the time of applying.
\end{flushleft}


\begin{flushleft}
(Senate approved a one-time relaxation of this criterion to up to a 36 credit backlog in either OE or non-OE credits from
\end{flushleft}


\begin{flushleft}
the UG programme at the time of applying for students of the 2011 and 2012 batches)
\end{flushleft}


\begin{flushleft}
iii. For students applying under Category B or C, there may be additional norms, such as interviews or written tests, in
\end{flushleft}


\begin{flushleft}
certain departments. Students may approach the concerned department / programme for details regarding such
\end{flushleft}


\begin{flushleft}
additional eligibility norm.
\end{flushleft}


\begin{flushleft}
iv. Admission into the dual degree programme is subject to the fulfillment of eligibility criteria and availability of seats in the
\end{flushleft}


\begin{flushleft}
concerned department / programme. Details regarding seat availability are given in Section 10.6.2.
\end{flushleft}


\begin{flushleft}
v. Students opting for the Dual Degree programme will not be allowed to do a Double Major.
\end{flushleft}


\begin{flushleft}
10.2.2 Application Process
\end{flushleft}


\begin{flushleft}
i.
\end{flushleft}


\begin{flushleft}
ii.
\end{flushleft}


\begin{flushleft}
iii.
\end{flushleft}


\begin{flushleft}
iv.
\end{flushleft}


\begin{flushleft}
v.
\end{flushleft}


\begin{flushleft}
vi.
\end{flushleft}





\begin{flushleft}
For Category A dual degree programme, students may apply after their fifth, sixth, or seventh semester.
\end{flushleft}


\begin{flushleft}
For Category B and C, students may apply only once after their sixth semester.
\end{flushleft}


\begin{flushleft}
The DoAA office will call for Dual Degree applications under all three categories twice a year in April and November.
\end{flushleft}


\begin{flushleft}
Eligible students may apply on the prescribed Dual Degree application form addressed to the Chairperson, SUGC.
\end{flushleft}


\begin{flushleft}
Completed application forms should be submitted in the UG section of the DoAA office by the given deadline.
\end{flushleft}


\begin{flushleft}
Completed application forms received by the deadline will be processed by the DoAA office after the results of that
\end{flushleft}


\begin{flushleft}
semester have been declared, and forwarded to the concerned departments. Departments will be expected to convey
\end{flushleft}


\begin{flushleft}
the results to the DoAA office latest by the day before registration for the next semester is due to begin.
\end{flushleft}


\begin{flushleft}
vii. Students who have been granted admission into the Dual Degree programme will be informed accordingly before the
\end{flushleft}


\begin{flushleft}
date for Final Registration in the semester following the submission of the application.
\end{flushleft}





\begin{flushleft}
10.2.3 Academic Road-Map: Details of Dual Degree course work is available in UG course templates of each department.
\end{flushleft}


\begin{flushleft}
i.
\end{flushleft}





\begin{flushleft}
ii.
\end{flushleft}


\begin{flushleft}
iii.
\end{flushleft}


\begin{flushleft}
iv.
\end{flushleft}


\begin{flushleft}
v.
\end{flushleft}


\begin{flushleft}
vi.
\end{flushleft}


\begin{flushleft}
vii.
\end{flushleft}


\begin{flushleft}
viii.
\end{flushleft}





\begin{flushleft}
Dual Degree students will be allowed to use OE slots and overloads (overload rules are given in Section 6.1.5 and 6.1.6)
\end{flushleft}


\begin{flushleft}
to complete their dual degree course requirements. The use of such slots should be done in consultation with the DUGC
\end{flushleft}


\begin{flushleft}
convener of the parent (UG) as well as the host (PG) department / programme.
\end{flushleft}


\begin{flushleft}
A maximum of 36 OE credits may be waived from the UG graduation requirement to be used for PG requirements by
\end{flushleft}


\begin{flushleft}
Dual Degree students. Details regarding all such waivers for Dual Degree students are given in the course templates.
\end{flushleft}


\begin{flushleft}
Dual Degree students may be allowed to take relevant courses in the Summer Term, if offered, (up to 25 credits) after
\end{flushleft}


\begin{flushleft}
the eighth semester.
\end{flushleft}


\begin{flushleft}
OARS should allow registration for mandatory laboratory courses, if any, if the instructor arranges alternate times for the
\end{flushleft}


\begin{flushleft}
dual degree students without changing the normal schedule of the laboratories.
\end{flushleft}


\begin{flushleft}
Migration to the Masters part of the Dual Degree programme will be done only when the student has completed all the
\end{flushleft}


\begin{flushleft}
mandatory credit requirements from the undergraduate part of the programme up to the seventh semester.
\end{flushleft}


\begin{flushleft}
At the end of the eighth semester a student should apply for any swapping of completed courses between their UG and
\end{flushleft}


\begin{flushleft}
PG parts of the programme. Such swapping may be done only once during the dual degree programme.
\end{flushleft}


\begin{flushleft}
Upon migration to the Masters part of the Dual Degree programme, the student will be issued a new roll number for the
\end{flushleft}


\begin{flushleft}
PG part of the programme.
\end{flushleft}


\begin{flushleft}
A Dual Degree student will be governed by existing APEC rules for UG students until s/he officially migrates to the
\end{flushleft}


\begin{flushleft}
Masters part of the programme. In the PG part of the programme, the student will be governed by existing APEC rules for
\end{flushleft}


\begin{flushleft}
PG students.
\end{flushleft}





25





\begin{flushleft}
\newpage
10.2.4 Withdrawal from the Bachelors-Masters Dual Degree Programme
\end{flushleft}


\begin{flushleft}
i.
\end{flushleft}





\begin{flushleft}
Request for withdrawal from the PG part of the programme will be entertained at any time during the course of the Dual
\end{flushleft}


\begin{flushleft}
Degree programme. The request should be made to the Chairperson SUGC, through the DUGC conveners of both (UG
\end{flushleft}


\begin{flushleft}
and PG) departments, as well as the thesis supervisor (if one has been assigned). Permission to withdraw from the PG
\end{flushleft}


\begin{flushleft}
part of the programme is subject to the approval of the Chairperson, SUGC.
\end{flushleft}


\begin{flushleft}
ii. In case the PG part of the programme is withdrawn, the student will be required to complete all mandatory graduation
\end{flushleft}


\begin{flushleft}
requirements of the undergraduate programme in the parent department. However, courses taken from the PG
\end{flushleft}


\begin{flushleft}
template in lieu of OE options may be counted as OE credits for the UG graduation requirements.
\end{flushleft}


\begin{flushleft}
iii. If the PG part of the programme is withdrawn, the UG degree may be awarded to the student upon completion of all
\end{flushleft}


\begin{flushleft}
graduation requirements for the concerned programme.
\end{flushleft}


\begin{flushleft}
10.2.5 Termination of PG Part of the Bachelors-Masters Dual Degree Programme
\end{flushleft}


\begin{flushleft}
i.
\end{flushleft}





\begin{flushleft}
The PG part of the programme will be terminated if a student is more than 40 credits short of the total credit
\end{flushleft}


\begin{flushleft}
requirements of her/his UG programme at the end of the eighth semester.
\end{flushleft}


\begin{flushleft}
ii. In case the PG part of the programme is terminated, the student will be required to complete all mandatory graduation
\end{flushleft}


\begin{flushleft}
requirements of the undergraduate programme in the parent department. However, courses taken from the PG
\end{flushleft}


\begin{flushleft}
template in lieu of OE options may be counted as OE credits for the UG graduation requirements.
\end{flushleft}


\begin{flushleft}
iii. If the PG part of the programme is terminated, the UG degree may be awarded to the student upon completion of all
\end{flushleft}


\begin{flushleft}
graduation requirements for the concerned programme.
\end{flushleft}





\begin{flushleft}
10.3 Double Major
\end{flushleft}


\begin{flushleft}
10.3.1 Eligibility
\end{flushleft}


\begin{flushleft}
i. Students should have a minimum CPI of 7.0 at the time of applying.
\end{flushleft}


\begin{flushleft}
ii. Students opting for Double Major are not allowed to change to the Dual Degree programme.
\end{flushleft}


\begin{flushleft}
iii. Students opting for Double Major may not apply for formal admission into a Minor. However, they are free to take
\end{flushleft}


\begin{flushleft}
courses for a Minor of their choice, provided the Minor is not offered by their parent or double major department. If
\end{flushleft}


\begin{flushleft}
they can register for the courses for such a Minor even without being formally admitted to the Minor and can complete
\end{flushleft}


\begin{flushleft}
all the required courses for that Minor in this fashion, s/he may apply to get a retrospective Minor at the time of
\end{flushleft}


\begin{flushleft}
graduation.
\end{flushleft}


\begin{flushleft}
iv. Admission to the Double Major programme is subject to departmental CPI criteria, overall CPI ranking and availability of
\end{flushleft}


\begin{flushleft}
seats. Details regarding seat availability are given in section 10.6.3.
\end{flushleft}


\begin{flushleft}
10.3.2 Application Process
\end{flushleft}


\begin{flushleft}
i.
\end{flushleft}


\begin{flushleft}
ii.
\end{flushleft}


\begin{flushleft}
iii.
\end{flushleft}


\begin{flushleft}
iv.
\end{flushleft}


\begin{flushleft}
v.
\end{flushleft}





\begin{flushleft}
Students may apply for the Double Major programme only once towards the end of their fourth semester.
\end{flushleft}


\begin{flushleft}
The DoAA office will call for Double Major applications once every year in April.
\end{flushleft}


\begin{flushleft}
Eligible students may apply on the prescribed Double Major application form.
\end{flushleft}


\begin{flushleft}
Completed application forms should be submitted in the UG section of the DoAA office by the given deadline.
\end{flushleft}


\begin{flushleft}
Completed application forms received by the deadline will be processed by the DoAA office after the results of that
\end{flushleft}


\begin{flushleft}
semester have been declared, and forwarded to the concerned departments. Departments will be expected to convey
\end{flushleft}


\begin{flushleft}
the results to the DoAA office latest by the day before registration for the next semester is due to begin.
\end{flushleft}


\begin{flushleft}
vi. Students who have been granted admission into the Double Major programme will be informed accordingly before the
\end{flushleft}


\begin{flushleft}
date for Final Registration in the semester following the submission of the application.
\end{flushleft}


\begin{flushleft}
10.3.3 Academic Road-Map: Detailed list of courses that need to be completed for a Double Major are given alongside the
\end{flushleft}


\begin{flushleft}
course templates.
\end{flushleft}


\begin{flushleft}
i.
\end{flushleft}





\begin{flushleft}
Double Major students will be allowed to use OE slots and overloads to complete requirements for their second major.
\end{flushleft}


\begin{flushleft}
The use of such slots should be done in consultation with the DUGC convener of the parent as well as the host (second
\end{flushleft}


\begin{flushleft}
major) department / programme.
\end{flushleft}





26





\begin{flushleft}
\newpage
ii.
\end{flushleft}





\begin{flushleft}
A maximum of 36 OE credits may be waived from the parent department graduation requirements to be used for the
\end{flushleft}


\begin{flushleft}
second major requirements of Double Major students.
\end{flushleft}


\begin{flushleft}
iii. Double Major students may be allowed to take relevant courses (up to 25 credits) in the summer term, if offered.
\end{flushleft}


\begin{flushleft}
iv. Once a student is admitted into the Double Major programme, s/he will be advised by the DUGC convener of both the
\end{flushleft}


\begin{flushleft}
parent and the second Major departments.
\end{flushleft}


\begin{flushleft}
10.3.4 Withdrawal from the Double Major Programme
\end{flushleft}


\begin{flushleft}
A student may withdraw at any time from the Double Major programme. To do so, the student needs to submit an
\end{flushleft}


\begin{flushleft}
application stating the intention to withdraw from the programme to the Chairperson SUGC. The application should be
\end{flushleft}


\begin{flushleft}
forwarded by the DUGC conveners of both departments - the parent department as well as the host (second major)
\end{flushleft}


\begin{flushleft}
department.
\end{flushleft}


\begin{flushleft}
In case the student withdraws from the Double Major programme, the student will graduate with a Bachelor's degree in
\end{flushleft}


\begin{flushleft}
the parent department only. In this case, all credits taken towards the second major will be treated as OE credits in the
\end{flushleft}


\begin{flushleft}
parent department. These credits may also be counted towards a Minor if applicable.
\end{flushleft}





\begin{flushleft}
10.3.5 Termination of the Double Major Programme
\end{flushleft}


\begin{flushleft}
In case a student is unable to complete all the requirements for the Double Major in the stipulated maximum period of
\end{flushleft}


\begin{flushleft}
12 semesters, the second major part of the student's programme will be terminated.
\end{flushleft}


\begin{flushleft}
If the student has completed all the parent department graduation requirements when the second major part of the
\end{flushleft}


\begin{flushleft}
programme is terminated, the student may graduate with a Bachelor's degree in the parent department only.
\end{flushleft}


\begin{flushleft}
th
\end{flushleft}





\begin{flushleft}
If the student has not even completed the parent department graduation requirements by the end of the 12 semester,
\end{flushleft}


\begin{flushleft}
she/he may apply for an extension of one semester to complete the requirements of the parent department ONLY.
\end{flushleft}


\begin{flushleft}
If the double major is terminated, all credits taken towards the second major will be treated as OE credits in the parent
\end{flushleft}


\begin{flushleft}
department. These credits may also be counted towards a Minor if applicable.
\end{flushleft}





\begin{flushleft}
10.4. MINOR
\end{flushleft}


\begin{flushleft}
10.4.1 Eligibility
\end{flushleft}


\begin{flushleft}
i.
\end{flushleft}


\begin{flushleft}
ii.
\end{flushleft}





\begin{flushleft}
Students doing a Double Major can only get a retrospective Minor.
\end{flushleft}


\begin{flushleft}
All other students may apply for a Minor in any department except their own. Economics students may also apply for a
\end{flushleft}


\begin{flushleft}
Minor in another discipline within the HSS department.
\end{flushleft}


\begin{flushleft}
iii. There is no CPI criterion for Minors. Admission is based only on availability of seats as detailed in section 10.6.4.
\end{flushleft}


\begin{flushleft}
iv. A student who manages to complete all the courses required for a Minor without being formally admitted to the Minor
\end{flushleft}


\begin{flushleft}
may also apply for a Minor retrospectively in her/his final semester.
\end{flushleft}


\begin{flushleft}
v. A student can get more than one Minor.
\end{flushleft}


\begin{flushleft}
10.4.2 Application Process
\end{flushleft}


\begin{flushleft}
i.
\end{flushleft}


\begin{flushleft}
ii.
\end{flushleft}





\begin{flushleft}
Students (except those in the Double Major programme) may apply for a Minor during their 4th, 5th, or 6th semester.
\end{flushleft}


\begin{flushleft}
The DoAA office will call for applications for Minors in every semester at least two weeks before the dates for preregistration as given in the Academic Calendar.
\end{flushleft}


\begin{flushleft}
iii. Eligible students may apply on the prescribed Minor application form.
\end{flushleft}


\begin{flushleft}
iv. Completed application forms should be submitted in the UG section of the DoAA office by the given deadline.
\end{flushleft}


\begin{flushleft}
v. Completed application forms received by the deadline will be processed by the DoAA office through the relevant
\end{flushleft}


\begin{flushleft}
departments. Departments will be expected to convey the results to the DoAA office latest by the day before preregistration for the next semester is due to begin.
\end{flushleft}





27





\begin{flushleft}
\newpage
vi. Students who have been granted admission to a Minor will be informed accordingly before pre-registration begins for
\end{flushleft}


\begin{flushleft}
the next semester.
\end{flushleft}


\begin{flushleft}
vii. A student may apply for a Minor after completing some of the courses in the Minor.
\end{flushleft}


\begin{flushleft}
10.4.3 Retrospective Minor
\end{flushleft}


\begin{flushleft}
i. The DOAA office will call for applications for retrospective Minors once a year in April.
\end{flushleft}


\begin{flushleft}
ii. Eligible students may apply on the prescribed Minor form.
\end{flushleft}


\begin{flushleft}
iii. Double Major students who have managed to complete all the courses for a Minor may apply for the same on the
\end{flushleft}


\begin{flushleft}
prescribed retrospective Minor form.
\end{flushleft}


\begin{flushleft}
iv. Completed application forms should be submitted to the UG section of the DOAA office by the given deadline.
\end{flushleft}


\begin{flushleft}
v. Completed application forms received by the deadline will be processed by the DOAA office through the departments
\end{flushleft}


\begin{flushleft}
after the grade submission deadline for that semester.
\end{flushleft}


\begin{flushleft}
vi. Approved retrospective Minors will appear in the student's final grade sheet.
\end{flushleft}


\begin{flushleft}
10.4.4 Academic Road-Map
\end{flushleft}


\begin{flushleft}
i. Details regarding required courses for each Minor are given alongside departmental course templates.
\end{flushleft}


\begin{flushleft}
ii. A Minor entails the completion of 27-36 credits through specified courses within a discipline / programme.
\end{flushleft}


\begin{flushleft}
iii. A student may take Minor courses in any OE, HSS, ESO, or DE slot, as advised by the parent department's DUGC
\end{flushleft}


\begin{flushleft}
convener.
\end{flushleft}


\begin{flushleft}
iv. Rules regarding continuation of a Minor in case of failing and/or dropping a required Minor course are specific to the
\end{flushleft}


\begin{flushleft}
department / programme offering the Minor. Students need to contact the concerned department's DUGC convener for
\end{flushleft}


\begin{flushleft}
details in this regard.
\end{flushleft}


\begin{flushleft}
10.4.5 Withdrawal from a Minor
\end{flushleft}


\begin{flushleft}
A student may withdraw from a Minor at any time by submitting an application to this effect with the approval of the Minor
\end{flushleft}


\begin{flushleft}
department's DUGC convener, to the DoAA office.
\end{flushleft}





\begin{flushleft}
10.5 MSPD Dual Degree Programme
\end{flushleft}


\begin{flushleft}
Students in the MSc two-year programme may apply for this dual degree programme in departments where it is available.
\end{flushleft}


\begin{flushleft}
10.5.1 Eligibility
\end{flushleft}


\begin{flushleft}
i. Students should have a CPI of 7.0 at the time of applying.
\end{flushleft}


\begin{flushleft}
ii. Students should have no backlog of courses at the time of applying.
\end{flushleft}


\begin{flushleft}
iii. Admission into the MSPD dual degree programme is subject to fulfillment of eligibility criteria and availability of seats in
\end{flushleft}


\begin{flushleft}
the programme.
\end{flushleft}


\begin{flushleft}
10.5.2 Application Process
\end{flushleft}


\begin{flushleft}
i. Students should consult the DUGC to confirm details of the application process within their own department.
\end{flushleft}


\begin{flushleft}
ii. Students may apply at the end of their second or third semester.
\end{flushleft}


\begin{flushleft}
iii. Students should include two letters of recommendation with the application. At least one of the letters of
\end{flushleft}


\begin{flushleft}
recommendation should be from an instructor with whom the student has done a course at IIT Kanpur.
\end{flushleft}


\begin{flushleft}
10.5.3 Academic Roadmap
\end{flushleft}


\begin{flushleft}
i.
\end{flushleft}


\begin{flushleft}
ii.
\end{flushleft}





\begin{flushleft}
After migration to the PG part of the programme, whenever it occurs, a student shall be considered a PhD student.
\end{flushleft}


\begin{flushleft}
During the PG part of the programme, students will be governed by the rules and regulations specified in the PG manual,
\end{flushleft}


\begin{flushleft}
including SPI / CPI rules, fee structure, scholarships and other assistantships.
\end{flushleft}


\begin{flushleft}
iii. Students are required to clear the comprehensive examination by the end of the sixth semester. If a student fails to do
\end{flushleft}


\begin{flushleft}
so, her/his PhD programme is terminated as per rules given in the PG manual.
\end{flushleft}





28





\begin{flushleft}
\newpage
iv. A student who fails to clear the comprehensive examination may be permitted, depending on the merits of the case, to
\end{flushleft}


\begin{flushleft}
revert to the MSc two-year programme.
\end{flushleft}


\begin{flushleft}
v. Students shall be awarded the MSc degree at the end of the sixth semester provided they have completed all the course
\end{flushleft}


\begin{flushleft}
requirements of the two-year programme as specified in the template with a CPI of 6.0.
\end{flushleft}


\begin{flushleft}
10.5.4 Withdrawal from the MSPD Dual Degree Programme
\end{flushleft}


\begin{flushleft}
There is no provision for withdrawal from this dual degree programme to revert to the MSc two-year programme
\end{flushleft}


\begin{flushleft}
except in cases where the PhD part of the programme is terminated as per the rules specified above.
\end{flushleft}





\begin{flushleft}
10.6 Calculation of Seat Availability
\end{flushleft}


\begin{flushleft}
10.6.1 Branch Change
\end{flushleft}


\begin{flushleft}
i.
\end{flushleft}





\begin{flushleft}
The vacancies in various programs, and allotment of branch changes, will be computed irrespective of all categories,
\end{flushleft}


\begin{flushleft}
except where Senate has specifically mandated the preference to be given to specific categories (GN/SC/ST/OBC/PD
\end{flushleft}


\begin{flushleft}
etc.).
\end{flushleft}


\begin{flushleft}
ii. Seat availability will be calculated such that no programme exceeds the larger of E and S + 2, where E is its existing and S
\end{flushleft}


\begin{flushleft}
is the sanctioned strength. At the same time, no programme should fall below 55\% of its sanctioned strength as the
\end{flushleft}


\begin{flushleft}
result of branch change allotments.
\end{flushleft}


\begin{flushleft}
iii. Existing and Sanctioned strength of a programme are calculated as follows:
\end{flushleft}


\begin{flushleft}
Existing Strength of batch is L + A -- T, where
\end{flushleft}


\begin{flushleft}
L = Number of students registered on the last date of the semester
\end{flushleft}


\begin{flushleft}
A = Number of students who are on authorized leave for that semester
\end{flushleft}


\begin{flushleft}
T = Number of students whose programs have been terminated at the end of that semester, and whose
\end{flushleft}


\begin{flushleft}
appeals have not been accepted by Senate.
\end{flushleft}


\begin{flushleft}
Sanctioned strength for batch is St + E + C, where
\end{flushleft}


\begin{flushleft}
St = Larger of the sanctioned strength approved by Senate and the actual number of students admitted
\end{flushleft}


\begin{flushleft}
E = Number of extra seat(s) created by Senate for this batch in the previous semester(s)
\end{flushleft}


\begin{flushleft}
C = Number of extra seat(s) created by Senate as special cases in previous semester(s)
\end{flushleft}


\begin{flushleft}
Note
\end{flushleft}


\begin{flushleft}
a) Seats fallen vacant in the parent department due to seats created in other department for branch change of students
\end{flushleft}


\begin{flushleft}
securing 10.0 CPI will be considered as vacant for the purpose of branch change.
\end{flushleft}


\begin{flushleft}
b) Seats fallen vacant in the parent department due to TIE among two or more students given branch change/permanent
\end{flushleft}


\begin{flushleft}
withdrawal by any student or due to death of any student will be considered as vacant for the purpose of branch change.
\end{flushleft}


\begin{flushleft}
c) Seat of terminated student will be considered as vacant for the purpose of branch change only after Senate has turned
\end{flushleft}


\begin{flushleft}
down her/his appeal or the student has not appealed.
\end{flushleft}


\begin{flushleft}
d) Extra seats created for students securing 10.0 CPI and/or for allotting to the students among TIE, shall not be added to
\end{flushleft}


\begin{flushleft}
the actual sanctioned strength for the purpose of branch change.
\end{flushleft}


\begin{flushleft}
10.6.2 Bachelors-Masters Dual Degree
\end{flushleft}


\begin{flushleft}
i.
\end{flushleft}


\begin{flushleft}
ii.
\end{flushleft}





\begin{flushleft}
For Dual Degree Category A (Master's degree in the same department as the Bachelor's), there is no limit on the number
\end{flushleft}


\begin{flushleft}
of available seats.
\end{flushleft}


\begin{flushleft}
For Dual Degree Category B and C (Master's degree in a department / programme different from the Bachelors),
\end{flushleft}


\begin{flushleft}
available seats are given in the table below:
\end{flushleft}


\begin{flushleft}
Name of Department
\end{flushleft}


\begin{flushleft}
Aerospace Engineering
\end{flushleft}


\begin{flushleft}
Biological Sciences and Bio-Engineering
\end{flushleft}


\begin{flushleft}
Chemical Engineering
\end{flushleft}


\begin{flushleft}
Chemistry
\end{flushleft}


\begin{flushleft}
Civil Engineering
\end{flushleft}





\begin{flushleft}
Seats available for Bachelors-Masters Dual Degree Category B / C
\end{flushleft}


10


10


\begin{flushleft}
Flexible
\end{flushleft}


\begin{flushleft}
Flexible
\end{flushleft}


20





29





\begin{flushleft}
\newpage
Computer Science and Engineering
\end{flushleft}


\begin{flushleft}
Design Programme
\end{flushleft}


\begin{flushleft}
Economics
\end{flushleft}


\begin{flushleft}
Electrical Engineering
\end{flushleft}


\begin{flushleft}
Envir. Engg. and Mgmt. Programme (EEM)
\end{flushleft}


\begin{flushleft}
Industrial and Management Engineering
\end{flushleft}


\begin{flushleft}
Materials Science and Engineering
\end{flushleft}


\begin{flushleft}
Materials Science Programme
\end{flushleft}


\begin{flushleft}
Mathematics and Statistics
\end{flushleft}


\begin{flushleft}
Mechanical Engineering
\end{flushleft}


\begin{flushleft}
Nuclear Engineering and Technology
\end{flushleft}


\begin{flushleft}
Photonics Science and Engg. Programme
\end{flushleft}


\begin{flushleft}
Physics
\end{flushleft}





\begin{flushleft}
None
\end{flushleft}


10


15


\begin{flushleft}
None
\end{flushleft}


10


\begin{flushleft}
Flexible (for M Tech and MBA)
\end{flushleft}


\begin{flushleft}
Flexible
\end{flushleft}


\begin{flushleft}
Not specified
\end{flushleft}


10


\begin{flushleft}
20\% of sanctioned strength
\end{flushleft}


\begin{flushleft}
Not specified
\end{flushleft}


\begin{flushleft}
Flexible
\end{flushleft}


05





\begin{flushleft}
10.6.3 Double Major
\end{flushleft}


\begin{flushleft}
Name of the Department
\end{flushleft}


\begin{flushleft}
Aerospace Engineering
\end{flushleft}


\begin{flushleft}
Biological Sciences and Bio-Engineering
\end{flushleft}


\begin{flushleft}
Chemical Engineering
\end{flushleft}


\begin{flushleft}
Chemistry
\end{flushleft}


\begin{flushleft}
Civil Engineering
\end{flushleft}


\begin{flushleft}
Computer Science and Engineering
\end{flushleft}


\begin{flushleft}
Economics
\end{flushleft}


\begin{flushleft}
Electrical Engineering
\end{flushleft}


\begin{flushleft}
Materials Science and Engineering
\end{flushleft}


\begin{flushleft}
Mathematics and Statistics
\end{flushleft}


\begin{flushleft}
Mechanical Engineering
\end{flushleft}


\begin{flushleft}
Physics
\end{flushleft}





\begin{flushleft}
Seats Available for Double Major
\end{flushleft}


15


04


08


10


20


10


6


10


\begin{flushleft}
Not specified
\end{flushleft}


05


\begin{flushleft}
10\% of sanctioned strength
\end{flushleft}


10





\begin{flushleft}
10.6.4 Minor
\end{flushleft}


\begin{flushleft}
If the first course of a Minor programme in a department is a compulsory departmental course then the number of students
\end{flushleft}


\begin{flushleft}
admitted per year in that Minor programme will be at least (i) 10 or (ii) 20\% of the department's existing batch strength,
\end{flushleft}


\begin{flushleft}
whichever is smaller.
\end{flushleft}


\begin{flushleft}
If the first course of the Minor is not a compulsory department course, the department will admit up to 20\% of its existing
\end{flushleft}


\begin{flushleft}
batch strength. If the number of students exceeds the number that a department can accommodate, the department should
\end{flushleft}


\begin{flushleft}
clearly state the criterion it is going to apply to limit the number of students. This limitation in number of students applies
\end{flushleft}


\begin{flushleft}
only in the first course of a Minor. In subsequent courses this limit does not apply to students who have already been
\end{flushleft}


\begin{flushleft}
admitted to the Minor.
\end{flushleft}





30





\begin{flushleft}
\newpage
Chapter 11
\end{flushleft}





\begin{flushleft}
Leave of Absence
\end{flushleft}


\begin{flushleft}
11.1 Mid-Semester Recess and Vacation
\end{flushleft}


\begin{flushleft}
Undergraduate students are entitled to avail the mid-semester recess, winter and summer vacations as specified in the
\end{flushleft}


\begin{flushleft}
Academic Calendar without seeking any permission.
\end{flushleft}





\begin{flushleft}
11.2 Short Leave
\end{flushleft}


\begin{flushleft}
Leave of absence during the semester is discouraged for all registered students. However, for bona fide reasons, a student
\end{flushleft}


\begin{flushleft}
may apply for leave using the online registration system. The extent of this leave for medical reasons can be a maximum of
\end{flushleft}


\begin{flushleft}
ten working days. A maximum of five working days of leave may also be granted for any other valid reason. In no case may a
\end{flushleft}


\begin{flushleft}
student be granted leave of absence in excess of fifteen working days in a semester.
\end{flushleft}


\begin{flushleft}
The leave of absence in the summer term shall correspondingly be five working days (medical) and three working days
\end{flushleft}


\begin{flushleft}
(others), i.e., eight working days total.
\end{flushleft}


\begin{flushleft}
Application for leave of absence may be submitted online through OARS.
\end{flushleft}





\begin{flushleft}
11.3 Temporary Withdrawal/ Semester Leave
\end{flushleft}


\begin{flushleft}
i.
\end{flushleft}





\begin{flushleft}
A student may be allowed a leave of absence for a whole semester (temporary withdrawal) for bona fide reasons. Such
\end{flushleft}


\begin{flushleft}
leave of absence shall ordinarily not exceed two semesters with or without break during the entire period of the
\end{flushleft}


\begin{flushleft}
academic programme.
\end{flushleft}


\begin{flushleft}
ii. An application for temporary withdrawal should be made before the date of registration for the semester as mentioned
\end{flushleft}


\begin{flushleft}
in the Academic Calendar. However, under exceptional circumstances, a student may apply for withdrawal anytime
\end{flushleft}


\begin{flushleft}
during the semester.
\end{flushleft}


\begin{flushleft}
iii. Application for temporary withdrawal should be addressed to the Chairperson, SUGC, and routed through the DUGC
\end{flushleft}


\begin{flushleft}
convener. It should be submitted to the Undergraduate office with supporting documents such as a medical certificate
\end{flushleft}


\begin{flushleft}
(in original) in case of an illness.
\end{flushleft}


\begin{flushleft}
iv. A student who remains on authorized leave of absence due to ill health shall be required to submit a certificate from a
\end{flushleft}


\begin{flushleft}
Registered Medical Practitioner to the effect that s/he is sufficiently cured and is fit to resume her/his studies. The
\end{flushleft}


\begin{flushleft}
Institute may constitute a Medical Board to determine the fitness of the student before registration. The registration of
\end{flushleft}


\begin{flushleft}
the student shall be provisional till the Board certifies the fitness. In the event that the Board recommends that the
\end{flushleft}


\begin{flushleft}
student is not yet fit to resume studies, the registration may be cancelled.
\end{flushleft}





\begin{flushleft}
11.4 Penalty for Unsanctioned or Excessive Leave
\end{flushleft}


\begin{flushleft}
If a student is found to be absent from class without sanctioned leave, then the course instructor may recommend deregistration of the student from the course. The policy regarding unsanctioned leave leading to de-registration or any other
\end{flushleft}


\begin{flushleft}
consequence must be declared by the instructor at the beginning of the course. This rule applies to regular as well as modular
\end{flushleft}


\begin{flushleft}
courses and in regular semesters as well as in the summer term.
\end{flushleft}


\begin{flushleft}
If a student is found to be absent from a majority of lectures, tutorials and laboratory sessions for more than 20 working days
\end{flushleft}


\begin{flushleft}
(not necessarily contiguous) in a semester, with or without sanction, then her/his registration for all the courses in that
\end{flushleft}


\begin{flushleft}
semester may be cancelled by Senate on recommendation of SUGC resulting in a forced semester drop.
\end{flushleft}


\begin{flushleft}
If a student is found to be absent from all academic activities in a semester without authorization for more than 30 working
\end{flushleft}


\begin{flushleft}
days contiguously or s/he does not appear, without a compelling reason, for the end-semester examinations in all the courses
\end{flushleft}


\begin{flushleft}
in which s/he is registered, then her/his programme will be terminated.
\end{flushleft}





31





\begin{flushleft}
\newpage
11.5 Permission to proceed to other Institutions
\end{flushleft}


\begin{flushleft}
In order to help students to broaden their horizons and gain course-work experience, they may be permitted to proceed to
\end{flushleft}


\begin{flushleft}
other academic institutions in India or abroad as a non-degree student. The following guidelines and procedures apply for
\end{flushleft}


\begin{flushleft}
this purpose:
\end{flushleft}


\begin{flushleft}
i. A student who satisfies the minimum eligibility conditions given below may spend up to two semesters and/or two
\end{flushleft}


\begin{flushleft}
summer terms in any academic institution of repute in India or abroad with prior permission of the SUGC.
\end{flushleft}


\begin{flushleft}
ii. The semester spent as a non-degree student will be counted as a part of the time spent in pursuit of the degree.
\end{flushleft}


\begin{flushleft}
11.5.1 Eligibility
\end{flushleft}


\begin{flushleft}
i.
\end{flushleft}


\begin{flushleft}
ii.
\end{flushleft}





\begin{flushleft}
Completion of 200 credits of course work
\end{flushleft}


\begin{flushleft}
CPI of at least 7.0
\end{flushleft}





\begin{flushleft}
11.5.2 Application Procedure
\end{flushleft}


\begin{flushleft}
i.
\end{flushleft}





\begin{flushleft}
The student shall make an application to SUGC through the concerned DUGC, giving details of the proposed programme
\end{flushleft}


\begin{flushleft}
and shall submit a statement of purpose with sufficient information about the Institution where s/he has chosen to
\end{flushleft}


\begin{flushleft}
spend time as a non-degree student.
\end{flushleft}


\begin{flushleft}
ii. The DUGC shall examine the student's proposal to determine whether the proposed programme is of a nature that the
\end{flushleft}


\begin{flushleft}
student will benefit from the exposure.
\end{flushleft}


\begin{flushleft}
iii. On the recommendation of the DUGC, SUGC may approve the proposal and grant permission, with leave of absence, to
\end{flushleft}


\begin{flushleft}
the student to proceed as a non-degree student to the selected Institution.
\end{flushleft}


\begin{flushleft}
iv. Any application for waiver of credits at IIT Kanpur or transfer of credits from the other Institution shall be decided in
\end{flushleft}


\begin{flushleft}
accordance with the procedure given in section 11.5.3.
\end{flushleft}


\begin{flushleft}
11.5.3 Transfer of Credits and Waiver in-lieu thereof
\end{flushleft}


\begin{flushleft}
i.
\end{flushleft}





\begin{flushleft}
Permission to proceed to another institution as a non-degree student does not imply that the student will automatically
\end{flushleft}


\begin{flushleft}
get waiver from the academic and other requirements of her/his ongoing undergraduate programme at the Institute.
\end{flushleft}


\begin{flushleft}
ii. On return, s/he may apply for waiver of courses from her/his program template which s/he thinks are equivalent to the
\end{flushleft}


\begin{flushleft}
courses successfully completed at the visited Institute as a non-degree student. With the application the student must
\end{flushleft}


\begin{flushleft}
submit an official transcript of the grades obtained by her/him at the visited Institute as a non-degree student and other
\end{flushleft}


\begin{flushleft}
documents/material that the concerned DUGC may require for evaluation. The DUGC will determine, by whatever means
\end{flushleft}


\begin{flushleft}
it deems fit, the equivalent courses (credits) and/or requirements for which the student may be given a waiver in her/his
\end{flushleft}


\begin{flushleft}
undergraduate programme at IIT Kanpur.
\end{flushleft}


\begin{flushleft}
iii. On the recommendation of the DUGC, SUGC may allow a student a waiver for a maximum of 100 credits against the
\end{flushleft}


\begin{flushleft}
course work completed elsewhere as a non-degree student.
\end{flushleft}


\begin{flushleft}
iv. Against each course or requirement for which a waiver is granted, symbol {``}W'' would appear on the Grade Report with
\end{flushleft}


\begin{flushleft}
an explanatory note that it stands for waiver granted due to courses taken and/or work done at the selected Institution
\end{flushleft}


\begin{flushleft}
elsewhere. All such courses and/or requirements will be deemed to carry zero credits for SPI/CPI calculation.
\end{flushleft}


\begin{flushleft}
v. Under no conditions will the grades earned at any other Institution appear on the Grade Report.
\end{flushleft}


\begin{flushleft}
Those students who are selected by the Institute, using prescribed rules and procedures, to proceed on any Institutional
\end{flushleft}


\begin{flushleft}
exchange programme will also be governed by these clauses for the transfer of academic credits, waiver, etc.
\end{flushleft}





32





\begin{flushleft}
\newpage
Chapter 12
\end{flushleft}





\begin{flushleft}
Scholarships, Awards and Medals
\end{flushleft}


\begin{flushleft}
12.1 Scholarships
\end{flushleft}


\begin{flushleft}
A number of Merit-cum-Means scholarships, free ships (i.e., tuition waiver), free basic messing and pocket allowance (for SC
\end{flushleft}


\begin{flushleft}
and ST categories), and endowment scholarships/fellowships are awarded to undergraduate students according to the rules
\end{flushleft}


\begin{flushleft}
and procedures laid down by the Senate. These scholarships/fellowships are administered by the Senate Scholarships and
\end{flushleft}


\begin{flushleft}
Prizes Committee (SSPC). More details about these scholarships can be found from the Dean of Students' Affairs (DoSA) office
\end{flushleft}


\begin{flushleft}
and DoSA webpage.
\end{flushleft}


\begin{flushleft}
The scholarships, etc. are paid up to the month in which a student completes all the requirements of her/his programme.
\end{flushleft}





\begin{flushleft}
12.2 Withdrawal of Scholarship
\end{flushleft}


\begin{flushleft}
These scholarships, etc. are liable to be withdrawn, partially or wholly, in case of misconduct, deliberate concealment of
\end{flushleft}


\begin{flushleft}
material facts and/or giving false information.
\end{flushleft}


\begin{flushleft}
A student leaving the Institute on her/his own accord without completing the programme of study may be required to refund
\end{flushleft}


\begin{flushleft}
the amount of scholarship, etc. received during the academic session in which s/he leaves the Institute.
\end{flushleft}





\begin{flushleft}
12.3 Awards and Medals
\end{flushleft}


\begin{flushleft}
To promote and recognize academic excellence, constructive leadership and overall growth and development of students,
\end{flushleft}


\begin{flushleft}
the Senate awards a number of prizes and medals established by the Institute on its own or through endowments/grants
\end{flushleft}


\begin{flushleft}
made by donors, with the approval of the Board of Governors. Details of the same can be found at the DoSA webpage. All
\end{flushleft}


\begin{flushleft}
matters related to awards and medals are handled by the Senate Scholarships and Prizes Committee (SSPC).
\end{flushleft}





33





\begin{flushleft}
\newpage
Chapter 13
\end{flushleft}





\begin{flushleft}
Conduct and Discipline
\end{flushleft}


\begin{flushleft}
13.1 Code of Conduct
\end{flushleft}


\begin{flushleft}
Students are expected to conduct themselves with integrity and proper consideration for others at all times. Students are
\end{flushleft}


\begin{flushleft}
expected to exhibit proper respect for others in their personal behavior and interpersonal interactions, both within and
\end{flushleft}


\begin{flushleft}
outside the campus. The institute strictly prohibits ragging and sexual harassment; any instance of either should be reported
\end{flushleft}


\begin{flushleft}
immediately and will be dealt with as a serious offense.
\end{flushleft}


\begin{flushleft}
In academic matters, absolute honesty is mandatory. The institute has a zero tolerance policy for any adoption of unfair
\end{flushleft}


\begin{flushleft}
means during examinations. In every other respect also, students are expected to do their academic work with integrity, with
\end{flushleft}


\begin{flushleft}
proper acknowledgement if material from other sources is included in their own work. Plagiarism, whether intended or not, is
\end{flushleft}


\begin{flushleft}
an act of academic dishonesty and penalized as such. If there is any doubt about what constitutes plagiarism, students should
\end{flushleft}


\begin{flushleft}
consult their instructors to ensure maintenance of academic honesty in their work. Any case of cheating will be dealt with
\end{flushleft}


\begin{flushleft}
strictly by the Institute.
\end{flushleft}


\begin{flushleft}
Students are expected to respect Institute property and follow all institute rules and regulations at all times.
\end{flushleft}


\begin{flushleft}
If students feel victimized by the conduct, academic or personal, of any other member of the Institute, they may register a
\end{flushleft}


\begin{flushleft}
complaint with the Ombudsperson for whom the contact information is available from the DoSA website. In case of any
\end{flushleft}


\begin{flushleft}
complaint related to sexual harassment, students should contact the Internal Complaints Committee (ICC) or Women's Cell
\end{flushleft}


\begin{flushleft}
(women\_cell@iitk.ac.in).
\end{flushleft}





\begin{flushleft}
13.2 Disciplinary Action and Related Matters
\end{flushleft}


\begin{flushleft}
Any violation of the Code of Conduct shall invite disciplinary action which may include punishments such as reprimand,
\end{flushleft}


\begin{flushleft}
disciplinary probation, fine, debarring from examinations, withdrawal of scholarship and/or placement services, withholding
\end{flushleft}


\begin{flushleft}
of grades and/or degrees, cancellation of registration and even expulsion from the Institute. In certain cases, the student may
\end{flushleft}


\begin{flushleft}
be barred from applying for a change of programme.
\end{flushleft}


\begin{flushleft}
The Instructor-in-Charge of a course may debar a student from the examination in which s/he is detected to be using unfair
\end{flushleft}


\begin{flushleft}
means. The Instructor/Tutor may take appropriate action against a student who misbehaves in her/his class. In all such cases,
\end{flushleft}


\begin{flushleft}
the Instructor/Tutor shall inform the DoAA office of all concerned information for record.
\end{flushleft}


\begin{flushleft}
The Warden-in-Charge of a Hall of Residence may reprimand, impose fine or take any other suitable measure against a
\end{flushleft}


\begin{flushleft}
resident who violates either the Code of Conduct or rules and regulations pertaining to the concerned Hall of Residence. In all
\end{flushleft}


\begin{flushleft}
such cases, the Warden-in-Charge shall inform the DoSA office of all the details for record.
\end{flushleft}


\begin{flushleft}
Involvement of a student in ragging may lead to his/her expulsion from the Institute.
\end{flushleft}


\begin{flushleft}
The Senate Student Affairs Committee (S-SAC) investigates alleged misdemeanors, complaints, etc. and recommends a
\end{flushleft}


\begin{flushleft}
suitable course of action. Violation of the Code of Conduct by an individual or of a group of students can be referred to this
\end{flushleft}


\begin{flushleft}
committee by a student, teacher or other functionary of the Institute. Further, in exceptional circumstances, the Chairperson,
\end{flushleft}


\begin{flushleft}
Senate, may appoint a special committee to investigate and/or recommend appropriate action for any act of gross
\end{flushleft}


\begin{flushleft}
indiscipline involving an individual or a number of students, which, in her/his view, may tarnish the image of the Institute.
\end{flushleft}


\begin{flushleft}
The recommendations of S-SAC are submitted to Chairperson, Senate for approval. In cases when the expulsion of a student
\end{flushleft}


\begin{flushleft}
from the Institute has been recommended, the matter is sent to the Senate for the final decision.
\end{flushleft}


\begin{flushleft}
A student, who feels aggrieved with the punishment awarded, may appeal to the Chairperson, Senate, stating clearly the case
\end{flushleft}


\begin{flushleft}
and explaining her/his position, and seeking reconsideration of the decision.
\end{flushleft}





34





\begin{flushleft}
\newpage
The Senate may not recommend a student who is found guilty of a major offense to the Board of Governors for the award of
\end{flushleft}


\begin{flushleft}
a degree/diploma/certificate even if s/he has satisfactorily completed all the academic requirements.
\end{flushleft}





35





\begin{flushleft}
\newpage
Chapter 14
\end{flushleft}





\begin{flushleft}
A Quick Guide for Students
\end{flushleft}


\begin{flushleft}
Important Information:
\end{flushleft}


\begin{flushleft}
1) The Academic Calendar is available on the DoAA website. It contains all the important dates for the calendar year, such
\end{flushleft}


\begin{flushleft}
as pre-registration, last date for dropping of courses, exams, and vacations.
\end{flushleft}


\begin{flushleft}
2) The course templates for all UG programmes are available through the DoAA website.
\end{flushleft}


\begin{flushleft}
3) All students have to do online academic pre-registration for courses to be taken next semester during the specified preregistration period.
\end{flushleft}


\begin{flushleft}
4) The minimum academic load in a regular semester is 35 credits; the maximum is 65 credits.
\end{flushleft}


\begin{flushleft}
5) During the summer term, eligible students may register for a maximum of 25 course credits.
\end{flushleft}


\begin{flushleft}
6) Parent programme refers to the basic four-year programme to which a student has been admitted (through JEE or
\end{flushleft}


\begin{flushleft}
Branch Change). For any questions regarding this programme, consult the parent programme DUGC.
\end{flushleft}


\begin{flushleft}
7) For questions regarding change / addition to the parent programme (dual degree, double major, minor), consult the
\end{flushleft}


\begin{flushleft}
Associate Dean of UG affairs (ADUG).
\end{flushleft}


\begin{flushleft}
8) Students may apply for short leave online through OARS.
\end{flushleft}


\begin{flushleft}
9) Information regarding financial aid and scholarships is available from the DoSA website.
\end{flushleft}


\begin{flushleft}
10) Students may register a complaint with the Ombudsperson (contact information available through DoSA website) if s/he
\end{flushleft}


\begin{flushleft}
feels victimized by the conduct, academic or personal, of any member of the Institute. In case of any complaint related to
\end{flushleft}


\begin{flushleft}
sexual harassment, students should contact the Internal Complaints Committee (ICC) or Women's Cell
\end{flushleft}


\begin{flushleft}
(women\_cell@iitk.ac.in).
\end{flushleft}


\begin{flushleft}
Here are answers to some of the most frequently asked questions regarding UG programmes:
\end{flushleft}


\begin{flushleft}
Q.1 Do I have to register for courses strictly as per the template?
\end{flushleft}


\begin{flushleft}
Course templates are advisory in terms of when specific courses need to be done. However, for compulsory Institute core
\end{flushleft}


\begin{flushleft}
courses, there is no guarantee that the instructor will allow you to register if you are taking it at a time other than the one
\end{flushleft}


\begin{flushleft}
specified in the template. For ESO and HSS, you may shift around courses, depending on their availability. For Department
\end{flushleft}


\begin{flushleft}
Core and Department Electives, please consult your DUGC before shifting any slots. You will NOT be automatically entitled to
\end{flushleft}


\begin{flushleft}
any required course if you are not taking it in the scheduled semester. Also, you need to ensure that you complete any prerequisites in time to do the succeeding courses.
\end{flushleft}


\begin{flushleft}
Q.2 What happens if I do not do academic pre-registration?
\end{flushleft}


\begin{flushleft}
You may do academic registration during Final Registration as specified in the Academic Calendar after paying a penalty fee
\end{flushleft}


\begin{flushleft}
for not doing academic pre-registration. In special circumstances this penalty fee may be waived by the SUGC if you apply for
\end{flushleft}


\begin{flushleft}
such a waiver within one month of the conclusion of the pre-registration period.
\end{flushleft}


\begin{flushleft}
Q.3 When and how can I apply for branch change?
\end{flushleft}


\begin{flushleft}
You may apply for Branch Change at the end of your second, third, or fourth semester. The DoAA office will send out a call for
\end{flushleft}


\begin{flushleft}
applications, and you need to apply on the specified form by the given deadline.
\end{flushleft}


\begin{flushleft}
Q.4 When and how can I apply for Double Major, Dual Degree or Minor?
\end{flushleft}


\begin{flushleft}
Double Major: At the end of your fourth semester if you have a CPI of 7.0 or above.
\end{flushleft}


\begin{flushleft}
Dual Degree: For category A (within the same department), at the end of your fifth, sixth, or seventh semesters; for category
\end{flushleft}


\begin{flushleft}
B/C (in another department/programme), at the end of your sixth semester only.
\end{flushleft}


\begin{flushleft}
MSPD Dual Degree: Students in the departments offering this programme may apply after the second or third semester if
\end{flushleft}


\begin{flushleft}
they have a CPI of 7.0 or above. Details regarding the application process should be confirmed from the DUGC.
\end{flushleft}


\begin{flushleft}
Minor: During your fourth, fifth or sixth semester. The DOAA office will send out calls for applications at appropriate times,
\end{flushleft}


\begin{flushleft}
and you need to apply on the specified forms by the given deadline. You CANNOT apply for a Minor if you are doing a Double
\end{flushleft}





36





\begin{flushleft}
\newpage
Major, though you may claim a retrospective Minor if you manage to complete all the required courses without ever being
\end{flushleft}


\begin{flushleft}
admitted to the Minor officially.
\end{flushleft}


\begin{flushleft}
Q.5 How are grades determined in a course?
\end{flushleft}


\begin{flushleft}
Each instructor will announce the course's grading policy at the beginning of the semester.
\end{flushleft}


\begin{flushleft}
Q.6 What is the passing grade, and what is the minimum CPI for graduation?
\end{flushleft}


\begin{flushleft}
The passing grade is D. For all B Tech /BS programmes, the only graduation requirement is successful completion of all the
\end{flushleft}


\begin{flushleft}
credit requirements specified in your template. For MSc 2-year, graduation requirement is completion of all required courses
\end{flushleft}


\begin{flushleft}
and a minimum CPI of 6.0. For all dual-degree students, the graduation requirements for the PG part of the programme are
\end{flushleft}


\begin{flushleft}
as per the PG manual.
\end{flushleft}


\begin{flushleft}
Q.7 Can I repeat a course to improve my grade?
\end{flushleft}


\begin{flushleft}
No, if you have passed a course with a D grade, you may not repeat it. You are required to repeat a course if you receive an E
\end{flushleft}


\begin{flushleft}
or F grade in it, since both are failing grades.
\end{flushleft}


\begin{flushleft}
Q.8 Can I drop a course if I am not performing well in it?
\end{flushleft}


\begin{flushleft}
You may request a course drop until the last date specified for this in the Academic Calendar. However, a course drop
\end{flushleft}


\begin{flushleft}
request will be accepted only if it is approved by the course instructor-in-charge as well as the DUGC convener.
\end{flushleft}


\begin{flushleft}
Q.9 Can I take courses in the summer term?
\end{flushleft}


\begin{flushleft}
Summer courses are primarily for clearing backlogs to ensure timely graduation and for Double Major / Bachelors-Masters
\end{flushleft}


\begin{flushleft}
Dual Degree students. If any seats are left over in the offered courses after admitting all students in the above categories,
\end{flushleft}


\begin{flushleft}
they may be given to students who are attempting to do courses in advance.
\end{flushleft}


\begin{flushleft}
Q.10 Can I take leave during a semester?
\end{flushleft}


\begin{flushleft}
In a semester, you may apply online for a maximum of 10 days of leave for bona fide medical reasons, and a maximum of 5
\end{flushleft}


\begin{flushleft}
days for family emergencies. You are responsible for making up any missed work during this period; the instructor is not
\end{flushleft}


\begin{flushleft}
obliged to provide any make-up assignments, quizzes, exams, etc. for those missed while you are on leave. For end-semester
\end{flushleft}


\begin{flushleft}
examinations, a make-up will be provided for those with CMO (IITK Chief Medical Officer) certified leave or any other
\end{flushleft}


\begin{flushleft}
compelling reason for absence during the exam period.
\end{flushleft}





37





\begin{flushleft}
\newpage
Chapter 16
\end{flushleft}





\begin{flushleft}
Waiver and Amendments
\end{flushleft}


\begin{flushleft}
16.1 Waiver
\end{flushleft}


\begin{flushleft}
The procedures and requirements set out in this manual, other than those in Chapters 3, 8, 9, and 10, may be relaxed or
\end{flushleft}


\begin{flushleft}
waived in special circumstances by SUGC. All such exceptions are reported to the Senate.
\end{flushleft}





\begin{flushleft}
16.2 Amendments
\end{flushleft}


\begin{flushleft}
Notwithstanding anything contained in this manual, the Senate of the Indian Institute of Technology Kanpur reserves the
\end{flushleft}


\begin{flushleft}
right to modify/amend without notice, the curricula, procedures, requirements, and rules pertaining to its undergraduate
\end{flushleft}


\begin{flushleft}
programmes.
\end{flushleft}





38





\newpage



\end{document}
