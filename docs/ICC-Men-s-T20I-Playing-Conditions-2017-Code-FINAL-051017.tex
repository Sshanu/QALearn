%% ================================================================================
%% This LaTeX file was created by AbiWord.                                         
%% AbiWord is a free, Open Source word processor.                                  
%% More information about AbiWord is available at http://www.abisource.com/        
%% ================================================================================

\documentclass[a4paper,portrait,12pt]{article}
\usepackage[latin1]{inputenc}
\usepackage{calc}
\usepackage{setspace}
\usepackage{fixltx2e}
\usepackage{graphicx}
\usepackage{multicol}
\usepackage[normalem]{ulem}
%% Please revise the following command, if your babel
%% package does not support en-IN
\usepackage[en]{babel}
\usepackage{color}
\usepackage{hyperref}
 
\begin{document}


\begin{flushleft}
ICC Men's Twenty20 International Playing Conditions
\end{flushleft}


\begin{flushleft}
(incorporating the 2017 Code of the MCC Laws of Cricket)
\end{flushleft}


\begin{flushleft}
Effective 28th September 2017
\end{flushleft}





\begin{flushleft}
\newpage
Contents
\end{flushleft}


1





\begin{flushleft}
THE PLAYERS ......................................................................................................................................................1
\end{flushleft}





2





\begin{flushleft}
THE UMPIRES ......................................................................................................................................................2
\end{flushleft}





3





\begin{flushleft}
THE SCORERS ....................................................................................................................................................6
\end{flushleft}





4





\begin{flushleft}
THE BALL ............................................................................................................................................................. 7
\end{flushleft}





5





\begin{flushleft}
THE BAT ............................................................................................................................................................... 7
\end{flushleft}





6





\begin{flushleft}
THE PITCH ...........................................................................................................................................................9
\end{flushleft}





7





\begin{flushleft}
THE CREASES ................................................................................................................................................... 10
\end{flushleft}





8





\begin{flushleft}
THE WICKETS .................................................................................................................................................... 11
\end{flushleft}





9





\begin{flushleft}
PREPARATION AND MAINTENANCE OF THE PLAYING AREA ...................................................................... 11
\end{flushleft}





10





\begin{flushleft}
COVERING THE PITCH ..................................................................................................................................... 13
\end{flushleft}





11





\begin{flushleft}
INTERVALS ........................................................................................................................................................ 13
\end{flushleft}





12





\begin{flushleft}
START OF PLAY; CESSATION OF PLAY .......................................................................................................... 14
\end{flushleft}





13





\begin{flushleft}
INNINGS ............................................................................................................................................................. 16
\end{flushleft}





14





\begin{flushleft}
THE FOLLOW-ON .............................................................................................................................................. 18
\end{flushleft}





15





\begin{flushleft}
DECLARATION AND FORFEITURE .................................................................................................................. 19
\end{flushleft}





16





\begin{flushleft}
THE RESULT ...................................................................................................................................................... 19
\end{flushleft}





17





\begin{flushleft}
THE OVER .......................................................................................................................................................... 21
\end{flushleft}





18





\begin{flushleft}
SCORING RUNS ................................................................................................................................................ 22
\end{flushleft}





19





\begin{flushleft}
BOUNDARIES ..................................................................................................................................................... 25
\end{flushleft}





20





\begin{flushleft}
DEAD BALL......................................................................................................................................................... 27
\end{flushleft}





21





\begin{flushleft}
NO BALL ............................................................................................................................................................. 28
\end{flushleft}





22





\begin{flushleft}
WIDE BALL ......................................................................................................................................................... 31
\end{flushleft}





23





\begin{flushleft}
BYE AND LEG BYE ............................................................................................................................................ 32
\end{flushleft}





24





\begin{flushleft}
FIELDER'S ABSENCE; SUBSTITUTES ............................................................................................................. 33
\end{flushleft}





25





\begin{flushleft}
BATSMAN'S INNINGS ........................................................................................................................................ 35
\end{flushleft}





26





\begin{flushleft}
PRACTICE ON THE FIELD ................................................................................................................................. 36
\end{flushleft}





27





\begin{flushleft}
THE WICKET-KEEPER....................................................................................................................................... 37
\end{flushleft}





28





\begin{flushleft}
THE FIELDER ..................................................................................................................................................... 38
\end{flushleft}





29





\begin{flushleft}
THE WICKET IS DOWN...................................................................................................................................... 40
\end{flushleft}





30





\begin{flushleft}
BATSMAN OUT OF HIS GROUND ..................................................................................................................... 41
\end{flushleft}





31





\begin{flushleft}
APPEALS ............................................................................................................................................................ 42
\end{flushleft}





32





\begin{flushleft}
BOWLED ............................................................................................................................................................. 43
\end{flushleft}





33





\begin{flushleft}
CAUGHT ............................................................................................................................................................. 43
\end{flushleft}





34





\begin{flushleft}
HIT THE BALL TWICE ........................................................................................................................................ 44
\end{flushleft}





35





\begin{flushleft}
HIT WICKET........................................................................................................................................................ 45
\end{flushleft}





36





\begin{flushleft}
LEG BEFORE WICKET....................................................................................................................................... 45
\end{flushleft}





37





\begin{flushleft}
OBSTRUCTING THE FIELD ............................................................................................................................... 46
\end{flushleft}





38





\begin{flushleft}
RUN OUT ............................................................................................................................................................ 47
\end{flushleft}





39





\begin{flushleft}
STUMPED ........................................................................................................................................................... 48
\end{flushleft}





\begin{flushleft}
i
\end{flushleft}





\newpage
40





\begin{flushleft}
TIMED OUT......................................................................................................................................................... 48
\end{flushleft}





41





\begin{flushleft}
UNFAIR PLAY ..................................................................................................................................................... 48
\end{flushleft}





42





\begin{flushleft}
PLAYERS' CONDUCT ........................................................................................................................................ 58
\end{flushleft}





\begin{flushleft}
Appendix A Definitions ................................................................................................................................................ 61
\end{flushleft}


1





\begin{flushleft}
The match ........................................................................................................................................................... 61
\end{flushleft}





2





\begin{flushleft}
Implements and equipment ................................................................................................................................. 61
\end{flushleft}





3





\begin{flushleft}
The playing area.................................................................................................................................................. 62
\end{flushleft}





4





\begin{flushleft}
Positioning ........................................................................................................................................................... 62
\end{flushleft}





5





\begin{flushleft}
Umpires and decision-making ............................................................................................................................. 62
\end{flushleft}





6





\begin{flushleft}
Batsmen .............................................................................................................................................................. 63
\end{flushleft}





7





\begin{flushleft}
Fielders................................................................................................................................................................ 64
\end{flushleft}





8





\begin{flushleft}
Substitutes........................................................................................................................................................... 64
\end{flushleft}





9





\begin{flushleft}
Bowlers................................................................................................................................................................ 64
\end{flushleft}





10





\begin{flushleft}
The ball................................................................................................................................................................ 64
\end{flushleft}





11





\begin{flushleft}
Runs .................................................................................................................................................................... 65
\end{flushleft}





12





\begin{flushleft}
The person .......................................................................................................................................................... 65
\end{flushleft}





13





\begin{flushleft}
Off side / on side; in front of / behind the popping crease.................................................................................... 66
\end{flushleft}





\begin{flushleft}
Appendix B Equipment................................................................................................................................................ 67
\end{flushleft}


1





\begin{flushleft}
The Bat ................................................................................................................................................................ 67
\end{flushleft}





2





\begin{flushleft}
The wickets ......................................................................................................................................................... 69
\end{flushleft}





3





\begin{flushleft}
Wicket-keeping gloves......................................................................................................................................... 69
\end{flushleft}





\begin{flushleft}
Appendix C The venue ................................................................................................................................................ 71
\end{flushleft}


1





\begin{flushleft}
The pitch and the creases ................................................................................................................................... 71
\end{flushleft}





2





\begin{flushleft}
Restriction on the placement of fielders............................................................................................................... 71
\end{flushleft}





3





\begin{flushleft}
Advertising on grounds, perimeter boards and sight-screens.............................................................................. 72
\end{flushleft}





4





\begin{flushleft}
Markings on outfield ............................................................................................................................................ 72
\end{flushleft}





\begin{flushleft}
Appendix D Decision Review System (DRS) and Third Umpire Protocol .................................................................... 73
\end{flushleft}


1





\begin{flushleft}
General................................................................................................................................................................ 73
\end{flushleft}





2





\begin{flushleft}
Umpire Review .................................................................................................................................................... 74
\end{flushleft}





3





\begin{flushleft}
Player Review ..................................................................................................................................................... 76
\end{flushleft}





4





\begin{flushleft}
Interpretation of Playing Conditions ..................................................................................................................... 83
\end{flushleft}





\begin{flushleft}
Appendix E Calculations ............................................................................................................................................. 84
\end{flushleft}


\begin{flushleft}
Appendix F Procedure for the Super Over .................................................................................................................. 87
\end{flushleft}





\begin{flushleft}
ii
\end{flushleft}





\begin{flushleft}
\newpage
ICC Men's Twenty20 International
\end{flushleft}


\begin{flushleft}
Playing Conditions
\end{flushleft}


\begin{flushleft}
(incorporating the 2017 Code of the MCC Laws of Cricket)
\end{flushleft}


\begin{flushleft}
Preamble - The Spirit of Cricket
\end{flushleft}


\begin{flushleft}
Cricket owes much of its appeal and enjoyment to the fact that it should be played not only according to the Laws
\end{flushleft}


\begin{flushleft}
(which are incorporated within these Playing Conditions), but also within the Spirit of Cricket.
\end{flushleft}


\begin{flushleft}
The major responsibility for ensuring fair play rests with the captains, but extends to all players, umpires and,
\end{flushleft}


\begin{flushleft}
especially in junior cricket, teachers, coaches and parents.
\end{flushleft}


\begin{flushleft}
Respect is central to the Spirit of Cricket.
\end{flushleft}


\begin{flushleft}
Respect your captain, team-mates, opponents and the authority of the umpires.
\end{flushleft}


\begin{flushleft}
Play hard and play fair.
\end{flushleft}


\begin{flushleft}
Accept the umpire's decision.
\end{flushleft}


\begin{flushleft}
Create a positive atmosphere by your own conduct, and encourage others to do likewise.
\end{flushleft}


\begin{flushleft}
Show self-discipline, even when things go against you.
\end{flushleft}


\begin{flushleft}
Congratulate the opposition on their successes, and enjoy those of your own team.
\end{flushleft}


\begin{flushleft}
Thank the officials and your opposition at the end of the match, whatever the result.
\end{flushleft}


\begin{flushleft}
Cricket is an exciting game that encourages leadership, friendship and teamwork, which brings together people from
\end{flushleft}


\begin{flushleft}
different nationalities, cultures and religions, especially when played within the Spirit of Cricket.
\end{flushleft}





\begin{flushleft}
1 THE PLAYERS
\end{flushleft}


1.1





\begin{flushleft}
Number of players
\end{flushleft}





\begin{flushleft}
A match is played between two sides, each of eleven players, one of whom shall be captain.
\end{flushleft}





1.2





\begin{flushleft}
Nomination and replacement of players
\end{flushleft}





1.2.1





\begin{flushleft}
Each captain shall nominate 11 players plus a maximum of 4 substitute fielders in writing to the ICC Match
\end{flushleft}


\begin{flushleft}
Referee before the toss. No player (member of the playing eleven) may be changed after the nomination
\end{flushleft}


\begin{flushleft}
without the consent of the opposing captain.
\end{flushleft}





1.2.2





\begin{flushleft}
Only those nominated as substitute fielders shall be entitled to act as substitute fielders during the match,
\end{flushleft}


\begin{flushleft}
unless the ICC Match Referee, in exceptional circumstances, allows subsequent additions.
\end{flushleft}





1.2.3





\begin{flushleft}
All those nominated including those nominated as substitute fielders, must be eligible to play for that
\end{flushleft}


\begin{flushleft}
particular team and by such nomination the nominees shall warrant that they are so eligible.
\end{flushleft}





1.2.4





\begin{flushleft}
In addition, by their nomination, the nominees shall be deemed to have agreed to abide by all the applicable
\end{flushleft}


\begin{flushleft}
ICC Regulations pertaining to international cricket and in particular, the Clothing and Equipment
\end{flushleft}


\begin{flushleft}
Regulations, the Code of Conduct for Players and Player Support Personnel (hereafter referred to as the
\end{flushleft}


\begin{flushleft}
ICC Code of Conduct), the Anti-Racism Code for Players and Player Support Personnel, the Anti-Doping
\end{flushleft}


\begin{flushleft}
Code and the Anti-Corruption Code.
\end{flushleft}





1.2.5





\begin{flushleft}
A player or player support personnel who has been suspended from participating in a match shall not, from
\end{flushleft}


\begin{flushleft}
the toss of the coin and for the remainder of the match thereafter:
\end{flushleft}


1.2.5.1





\begin{flushleft}
Be nominated as, or carry out any of the duties or responsibilities of a substitute fielder, or
\end{flushleft}





1





\newpage
1.2.5.2





\begin{flushleft}
Enter any part of the playing area (which shall include the field of play and the area between the
\end{flushleft}


\begin{flushleft}
boundary and the perimeter boards) at any time, including any scheduled or unscheduled
\end{flushleft}


\begin{flushleft}
breaks in play.
\end{flushleft}





\begin{flushleft}
A player who has been suspended from participating in a match shall be permitted from the toss of the coin
\end{flushleft}


\begin{flushleft}
and for the remainder of the match thereafter be permitted to enter the players' dressing room provided that
\end{flushleft}


\begin{flushleft}
the players' dressing room (or any part thereof) for the match is not within the playing area described in
\end{flushleft}


\begin{flushleft}
clause 1.2.5.2 above (for example, the player is not permitted to enter the on-field {`}dug-out').
\end{flushleft}





1.3





\begin{flushleft}
Captain
\end{flushleft}





1.3.1





\begin{flushleft}
If at any time the captain is not available, a deputy shall act for him.
\end{flushleft}





1.3.2





\begin{flushleft}
If a captain is not available to nominate the players, then any person associated with that team may act as
\end{flushleft}


\begin{flushleft}
his deputy to do so. See clause 1.2.
\end{flushleft}





1.3.3





\begin{flushleft}
At any time after the nomination of the players, only a nominated player can act as deputy in discharging the
\end{flushleft}


\begin{flushleft}
duties and responsibilities of the captain as stated in these Playing Conditions, including at the toss. See
\end{flushleft}


\begin{flushleft}
clause 13.4 (The toss).
\end{flushleft}





1.3.4





\begin{flushleft}
Each Member Board must nominate its {`}T20I Team Captain' to the ICC when appointed.
\end{flushleft}





1.3.5





\begin{flushleft}
If the T20I Team Captain' is not participating in a series, the relevant Home Board must nominate a
\end{flushleft}


\begin{flushleft}
replacement {`}T20I Team Captain' for the series. The Home Board shall advise the series Match Referee.
\end{flushleft}





1.3.6





\begin{flushleft}
If the {`}T20I Team Captain' plays in a match without being the nominated captain for that match, he will be
\end{flushleft}


\begin{flushleft}
deemed to be the captain should any penalties be applied for over rate breaches under the ICC Code of
\end{flushleft}


\begin{flushleft}
Conduct.
\end{flushleft}





1.4





\begin{flushleft}
Responsibility of captains
\end{flushleft}





\begin{flushleft}
The captains are responsible at all times for ensuring that play is conducted within the Spirit of Cricket as well as
\end{flushleft}


\begin{flushleft}
within these Playing Conditions.
\end{flushleft}





\begin{flushleft}
2 THE UMPIRES
\end{flushleft}


2.1





\begin{flushleft}
Appointment and attendance
\end{flushleft}





\begin{flushleft}
The following rules for the selection and appointment of T20I umpires shall be followed as far as it is practicable to do
\end{flushleft}


\begin{flushleft}
so:
\end{flushleft}


2.1.1





\begin{flushleft}
The umpires shall control the game as required by these Playing Conditions, with absolute impartiality and
\end{flushleft}


\begin{flushleft}
shall be present at the ground at least two hours before the scheduled start of play,
\end{flushleft}





2.1.2





\begin{flushleft}
The ICC shall establish an {`}Elite Panel' of umpires who shall be contracted to the ICC.
\end{flushleft}





2.1.3





\begin{flushleft}
Each Full Member shall nominate from its panel of first class umpires up to four umpires to an {`}International
\end{flushleft}


\begin{flushleft}
Panel'
\end{flushleft}





2.1.4





\begin{flushleft}
The Home Board shall appoint both umpires to stand in each T20I match. Such umpires shall be selected
\end{flushleft}


\begin{flushleft}
from its umpires on the Elite or International Panel.
\end{flushleft}





2.1.5





\begin{flushleft}
In all T20I matches, the third umpire will be appointed by the Home Board and shall act as the emergency
\end{flushleft}


\begin{flushleft}
on-field umpire and officiate in regard to TV replays. Such appointment shall be made from the {`}Elite Panel'
\end{flushleft}


\begin{flushleft}
or the {`}International Panel'.
\end{flushleft}


2.1.5.1





2.1.6





\begin{flushleft}
The playing conditions governing the use of the DRS and the third umpire are included in
\end{flushleft}


\begin{flushleft}
Appendix D.
\end{flushleft}





\begin{flushleft}
The Home Board shall also appoint a fourth umpire for each T20I match from its panel of first class umpires.
\end{flushleft}


\begin{flushleft}
The fourth umpire shall act as the emergency third umpire. In {`}DRS' T20I matches the fourth umpire shall be
\end{flushleft}


\begin{flushleft}
appointed from the {``}International Panel''
\end{flushleft}





2





\newpage
2.1.7





\begin{flushleft}
The ICC shall appoint the match referee for all matches (ICC Match Referee).
\end{flushleft}





2.1.8





\begin{flushleft}
The ICC Match Referee shall not be from the same country as the participating teams.
\end{flushleft}





2.1.9





\begin{flushleft}
Neither team will have a right of objection to the appointment of any umpire or match referee.
\end{flushleft}





2.2





\begin{flushleft}
Change of umpire
\end{flushleft}





\begin{flushleft}
An umpire shall not be changed during the match, other than in exceptional circumstances, unless he/she is injured
\end{flushleft}


\begin{flushleft}
or ill.
\end{flushleft}





2.3





\begin{flushleft}
Consultation with Home Board
\end{flushleft}





\begin{flushleft}
Before the match the umpires shall consult with the Home Board to determine;
\end{flushleft}


2.3.1





\begin{flushleft}
the balls to be used during the match. See clause 4 (The ball).
\end{flushleft}





2.3.2





\begin{flushleft}
the hours of play and the times and durations of any agreed intervals.
\end{flushleft}





2.3.3





\begin{flushleft}
which clock or watch and back-up time piece is to be used during the match.
\end{flushleft}





2.3.4





\begin{flushleft}
the boundary of the field of play. See clause 19 (Boundaries).
\end{flushleft}





2.3.5





\begin{flushleft}
the use of covers. See clause 10 (Covering the pitch).
\end{flushleft}





2.3.6





\begin{flushleft}
any special conditions of play affecting the conduct of the match.
\end{flushleft}





\begin{flushleft}
inform the scorers of agreements in 2.3.2, 2.3.3, 2.3.4 and 2.3.6.
\end{flushleft}





2.4





\begin{flushleft}
The wickets, creases and boundaries
\end{flushleft}





\begin{flushleft}
Before the toss and during the match, the umpires shall satisfy themselves that
\end{flushleft}


2.4.1





\begin{flushleft}
the wickets are properly pitched. See clause 8 (The wickets)
\end{flushleft}





2.4.2





\begin{flushleft}
the creases are correctly marked. See clause 7 (The creases).
\end{flushleft}





2.4.3





\begin{flushleft}
the boundary of the field of play complies with the requirements of clauses 19.1 (Determining the boundary
\end{flushleft}


\begin{flushleft}
of the field of play), 19.2 (Identifying and marking the boundary) and 19.3 (Restoring the boundary).
\end{flushleft}





2.5





\begin{flushleft}
Conduct of the match, implements and equipment
\end{flushleft}





\begin{flushleft}
Before the toss and during the match, the umpires shall satisfy themselves that
\end{flushleft}


2.5.1





\begin{flushleft}
the conduct of the match is strictly in accordance with these Playing Conditions.
\end{flushleft}





2.5.2





\begin{flushleft}
the implements used in the match conform to the following
\end{flushleft}


2.5.2.1





\begin{flushleft}
clause 4 (The ball).
\end{flushleft}





2.5.2.2





\begin{flushleft}
externally visible requirements of clause 5 (The bat) and paragraph 1 of Appendix B.
\end{flushleft}





2.5.2.3





\begin{flushleft}
either clauses 8.2 (Size of stumps) and 8.3 (The bails).
\end{flushleft}





2.5.3





\begin{flushleft}
no player uses equipment other than that permitted. See paragraph 2 of Appendix A. Note particularly
\end{flushleft}


\begin{flushleft}
therein the interpretation of {`}protective helmet'.
\end{flushleft}





2.5.4





\begin{flushleft}
the wicket-keeper's gloves comply with the requirements of clause 27.2 (Gloves).
\end{flushleft}





2.6





\begin{flushleft}
Fair and unfair play
\end{flushleft}





\begin{flushleft}
The umpires shall be the sole judges of fair and unfair play.
\end{flushleft}





2.7





\begin{flushleft}
Fitness for play
\end{flushleft}





2.7.1





\begin{flushleft}
It is solely for the umpires together to decide whether either conditions of ground, weather or light or
\end{flushleft}


\begin{flushleft}
exceptional circumstances mean that it would be dangerous or unreasonable for play to take place.
\end{flushleft}





3





\begin{flushleft}
\newpage
Conditions shall not be regarded as either dangerous or unreasonable merely because they are not ideal.
\end{flushleft}


\begin{flushleft}
The fact that the grass and the ball are wet does not warrant the ground conditions being regarded as
\end{flushleft}


\begin{flushleft}
unreasonable or dangerous.
\end{flushleft}


2.7.2





\begin{flushleft}
Conditions shall be regarded as dangerous if there is actual and foreseeable risk to the safety of any player
\end{flushleft}


\begin{flushleft}
or umpire.
\end{flushleft}





2.7.3





\begin{flushleft}
Conditions shall be regarded as unreasonable if, although posing no risk to safety, it would not be sensible
\end{flushleft}


\begin{flushleft}
for play to proceed.
\end{flushleft}





2.7.4





\begin{flushleft}
If the umpires consider the ground is so wet or slippery as to deprive the bowler of a reasonable foothold,
\end{flushleft}


\begin{flushleft}
the fielders of the power of free movement, or the batsmen of the ability to play their strokes or to run
\end{flushleft}


\begin{flushleft}
between the wickets, then these conditions shall be regarded as so bad that it would be dangerous and
\end{flushleft}


\begin{flushleft}
unreasonable for play to take place.
\end{flushleft}





2.8





\begin{flushleft}
Suspension of play in dangerous or unreasonable circumstances
\end{flushleft}





2.8.1





\begin{flushleft}
All references to ground include the pitch. See clause 6.1 (Area of pitch).
\end{flushleft}





2.8.2





\begin{flushleft}
If at any time the umpires together agree that the conditions of ground, weather or light, or any other
\end{flushleft}


\begin{flushleft}
circumstances are dangerous or unreasonable, they shall immediately suspend play, or not allow play to
\end{flushleft}


\begin{flushleft}
start or to recommence. The decision as to whether conditions are so bad as to warrant such action is one
\end{flushleft}


\begin{flushleft}
for the umpires alone to make, following consultation with the ICC Match Referee.
\end{flushleft}





2.8.3





\begin{flushleft}
If circumstances are warranted, the umpires shall stop play and instruct the Ground Authority to take
\end{flushleft}


\begin{flushleft}
whatever action they can and use whatever equipment is necessary to remove as much dew as possible
\end{flushleft}


\begin{flushleft}
from the outfield when conditions become unreasonable or dangerous. The umpires may also instruct the
\end{flushleft}


\begin{flushleft}
ground staff to take such action during scheduled and unscheduled breaks in play.
\end{flushleft}





2.8.4





\begin{flushleft}
The umpires shall disregard any shadow on the pitch from the stadium or from any permanent object on the
\end{flushleft}


\begin{flushleft}
ground.
\end{flushleft}





2.8.5





\begin{flushleft}
Light Meters
\end{flushleft}


\begin{flushleft}
It is the responsibility of the ICC to supply light meters to the match officials to be used in accordance with
\end{flushleft}


\begin{flushleft}
these playing conditions.
\end{flushleft}





2.8.6





2.8.5.1





\begin{flushleft}
All light meters shall be uniformly calibrated.
\end{flushleft}





2.8.5.2





\begin{flushleft}
The umpires shall be entitled to use light meter readings as a guideline for determining whether
\end{flushleft}


\begin{flushleft}
the light is fit for play in accordance with the criteria set out in clause 2.8.2 above.
\end{flushleft}





2.8.5.3





\begin{flushleft}
Light meter readings may accordingly be used by the umpires:
\end{flushleft}


2.8.5.3.1





\begin{flushleft}
To determine whether there has been at any stage a deterioration or
\end{flushleft}


\begin{flushleft}
improvement in the light.
\end{flushleft}





2.8.5.3.2





\begin{flushleft}
As benchmarks for the remainder of a match.
\end{flushleft}





\begin{flushleft}
Use of artificial lights
\end{flushleft}


\begin{flushleft}
If in the opinion of the umpires, natural light is deteriorating to an unfit level, they shall authorize the Ground
\end{flushleft}


\begin{flushleft}
Authority to use the available artificial lighting so that the match can commence or continue in acceptable
\end{flushleft}


\begin{flushleft}
conditions.
\end{flushleft}


\begin{flushleft}
In the event of power failure or lights malfunction, the provisions relating to the delay or interruption of play
\end{flushleft}


\begin{flushleft}
due to bad weather or light shall apply.
\end{flushleft}





2.8.7





\begin{flushleft}
When there is a suspension of play it is the responsibility of the umpires to monitor conditions. They shall
\end{flushleft}


\begin{flushleft}
make inspections as often as appropriate, unaccompanied by any players or officials. Immediately the
\end{flushleft}


\begin{flushleft}
umpires together agree that the conditions are no longer dangerous or unreasonable they shall call upon the
\end{flushleft}


\begin{flushleft}
players to resume play.
\end{flushleft}





4





\newpage
2.8.8





\begin{flushleft}
The safety of all persons within the ground is of paramount importance to the ICC. In the event that of any
\end{flushleft}


\begin{flushleft}
threatening circumstance, whether actual or perceived (including for example weather, pitch invasions, act
\end{flushleft}


\begin{flushleft}
of God, etc.), then the umpires, on the advice of the ICC Match Referee, should suspend play and all
\end{flushleft}


\begin{flushleft}
players and officials should immediately be asked to leave the field of play in a safe and orderly manner and
\end{flushleft}


\begin{flushleft}
to relocate to a secure and safe area (depending on each particular threat) pending the satisfactory passing
\end{flushleft}


\begin{flushleft}
or resolution of such threat or risk to the reasonable satisfaction of the umpires, ICC Match Referee, the
\end{flushleft}


\begin{flushleft}
head of the relevant Ground Authority, the head of ground security and/or the police as the circumstances
\end{flushleft}


\begin{flushleft}
may require.
\end{flushleft}





2.8.9





\begin{flushleft}
Where play is suspended under clause 2.8.8 above the decision to abandon or resume play shall be the
\end{flushleft}


\begin{flushleft}
responsibility of the ICC Match Referee who shall act only after consultation with the head of ground security
\end{flushleft}


\begin{flushleft}
and the police.
\end{flushleft}





2.9





\begin{flushleft}
Position of umpires
\end{flushleft}





\begin{flushleft}
The umpires shall stand where they can best see any act upon which their decision may be required.
\end{flushleft}


\begin{flushleft}
Subject to this over-riding consideration, the bowler's end umpire shall stand in a position so as not to interfere with
\end{flushleft}


\begin{flushleft}
either the bowler's run-up or the striker's view.
\end{flushleft}


\begin{flushleft}
The striker's end umpire may elect to stand on the off side instead of the on side of the pitch, provided he/she informs
\end{flushleft}


\begin{flushleft}
the captain of the fielding side, the striker and the other umpire.
\end{flushleft}





2.10





\begin{flushleft}
Umpires changing ends
\end{flushleft}





\begin{flushleft}
Shall not apply.
\end{flushleft}





2.11





\begin{flushleft}
Disagreement and dispute
\end{flushleft}





\begin{flushleft}
Where there is disagreement or dispute about any matter, the umpires together shall make the final decision. See
\end{flushleft}


\begin{flushleft}
also clause 31.6 (Consultation by umpires).
\end{flushleft}





2.12





\begin{flushleft}
Umpire's decision
\end{flushleft}





\begin{flushleft}
An umpire may alter any decision provided that such alteration is made promptly. This apart, an umpire's decision,
\end{flushleft}


\begin{flushleft}
once made, is final.
\end{flushleft}





2.13





\begin{flushleft}
Signals
\end{flushleft}





2.13.1





\begin{flushleft}
The following code of signals shall be used by umpires.
\end{flushleft}


2.13.1.1





\begin{flushleft}
Signals made while the ball is in play
\end{flushleft}


\begin{flushleft}
No ball - by extending one arm horizontally.
\end{flushleft}


\begin{flushleft}
Out - by raising an index finger above the head. (If not out, the umpire shall call Not out.)
\end{flushleft}


\begin{flushleft}
Wide - by extending both arms horizontally.
\end{flushleft}


\begin{flushleft}
Dead ball - by crossing and re-crossing the wrists below the waist.
\end{flushleft}





2.13.1.2





\begin{flushleft}
When the ball is dead, the bowler's end umpire shall repeat the signals in clause 2.13.1.1, with
\end{flushleft}


\begin{flushleft}
the exception of the signal for Out, to the scorers.
\end{flushleft}





2.13.1.3





\begin{flushleft}
The signals listed below shall be made to the scorers only when the ball is dead.
\end{flushleft}


\begin{flushleft}
Boundary 4 - by waving an arm from side to side finishing with the arm across the chest
\end{flushleft}


\begin{flushleft}
Boundary 6 - by raising both arms above the head.
\end{flushleft}


\begin{flushleft}
Bye - by raising an open hand above the head.
\end{flushleft}


\begin{flushleft}
Five Penalty runs awarded to the batting side - by repeated tapping of one shoulder with the
\end{flushleft}


\begin{flushleft}
opposite hand.
\end{flushleft}


\begin{flushleft}
Five Penalty runs awarded to the fielding side - by placing one hand on the opposite shoulder.
\end{flushleft}





5





\begin{flushleft}
\newpage
Leg bye - by touching a raised knee with the hand.
\end{flushleft}


\begin{flushleft}
Revoke last signal - by touching both shoulders, each with the opposite hand.
\end{flushleft}


\begin{flushleft}
Short run - by bending one arm upwards and touching the nearer shoulder with the tips of the
\end{flushleft}


\begin{flushleft}
fingers.
\end{flushleft}


\begin{flushleft}
Free Hit -- after signaling the No ball, the bowler's end umpire extends one arm straight upwards
\end{flushleft}


\begin{flushleft}
and moves it in a circular motion.
\end{flushleft}


\begin{flushleft}
Powerplay Over -- by rotating his arm in a large circle.
\end{flushleft}


\begin{flushleft}
The following signal is for Level 4 player conduct offences. The signal has two parts, both of which
\end{flushleft}


\begin{flushleft}
should be acknowledged separately by the scorers.
\end{flushleft}


\begin{flushleft}
Level 4 conduct
\end{flushleft}





\begin{flushleft}
Part 1 - by putting one arm out to the side of the body and repeatedly
\end{flushleft}


\begin{flushleft}
raising it and lowering it.
\end{flushleft}


\begin{flushleft}
Part 2 - by raising an index finger, held at shoulder height, to the side of
\end{flushleft}


\begin{flushleft}
the body.
\end{flushleft}





2.13.1.4





2.13.2





\begin{flushleft}
All the signals in clause 2.13.1.3 are to be made by the bowler's end umpire except that for
\end{flushleft}


\begin{flushleft}
Short run, which is to be signalled by the umpire at the end where short running occurs.
\end{flushleft}


\begin{flushleft}
However, the bowler's end umpire shall be responsible both for the final signal of Short run to
\end{flushleft}


\begin{flushleft}
the scorers and, if more than one run is short, for informing them as to the number of runs to be
\end{flushleft}


\begin{flushleft}
recorded.
\end{flushleft}





\begin{flushleft}
The umpire shall wait until each signal to the scorers has been separately acknowledged by a scorer before
\end{flushleft}


\begin{flushleft}
allowing play to proceed.
\end{flushleft}


\begin{flushleft}
If several signals are to be used, they should be given in the order that the events occurred.
\end{flushleft}





2.14





\begin{flushleft}
Informing the umpires
\end{flushleft}





\begin{flushleft}
Wherever the umpires are to receive information from captains or other players under these Playing Conditions, it will
\end{flushleft}


\begin{flushleft}
be sufficient for one umpire to be so informed and for him/her to inform the other umpire.
\end{flushleft}





2.15





\begin{flushleft}
Correctness of scores
\end{flushleft}





\begin{flushleft}
Consultation between umpires and scorers on doubtful points is essential. The umpires shall, throughout the match,
\end{flushleft}


\begin{flushleft}
satisfy themselves as to the correctness of the number of runs scored, the wickets that have fallen and, where
\end{flushleft}


\begin{flushleft}
appropriate, the number of overs bowled.
\end{flushleft}


\begin{flushleft}
The umpires shall ensure that they are able to contact the scorers at any time during the match and at its conclusion
\end{flushleft}


\begin{flushleft}
to address any issues relating to the correctness of scores.
\end{flushleft}





\begin{flushleft}
3 THE SCORERS
\end{flushleft}


3.1





\begin{flushleft}
Appointment of scorers
\end{flushleft}





\begin{flushleft}
Two scorers shall be appointed to record all runs scored, all wickets taken and, where appropriate, number of overs
\end{flushleft}


\begin{flushleft}
bowled.
\end{flushleft}





3.2





\begin{flushleft}
Correctness of scores
\end{flushleft}





\begin{flushleft}
The scorers shall frequently check to ensure that their records agree and consult with the umpires if necessary. See
\end{flushleft}


\begin{flushleft}
clause 2.15 (Correctness of scores).
\end{flushleft}





3.3





\begin{flushleft}
Acknowledging signals
\end{flushleft}





\begin{flushleft}
The scorers shall accept all instructions and signals given to them by the umpires and shall immediately acknowledge
\end{flushleft}


\begin{flushleft}
each separate signal.
\end{flushleft}





6





\begin{flushleft}
\newpage
4 THE BALL
\end{flushleft}


4.1





\begin{flushleft}
Weight and size
\end{flushleft}





\begin{flushleft}
The ball, when new, shall weigh not less than 5.5 ounces/155.9 g, nor more than 5.75 ounces/163 g, and shall
\end{flushleft}


\begin{flushleft}
measure not less than 8.81 in/22.4 cm, nor more than 9 in/22.9 cm in circumference.
\end{flushleft}





4.2





\begin{flushleft}
Approval and control of balls
\end{flushleft}





4.2.1





\begin{flushleft}
The Home Board shall provide white cricket balls of an approved standard for T20I cricket and spare used
\end{flushleft}


\begin{flushleft}
balls for changing during a match, which shall also be of the same brand. Note: The Home Board shall be
\end{flushleft}


\begin{flushleft}
required to advise the Visiting Board of the brand of ball to be used in the match(es) at least 30 days prior to
\end{flushleft}


\begin{flushleft}
the start of the match(es).
\end{flushleft}





4.2.2





\begin{flushleft}
The fielding captain or his nominee may select the ball with which he wishes to bowl from the supply
\end{flushleft}


\begin{flushleft}
provided by the Home Board. The fourth umpire shall take a box containing at least 6 new balls to the
\end{flushleft}


\begin{flushleft}
dressing room and supervise the selection of the ball.
\end{flushleft}





4.2.3





\begin{flushleft}
The umpires shall retain possession of the match ball(s) throughout the duration of the match when play is
\end{flushleft}


\begin{flushleft}
not actually taking place.
\end{flushleft}





4.2.4





\begin{flushleft}
During play umpires shall periodically and irregularly inspect the condition of the ball and shall retain
\end{flushleft}


\begin{flushleft}
possession of it at the fall of a wicket or any other disruption in play.
\end{flushleft}





4.3





\begin{flushleft}
New ball
\end{flushleft}





4.3.1





\begin{flushleft}
One new ball shall be used at the start of each innings.
\end{flushleft}





4.4





\begin{flushleft}
Ball lost or becoming unfit for play
\end{flushleft}





\begin{flushleft}
If, during play, the ball cannot be found or recovered or the umpires agree that it has become unfit for play through
\end{flushleft}


\begin{flushleft}
normal use, the umpires shall replace it with a ball which has had wear comparable with that which the previous ball
\end{flushleft}


\begin{flushleft}
had received before the need for its replacement. When the ball is replaced, the umpire shall inform the batsmen and
\end{flushleft}


\begin{flushleft}
the fielding captain.
\end{flushleft}





\begin{flushleft}
5 THE BAT
\end{flushleft}


5.1





\begin{flushleft}
The bat
\end{flushleft}





5.1.1





\begin{flushleft}
The bat consists of two parts, a handle and a blade.
\end{flushleft}





5.1.2





\begin{flushleft}
The basic requirements and measurements of the bat are set out in this clause with detailed specifications in
\end{flushleft}


\begin{flushleft}
paragraph 1 of Appendix B.
\end{flushleft}





5.2





\begin{flushleft}
The handle
\end{flushleft}





5.2.1





\begin{flushleft}
The handle is to be made principally of cane and/or wood.
\end{flushleft}





5.2.2





\begin{flushleft}
The part of the handle that is wholly outside the blade is defined to be the upper portion of the handle. It is a
\end{flushleft}


\begin{flushleft}
straight shaft for holding the bat.
\end{flushleft}





5.2.3





\begin{flushleft}
The upper portion of the handle may be covered with a grip as defined in paragraph 1.2.2 of Appendix B.
\end{flushleft}





5.3





\begin{flushleft}
The blade
\end{flushleft}





5.3.1





\begin{flushleft}
The blade comprises the whole of the bat apart from the handle as defined in clause 5.2 and in paragraph
\end{flushleft}


\begin{flushleft}
1.3 of Appendix B.
\end{flushleft}





5.3.2





\begin{flushleft}
The blade shall consist solely of wood.
\end{flushleft}





5.4





\begin{flushleft}
Protection and repair
\end{flushleft}





\begin{flushleft}
Subject to the specifications in paragraph 1.4 of Appendix B. and providing clause 5.5 is not contravened,
\end{flushleft}





7





\newpage
5.4.1





\begin{flushleft}
solely for the purposes of
\end{flushleft}


\begin{flushleft}
either
\end{flushleft}





\begin{flushleft}
protection from surface damage to the face, sides and shoulders of the blade
\end{flushleft}





\begin{flushleft}
or
\end{flushleft}





\begin{flushleft}
repair to the blade after surface damage,
\end{flushleft}





\begin{flushleft}
material that is not rigid, either at the time of its application to the blade or subsequently, may be placed on
\end{flushleft}


\begin{flushleft}
these surfaces.
\end{flushleft}


5.4.2





\begin{flushleft}
for repair of the blade after damage other than surface damage
\end{flushleft}


5.4.2.1





\begin{flushleft}
solid material may be inserted into the blade.
\end{flushleft}





5.4.2.2





\begin{flushleft}
The only material permitted for any insertion is wood with minimal essential adhesives.
\end{flushleft}





5.4.3





\begin{flushleft}
to prevent damage to the toe, material may be placed on that part of the blade but shall not extend over any
\end{flushleft}


\begin{flushleft}
part of the face, back or sides of the blade.
\end{flushleft}





5.5





\begin{flushleft}
Damage to the ball
\end{flushleft}





5.5.1





\begin{flushleft}
For any part of the bat, covered or uncovered, the hardness of the constituent materials and the surface
\end{flushleft}


\begin{flushleft}
texture thereof shall not be such that either or both could cause unacceptable damage to the ball.
\end{flushleft}





5.5.2





\begin{flushleft}
Any material placed on any part of the bat, for whatever purpose, shall similarly not be such that it could
\end{flushleft}


\begin{flushleft}
cause unacceptable damage to the ball.
\end{flushleft}





5.5.3





\begin{flushleft}
For the purpose of this clause, unacceptable damage is any change that is greater than normal wear and
\end{flushleft}


\begin{flushleft}
tear caused by the ball striking the uncovered wooden surface of the blade.
\end{flushleft}





5.6





\begin{flushleft}
Contact with the ball
\end{flushleft}





\begin{flushleft}
In these clauses,
\end{flushleft}


5.6.1





\begin{flushleft}
reference to the bat shall imply that the bat is held in the batsman's hand or a glove worn on his hand,
\end{flushleft}


\begin{flushleft}
unless stated otherwise.
\end{flushleft}





5.6.2





\begin{flushleft}
contact between the ball and any of 5.6.2.1 to 5.6.2.4
\end{flushleft}


5.6.2.1





\begin{flushleft}
the bat itself
\end{flushleft}





5.6.2.2





\begin{flushleft}
the batsman's hand holding the bat
\end{flushleft}





5.6.2.3





\begin{flushleft}
any part of a glove worn on the batsman's hand holding the bat
\end{flushleft}





5.6.2.4





\begin{flushleft}
any additional materials permitted under 5.4
\end{flushleft}





\begin{flushleft}
shall be regarded as the ball striking or touching the bat or being struck by the bat.
\end{flushleft}





5.7





\begin{flushleft}
Bat size limits
\end{flushleft}





5.7.1





\begin{flushleft}
The overall length of the bat, when the lower portion of the handle is inserted, shall not be more than 38
\end{flushleft}


\begin{flushleft}
in/96.52 cm.
\end{flushleft}





5.7.2





\begin{flushleft}
The blade of the bat shall not exceed the following dimensions:
\end{flushleft}


\begin{flushleft}
Width: 4.25in / 10.8 cm
\end{flushleft}


\begin{flushleft}
Depth: 2.64in / 6.7 cm
\end{flushleft}


\begin{flushleft}
Edges: 1.56in / 4.0cm.
\end{flushleft}


\begin{flushleft}
Furthermore, it should also be able to pass through a bat gauge as described in paragraph 1.6 of Appendix
\end{flushleft}


\begin{flushleft}
B.
\end{flushleft}





5.7.3





\begin{flushleft}
The handle shall not exceed 52\% of the overall length of the bat.
\end{flushleft}





5.7.4





\begin{flushleft}
The material permitted for covering the blade in clause 5.4.1 shall not exceed 0.04 in/0.1 cm in thickness.
\end{flushleft}





8





\newpage
5.7.5





\begin{flushleft}
The maximum permitted thickness of protective material placed on the toe of the blade is 0.12 in/0.3 cm.
\end{flushleft}





5.8





\begin{flushleft}
Categories of bat
\end{flushleft}





5.8.1





\begin{flushleft}
Type A bats conform to clauses 5.1 to 5.7 inclusive.
\end{flushleft}





5.8.2





\begin{flushleft}
Only Type A bats may be used in T20I matches.
\end{flushleft}





\begin{flushleft}
6 THE PITCH
\end{flushleft}


6.1





\begin{flushleft}
Area of pitch
\end{flushleft}





\begin{flushleft}
The pitch is a rectangular area of the ground 22 yards/20.12 m in length and 10 ft/3.05 m in width. It is bounded at
\end{flushleft}


\begin{flushleft}
either end by the bowling creases and on either side by imaginary lines, one each side of the imaginary line joining
\end{flushleft}


\begin{flushleft}
the centres of the two middle stumps, each parallel to it and 5 ft/1.52 m from it. If the pitch is next to an artificial pitch
\end{flushleft}


\begin{flushleft}
which is closer than 5 ft/1.52 m from the middle stumps, the pitch on that side will extend only to the junction of the
\end{flushleft}


\begin{flushleft}
two surfaces. See clauses 8.1 (Description, width and pitching) and 7.2 (The bowling crease).
\end{flushleft}





6.2





\begin{flushleft}
Fitness of pitch for play
\end{flushleft}





\begin{flushleft}
The umpires shall be the sole judges of the fitness of the pitch for play. See clauses 2.7 (Fitness for play) and 2.8
\end{flushleft}


\begin{flushleft}
(Suspension of play in dangerous or unreasonable conditions).
\end{flushleft}





6.3





\begin{flushleft}
Selection and preparation
\end{flushleft}





\begin{flushleft}
Before the match, the Ground Authority shall be responsible for the selection and preparation of the pitch. During the
\end{flushleft}


\begin{flushleft}
match, the umpires shall control its use and maintenance.
\end{flushleft}


6.3.1





\begin{flushleft}
The Ground Authority shall ensure that during the period prior to the start of play and during intervals, the
\end{flushleft}


\begin{flushleft}
pitch area shall be roped off so as to prevent unauthorised access. (The pitch area shall include an area at
\end{flushleft}


\begin{flushleft}
least 2 metres beyond the rectangle made by the crease markings at both ends of the pitch).
\end{flushleft}





6.3.2





\begin{flushleft}
The fourth umpire shall ensure that, prior to the start of play and during any intervals, only authorised staff,
\end{flushleft}


\begin{flushleft}
the ICC match officials, players, team coaches and authorised television personnel shall be allowed access
\end{flushleft}


\begin{flushleft}
to the pitch area. Such access shall be subject to the following limitations:
\end{flushleft}


6.3.2.1





\begin{flushleft}
Only captains and team coaches may walk on the actual playing surface of the pitch area
\end{flushleft}


\begin{flushleft}
(outside of the crease markings).
\end{flushleft}





6.3.2.2





\begin{flushleft}
Access to the pitch area by television personnel shall be restricted to one camera crew
\end{flushleft}


\begin{flushleft}
(including one or two television commentators) of the official licensed television broadcaster(s)
\end{flushleft}


\begin{flushleft}
(but not news crews).
\end{flushleft}





6.3.2.3





\begin{flushleft}
No spiked footwear shall be permitted.
\end{flushleft}





6.3.2.4





\begin{flushleft}
No one shall be permitted to bounce a ball on the pitch, strike it with a bat or cause damage to
\end{flushleft}


\begin{flushleft}
the pitch in any other way.
\end{flushleft}





6.3.2.5





\begin{flushleft}
Access shall not interfere with pitch preparation.
\end{flushleft}





6.3.3





\begin{flushleft}
In the event of any dispute, the ICC Match Referee will rule and his ruling will be final.
\end{flushleft}





6.4





\begin{flushleft}
Changing the pitch
\end{flushleft}





6.4.1





\begin{flushleft}
If the on-field umpires decide that it is dangerous or unreasonable for play to continue on the match pitch,
\end{flushleft}


\begin{flushleft}
they shall stop play and immediately advise the ICC Match Referee.
\end{flushleft}





6.4.2





\begin{flushleft}
The on-field umpires and the ICC Match Referee shall then consult with both captains.
\end{flushleft}





6.4.3





\begin{flushleft}
If the captains agree to continue, play shall resume.
\end{flushleft}





6.4.4





\begin{flushleft}
If the decision is not to resume play, the on-field umpires together with the ICC Match Referee shall consider
\end{flushleft}


\begin{flushleft}
whether the existing pitch can be repaired and the match resumed from the point it was stopped. In
\end{flushleft}





9





\begin{flushleft}
\newpage
considering whether to authorise such repairs, the ICC Match Referee must consider whether this would
\end{flushleft}


\begin{flushleft}
place either side at an unfair advantage, given the play that had already taken place on the dangerous pitch.
\end{flushleft}


6.4.5





\begin{flushleft}
If the decision is that the existing pitch cannot be repaired, then the match is to be abandoned with the
\end{flushleft}


\begin{flushleft}
following consequences:
\end{flushleft}


6.4.5.1





\begin{flushleft}
In the event of the required number of overs to constitute a match having been completed at the
\end{flushleft}


\begin{flushleft}
time the match is abandoned, the result shall be determined according to the provisions of
\end{flushleft}


\begin{flushleft}
clause 16.4.2.
\end{flushleft}





6.4.5.2





\begin{flushleft}
In the event of the required number of overs to constitute a match not having been completed,
\end{flushleft}


\begin{flushleft}
the match will be abandoned as a no result.
\end{flushleft}





6.4.6





\begin{flushleft}
If the abandonment occurs on the day of the match, the ICC Match Referee shall consult with the Home
\end{flushleft}


\begin{flushleft}
Board with the objective of finding a way for a new match (including a new nomination of teams and toss) to
\end{flushleft}


\begin{flushleft}
commence on the same date and venue. Such a match may be played either on the repaired pitch or on
\end{flushleft}


\begin{flushleft}
another pitch, subject to the ICC Match Referee and the relevant Ground Authority both being satisfied that
\end{flushleft}


\begin{flushleft}
the new pitch will be of the required T20I standard. The playing time lost between the scheduled start time of
\end{flushleft}


\begin{flushleft}
the original match and the actual start time of the new match will be covered by the provisions of clause 12.
\end{flushleft}





6.4.7





\begin{flushleft}
If it is not possible to start a new match on the scheduled day of the match, the relevant officials from the
\end{flushleft}


\begin{flushleft}
participating Boards shall agree on whether the match can be replayed within the existing tour schedule.
\end{flushleft}





6.4.8





\begin{flushleft}
Throughout the above decision making processes, the ICC Match Referee shall keep informed both
\end{flushleft}


\begin{flushleft}
captains and the head of the Ground Authority. The head of the Ground Authority shall ensure that suitable
\end{flushleft}


\begin{flushleft}
and prompt public announcements are made.
\end{flushleft}





6.5





\begin{flushleft}
Non-turf pitches
\end{flushleft}





\begin{flushleft}
All T20I matches shall be played on natural turf pitches. The use of PVA and other adhesives in the preparation of
\end{flushleft}


\begin{flushleft}
pitches is not permitted.
\end{flushleft}





\begin{flushleft}
7 THE CREASES
\end{flushleft}


7.1





\begin{flushleft}
The creases
\end{flushleft}





\begin{flushleft}
The positions of a bowling crease, a popping crease and two return creases shall be marked by white lines, as set
\end{flushleft}


\begin{flushleft}
out in clauses 7.2, 7.3 and 7.4, at each end of the pitch. See paragraph 1 of Appendix C.
\end{flushleft}





7.2





\begin{flushleft}
The bowling crease
\end{flushleft}





\begin{flushleft}
The bowling crease, which is the back edge of the crease marking, is the line that marks the end of the pitch, as in
\end{flushleft}


\begin{flushleft}
clause 6.1 (Area of pitch). It shall be 8 ft 8 in/2.64 m in length.
\end{flushleft}





7.3





\begin{flushleft}
The popping crease
\end{flushleft}





\begin{flushleft}
The popping crease, which is the back edge of the crease marking, shall be in front of and parallel to the bowling
\end{flushleft}


\begin{flushleft}
crease and shall be 4 ft/1.22 m from it. The popping crease shall be marked to a minimum of 15 yards/13.71 m on
\end{flushleft}


\begin{flushleft}
either side of the imaginary line joining the centres of the two middle stumps and shall be considered to be unlimited
\end{flushleft}


\begin{flushleft}
in length.
\end{flushleft}





7.4





\begin{flushleft}
The return creases
\end{flushleft}





\begin{flushleft}
The return creases, which are the inside edges of the crease markings, shall be at right angles to the popping crease
\end{flushleft}


\begin{flushleft}
at a distance of 4 ft 4 in/1.32 m either side of the imaginary line joining the centres of the two middle stumps. Each
\end{flushleft}


\begin{flushleft}
return crease shall be marked from the popping crease to a minimum of 8 ft/2.44 m behind it and shall be considered
\end{flushleft}


\begin{flushleft}
to be unlimited in length.
\end{flushleft}





7.5





\begin{flushleft}
Additional Crease Markings
\end{flushleft}





\begin{flushleft}
As a guideline to the umpires for the calling of Wides on the offside, the crease markings detailed in paragraph 1 of
\end{flushleft}


\begin{flushleft}
Appendix C shall be marked in white at each end of the pitch.
\end{flushleft}





10





\begin{flushleft}
\newpage
8 THE WICKETS
\end{flushleft}


8.1





\begin{flushleft}
Description, width and pitching
\end{flushleft}





\begin{flushleft}
Two sets of wickets shall be pitched opposite and parallel to each other in the centres of the bowling creases. Each
\end{flushleft}


\begin{flushleft}
set shall be 9 in/22.86 cm wide and shall consist of three wooden stumps with two wooden bails on top. See
\end{flushleft}


\begin{flushleft}
paragraph 2 of Appendix B.
\end{flushleft}





8.2





\begin{flushleft}
Size of stumps
\end{flushleft}





\begin{flushleft}
The tops of the stumps shall be 28 in/71.12 cm above the playing surface and shall be dome shaped except for the
\end{flushleft}


\begin{flushleft}
bail grooves. The portion of a stump above the playing surface shall be cylindrical apart from the domed top, with
\end{flushleft}


\begin{flushleft}
circular section of diameter not less than 1.38 in/3.50 cm nor more than 1.5 in/3.81 cm. See paragraph 2 of Appendix
\end{flushleft}


\begin{flushleft}
B.
\end{flushleft}


\begin{flushleft}
For televised matches the Home Board may provide a slightly larger cylindrical stump to accommodate the stump
\end{flushleft}


\begin{flushleft}
camera. When the larger stump is used, all three stumps must be exactly the same size.
\end{flushleft}





8.3





\begin{flushleft}
The bails
\end{flushleft}





8.3.1





\begin{flushleft}
The bails, when in position on top of the stumps,
\end{flushleft}


\begin{flushleft}
- shall not project more than 0.5 in/1.27 cm above them.
\end{flushleft}


\begin{flushleft}
- shall fit between the stumps without forcing them out of the vertical.
\end{flushleft}





8.3.2





\begin{flushleft}
Each bail shall conform to the following specifications (see paragraph 2 of Appendix B).
\end{flushleft}


\begin{flushleft}
Overall length 4.31 in/10.95 cm
\end{flushleft}


\begin{flushleft}
Length of barrel 2.13 in /5.40 cm
\end{flushleft}


\begin{flushleft}
Longer spigot 1.38 in/3.50 cm
\end{flushleft}


\begin{flushleft}
Shorter spigot 0.81 in/2.06 cm.
\end{flushleft}





8.3.3





\begin{flushleft}
The two spigots and the barrel shall have the same centre line.
\end{flushleft}





8.3.4





\begin{flushleft}
Devices aimed at protecting player safety by limiting the distance that a bail can travel off the stumps will be
\end{flushleft}


\begin{flushleft}
allowed, subject to the approval of the Home Board and the ICC.
\end{flushleft}





8.4





\begin{flushleft}
Dispensing with bails
\end{flushleft}





\begin{flushleft}
The umpires may agree to dispense with the use of bails, if necessary. If they so agree then no bails shall be used at
\end{flushleft}


\begin{flushleft}
either end. The use of bails shall be resumed as soon as conditions permit. See clause 29.4 (Dispensing with bails).
\end{flushleft}





8.5





\begin{flushleft}
LED Wickets
\end{flushleft}





\begin{flushleft}
The use of approved LED Wickets is permitted. Refer also to paragraphs 3.8.1.6 and 4.2 of Appendix D.
\end{flushleft}





\begin{flushleft}
9 PREPARATION AND MAINTENANCE OF THE PLAYING
\end{flushleft}


\begin{flushleft}
AREA
\end{flushleft}


9.1





\begin{flushleft}
Rolling
\end{flushleft}





\begin{flushleft}
The pitch shall not be rolled during the match except as permitted in clauses 9.1.1 and 9.1.2.
\end{flushleft}


9.1.1





\begin{flushleft}
Frequency and duration of rolling
\end{flushleft}


\begin{flushleft}
During the match the pitch may be rolled at the request of the captain of the side batting second, for a period
\end{flushleft}


\begin{flushleft}
of not more than 7 minutes, before the start of the second innings.
\end{flushleft}





11





\newpage
9.1.2





\begin{flushleft}
Rolling after a delayed start
\end{flushleft}


\begin{flushleft}
In addition to the rolling permitted above, if, after the toss and before the first innings of the match, the start
\end{flushleft}


\begin{flushleft}
is delayed, the captain of the batting side may request that the pitch be rolled for not more than 7 minutes.
\end{flushleft}


\begin{flushleft}
However, if the umpires together agree that the delay has had no significant effect on the state of the pitch,
\end{flushleft}


\begin{flushleft}
they shall refuse such request for rolling of the pitch.
\end{flushleft}





9.1.3





\begin{flushleft}
Choice of rollers
\end{flushleft}


\begin{flushleft}
If there is more than one roller available the captain of the batting side shall choose which one is to be used.
\end{flushleft}





\begin{flushleft}
The following shall apply in addition to clause 9.1:
\end{flushleft}


9.1.4





\begin{flushleft}
Prior to the scheduled time for the toss, the artificial drying of the pitch and outfield shall be at the discretion
\end{flushleft}


\begin{flushleft}
of the Ground Authority. Thereafter and throughout the match the drying of the outfield may be undertaken
\end{flushleft}


\begin{flushleft}
at any time by the Ground Authority, but the drying of the affected area of the pitch shall be carried out only
\end{flushleft}


\begin{flushleft}
on the instructions and under the supervision of the umpires. The umpires shall be empowered to have the
\end{flushleft}


\begin{flushleft}
pitch dried without reference to the captains at any time they are of the opinion that it is unfit for play.
\end{flushleft}





9.1.5





\begin{flushleft}
The umpires may instruct the Ground Authority to use any available equipment, including any roller for the
\end{flushleft}


\begin{flushleft}
purpose of drying the pitch and making it fit for play.
\end{flushleft}





9.1.6





\begin{flushleft}
An absorbent roller may be used to remove water from the covers including the cover on the match pitch.
\end{flushleft}





9.2





\begin{flushleft}
Clearing debris from the pitch
\end{flushleft}





9.2.1





\begin{flushleft}
The pitch shall be cleared of any debris
\end{flushleft}


9.2.1.1





\begin{flushleft}
between innings. This shall precede rolling if any is to take place.
\end{flushleft}





9.2.2





\begin{flushleft}
The clearance of debris in clause 9.2.1 shall be done by sweeping, except where the umpires consider that
\end{flushleft}


\begin{flushleft}
this may be detrimental to the surface of the pitch. In this case the debris must be cleared from that area by
\end{flushleft}


\begin{flushleft}
hand, without sweeping.
\end{flushleft}





9.2.3





\begin{flushleft}
In addition to clause 9.2.1, debris may be cleared from the pitch by hand, without sweeping, before mowing
\end{flushleft}


\begin{flushleft}
and whenever either umpire considers it necessary.
\end{flushleft}





9.3





\begin{flushleft}
Mowing
\end{flushleft}





9.3.1





\begin{flushleft}
Responsibility for mowing
\end{flushleft}


9.3.1.1





9.4





\begin{flushleft}
All mowings which are carried out before the match shall be the sole responsibility of the
\end{flushleft}


\begin{flushleft}
Ground Authority.
\end{flushleft}





\begin{flushleft}
Watering the pitch
\end{flushleft}





\begin{flushleft}
The pitch shall not be watered during the match.
\end{flushleft}





9.5





\begin{flushleft}
Re-marking creases
\end{flushleft}





\begin{flushleft}
Creases shall be re-marked whenever either umpire considers it necessary.
\end{flushleft}





9.6





\begin{flushleft}
Maintenance of footholes
\end{flushleft}





\begin{flushleft}
The umpires shall ensure that the holes made by the bowlers and batsmen are cleaned out and dried whenever
\end{flushleft}


\begin{flushleft}
necessary to facilitate play.
\end{flushleft}


\begin{flushleft}
The umpires shall allow, if necessary, the returfing of footholes made by the bowlers in their delivery strides, or the
\end{flushleft}


\begin{flushleft}
use of quick-setting fillings for the same purpose.
\end{flushleft}


\begin{flushleft}
In addition, the umpires shall see that wherever possible and whenever it is considered necessary, action is taken
\end{flushleft}


\begin{flushleft}
during all intervals in play to do whatever is practicable to improve the bowler's footholes.
\end{flushleft}





12





\newpage
9.7





\begin{flushleft}
Securing of footholds and maintenance of pitch
\end{flushleft}





\begin{flushleft}
During play, umpires shall allow the players to secure their footholds by the use of sawdust provided that no damage
\end{flushleft}


\begin{flushleft}
to the pitch is caused and that clause 41 (Unfair play) is not contravened.
\end{flushleft}





9.8





\begin{flushleft}
Protection and preparation of adjacent pitches during matches
\end{flushleft}





\begin{flushleft}
The protection (by way of an appropriate cover) and preparation of pitches which are adjacent to the match pitch will
\end{flushleft}


\begin{flushleft}
be permitted during the match subject to the following:
\end{flushleft}


9.8.1





\begin{flushleft}
Such measures will only be possible if requested by the Ground Authority and approved by the umpires
\end{flushleft}


\begin{flushleft}
before the start of the match.
\end{flushleft}





9.8.2





\begin{flushleft}
Approval should only be granted where such measures are unavoidable and will not compromise the safety
\end{flushleft}


\begin{flushleft}
of the players or their ability to execute their actions with complete freedom.
\end{flushleft}





9.8.3





\begin{flushleft}
The preparation work shall be carried out under the supervision of the fourth umpire.
\end{flushleft}





9.8.4





\begin{flushleft}
The consent of the captains is not required but the umpires shall advise both captains and the ICC Match
\end{flushleft}


\begin{flushleft}
Referee before the start of the match on what has been agreed.
\end{flushleft}





\begin{flushleft}
10 COVERING THE PITCH
\end{flushleft}


10.1





\begin{flushleft}
Before the match
\end{flushleft}





\begin{flushleft}
The use of covers before the match is the responsibility of the Ground Authority and may include full covering if
\end{flushleft}


\begin{flushleft}
required.
\end{flushleft}


\begin{flushleft}
The pitch shall be entirely protected against rain up to the commencement of play.
\end{flushleft}


\begin{flushleft}
However, the Ground Authority shall grant suitable facility to the captains to inspect the pitch before the nomination of
\end{flushleft}


\begin{flushleft}
their players and to the umpires to discharge their duties as laid down in clauses 2 (The umpires), 6 (The pitch), 7
\end{flushleft}


\begin{flushleft}
(The creases), 8 (The wickets), and 9 (Preparation and maintenance of the playing area).
\end{flushleft}





10.2





\begin{flushleft}
During the match
\end{flushleft}





\begin{flushleft}
The pitch shall be entirely protected against rain up to the commencement of play, and for the duration of the period
\end{flushleft}


\begin{flushleft}
of the match.
\end{flushleft}


\begin{flushleft}
The covers must totally protect the pitch and also the pitch surroundings, to a minimum of 5 metres either side of the
\end{flushleft}


\begin{flushleft}
pitch, and any worn or soft areas in the outfield.
\end{flushleft}


\begin{flushleft}
The bowlers' run-ups shall be covered during inclement weather, in order to keep them dry, to a distance of at least
\end{flushleft}


\begin{flushleft}
10 x 10 metres.
\end{flushleft}





10.3





\begin{flushleft}
Removal of covers
\end{flushleft}





\begin{flushleft}
All covers (including {``}hessian'' or {``}scrim'' covers used to protect the pitch against the sun) shall be removed not later
\end{flushleft}


\begin{flushleft}
than 2 � hours before the scheduled start of play provided it is not raining at the time, but the pitch will be covered
\end{flushleft}


\begin{flushleft}
again if rain falls prior to the commencement of play.
\end{flushleft}





\begin{flushleft}
11 INTERVALS
\end{flushleft}


11.1





\begin{flushleft}
An interval
\end{flushleft}





11.1.1





\begin{flushleft}
The following shall be classed as intervals.
\end{flushleft}


\begin{flushleft}
- Intervals between innings.
\end{flushleft}


\begin{flushleft}
- Any other agreed interval.
\end{flushleft}





11.1.2





\begin{flushleft}
Only these intervals shall be considered as scheduled breaks for the purposes of clause 24.2.6.
\end{flushleft}





13





\newpage
11.2





\begin{flushleft}
Duration of interval
\end{flushleft}





11.2.1





\begin{flushleft}
There shall be a 20 minute interval between innings, taken from the call of Time before the interval until the
\end{flushleft}


\begin{flushleft}
call of Play on resumption after the interval.
\end{flushleft}





11.3





\begin{flushleft}
Allowance for interval between innings
\end{flushleft}


\begin{flushleft}
Law 11.3 of the Laws of Cricket shall not apply.
\end{flushleft}





11.4





\begin{flushleft}
Changing agreed times of intervals
\end{flushleft}





11.4.1





\begin{flushleft}
If the innings of the team batting first is completed prior to the scheduled time for the interval, the interval
\end{flushleft}


\begin{flushleft}
shall take place immediately and the innings of the team batting second will commence correspondingly
\end{flushleft}


\begin{flushleft}
earlier. In circumstances where the side bowling first has not completed the allotted number of overs by the
\end{flushleft}


\begin{flushleft}
scheduled or re-scheduled cessation time for the first innings, the umpires shall reduce the length of the
\end{flushleft}


\begin{flushleft}
interval by the amount of time that the first innings over-ran. The minimum time for the interval will be 10
\end{flushleft}


\begin{flushleft}
minutes.
\end{flushleft}





11.4.2





\begin{flushleft}
However, following a lengthy delay or interruption prior to the completion of the innings of the team batting
\end{flushleft}


\begin{flushleft}
first, the Match Referee may, at his discretion, reduce the interval between innings from 20 minutes to not
\end{flushleft}


\begin{flushleft}
less than 10 minutes.
\end{flushleft}





11.4.3





\begin{flushleft}
Such discretion should only be exercised after determining the adjusted overs per side based on a 20
\end{flushleft}


\begin{flushleft}
minute interval. If having exercised this discretion, the rescheduled finishing time for the match is earlier than
\end{flushleft}


\begin{flushleft}
the latest possible finishing time, then these minutes should be deducted from the length of any interruption
\end{flushleft}


\begin{flushleft}
during the second innings before determining the overs remaining.
\end{flushleft}





11.5





\begin{flushleft}
Intervals for drinks
\end{flushleft}





11.5.1





\begin{flushleft}
No drinks intervals shall be permitted.
\end{flushleft}





11.5.2





\begin{flushleft}
An individual player may be given a drink either on the boundary edge or at the fall of a wicket, on the field,
\end{flushleft}


\begin{flushleft}
provided that no playing time is wasted. No other drinks shall be taken onto the field without the permission
\end{flushleft}


\begin{flushleft}
of the umpires. Any player taking drinks onto the field shall be dressed in proper cricket attire (subject to the
\end{flushleft}


\begin{flushleft}
wearing of bibs -- refer to the note in clause 24.1.4).
\end{flushleft}





11.6





\begin{flushleft}
Scorers to be informed
\end{flushleft}





\begin{flushleft}
The umpires shall ensure that the scorers are informed of all agreements about hours of play and intervals and of any
\end{flushleft}


\begin{flushleft}
changes made thereto as permitted under this clause.
\end{flushleft}





\begin{flushleft}
12 START OF PLAY; CESSATION OF PLAY
\end{flushleft}


12.1





\begin{flushleft}
Call of Play
\end{flushleft}





\begin{flushleft}
The bowler's end umpire shall call Play before the first ball of the match and on the resumption of play after any
\end{flushleft}


\begin{flushleft}
interval or interruption.
\end{flushleft}





12.2





\begin{flushleft}
Call of Time
\end{flushleft}





\begin{flushleft}
The bowler's end umpire shall call Time, when the ball is dead, at the end of any session of play or as required by
\end{flushleft}


\begin{flushleft}
these Playing Conditions. See also clause 20.3 (Call of Over or Time).
\end{flushleft}





12.3





\begin{flushleft}
Removal of bails
\end{flushleft}





\begin{flushleft}
After the call of Time, the bails shall be removed from both wickets.
\end{flushleft}





12.4





\begin{flushleft}
Starting a new over
\end{flushleft}





\begin{flushleft}
Another over shall always be started at any time during the match, unless an interval is to be taken in the
\end{flushleft}


\begin{flushleft}
circumstances set out in clause 12.5.2, if the umpire, walking at normal pace, has arrived at the position behind the
\end{flushleft}


\begin{flushleft}
stumps at the bowler's end before the time agreed for the next interval has been reached.
\end{flushleft}





14





\newpage
12.5





\begin{flushleft}
Completion of an over
\end{flushleft}





\begin{flushleft}
Other than at the end of the match,
\end{flushleft}


12.5.1





\begin{flushleft}
if the agreed time for an interval is reached during an over, the over shall be completed before the interval is
\end{flushleft}


\begin{flushleft}
taken, except as provided for in clause 12.5.2.
\end{flushleft}





12.5.2





\begin{flushleft}
when less than 3 minutes remains before the time agreed for the next interval, the interval shall be taken
\end{flushleft}


\begin{flushleft}
immediately if
\end{flushleft}


\begin{flushleft}
either a batsman is dismissed or retires, or
\end{flushleft}


\begin{flushleft}
the players have occasion to leave the field,
\end{flushleft}


\begin{flushleft}
whether this occurs during an over or at the end of an over. Except at the end of an innings, if an over is
\end{flushleft}


\begin{flushleft}
thus interrupted it shall be completed on the resumption of play.
\end{flushleft}





12.6





\begin{flushleft}
Conclusion of match
\end{flushleft}





12.6.1





\begin{flushleft}
The match is concluded
\end{flushleft}


12.6.1.1





\begin{flushleft}
as soon as a result as defined in clauses 16.1 to 16.5 (The result) is reached.
\end{flushleft}





12.6.1.2





\begin{flushleft}
as soon as the prescribed number of overs have been completed
\end{flushleft}





12.6.2





\begin{flushleft}
The match is concluded if, without a conclusion having been reached under 12.6.1, the players leave the
\end{flushleft}


\begin{flushleft}
field for adverse conditions of ground, weather or light, or in exceptional circumstances, and no further play
\end{flushleft}


\begin{flushleft}
is possible.
\end{flushleft}





12.7





\begin{flushleft}
Hours of Play; Minimum Overs Requirement
\end{flushleft}





12.7.1





\begin{flushleft}
To be determined by the Home Board subject to there being 2 sessions of 1 hour 25 minutes each,
\end{flushleft}


\begin{flushleft}
separated by a 20 minute interval between innings.
\end{flushleft}





12.8





\begin{flushleft}
Minimum Over Rates
\end{flushleft}





12.8.1





\begin{flushleft}
The minimum over rate to be achieved in T20I Matches shall be 14.11 overs per hour.
\end{flushleft}





12.8.2





\begin{flushleft}
The actual over rate shall be calculated at the end of each innings by the umpires.
\end{flushleft}





12.8.3





\begin{flushleft}
In calculating the actual over rate for the match, allowances shall be given as follows:
\end{flushleft}


12.8.3.1





\begin{flushleft}
The time lost as a result of treatment given to a player by an authorised medical personnel on
\end{flushleft}


\begin{flushleft}
the field of play;
\end{flushleft}





12.8.3.2





\begin{flushleft}
The time lost as a result of a player being required to leave the field as a result of a serious
\end{flushleft}


\begin{flushleft}
injury;
\end{flushleft}





12.8.3.3





\begin{flushleft}
The time taken for all third umpire referrals and consultations and any umpire or player reviews;
\end{flushleft}





12.8.3.4





\begin{flushleft}
The time lost as a result of time wasting by the batting side; and
\end{flushleft}





12.8.3.5





\begin{flushleft}
The time lost due to all other circumstances that are beyond the control of the fielding side.
\end{flushleft}





12.8.4





\begin{flushleft}
In the event of any time allowances being granted to the fielding team under clause 12.8.3.4 above (time
\end{flushleft}


\begin{flushleft}
wasting by batting team), then such time shall be deducted from the allowances granted to such batting
\end{flushleft}


\begin{flushleft}
team in the determination of its over rate.
\end{flushleft}





12.8.5





\begin{flushleft}
In addition to the allowances as provided for above,
\end{flushleft}


12.8.5.1





\begin{flushleft}
in the case of an innings that has been reduced due to any delay or interruption in play, an
\end{flushleft}


\begin{flushleft}
additional allowance of 1 minute for every full 3 overs by which the innings is reduced will be
\end{flushleft}


\begin{flushleft}
granted.
\end{flushleft}





12.8.5.2





\begin{flushleft}
an additional allowance of 1 minute will be given for each of the 6th, 7th, 8th and 9th wickets
\end{flushleft}


\begin{flushleft}
taken during an innings.
\end{flushleft}





15





\newpage
12.8.6





\begin{flushleft}
If a batting team is bowled out within the time determined for that innings pursuant to these playing
\end{flushleft}


\begin{flushleft}
conditions (taking into account all of the time allowances set out above), the fielding side shall be deemed to
\end{flushleft}


\begin{flushleft}
have complied with the required minimum over rate.
\end{flushleft}





12.8.7





\begin{flushleft}
The current over rate of the fielding team (+/- overs compared to the minimum rate required), to be advised
\end{flushleft}


\begin{flushleft}
by the 3rd umpire every 30 minutes as a minimum, shall be displayed on a scoreboard or replay screen.
\end{flushleft}





\begin{flushleft}
13 INNINGS
\end{flushleft}


13.1





\begin{flushleft}
Number of innings
\end{flushleft}





13.1.1





\begin{flushleft}
A match shall be one innings for each side.
\end{flushleft}





13.2





\begin{flushleft}
Alternate innings
\end{flushleft}





\begin{flushleft}
Each side shall take their innings alternately.
\end{flushleft}





13.3





\begin{flushleft}
Completed innings
\end{flushleft}





\begin{flushleft}
A side's innings is to be considered as completed if any of the following applies
\end{flushleft}


13.3.1





\begin{flushleft}
the side is all out.
\end{flushleft}





13.3.2





\begin{flushleft}
at the fall of a wicket or the retirement of a batsman, further balls remain to be bowled but no further
\end{flushleft}


\begin{flushleft}
batsman is available to come in.
\end{flushleft}





13.3.3





\begin{flushleft}
the prescribed number of overs have been bowled to the batting side.
\end{flushleft}





13.4





\begin{flushleft}
The toss
\end{flushleft}





\begin{flushleft}
The captains shall toss a coin for the choice of innings, on the field of play and under the supervision of the ICC
\end{flushleft}


\begin{flushleft}
Match Referee, not earlier than 30 minutes, nor later than 15 minutes before the scheduled or any rescheduled time
\end{flushleft}


\begin{flushleft}
for the start of play. Note, however, the provisions of clause 1.3 (Captain).
\end{flushleft}





13.5





\begin{flushleft}
Decision to be notified
\end{flushleft}





\begin{flushleft}
As soon as the toss is completed, the captain of the side winning the toss shall decide whether to bat or to field and
\end{flushleft}


\begin{flushleft}
shall notify the opposing captain and the umpires of this decision. Once notified, the decision cannot be changed.
\end{flushleft}





13.6





\begin{flushleft}
Duration of Match
\end{flushleft}





13.6.1





\begin{flushleft}
All matches will consist of one innings per side, each innings being limited to a maximum of 20 overs. All
\end{flushleft}


\begin{flushleft}
matches shall be of one day's scheduled duration.
\end{flushleft}





13.7





\begin{flushleft}
Length of Innings
\end{flushleft}





13.7.1





\begin{flushleft}
Uninterrupted Matches.
\end{flushleft}


13.7.1.1





\begin{flushleft}
Each team shall bat for 20 overs unless all out earlier.
\end{flushleft}





13.7.1.2





\begin{flushleft}
If the team fielding first fails to bowl the required number of overs by the scheduled time for
\end{flushleft}


\begin{flushleft}
cessation of the first innings, play shall continue until the required number of overs has been
\end{flushleft}


\begin{flushleft}
bowled. The interval shall not be extended and the second session shall commence at the
\end{flushleft}


\begin{flushleft}
scheduled time. The team batting second shall receive its full quota of 20 overs irrespective of
\end{flushleft}


\begin{flushleft}
the number of overs it bowled in the scheduled time for the cessation of the first innings.
\end{flushleft}





13.7.1.3





\begin{flushleft}
If the team batting first is dismissed in less than 20 overs, the team batting second shall be
\end{flushleft}


\begin{flushleft}
entitled to bat for 20 overs.
\end{flushleft}





13.7.1.4





\begin{flushleft}
If the team fielding second fails to bowl 20 overs by the scheduled cessation time, the hours of
\end{flushleft}


\begin{flushleft}
play shall be extended until the required number of overs has been bowled or a result is
\end{flushleft}


\begin{flushleft}
achieved.
\end{flushleft}





13.7.1.5





\begin{flushleft}
Penalties shall apply for slow over rates (refer to the ICC Code of Conduct).
\end{flushleft}





16





\newpage
13.7.2





\begin{flushleft}
Delayed or Interrupted Matches
\end{flushleft}


13.7.2.1





13.7.2.2





\begin{flushleft}
Delay or Interruption to the Innings of the Team Batting First (see paragraph 1 of Appendix E)
\end{flushleft}


13.7.2.1.1





\begin{flushleft}
When playing time has been lost the revised number of overs to be bowled in the
\end{flushleft}


\begin{flushleft}
match shall be based on a rate of 14.11 overs per hour in the total remaining time
\end{flushleft}


\begin{flushleft}
available for play.
\end{flushleft}





13.7.2.1.2





\begin{flushleft}
The revision of the number of overs should ensure, whenever possible, that both
\end{flushleft}


\begin{flushleft}
teams have the opportunity of batting for the same number of overs. The team
\end{flushleft}


\begin{flushleft}
batting second shall not bat for a greater number of overs than the first team
\end{flushleft}


\begin{flushleft}
unless the latter completed its innings in less than its allocated overs. To
\end{flushleft}


\begin{flushleft}
constitute a match, a minimum of 5 overs have to be bowled to the side batting
\end{flushleft}


\begin{flushleft}
second, subject to a result not being achieved earlier.
\end{flushleft}





13.7.2.1.3





\begin{flushleft}
As soon as the total minutes of playing time remaining is less than the completed
\end{flushleft}


\begin{flushleft}
overs faced by Team 1 multiplied by 4.25, then the first innings is terminated and
\end{flushleft}


\begin{flushleft}
the provisions of 13.7.2.2 below take effect.
\end{flushleft}





13.7.2.1.4





\begin{flushleft}
A fixed time will be specified for the commencement of the interval, and also the
\end{flushleft}


\begin{flushleft}
close of play for the match, by applying a rate of 14.11 overs per hour. When
\end{flushleft}


\begin{flushleft}
calculating the length of playing time available for the match, or the length of
\end{flushleft}


\begin{flushleft}
either innings, the timing and duration of all relative delays, extensions in playing
\end{flushleft}


\begin{flushleft}
hours, interruptions in play, and intervals will be taken into consideration. This
\end{flushleft}


\begin{flushleft}
calculation must not cause the match to finish earlier than the original or
\end{flushleft}


\begin{flushleft}
rescheduled time for cessation of play on the final scheduled day for play. If
\end{flushleft}


\begin{flushleft}
required the original time shall be extended to allow for one extra over for each
\end{flushleft}


\begin{flushleft}
team.
\end{flushleft}





13.7.2.1.5





\begin{flushleft}
If the team fielding first fails to bowl the revised number of overs by the specified
\end{flushleft}


\begin{flushleft}
time, play shall continue until the required number of overs have been bowled or
\end{flushleft}


\begin{flushleft}
the innings is completed.
\end{flushleft}





13.7.2.1.6





\begin{flushleft}
Penalties shall apply for slow over rates (refer to the ICC Code of Conduct).
\end{flushleft}





\begin{flushleft}
Delay or Interruption to the innings of the Team Batting Second (see paragraph 2 of Appendix
\end{flushleft}


\begin{flushleft}
E)
\end{flushleft}





17





\newpage
13.8





13.7.2.2.1





\begin{flushleft}
When playing time has been lost and, as a result, it is not possible for the team
\end{flushleft}


\begin{flushleft}
batting second to have the opportunity of receiving its allocated, or revised
\end{flushleft}


\begin{flushleft}
allocation of overs in the playing time available, the number of overs shall be
\end{flushleft}


\begin{flushleft}
reduced at a rate of 14.11 overs per hour in respect of the lost playing time.
\end{flushleft}


\begin{flushleft}
Should the calculations result in a fraction of an over the fraction shall be ignored.
\end{flushleft}





13.7.2.2.2





\begin{flushleft}
In addition, should the innings of the team batting first have been completed prior
\end{flushleft}


\begin{flushleft}
to the scheduled, or re-scheduled time for the commencement of the interval,
\end{flushleft}


\begin{flushleft}
then any calculation relating to the revision of overs shall not be effective until an
\end{flushleft}


\begin{flushleft}
amount of time equivalent to that by which the second innings started early has
\end{flushleft}


\begin{flushleft}
elapsed.
\end{flushleft}





13.7.2.2.3





\begin{flushleft}
To constitute a match, a minimum of 5 overs have to be bowled to the team
\end{flushleft}


\begin{flushleft}
batting second subject to a result not being achieved earlier.
\end{flushleft}





13.7.2.2.4





\begin{flushleft}
The team batting second shall not bat for a greater number of overs than the first
\end{flushleft}


\begin{flushleft}
team unless the latter completed its innings in less than its allocated overs.
\end{flushleft}





13.7.2.2.5





\begin{flushleft}
A fixed time will be specified for the close of play by applying a rate of 14.11 overs
\end{flushleft}


\begin{flushleft}
per hour. The timing and duration of all relative delays, extensions in playing
\end{flushleft}


\begin{flushleft}
hours and interruptions in play will be taken into consideration in specifying this
\end{flushleft}


\begin{flushleft}
time.
\end{flushleft}





13.7.2.2.6





\begin{flushleft}
If the team fielding second fails to bowl the revised overs by the scheduled or rescheduled close of play, the hours of play shall be extended until the overs have
\end{flushleft}


\begin{flushleft}
been bowled or a result achieved.
\end{flushleft}





13.7.2.2.7





\begin{flushleft}
Penalties shall apply for slow over rates (refer to the ICC Code of Conduct).
\end{flushleft}





\begin{flushleft}
Extra Time
\end{flushleft}





\begin{flushleft}
The participating countries may agree to provide for extra time where the start of play is delayed or play is
\end{flushleft}


\begin{flushleft}
suspended. For clarity, the changeover period (maximum 10 mins) for a Super Over after the main match is not to be
\end{flushleft}


\begin{flushleft}
taken into account when applying any permitted extra time available.
\end{flushleft}





13.9





\begin{flushleft}
Number of Overs per Bowler
\end{flushleft}





13.9.1





\begin{flushleft}
No bowler shall bowl more than 4 overs in an innings.
\end{flushleft}





13.9.2





\begin{flushleft}
In a delayed or interrupted match where the overs are reduced for both teams or for the team bowling
\end{flushleft}


\begin{flushleft}
second;
\end{flushleft}


13.9.2.1





\begin{flushleft}
for innings of rescheduled length of at least 10 overs, no bowler may bowl more than one-fifth of
\end{flushleft}


\begin{flushleft}
the total overs allowed. Where the total overs is not divisible by 5, one additional over shall be
\end{flushleft}


\begin{flushleft}
allowed to the maximum number per bowler necessary to make up the balance.
\end{flushleft}





13.9.2.2





\begin{flushleft}
for innings of rescheduled length of between 5 and 9 overs, no bowler may bowl more than two
\end{flushleft}


\begin{flushleft}
overs.
\end{flushleft}





13.9.3





\begin{flushleft}
In the event of a bowler breaking down and being unable to complete an over, the remaining balls will be
\end{flushleft}


\begin{flushleft}
allowed by another bowler. Such part of an over will count as a full over only in so far as each bowler's limit
\end{flushleft}


\begin{flushleft}
is concerned.
\end{flushleft}





13.9.4





\begin{flushleft}
The scoreboard shall show the total number of overs bowled and the number of overs bowled by each
\end{flushleft}


\begin{flushleft}
bowler.
\end{flushleft}





\begin{flushleft}
14 THE FOLLOW-ON
\end{flushleft}


\begin{flushleft}
Shall not apply.
\end{flushleft}





18





\begin{flushleft}
\newpage
15 DECLARATION AND FORFEITURE
\end{flushleft}


\begin{flushleft}
Shall not apply.
\end{flushleft}





\begin{flushleft}
16 THE RESULT
\end{flushleft}


16.1





\begin{flushleft}
A Win
\end{flushleft}





16.1.1





\begin{flushleft}
The side which has scored in its one innings a total of runs in excess of that scored by the opposing side in
\end{flushleft}


\begin{flushleft}
its one completed innings shall win the match. See clause 13.3 (Completed innings). Note also clause 16.4
\end{flushleft}


\begin{flushleft}
(Winning hit or extras).
\end{flushleft}





16.1.2





\begin{flushleft}
Save for circumstances where a match is awarded to a team as a consequence of the opposing team's
\end{flushleft}


\begin{flushleft}
refusal to play (Clause 16.2), a result can be achieved only if both teams have had the opportunity of batting
\end{flushleft}


\begin{flushleft}
for at least 5 overs, unless one team has been all out in less than 5 overs or unless the team batting second
\end{flushleft}


\begin{flushleft}
scores enough runs to win in less than 5 overs.
\end{flushleft}





16.1.3





\begin{flushleft}
Save for circumstances where a match is awarded to a team as a consequence of the opposing team's
\end{flushleft}


\begin{flushleft}
refusal to play (Clause 16.2), all matches in which both teams have not had an opportunity of batting for a
\end{flushleft}


\begin{flushleft}
minimum of 5 overs, shall be declared a No Result.
\end{flushleft}





16.2





\begin{flushleft}
ICC Match Referee awarding a match
\end{flushleft}





16.2.1





\begin{flushleft}
A match shall be lost by a side which either
\end{flushleft}


16.2.1.1





\begin{flushleft}
concedes defeat or
\end{flushleft}





16.2.1.2





\begin{flushleft}
in the opinion of the ICC Match Referee refuses to play and the ICC Match Referee shall award
\end{flushleft}


\begin{flushleft}
the match to the other side.
\end{flushleft}





16.2.2





\begin{flushleft}
If an umpire considers that an action by any player or players might constitute a refusal by either side to play
\end{flushleft}


\begin{flushleft}
then the umpires together shall inform the ICC Match Referee of this fact. The ICC Match Referee shall
\end{flushleft}


\begin{flushleft}
together with the umpires ascertain the cause of the action. If the ICC Match Referee, after due consultation
\end{flushleft}


\begin{flushleft}
with the umpires, then decides that this action does constitute a refusal to play by one side, he/she shall so
\end{flushleft}


\begin{flushleft}
inform the captain of that side. If the captain persists in the action the ICC Match Referee shall award the
\end{flushleft}


\begin{flushleft}
match in accordance with clause 16.2.1.2 above.
\end{flushleft}





16.2.3





\begin{flushleft}
If action as in clause 16.2.2 above takes place after play has started and does not constitute a refusal to
\end{flushleft}


\begin{flushleft}
play the delay or interruption in play shall be dealt with in the same manner as provided for in clauses 13.7.2
\end{flushleft}


\begin{flushleft}
(Delayed and Interrupted Matches) and 11.4 (Changing agreed times for intervals) above.
\end{flushleft}


\begin{flushleft}
Note: In addition to the consequences of any refusal to play prescribed under this clause, any such
\end{flushleft}


\begin{flushleft}
refusal, whether temporary or final, may result in disciplinary action being taken against the captain and
\end{flushleft}


\begin{flushleft}
team responsible under the ICC Code of Conduct.
\end{flushleft}





16.3





\begin{flushleft}
All other matches -- A Tie or No Result
\end{flushleft}





16.3.1





\begin{flushleft}
A Tie
\end{flushleft}


\begin{flushleft}
The result of a match shall be a Tie when all innings have been completed and the scores are equal.
\end{flushleft}


\begin{flushleft}
If the scores are equal, the result shall be a tie and no account shall be taken of the number of wickets that
\end{flushleft}


\begin{flushleft}
have fallen. In the event of a tied match the teams shall compete in a Super Over to determine the winner.
\end{flushleft}


\begin{flushleft}
Refer to Appendix F.
\end{flushleft}





16.3.2





\begin{flushleft}
No Result
\end{flushleft}


\begin{flushleft}
See 16.1.3 above.
\end{flushleft}





16.4





\begin{flushleft}
Prematurely Terminated Matches - Calculation of the Target Score
\end{flushleft}





16.4.1





\begin{flushleft}
Interrupted Matches - Calculation of the Target Score
\end{flushleft}





19





\newpage
16.4.1.1





16.4.2





\begin{flushleft}
If, due to suspension of play after the start of the match, the number of overs in the innings of
\end{flushleft}


\begin{flushleft}
either team has to be revised to a lesser number than originally allotted (minimum of 5 overs),
\end{flushleft}


\begin{flushleft}
then a revised target score (to win) should be set for the number of overs which the team batting
\end{flushleft}


\begin{flushleft}
second will have the opportunity of facing. This revised target is to be calculated using the
\end{flushleft}


\begin{flushleft}
current Duckworth/Lewis/Stern method. The target set will always be a whole number and one
\end{flushleft}


\begin{flushleft}
run less will constitute a Tie. (Refer Duckworth/Lewis/Stern Regulations)
\end{flushleft}





\begin{flushleft}
Prematurely Terminated Matches
\end{flushleft}


16.4.2.1





\begin{flushleft}
If the innings of the side batting second is suspended (with at least 5 overs bowled) and it is not
\end{flushleft}


\begin{flushleft}
possible for the match to be resumed, the match will be decided by comparison with the DLS
\end{flushleft}


\begin{flushleft}
{`}Par Score' determined at the instant of the suspension by the Duckworth/Lewis/Stern method
\end{flushleft}


\begin{flushleft}
(refer Duckworth/Lewis/Stern Regulations). If the score is equal to the par score, the match is a
\end{flushleft}


\begin{flushleft}
Tie. Otherwise the result is a victory, or defeat, by the margin of runs by which the score
\end{flushleft}


\begin{flushleft}
exceeds, or falls short of, the Par Score.
\end{flushleft}





16.5





\begin{flushleft}
Winning hit or extras
\end{flushleft}





16.5.1





\begin{flushleft}
As soon as a result is reached as defined in clauses 16.1, 16.2 or 16.3.1, the match is at an end. Nothing
\end{flushleft}


\begin{flushleft}
that happens thereafter, except as in clause 41.18.2 (Penalty runs), shall be regarded as part of it. Note also
\end{flushleft}


\begin{flushleft}
clause 16.8.
\end{flushleft}





16.5.2





\begin{flushleft}
The side batting last will have scored enough runs to win only if its total of runs is sufficient without including
\end{flushleft}


\begin{flushleft}
any runs completed by the batsmen before the completion of a catch, or the obstruction of a catch, from
\end{flushleft}


\begin{flushleft}
which the striker could be dismissed.
\end{flushleft}





16.5.3





\begin{flushleft}
If a boundary is scored before the batsmen have completed sufficient runs to win the match, the whole of the
\end{flushleft}


\begin{flushleft}
boundary allowance shall be credited to the side's total and, in the case of a hit by the bat, to the striker's
\end{flushleft}


\begin{flushleft}
score.
\end{flushleft}





16.6





\begin{flushleft}
Statement of result
\end{flushleft}





\begin{flushleft}
If the side batting last wins the match without losing all its wickets, the result shall be stated as a win by the number of
\end{flushleft}


\begin{flushleft}
wickets still then to fall.
\end{flushleft}


\begin{flushleft}
If, without having scored a total of runs in excess of the total scored by the opposing side, the side batting last has
\end{flushleft}


\begin{flushleft}
lost all its wickets, but as the result of an award of 5 Penalty runs its total of runs is then sufficient to win, the result
\end{flushleft}


\begin{flushleft}
shall be stated as a win to that side by Penalty runs.
\end{flushleft}


\begin{flushleft}
If the side fielding last wins the match, the result shall be stated as a win by runs.
\end{flushleft}


\begin{flushleft}
If the match is decided by one side conceding defeat or refusing to play, the result shall be stated as Match
\end{flushleft}


\begin{flushleft}
Conceded or Match Awarded, as the case may be.
\end{flushleft}





16.7





\begin{flushleft}
Correctness of result
\end{flushleft}





\begin{flushleft}
Any decision as to the correctness of the scores shall be the responsibility of the umpires. See clause 2.15
\end{flushleft}


\begin{flushleft}
(Correctness of scores).
\end{flushleft}





16.8





\begin{flushleft}
Mistakes in scoring
\end{flushleft}





\begin{flushleft}
If, after the players and umpires have left the field in the belief that the match has been concluded, the umpires
\end{flushleft}


\begin{flushleft}
discover that a mistake in scoring has occurred which affects the result then, subject to clause 16.9, they shall adopt
\end{flushleft}


\begin{flushleft}
the following procedure.
\end{flushleft}


16.8.1





\begin{flushleft}
If, when the players leave the field, the side batting last has not completed its innings and,
\end{flushleft}


\begin{flushleft}
either the number of overs to be bowled in that innings has not been completed, or
\end{flushleft}


\begin{flushleft}
the end of the innings has not been reached
\end{flushleft}


\begin{flushleft}
then, unless one side concedes defeat, the umpires shall order play to resume.
\end{flushleft}





20





\begin{flushleft}
\newpage
Unless a result is reached sooner, play will then continue, if conditions permit, until the prescribed number of
\end{flushleft}


\begin{flushleft}
overs has been completed. The number of overs shall be taken as they were at the call of Time for the
\end{flushleft}


\begin{flushleft}
supposed conclusion of the match. No account shall be taken of the time between that moment and the
\end{flushleft}


\begin{flushleft}
resumption of play.
\end{flushleft}


16.8.2





\begin{flushleft}
If, at this call of Time, the overs have been completed and no Playing time remains, or if the side batting last
\end{flushleft}


\begin{flushleft}
has completed its innings, the umpires shall immediately inform both captains of the necessary corrections
\end{flushleft}


\begin{flushleft}
to the scores and to the result.
\end{flushleft}





16.9





\begin{flushleft}
Result not to be changed
\end{flushleft}





\begin{flushleft}
Once the umpires have agreed with the scorers the correctness of the scores at the conclusion of the match -- see
\end{flushleft}


\begin{flushleft}
clauses 2.15 (Correctness of scores) and 3.2 (Correctness of scores) -- the result cannot thereafter be changed.
\end{flushleft}





\begin{flushleft}
16.10 Points
\end{flushleft}


\begin{flushleft}
A points system shall not apply.
\end{flushleft}





\begin{flushleft}
17 THE OVER
\end{flushleft}


17.1





\begin{flushleft}
Number of balls
\end{flushleft}





\begin{flushleft}
The ball shall be bowled from each end alternately in overs of 6 balls.
\end{flushleft}





17.2





\begin{flushleft}
Start of an over
\end{flushleft}





\begin{flushleft}
An over has started when the bowler starts his run-up or, if there is no run-up, starts his action for the first delivery of
\end{flushleft}


\begin{flushleft}
that over.
\end{flushleft}





17.3





\begin{flushleft}
Validity of balls
\end{flushleft}





17.3.1





\begin{flushleft}
A ball shall not count as one of the 6 balls of the over unless it is delivered, even though, as in clause 41.16
\end{flushleft}


\begin{flushleft}
(Non-striker leaving his ground early) a batsman may be dismissed or some other incident occurs without
\end{flushleft}


\begin{flushleft}
the ball having been delivered.
\end{flushleft}





17.3.2





\begin{flushleft}
A ball delivered by the bowler shall not count as one of the 6 balls of the over
\end{flushleft}


17.3.2.1





\begin{flushleft}
if it is called dead, or is to be considered dead, before the striker has had an opportunity to play
\end{flushleft}


\begin{flushleft}
it. See clause 20.6 (Dead ball; ball counting as one of over).
\end{flushleft}





17.3.2.2





\begin{flushleft}
if it is called dead in the circumstances of clause 20.4.2.6. Note also the special provisions of
\end{flushleft}


\begin{flushleft}
clause 20.4.2.5. (Umpire calling and signaling Dead ball).
\end{flushleft}





17.3.2.3





\begin{flushleft}
if it is a No ball. See clause 21 (No ball).
\end{flushleft}





17.3.2.4





\begin{flushleft}
if it is a Wide. See clause 22 (Wide ball).
\end{flushleft}





17.3.2.5





\begin{flushleft}
when any of clauses 24.4 (Player returning without permission), 28.2 (Fielding the ball), 41.4
\end{flushleft}


\begin{flushleft}
(Deliberate attempt to distract striker), or 41.5 (Deliberate distraction, deception or obstruction of
\end{flushleft}


\begin{flushleft}
batsman) is applied.
\end{flushleft}





17.3.3





\begin{flushleft}
Any deliveries other than those listed in clause 17.3.1 and 17.3.2 shall be known as valid balls. Only valid
\end{flushleft}


\begin{flushleft}
balls shall count towards the 6 balls of the over.
\end{flushleft}





17.4





\begin{flushleft}
Call of Over
\end{flushleft}





\begin{flushleft}
When 6 valid balls have been bowled and when the ball becomes dead, the umpire shall call Over before leaving the
\end{flushleft}


\begin{flushleft}
wicket. See also clause 20.3 (Call of Over or Time).
\end{flushleft}





17.5





\begin{flushleft}
Umpire miscounting
\end{flushleft}





17.5.1





\begin{flushleft}
If the umpire miscounts the number of valid balls, the over as counted by the umpire shall stand.
\end{flushleft}





21





\newpage
17.5.2





\begin{flushleft}
If, having miscounted, the umpire allows an over to continue after 6 valid balls have been bowled, he/she
\end{flushleft}


\begin{flushleft}
may subsequently call Over when the ball becomes dead after any delivery, even if that delivery is not a
\end{flushleft}


\begin{flushleft}
valid ball.
\end{flushleft}





17.5.3





\begin{flushleft}
Whenever possible, the third umpire shall liaise with the scorers and if possible inform the on-field umpires if
\end{flushleft}


\begin{flushleft}
the over has been miscounted.
\end{flushleft}





17.6





\begin{flushleft}
Bowler changing ends
\end{flushleft}





\begin{flushleft}
A bowler shall be allowed to change ends as often as desired, provided he does not bowl two overs consecutively,
\end{flushleft}


\begin{flushleft}
nor bowl parts of each of two consecutive overs, in the same innings.
\end{flushleft}





17.7





\begin{flushleft}
Finishing an over
\end{flushleft}





17.7.1





\begin{flushleft}
Other than at the end of an innings, a bowler shall finish an over in progress unless incapacitated or
\end{flushleft}


\begin{flushleft}
suspended under these Playing Conditions.
\end{flushleft}





17.7.2





\begin{flushleft}
If for any reason, other than the end of an innings, an over is left uncompleted at the start of an interval or
\end{flushleft}


\begin{flushleft}
interruption, it shall be completed on resumption of play.
\end{flushleft}





17.8





\begin{flushleft}
Bowler incapacitated or suspended during an over
\end{flushleft}





\begin{flushleft}
If for any reason a bowler is incapacitated while running up to deliver the first ball of an over, or is incapacitated or
\end{flushleft}


\begin{flushleft}
suspended during an over, the umpire shall call and signal Dead ball. Another bowler shall complete the over from
\end{flushleft}


\begin{flushleft}
the same end, provided that he does not bowl two overs consecutively, nor bowl parts of each of two consecutive
\end{flushleft}


\begin{flushleft}
overs, in that innings.
\end{flushleft}





\begin{flushleft}
18 SCORING RUNS
\end{flushleft}


18.1





\begin{flushleft}
A run
\end{flushleft}





\begin{flushleft}
The score shall be reckoned by runs. A run is scored
\end{flushleft}


18.1.1





\begin{flushleft}
so often as the batsmen, at any time while the ball is in play, have crossed and made good their ground from
\end{flushleft}


\begin{flushleft}
end to end.
\end{flushleft}





18.1.2





\begin{flushleft}
when a boundary is scored. See clause 19 (Boundaries).
\end{flushleft}





18.1.3





\begin{flushleft}
when Penalty runs are awarded. See clause18.6.
\end{flushleft}





18.2





\begin{flushleft}
Runs disallowed
\end{flushleft}





\begin{flushleft}
Wherever in these Playing Conditions provision is made for the scoring of runs or awarding of penalties, such runs
\end{flushleft}


\begin{flushleft}
and penalties will be subject to any provisions that may be applicable for the disallowance of runs or for the nonaward of penalties.
\end{flushleft}


\begin{flushleft}
When runs are disallowed, the one run penalty for No ball or Wide shall stand and 5 run penalties shall be allowed,
\end{flushleft}


\begin{flushleft}
except for Penalty runs under clause 28.3 (Protective helmets belonging to the fielding side).
\end{flushleft}





18.3





\begin{flushleft}
Short runs
\end{flushleft}





18.3.1





\begin{flushleft}
A run is short if a batsman fails to make good his ground in turning for a further run.
\end{flushleft}





18.3.2





\begin{flushleft}
Although a short run shortens the succeeding one, the latter if completed shall not be regarded as short. A
\end{flushleft}


\begin{flushleft}
striker setting off for the first run from in front of the popping crease may do so also without penalty.
\end{flushleft}





18.4





\begin{flushleft}
Unintentional short runs
\end{flushleft}





\begin{flushleft}
Except in the circumstances of clause 18.5,
\end{flushleft}


18.4.1





\begin{flushleft}
if either batsman runs a short run, the umpire concerned shall, unless a boundary is scored, call and signal
\end{flushleft}


\begin{flushleft}
Short run as soon as the ball becomes dead and that run shall not be scored.
\end{flushleft}





22





\newpage
18.4.2





\begin{flushleft}
if, after either or both batsmen run short, a boundary is scored the umpire concerned shall disregard the
\end{flushleft}


\begin{flushleft}
short running and shall not call or signal Short run.
\end{flushleft}





18.4.3





\begin{flushleft}
if both batsmen run short in one and the same run, this shall be regarded as only one short run.
\end{flushleft}





18.4.4





\begin{flushleft}
if more than one run is short then, subject to clauses 18.4.2 and 18.4.3, all runs called as short shall not be
\end{flushleft}


\begin{flushleft}
scored.
\end{flushleft}





18.4.5





\begin{flushleft}
if there has been more than one short run, the umpire shall inform the scorers as to the number of runs to be
\end{flushleft}


\begin{flushleft}
recorded.
\end{flushleft}





18.5





\begin{flushleft}
Deliberate short runs
\end{flushleft}





18.5.1





\begin{flushleft}
If either umpire considers that one or both batsmen deliberately ran short at that umpire's end, the umpire
\end{flushleft}


\begin{flushleft}
concerned shall, when the ball is dead, call and signal Short run and inform the other umpire of what has
\end{flushleft}


\begin{flushleft}
occurred and apply clause 18.5.2.
\end{flushleft}





18.5.2





\begin{flushleft}
The bowler's end umpire shall
\end{flushleft}


\begin{flushleft}
- disallow all runs to the batting side
\end{flushleft}


\begin{flushleft}
- return any not out batsman to his original end
\end{flushleft}


\begin{flushleft}
- signal No ball or Wide to the scorers, if applicable
\end{flushleft}


\begin{flushleft}
- award 5 Penalty runs to the fielding side
\end{flushleft}


\begin{flushleft}
- award any other 5-run Penalty that is applicable except for Penalty runs under clause 28.3 (Protective
\end{flushleft}


\begin{flushleft}
helmets belonging to the fielding side)
\end{flushleft}


\begin{flushleft}
- inform the scorers as to the number of runs to be recorded, and
\end{flushleft}


\begin{flushleft}
- inform the captain of the fielding side and, as soon as practicable, the captain of the batting side of the
\end{flushleft}


\begin{flushleft}
reason for this action.
\end{flushleft}





18.6





\begin{flushleft}
Runs awarded for penalties
\end{flushleft}





\begin{flushleft}
Runs shall be awarded for penalties under clause 18.5 (Deliberate short runs), 24.4 (Player returning without
\end{flushleft}


\begin{flushleft}
permission), 26.4 (Penalties for contravention), 21 (No ball), 22 (Wide ball), 28.2(Fielding the ball), 28.3 (Protective
\end{flushleft}


\begin{flushleft}
helmets belonging to the fielding side) 41 (Unfair play) and 42 (Players' conduct). Note, however, the restrictions on
\end{flushleft}


\begin{flushleft}
the award of Penalty runs in clauses 23.3 (Leg byes not to be awarded), 28.3 (Protective helmets belonging to the
\end{flushleft}


\begin{flushleft}
fielding side) and 34 (Hit the ball twice).
\end{flushleft}





18.7





\begin{flushleft}
Runs scored for boundaries
\end{flushleft}





\begin{flushleft}
Runs shall be scored for boundary allowances under clause 19 (Boundaries).
\end{flushleft}





18.8





\begin{flushleft}
Runs scored when a batsman is dismissed
\end{flushleft}





\begin{flushleft}
When a batsman is dismissed, any runs for penalties awarded to either side shall stand.
\end{flushleft}


\begin{flushleft}
No other runs shall be credited to the batting side, except as follows.
\end{flushleft}


18.8.1





\begin{flushleft}
If a batsman is dismissed Obstructing the field, the batting side shall also score any runs completed before
\end{flushleft}


\begin{flushleft}
the offence.
\end{flushleft}


\begin{flushleft}
If, however, the obstruction prevented a catch being made, no runs other than penalties shall be scored.
\end{flushleft}





18.8.2





\begin{flushleft}
If a batsman is dismissed Run out, the batting side shall also score any runs completed before the wicket
\end{flushleft}


\begin{flushleft}
was put down.
\end{flushleft}





23





\newpage
18.9





\begin{flushleft}
Runs scored when the ball becomes dead other than at the fall of a wicket
\end{flushleft}





\begin{flushleft}
When the ball becomes dead for any reason other than the fall of a wicket, or is called dead by an umpire, unless
\end{flushleft}


\begin{flushleft}
there is specific provision otherwise in these Playing Conditions, any runs for penalties awarded to either side shall
\end{flushleft}


\begin{flushleft}
be scored. Note however the provisions of clauses 23.3 (Leg byes not to be awarded) and 28.3 (Protective helmets
\end{flushleft}


\begin{flushleft}
belonging to the fielding side).
\end{flushleft}


\begin{flushleft}
Additionally the batting side shall be credited with all runs completed by the batsmen before the incident or call of
\end{flushleft}


\begin{flushleft}
Dead ball and the run in progress if the batsmen had already crossed at the instant of the incident or call of Dead ball.
\end{flushleft}


\begin{flushleft}
Note specifically, however, the provisions of clause 41.5.8 (Deliberate distraction, deception or obstruction of
\end{flushleft}


\begin{flushleft}
batsman).
\end{flushleft}





\begin{flushleft}
18.10 Crediting of runs scored
\end{flushleft}


\begin{flushleft}
Unless stated otherwise in these Playing Conditions,
\end{flushleft}


\begin{flushleft}
18.10.1 if the ball is struck by the bat, all runs scored by the batting side shall be credited to the striker, except for
\end{flushleft}


\begin{flushleft}
the following:
\end{flushleft}


\begin{flushleft}
- an award of 5 Penalty runs, which shall be scored as Penalty runs
\end{flushleft}


\begin{flushleft}
- the one run penalty for a No ball, which shall be scored as a No balls extra.
\end{flushleft}


\begin{flushleft}
18.10.2 if the ball is not struck by the bat, runs shall be scored as Penalty runs, Byes, Leg byes, No ball extras or
\end{flushleft}


\begin{flushleft}
Wides as the case may be. If Byes or Leg byes accrue from a No ball, only the one run penalty for No ball
\end{flushleft}


\begin{flushleft}
shall be scored as such, and the remainder as Byes or Leg byes as appropriate.
\end{flushleft}


\begin{flushleft}
18.10.3 the bowler shall be debited with:
\end{flushleft}


\begin{flushleft}
- all runs scored by the striker
\end{flushleft}


\begin{flushleft}
- all runs scored as No ball extras
\end{flushleft}


\begin{flushleft}
- all runs scored as Wides.
\end{flushleft}





\begin{flushleft}
18.11 Batsman returning to original end
\end{flushleft}


\begin{flushleft}
18.11.1 When the striker is dismissed in any of the circumstances in clauses 18.11.1.1 to 18.11.1.5, the not out
\end{flushleft}


\begin{flushleft}
batsman shall return to his original end.
\end{flushleft}


18.11.1.1





\begin{flushleft}
Bowled.
\end{flushleft}





18.11.1.2





\begin{flushleft}
Stumped.
\end{flushleft}





18.11.1.3





\begin{flushleft}
Hit the ball twice.
\end{flushleft}





18.11.1.4





\begin{flushleft}
LBW.
\end{flushleft}





18.11.1.5





\begin{flushleft}
Hit wicket.
\end{flushleft}





\begin{flushleft}
18.11.2 The batsmen shall return to their original ends in any of the cases of clauses 18.11.2.1 to 18.11.2.3.
\end{flushleft}


18.11.2.1





\begin{flushleft}
A boundary is scored.
\end{flushleft}





18.11.2.2





\begin{flushleft}
Runs are disallowed for any reason.
\end{flushleft}





18.11.2.3





\begin{flushleft}
A decision by the batsmen at the wicket to do so, under clause 41.5 (Deliberate distraction,
\end{flushleft}


\begin{flushleft}
deception or obstruction of batsman).
\end{flushleft}





\begin{flushleft}
18.12 Batsman returning to wicket he has left
\end{flushleft}


\begin{flushleft}
18.12.1 When a batsman is dismissed in any of the ways in clauses 18.12.1.1 to 18.12.1.3, the not out batsman
\end{flushleft}


\begin{flushleft}
shall return to the wicket he has left but only if the batsmen had not already crossed at the instant of the
\end{flushleft}


\begin{flushleft}
incident causing the dismissal. If runs are to be disallowed, however, the not out batsman shall return to his
\end{flushleft}


\begin{flushleft}
original end.
\end{flushleft}





24





\newpage
18.12.1.1





\begin{flushleft}
Caught
\end{flushleft}





18.12.1.2





\begin{flushleft}
Obstructing the field
\end{flushleft}





18.12.1.3





\begin{flushleft}
Run out.
\end{flushleft}





\begin{flushleft}
18.12.2 If, while a run is in progress, the ball becomes dead for any reason other than the dismissal of a batsman,
\end{flushleft}


\begin{flushleft}
the batsmen shall return to the wickets they had left, but only if they had not already crossed in running
\end{flushleft}


\begin{flushleft}
when the ball became dead. If, however, any of the circumstances of clauses 18.11.2.1 to 18.11.2.3 apply,
\end{flushleft}


\begin{flushleft}
the batsmen shall return to their original ends.
\end{flushleft}





\begin{flushleft}
19 BOUNDARIES
\end{flushleft}


19.1





\begin{flushleft}
Determining the boundary of the field of play
\end{flushleft}





19.1.1





\begin{flushleft}
Before the toss, the umpires shall determine the boundary of the field of play, which shall be fixed for the
\end{flushleft}


\begin{flushleft}
duration of the match. See clause 2.3.4 (Consultation with Home Board).
\end{flushleft}





19.1.2





\begin{flushleft}
The boundary shall be determined such that no part of any sight-screen, will, at any stage of the match, be
\end{flushleft}


\begin{flushleft}
within the field of play.
\end{flushleft}





19.1.3





\begin{flushleft}
The aim shall be to maximize the size of the playing area at each venue. With respect to the size of the
\end{flushleft}


\begin{flushleft}
boundaries, no boundary shall be longer than 90 yards (82.29 meters), and no boundary should be shorter
\end{flushleft}


\begin{flushleft}
than 65 yards (59.43 metres) from the centre of the pitch to be used.
\end{flushleft}





19.1.4





\begin{flushleft}
Any ground which has previously been approved to host international cricket which is unable to conform to
\end{flushleft}


\begin{flushleft}
the minimum boundary dimension shall be exempt. In such cases the boundary shall be positioned so as to
\end{flushleft}


\begin{flushleft}
maximize the size of the playing area.
\end{flushleft}





19.2





\begin{flushleft}
Identifying and marking the boundary
\end{flushleft}





19.2.1





\begin{flushleft}
All boundaries must be designated by a rope, or similar object of a minimum standard as authorised by the
\end{flushleft}


\begin{flushleft}
ICC from time to time. The rope should be positioned a required minimum distance (3 yards (2.74 metres)
\end{flushleft}


\begin{flushleft}
minimum) inside the perimeter fencing or advertising signs, or from any solid object located between the
\end{flushleft}


\begin{flushleft}
rope and the fence/signs. For grounds with a large playing area, the maximum length of boundary should be
\end{flushleft}


\begin{flushleft}
used before applying the minimum 3 yards (2.74 metres) between the boundary and the fence.
\end{flushleft}





19.2.2





\begin{flushleft}
If the boundary is marked by means of an object that is in contact with the ground the boundary will be the
\end{flushleft}


\begin{flushleft}
edge of the grounded part of the object which is nearest the pitch.
\end{flushleft}





19.2.3





\begin{flushleft}
An obstacle within the field of play shall not be regarded as a boundary unless so determined by the umpires
\end{flushleft}


\begin{flushleft}
before the toss. See clause 2.3.4 (Consultation with Home Board).
\end{flushleft}





19.2.4





\begin{flushleft}
If an unauthorized person enters the playing arena and handles the ball, the umpire at the bowler's end shall
\end{flushleft}


\begin{flushleft}
be the sole judge of whether the boundary allowance should be scored or the ball be treated as still in play
\end{flushleft}


\begin{flushleft}
or called dead ball if a batsman is liable to be out as a result of the unauthorized person handling the ball.
\end{flushleft}





19.3





\begin{flushleft}
Restoring the boundary
\end{flushleft}





\begin{flushleft}
If a solid object used to mark the boundary is disturbed for any reason, then:
\end{flushleft}


19.3.1





\begin{flushleft}
the boundary shall be considered to be in its original position.
\end{flushleft}





19.3.2





\begin{flushleft}
the object shall be returned to its original position as soon as is practicable; if play is taking place, this shall
\end{flushleft}


\begin{flushleft}
be as soon as the ball is dead.
\end{flushleft}





19.3.3





\begin{flushleft}
if some part of a fence or other marker has come within the field of play, that part shall be removed from the
\end{flushleft}


\begin{flushleft}
field of play as soon as is practicable; if play is taking place, this shall be as soon as the ball is dead.
\end{flushleft}





19.4





\begin{flushleft}
Ball grounded beyond the boundary
\end{flushleft}





19.4.1





\begin{flushleft}
The ball in play is grounded beyond the boundary if it touches
\end{flushleft}





25





\begin{flushleft}
\newpage
- the boundary or any part of an object used to mark the boundary;
\end{flushleft}


\begin{flushleft}
- the ground beyond the boundary;
\end{flushleft}


\begin{flushleft}
- any object that is grounded beyond the boundary.
\end{flushleft}


19.4.2





\begin{flushleft}
The ball in play is to be regarded as being grounded beyond the boundary if
\end{flushleft}


\begin{flushleft}
- a fielder, grounded beyond the boundary as in clause 19.5, touches the ball;
\end{flushleft}


\begin{flushleft}
- a fielder, after catching the ball within the boundary, becomes grounded beyond the boundary while in
\end{flushleft}


\begin{flushleft}
contact with the ball, before completing the catch.
\end{flushleft}





19.5





\begin{flushleft}
Fielder grounded beyond the boundary
\end{flushleft}





19.5.1





\begin{flushleft}
A fielder is grounded beyond the boundary if some part of his person is in contact with any of the following:
\end{flushleft}


\begin{flushleft}
- the boundary or any part of an object used to mark the boundary;
\end{flushleft}


\begin{flushleft}
- the ground beyond the boundary;
\end{flushleft}


\begin{flushleft}
- any object that is in contact with the ground beyond the boundary;
\end{flushleft}


\begin{flushleft}
- another fielder who is grounded beyond the boundary.
\end{flushleft}





19.5.2





\begin{flushleft}
A fielder who is not in contact with the ground is considered to be grounded beyond the boundary if his final
\end{flushleft}


\begin{flushleft}
contact with the ground, before his first contact with the ball after it has been delivered by the bowler, was
\end{flushleft}


\begin{flushleft}
not entirely within the boundary.
\end{flushleft}





19.6





\begin{flushleft}
Boundary allowances
\end{flushleft}





19.6.1





\begin{flushleft}
6 runs shall be allowed for a boundary 6; and 4 runs for a boundary 4. See also clause 19.7.
\end{flushleft}





19.7





\begin{flushleft}
Runs scored from boundaries
\end{flushleft}





19.7.1





\begin{flushleft}
A boundary 6 will be scored if and only if the ball has been struck by the bat and is first grounded beyond the
\end{flushleft}


\begin{flushleft}
boundary without having been in contact with the ground within the field of play. This shall apply even if the
\end{flushleft}


\begin{flushleft}
ball has previously touched a fielder.
\end{flushleft}





19.7.2





\begin{flushleft}
A boundary 4 will be scored when a ball that is grounded beyond the boundary
\end{flushleft}


\begin{flushleft}
- whether struck by the bat or not, was first grounded within the boundary, or
\end{flushleft}


\begin{flushleft}
- has not been struck by the bat.
\end{flushleft}





19.7.3





\begin{flushleft}
When a boundary is scored, the batting side, except in the circumstances of clause 19.8, shall be awarded
\end{flushleft}


\begin{flushleft}
whichever is the greater of
\end{flushleft}


19.7.3.1





\begin{flushleft}
the allowance for the boundary
\end{flushleft}





19.7.3.2





\begin{flushleft}
the runs completed by the batsmen together with the run in progress if they had already crossed
\end{flushleft}


\begin{flushleft}
at the instant the boundary is scored.
\end{flushleft}





19.7.4





\begin{flushleft}
When the runs in clause 19.7.3.2 exceed the boundary allowance they shall replace the boundary allowance
\end{flushleft}


\begin{flushleft}
for the purposes of clause 18.12.
\end{flushleft}





19.7.5





\begin{flushleft}
The scoring of Penalty runs by either side is not affected by the scoring of a boundary.
\end{flushleft}





19.8





\begin{flushleft}
Overthrow or wilful act of fielder
\end{flushleft}





\begin{flushleft}
If the boundary results from an overthrow or from the wilful act of a fielder, the runs scored shall be
\end{flushleft}


\begin{flushleft}
any runs for penalties awarded to either side
\end{flushleft}


\begin{flushleft}
and the allowance for the boundary
\end{flushleft}





26





\begin{flushleft}
\newpage
and the runs completed by the batsmen, together with the run in progress if they had already crossed at the instant of
\end{flushleft}


\begin{flushleft}
the throw or act.
\end{flushleft}


\begin{flushleft}
Clause 18.12.2 (Batsman returning to wicket he has left) shall apply as from the instant of the throw or act.
\end{flushleft}





\begin{flushleft}
20 DEAD BALL
\end{flushleft}


20.1





\begin{flushleft}
Ball is dead
\end{flushleft}





20.1.1





\begin{flushleft}
The ball becomes dead when
\end{flushleft}


20.1.1.1





\begin{flushleft}
it is finally settled in the hands of the wicket-keeper or of the bowler.
\end{flushleft}





20.1.1.2





\begin{flushleft}
a boundary is scored. See clause 19.7 (Runs scored from boundaries).
\end{flushleft}





20.1.1.3





\begin{flushleft}
a batsman is dismissed. The ball will be deemed to be dead from the instant of the incident
\end{flushleft}


\begin{flushleft}
causing the dismissal.
\end{flushleft}





20.1.1.4





\begin{flushleft}
whether played or not it becomes trapped between the bat and person of a batsman or between
\end{flushleft}


\begin{flushleft}
items of his clothing or equipment.
\end{flushleft}





20.1.1.5





\begin{flushleft}
whether played or not it lodges in the clothing or equipment of a batsman or the clothing of an
\end{flushleft}


\begin{flushleft}
umpire.
\end{flushleft}





20.1.1.6





\begin{flushleft}
there is an award of Penalty runs under either of clauses 24.4 (Player returning without
\end{flushleft}


\begin{flushleft}
permission) or 28.2 (Fielding the ball). The ball shall not count as one of the over.
\end{flushleft}





20.1.1.7





\begin{flushleft}
there is a contravention of clause 28.3 (Protective helmets belonging to the fielding side).
\end{flushleft}





20.1.1.8





\begin{flushleft}
the match is concluded in any of the ways stated in clause 12.6 (Conclusion of match).
\end{flushleft}





20.1.2





\begin{flushleft}
The ball shall be considered to be dead when it is clear to the bowler's end umpire that the fielding side and
\end{flushleft}


\begin{flushleft}
both batsmen at the wicket have ceased to regard it as in play.
\end{flushleft}





20.1.3





\begin{flushleft}
In a match where cameras are being used on or over the field of play (e.g. Spidercam), should a ball that
\end{flushleft}


\begin{flushleft}
has been hit by the batsman make contact, while still in play, with the camera, its apparatus or its cable,
\end{flushleft}


\begin{flushleft}
either umpire shall call and signal {`}dead ball'. The ball shall not count as one of the over and no runs shall be
\end{flushleft}


\begin{flushleft}
scored. If the delivery was called a No ball it shall count and the No ball penalty shall be applied. No other
\end{flushleft}


\begin{flushleft}
runs (including penalty runs) apart from the No ball penalty shall be scored.
\end{flushleft}





20.1.4





\begin{flushleft}
Should a ball thrown by a fielder make contact with a camera on or over the field of play, its apparatus or its
\end{flushleft}


\begin{flushleft}
cable, either umpire shall call and signal dead ball. Unless this was already a No ball or Wide, the ball shall
\end{flushleft}


\begin{flushleft}
count as one of the over. All runs scored to that point shall count, plus the run in progress if the batsmen
\end{flushleft}


\begin{flushleft}
have already crossed.
\end{flushleft}





20.1.5





\begin{flushleft}
Refer also to paragraph 2.6 of Appendix D.
\end{flushleft}





20.2





\begin{flushleft}
Ball finally settled
\end{flushleft}





\begin{flushleft}
Whether the ball is finally settled or not is a matter for the umpire alone to decide.
\end{flushleft}





20.3





\begin{flushleft}
Call of Over or Time
\end{flushleft}





\begin{flushleft}
Neither the call of Over (see clause 17.4), nor the call of Time (see clause 12.2) is to be made until the ball is dead,
\end{flushleft}


\begin{flushleft}
either under clauses 20.1 or 20.4.
\end{flushleft}





20.4





\begin{flushleft}
Umpire calling and signalling Dead ball
\end{flushleft}





20.4.1





\begin{flushleft}
When the ball has become dead under clause 20.1, the bowler's end umpire may call and signal Dead ball if
\end{flushleft}


\begin{flushleft}
it is necessary to inform the players.
\end{flushleft}





20.4.2





\begin{flushleft}
Either umpire shall call and signal Dead ball when
\end{flushleft}


20.4.2.1





\begin{flushleft}
intervening in a case of unfair play.
\end{flushleft}





27





\newpage
20.5





20.4.2.2





\begin{flushleft}
a possibly serious injury to a player or umpire occurs.
\end{flushleft}





20.4.2.3





\begin{flushleft}
leaving his/her normal position for consultation.
\end{flushleft}





20.4.2.4





\begin{flushleft}
one or both bails fall from the striker's wicket before the striker has had the opportunity of
\end{flushleft}


\begin{flushleft}
playing the ball.
\end{flushleft}





20.4.2.5





\begin{flushleft}
the striker is not ready for the delivery of the ball and, if the ball is delivered, makes no attempt
\end{flushleft}


\begin{flushleft}
to play it. Provided the umpire is satisfied that the striker had adequate reason for not being
\end{flushleft}


\begin{flushleft}
ready, the ball shall not count as one of the over.
\end{flushleft}





20.4.2.6





\begin{flushleft}
the striker is distracted by any noise or movement or in any other way while preparing to
\end{flushleft}


\begin{flushleft}
receive, or receiving a delivery. This shall apply whether the source of the distraction is within
\end{flushleft}


\begin{flushleft}
the match or outside it. Note also clause 20.4.2.7. The ball shall not count as one of the over.
\end{flushleft}





20.4.2.7





\begin{flushleft}
there is an instance of a deliberate attempt to distract under either of clauses 41.4 (Deliberate
\end{flushleft}


\begin{flushleft}
attempt to distract striker) or 41.5 (Deliberate distraction, deception or obstruction of batsman).
\end{flushleft}


\begin{flushleft}
The ball shall not count as one of the over.
\end{flushleft}





20.4.2.8





\begin{flushleft}
the bowler drops the ball accidentally before delivery.
\end{flushleft}





20.4.2.9





\begin{flushleft}
the ball does not leave the bowler's hand for any reason other than an attempt to run out the
\end{flushleft}


\begin{flushleft}
non-striker under clause 41.16 (Non-striker leaving his ground early).
\end{flushleft}





20.4.2.10





\begin{flushleft}
satisfied that the ball in play cannot be recovered.
\end{flushleft}





20.4.2.11





\begin{flushleft}
required to do so under any of the Playing Conditions not included above.
\end{flushleft}





\begin{flushleft}
Ball ceases to be dead
\end{flushleft}





\begin{flushleft}
The ball ceases to be dead -- that is, it comes into play -- when the bowler starts his run-up or, if there is no run-up,
\end{flushleft}


\begin{flushleft}
starts his bowling action.
\end{flushleft}





20.6





\begin{flushleft}
Dead ball; ball counting as one of over
\end{flushleft}





20.6.1





\begin{flushleft}
When a ball which has been delivered is called dead or is to be considered dead then, other than as in
\end{flushleft}


\begin{flushleft}
clause 20.6.2,
\end{flushleft}





20.6.2





20.6.1.1





\begin{flushleft}
it will not count in the over if the striker has not had an opportunity to play it.
\end{flushleft}





20.6.1.2





\begin{flushleft}
unless No ball or Wide ball has been called, it will be a valid ball if the striker has had an
\end{flushleft}


\begin{flushleft}
opportunity to play it, except in the circumstances of clauses 20.4.2.6 and 24.4 ( Player
\end{flushleft}


\begin{flushleft}
returning without permission), 28.2 (Fielding the ball), 41.4 (Deliberate attempt to distract striker)
\end{flushleft}


\begin{flushleft}
and 41.5 (Deliberate distraction, deception or obstruction of batsman).
\end{flushleft}





\begin{flushleft}
In clause 20.4.2.5, the ball will not count in the over only if both conditions of not attempting to play the ball
\end{flushleft}


\begin{flushleft}
and having an adequate reason for not being ready are met. Otherwise the delivery will be a valid ball.
\end{flushleft}





\begin{flushleft}
21 NO BALL
\end{flushleft}


21.1





\begin{flushleft}
Mode of delivery
\end{flushleft}





21.1.1





\begin{flushleft}
The umpire shall ascertain whether the bowler intends to bowl right handed or left handed, over or round the
\end{flushleft}


\begin{flushleft}
wicket, and shall so inform the striker.
\end{flushleft}


\begin{flushleft}
It is unfair if the bowler fails to notify the umpire of a change in his mode of delivery. In this case the umpire
\end{flushleft}


\begin{flushleft}
shall call and signal No ball.
\end{flushleft}





21.1.2





\begin{flushleft}
Underarm bowling shall not be permitted.
\end{flushleft}





21.2





\begin{flushleft}
Fair delivery -- the arm
\end{flushleft}





\begin{flushleft}
For a delivery to be fair in respect of the arm the ball must not be delivered with an Illegal Bowling Action.
\end{flushleft}





28





\begin{flushleft}
\newpage
An Illegal Bowling Action is defined as a bowling action where a bowler's Elbow Extension exceeds 15 degrees,
\end{flushleft}


\begin{flushleft}
measured from the point at which the bowling arm reaches the horizontal until the point at which the ball is released
\end{flushleft}


\begin{flushleft}
(any Elbow Hyperextension shall be discounted for the purposes of determining an Illegal Bowling Action).
\end{flushleft}


\begin{flushleft}
Should either umpire or the ICC Match Referee suspect that a bowler has used an Illegal Bowling Action, they shall
\end{flushleft}


\begin{flushleft}
complete the ICC Bowling Action Report Form at the conclusion of the match, as set out in the Illegal Bowling
\end{flushleft}


\begin{flushleft}
Regulations.
\end{flushleft}





21.3





\begin{flushleft}
Ball thrown or delivered underarm -- action by umpires
\end{flushleft}





21.3.1





\begin{flushleft}
If, in the opinion of either umpire, the ball has been thrown (where such mode of delivery does not
\end{flushleft}


\begin{flushleft}
correspond to the bowler's normal bowling action) or delivered underarm, he/she shall call and signal No ball
\end{flushleft}


\begin{flushleft}
and, when the ball is dead, inform the other umpire of the reason for the call.
\end{flushleft}


\begin{flushleft}
The bowler's end umpire shall then
\end{flushleft}


\begin{flushleft}
- warn the bowler, indicating that this is a first and final warning. This warning shall apply to that bowler
\end{flushleft}


\begin{flushleft}
throughout the innings.
\end{flushleft}


\begin{flushleft}
- inform the captain of the fielding side of the reason for this action.
\end{flushleft}


\begin{flushleft}
- inform the batsmen at the wicket of what has occurred.
\end{flushleft}





21.3.2





\begin{flushleft}
If either umpire considers that, in that innings, a further delivery by the same bowler is thrown (where such
\end{flushleft}


\begin{flushleft}
mode of delivery does not correspond to the bowler's normal bowling action) or delivered underarm, he/she
\end{flushleft}


\begin{flushleft}
shall call and signal No ball and when the ball is dead inform the other umpire of the reason for the call.
\end{flushleft}


\begin{flushleft}
The bowler's end umpire shall then
\end{flushleft}


\begin{flushleft}
- direct the captain of the fielding side to suspend the bowler immediately from bowling. The over shall, if
\end{flushleft}


\begin{flushleft}
applicable, be completed by another bowler, who shall neither have bowled the previous over or part thereof
\end{flushleft}


\begin{flushleft}
nor be allowed to bowl any part of the next over. The bowler thus suspended shall not bowl again in that
\end{flushleft}


\begin{flushleft}
innings.
\end{flushleft}


\begin{flushleft}
- inform the batsmen at the wicket and, as soon as practicable, the captain of the batting side of the reason
\end{flushleft}


\begin{flushleft}
for this action.
\end{flushleft}





21.3.3





\begin{flushleft}
The umpires together shall report the occurrence as soon as possible after the match to the ICC Match
\end{flushleft}


\begin{flushleft}
Referee, who shall take such action as is considered appropriate against the bowler concerned.
\end{flushleft}





21.4





\begin{flushleft}
Bowler throwing towards striker's end before delivery
\end{flushleft}





\begin{flushleft}
If the bowler throws the ball towards the striker's end before entering the delivery stride, either umpire shall call and
\end{flushleft}


\begin{flushleft}
signal No ball. See clause 41.17 (Batsmen stealing a run).
\end{flushleft}


\begin{flushleft}
However, the procedure stated in clause 21.3 of caution, informing, final warning, action against the bowler and
\end{flushleft}


\begin{flushleft}
reporting shall not apply.
\end{flushleft}





21.5





\begin{flushleft}
Fair delivery -- the feet
\end{flushleft}





\begin{flushleft}
For a delivery to be fair in respect of the feet, in the delivery stride
\end{flushleft}


21.5.1





\begin{flushleft}
the bowler's back foot must land within and not touching the return crease appertaining to his stated mode of
\end{flushleft}


\begin{flushleft}
delivery.
\end{flushleft}





21.5.2





\begin{flushleft}
the bowler's front foot must land with some part of the foot, whether grounded or raised
\end{flushleft}


\begin{flushleft}
- on the same side of the imaginary line joining the two middle stumps as the return crease described in
\end{flushleft}


\begin{flushleft}
clause 21.5.1, and
\end{flushleft}


\begin{flushleft}
- behind the popping crease.
\end{flushleft}





\begin{flushleft}
If the bowler's end umpire is satisfied that any of these three conditions have not been met, he/she shall call and
\end{flushleft}


\begin{flushleft}
signal No ball. See clause 41.8 (Bowling of deliberate front foot No ball).
\end{flushleft}





29





\newpage
21.6





\begin{flushleft}
Bowler breaking wicket in delivering ball
\end{flushleft}





\begin{flushleft}
Either umpire shall call and signal No ball if, other than in an attempt to run out the non-striker under clause 41.16,
\end{flushleft}


\begin{flushleft}
the bowler breaks the wicket at any time after the ball comes into play and before completion of the stride after the
\end{flushleft}


\begin{flushleft}
delivery stride. This shall include any clothing or other object that falls from his person and breaks the wicket.
\end{flushleft}





21.7





\begin{flushleft}
Ball bouncing more than once, rolling along the ground or pitching off the pitch
\end{flushleft}





\begin{flushleft}
The umpire shall call and signal No ball if a ball which he/she considers to have been delivered, without having
\end{flushleft}


\begin{flushleft}
previously touched bat or person of the striker,
\end{flushleft}


\begin{flushleft}
- bounces more than once
\end{flushleft}


\begin{flushleft}
- or rolls along the ground before it reaches the popping crease.
\end{flushleft}


\begin{flushleft}
- or pitches wholly or partially off the pitch as defined in clause 6.1 (Area of pitch) before it reaches the line of the
\end{flushleft}


\begin{flushleft}
striker's wicket.
\end{flushleft}





21.8





\begin{flushleft}
Ball coming to rest in front of striker's wicket
\end{flushleft}





\begin{flushleft}
If a ball delivered by the bowler comes to rest in front of the line of the striker's wicket, without having previously
\end{flushleft}


\begin{flushleft}
touched the bat or person of the striker, the umpire shall call and signal No ball and immediately call and signal Dead
\end{flushleft}


\begin{flushleft}
ball.
\end{flushleft}





21.9





\begin{flushleft}
Fielder intercepting a delivery
\end{flushleft}





\begin{flushleft}
If, except in the circumstances of clause 27.3 (Position of wicket-keeper) a ball delivered by the bowler, makes
\end{flushleft}


\begin{flushleft}
contact with any part of a fielder's person before it either makes contact with the striker's bat or person, or it passes
\end{flushleft}


\begin{flushleft}
the striker's wicket, the umpire shall call and signal No ball and immediately call and signal Dead ball.
\end{flushleft}





\begin{flushleft}
21.10 Ball bouncing over head height of striker
\end{flushleft}


\begin{flushleft}
See clauses 22.1.1.2 and 41.6.1.7.
\end{flushleft}





\begin{flushleft}
21.11 Call of No ball for infringement of other Playing Conditions
\end{flushleft}


\begin{flushleft}
In addition to the instances above, No ball is to be called and signalled as required by the following clauses:
\end{flushleft}


\begin{flushleft}
Clause 27.3 -- Position of wicket-keeper
\end{flushleft}


\begin{flushleft}
Clause 28.4 -- Limitation of on side fielders
\end{flushleft}


\begin{flushleft}
Clause 28.5 -- Fielders not to encroach on pitch
\end{flushleft}


\begin{flushleft}
Clause 41.6 -- Bowling of dangerous and unfair short pitched deliveries
\end{flushleft}


\begin{flushleft}
Clause 41.7 -- Bowling of dangerous and unfair non-pitching deliveries
\end{flushleft}


\begin{flushleft}
Clause 41.8 -- Bowling of deliberate front foot No ball.
\end{flushleft}





\begin{flushleft}
21.12 Revoking a call of No ball
\end{flushleft}


\begin{flushleft}
An umpire shall revoke the call of No ball if Dead ball is called under any of clauses 20.4.2.4 to 20.4.2.9 (Umpire
\end{flushleft}


\begin{flushleft}
calling and signaling Dead ball).
\end{flushleft}





\begin{flushleft}
21.13 No ball to over-ride Wide
\end{flushleft}


\begin{flushleft}
A call of No ball shall over-ride the call of Wide ball at any time. See clauses 22.1(Judging a Wide) and 22.2 (Call and
\end{flushleft}


\begin{flushleft}
signal of Wide ball).
\end{flushleft}





\begin{flushleft}
21.14 Ball not dead
\end{flushleft}


\begin{flushleft}
The ball does not become dead on the call of No ball.
\end{flushleft}





30





\begin{flushleft}
\newpage
21.15 Penalty for a No ball
\end{flushleft}


\begin{flushleft}
A penalty of one run shall be awarded instantly on the call of No ball. Unless the call is revoked, the penalty shall
\end{flushleft}


\begin{flushleft}
stand even if a batsman is dismissed. It shall be in addition to any other runs scored, any boundary allowance and
\end{flushleft}


\begin{flushleft}
any other runs awarded for penalties.
\end{flushleft}





\begin{flushleft}
21.16 Runs resulting from a No ball -- how scored
\end{flushleft}


\begin{flushleft}
The one run penalty shall be scored as a No ball extra and shall be debited against the bowler. If other Penalty runs
\end{flushleft}


\begin{flushleft}
have been awarded to either side these shall be scored as stated in clause 41.18 (Penalty runs). Any runs completed
\end{flushleft}


\begin{flushleft}
by the batsmen or any boundary allowance shall be credited to the striker if the ball has been struck by the bat;
\end{flushleft}


\begin{flushleft}
otherwise they shall also be scored as Byes or Leg byes as appropriate.
\end{flushleft}





\begin{flushleft}
21.17 No ball not to count
\end{flushleft}


\begin{flushleft}
A No ball shall not count as one of the over. See clause 17.3 (Validity of balls).
\end{flushleft}





\begin{flushleft}
21.18 Out from a No ball
\end{flushleft}


\begin{flushleft}
When No ball has been called, neither batsman shall be out under any of the Playing Conditions except clause 34
\end{flushleft}


\begin{flushleft}
(Hit the ball twice), clause 37 (Obstructing the field) or clause 38 (Run out).
\end{flushleft}





\begin{flushleft}
21.19 Free Hit
\end{flushleft}


\begin{flushleft}
21.19.1 In addition to the above, the delivery following a No ball called (all modes of No ball) shall be a free hit for
\end{flushleft}


\begin{flushleft}
whichever batsman is facing it. If the delivery for the free hit is not a legitimate delivery (any kind of No ball
\end{flushleft}


\begin{flushleft}
or a Wide) then the next delivery will become a free hit for whichever batsman is facing it.
\end{flushleft}


\begin{flushleft}
21.19.2 For any free hit, the striker can be dismissed only under the circumstances that apply for a No ball, even if
\end{flushleft}


\begin{flushleft}
the delivery for the free hit is called Wide.
\end{flushleft}


\begin{flushleft}
21.19.3 Neither field changes nor the exchange of individuals between fielding positions are permitted for free hit
\end{flushleft}


\begin{flushleft}
deliveries unless:
\end{flushleft}


21.19.3.1





\begin{flushleft}
There is a change of striker (the provisions of clause 41.2 shall apply), or
\end{flushleft}





21.19.3.2





\begin{flushleft}
The No ball was the result of a fielding restriction breach, in which case the field may be
\end{flushleft}


\begin{flushleft}
changed to the extent of correcting the breach.
\end{flushleft}





\begin{flushleft}
21.19.4 For clarity, the bowler can change his mode of delivery for the free hit delivery. In such circumstances
\end{flushleft}


\begin{flushleft}
Clause 21.1 shall apply.
\end{flushleft}


\begin{flushleft}
21.19.5 The umpires will signal a free hit by (after the normal No ball signal) extending one arm straight upwards and
\end{flushleft}


\begin{flushleft}
moving it in a circular motion.
\end{flushleft}





\begin{flushleft}
22 WIDE BALL
\end{flushleft}


22.1





\begin{flushleft}
Judging a Wide
\end{flushleft}





22.1.1





\begin{flushleft}
If the bowler bowls a ball, not being a No ball, the umpire shall adjudge it a Wide if, according to the
\end{flushleft}


\begin{flushleft}
definition in clause 22.1.2
\end{flushleft}


22.1.1.1





\begin{flushleft}
the ball passes wide of where the striker is standing and which also would have passed wide of
\end{flushleft}


\begin{flushleft}
the striker standing in a normal guard position.
\end{flushleft}





22.1.1.2





\begin{flushleft}
the ball passes above the head height of the striker standing upright at the popping crease.
\end{flushleft}





22.1.2





\begin{flushleft}
The ball will be considered as passing wide of the striker unless it is sufficiently within reach for him to be
\end{flushleft}


\begin{flushleft}
able to hit it with the bat by means of a normal cricket stroke.
\end{flushleft}





22.1.3





\begin{flushleft}
Umpires are instructed to apply very strict and consistent interpretation in regard to this clause in order to
\end{flushleft}


\begin{flushleft}
prevent negative bowling wide of the wicket.
\end{flushleft}





31





\newpage
22.2





\begin{flushleft}
Call and signal of Wide ball
\end{flushleft}





\begin{flushleft}
If the umpire adjudges a delivery to be a Wide he/she shall call and signal Wide ball as soon as the ball passes the
\end{flushleft}


\begin{flushleft}
striker's wicket. It shall, however, be considered to have been a Wide from the instant that the bowler entered his
\end{flushleft}


\begin{flushleft}
delivery stride, even though it cannot be called Wide until it passes the striker's wicket.
\end{flushleft}





22.3





\begin{flushleft}
Revoking a call of Wide ball
\end{flushleft}





22.3.1





\begin{flushleft}
The umpire shall revoke the call of Wide ball if there is then any contact between the ball and the striker's
\end{flushleft}


\begin{flushleft}
bat or person before the ball comes into contact with any fielder.
\end{flushleft}





22.3.2





\begin{flushleft}
The umpire shall revoke the call of Wide ball if a delivery is called a No ball. See clause 21.13 (No ball to
\end{flushleft}


\begin{flushleft}
over-ride Wide).
\end{flushleft}





22.4





\begin{flushleft}
Delivery not a Wide
\end{flushleft}





22.4.1





\begin{flushleft}
The umpire shall not adjudge a delivery as being a Wide, if the striker, by moving, either causes the ball to
\end{flushleft}


\begin{flushleft}
pass wide of him, as defined in clause 22.1.2 or brings the ball sufficiently within reach to be able to hit it by
\end{flushleft}


\begin{flushleft}
means of a normal cricket stroke.
\end{flushleft}





22.4.2





\begin{flushleft}
The umpire shall not adjudge a delivery as being a Wide if the ball touches the striker's bat or person, but
\end{flushleft}


\begin{flushleft}
only as the ball passes the striker.
\end{flushleft}





22.5





\begin{flushleft}
Ball not dead
\end{flushleft}





\begin{flushleft}
The ball does not become dead on the call of Wide ball.
\end{flushleft}





22.6





\begin{flushleft}
Penalty for a Wide
\end{flushleft}





\begin{flushleft}
A penalty of one run shall be awarded instantly on the call of Wide ball. Unless the call is revoked, see clause 22.3,
\end{flushleft}


\begin{flushleft}
this penalty shall stand even if a batsman is dismissed, and shall be in addition to any other runs scored, any
\end{flushleft}


\begin{flushleft}
boundary allowance and any other runs awarded for penalties.
\end{flushleft}





22.7





\begin{flushleft}
Runs resulting from a Wide -- how scored
\end{flushleft}





\begin{flushleft}
All runs completed by the batsmen or a boundary allowance, together with the penalty for the Wide, shall be scored
\end{flushleft}


\begin{flushleft}
as Wide balls. Apart from any award of 5 Penalty runs, all runs resulting from a Wide shall be debited against the
\end{flushleft}


\begin{flushleft}
bowler.
\end{flushleft}





22.8





\begin{flushleft}
Wide not to count
\end{flushleft}





\begin{flushleft}
A Wide shall not count as one of the over. See clause 17.3 (Validity of balls).
\end{flushleft}





22.9





\begin{flushleft}
Out from a Wide
\end{flushleft}





\begin{flushleft}
When Wide ball has been called, neither batsman shall be out under any of the Playing Conditions except clause 35
\end{flushleft}


\begin{flushleft}
(Hit wicket), clause 37 (Obstructing the field), clause 38 (Run out) or clause 39 (Stumped).
\end{flushleft}





\begin{flushleft}
23 BYE AND LEG BYE
\end{flushleft}


23.1





\begin{flushleft}
Byes
\end{flushleft}





\begin{flushleft}
If the ball, delivered by the bowler, not being a Wide, passes the striker without touching his bat or person, any runs
\end{flushleft}


\begin{flushleft}
completed by the batsmen from that delivery, or a boundary allowance, shall be credited as Byes to the batting side.
\end{flushleft}


\begin{flushleft}
Additionally, if the delivery is a No ball, the one run penalty for such a delivery shall be incurred.
\end{flushleft}





23.2





\begin{flushleft}
Leg byes
\end{flushleft}





23.2.1





\begin{flushleft}
If a ball delivered by the bowler first strikes the person of the striker, runs shall be scored only if the umpire
\end{flushleft}


\begin{flushleft}
is satisfied that the striker has
\end{flushleft}


\begin{flushleft}
either attempted to play the ball with the bat
\end{flushleft}





32





\begin{flushleft}
\newpage
or tried to avoid being hit by the ball.
\end{flushleft}


23.2.2





\begin{flushleft}
If the umpire is satisfied that either of these conditions has been met runs shall be scored as follows.
\end{flushleft}


23.2.2.1





\begin{flushleft}
If there is
\end{flushleft}


\begin{flushleft}
either no subsequent contact with the striker's bat or person, or
\end{flushleft}


\begin{flushleft}
only inadvertent contact with the striker's bat or person
\end{flushleft}


\begin{flushleft}
any runs completed by the batsmen or a boundary allowance shall be credited to the striker in
\end{flushleft}


\begin{flushleft}
the case of subsequent contact with his bat but otherwise to the batting side as in clause 23.2.3.
\end{flushleft}





23.2.2.2


23.2.3





\begin{flushleft}
If the striker wilfully makes a lawful second strike, clause 34.3 (Ball lawfully struck more than
\end{flushleft}


\begin{flushleft}
once) and clause 34.4 (Runs permitted from ball lawfully struck more than once) shall apply.
\end{flushleft}





\begin{flushleft}
The runs in clause 23.2.2.1, unless credited to the striker, shall be scored as Leg byes.
\end{flushleft}


\begin{flushleft}
Additionally, if the delivery is a No ball, the one run penalty for the No ball shall be incurred.
\end{flushleft}





23.3





\begin{flushleft}
Leg byes not to be awarded
\end{flushleft}





\begin{flushleft}
If in the circumstance of clause 23.2.1 the umpire considers that neither of the conditions therein has been met, then
\end{flushleft}


\begin{flushleft}
Leg byes shall not be awarded.
\end{flushleft}


\begin{flushleft}
If the ball does not become dead for any other reason, the umpire shall call and signal Dead ball as soon as the ball
\end{flushleft}


\begin{flushleft}
reaches the boundary or at the completion of the first run.
\end{flushleft}


\begin{flushleft}
The umpire shall then:
\end{flushleft}


\begin{flushleft}
- disallow all runs to the batting side;
\end{flushleft}


\begin{flushleft}
- return any not out batsman to his original end;
\end{flushleft}


\begin{flushleft}
- signal No ball to the scorers if applicable;
\end{flushleft}


\begin{flushleft}
- award any 5-run Penalty that is applicable except for Penalty runs under clause 28.3 (Protective helmets belonging
\end{flushleft}


\begin{flushleft}
to the fielding side).
\end{flushleft}





\begin{flushleft}
24 FIELDER'S ABSENCE; SUBSTITUTES
\end{flushleft}


24.1





\begin{flushleft}
Substitute fielders
\end{flushleft}





24.1.1





\begin{flushleft}
The umpires shall allow a substitute fielder
\end{flushleft}


24.1.1.1





\begin{flushleft}
if they are satisfied that a fielder has been injured or become ill and that this occurred during the
\end{flushleft}


\begin{flushleft}
match, or
\end{flushleft}





24.1.1.2





\begin{flushleft}
for any other wholly acceptable reason.
\end{flushleft}





\begin{flushleft}
In all other circumstances, a substitute is not allowed.
\end{flushleft}


24.1.2





\begin{flushleft}
A substitute shall not bowl or act as captain but may act as wicket-keeper only with the consent of the
\end{flushleft}


\begin{flushleft}
umpires. Note, however, clause 42.4.1.
\end{flushleft}





24.1.3





\begin{flushleft}
A nominated player may bowl or field even though a substitute has previously acted for him, subject to
\end{flushleft}


\begin{flushleft}
clauses 24.2 and 24.3.
\end{flushleft}





24.1.4





\begin{flushleft}
Squad members of the fielding or batting team who are not playing in the match and who are not acting as
\end{flushleft}


\begin{flushleft}
substitute fielders shall be required to wear a team training bib whilst on the playing area (including the area
\end{flushleft}


\begin{flushleft}
between the boundary and the perimeter fencing).
\end{flushleft}





33





\newpage
24.2





\begin{flushleft}
Fielder absent or leaving the field of play
\end{flushleft}





24.2.1





\begin{flushleft}
A player going briefly outside the boundary while carrying out any duties as a fielder is not absent from the
\end{flushleft}


\begin{flushleft}
field of play nor, for the purposes of this clause, is he to be regarded as having left the field of play.
\end{flushleft}





24.2.2





\begin{flushleft}
If a fielder fails to take the field at the start of play or at any later time, or leaves the field during play,
\end{flushleft}





24.2.3





24.2.2.1





\begin{flushleft}
an umpire shall be informed of the reason for this absence.
\end{flushleft}





24.2.2.2





\begin{flushleft}
he shall not thereafter come on to the field of play during a session of play without the consent
\end{flushleft}


\begin{flushleft}
of the umpire. See clause 24.4. The umpire shall give such consent as soon as it is practicable.
\end{flushleft}





\begin{flushleft}
If a player is absent from the field for longer than 8 minutes, the following restrictions shall apply to their
\end{flushleft}


\begin{flushleft}
future participation in the match:
\end{flushleft}


24.2.3.1





\begin{flushleft}
The player shall not be permitted to bowl in the match until he has either been able to field, or
\end{flushleft}


\begin{flushleft}
his team has subsequently been batting, for the total length of playing time for which the player
\end{flushleft}


\begin{flushleft}
was absent (hereafter referred to as Penalty time). A player's unexpired Penalty time shall be
\end{flushleft}


\begin{flushleft}
limited to a maximum of 40 minutes. If any unexpired Penalty time remains at the end of an
\end{flushleft}


\begin{flushleft}
innings, it is carried forward to the next and subsequent innings of the match.
\end{flushleft}





24.2.3.2





\begin{flushleft}
The player shall not be permitted to bat in the match until his team's batting innings has been in
\end{flushleft}


\begin{flushleft}
progress for the length of playing time that is equal to the unexpired Penalty time carried
\end{flushleft}


\begin{flushleft}
forward from the previous innings. However, once his side has lost five wickets in its batting
\end{flushleft}


\begin{flushleft}
innings, he may bat immediately. If any unexpired penalty time remains at the end of that batting
\end{flushleft}


\begin{flushleft}
innings, it is carried forward to the next and subsequent innings of the match.
\end{flushleft}





24.2.4





\begin{flushleft}
If the player leaves the field before having served all of his Penalty time, the balance is carried forward as
\end{flushleft}


\begin{flushleft}
unserved Penalty time.
\end{flushleft}





24.2.5





\begin{flushleft}
On any occasion of absence, the amount of playing time for which the player is off the field shall be added to
\end{flushleft}


\begin{flushleft}
any Penalty time that remains unserved, subject to a maximum cumulative Penalty time of 40 minutes, and
\end{flushleft}


\begin{flushleft}
that player shall not bowl until all of his Penalty time has been served.
\end{flushleft}





24.2.6





\begin{flushleft}
For the purposes of clauses 24.2.3.1 and 24.2.3.2, playing time shall comprise the time play is in progress
\end{flushleft}


\begin{flushleft}
excluding intervals between innings. For clarity, a player's Penalty time will continue to expire after he is
\end{flushleft}


\begin{flushleft}
dismissed, for the remainder of his team's batting innings.
\end{flushleft}





24.2.7





\begin{flushleft}
If there is an unscheduled break in play, the stoppage time shall count as Penalty time served, provided that,
\end{flushleft}


24.2.7.1





\begin{flushleft}
the fielder who was on the field of play at the start of the break either takes the field on the
\end{flushleft}


\begin{flushleft}
resumption of play, or his side is now batting.
\end{flushleft}





24.2.7.2





\begin{flushleft}
the fielder who was already off the field at the start of the break notifies an umpire in person as
\end{flushleft}


\begin{flushleft}
soon as he is able to participate, and either takes the field on the resumption of play, or his side
\end{flushleft}


\begin{flushleft}
is now batting. Stoppage time before an umpire has been so notified shall not count towards
\end{flushleft}


\begin{flushleft}
unserved Penalty time.
\end{flushleft}





24.2.8





\begin{flushleft}
Any unserved Penalty time shall be carried forward into the next innings of the match, as applicable.
\end{flushleft}





24.3





\begin{flushleft}
Penalty time not incurred
\end{flushleft}





\begin{flushleft}
A nominated player's absence will not incur Penalty time if,
\end{flushleft}


24.3.1





\begin{flushleft}
he has suffered an external blow during the match and, as a result, has justifiably left the field or is unable to
\end{flushleft}


\begin{flushleft}
take the field.
\end{flushleft}





24.3.2





\begin{flushleft}
in the opinion of the umpires, the player has been absent or has left the field for other wholly acceptable
\end{flushleft}


\begin{flushleft}
reasons, which shall not include illness or internal injury.
\end{flushleft}





24.3.3





\begin{flushleft}
the player is absent from the field for a period of 8 minutes or less.
\end{flushleft}





34





\newpage
24.4





\begin{flushleft}
Player returning without permission
\end{flushleft}





\begin{flushleft}
If a player comes on to the field of play in contravention of clause 24.2.2 and comes into contact with the ball while it
\end{flushleft}


\begin{flushleft}
is in play, the ball shall immediately become dead.
\end{flushleft}


\begin{flushleft}
- The umpire shall award 5 Penalty runs to the batting side.
\end{flushleft}


\begin{flushleft}
- Runs completed by the batsmen shall be scored together with the run in progress if they had already crossed at the
\end{flushleft}


\begin{flushleft}
instant of the offence.
\end{flushleft}


\begin{flushleft}
- The ball shall not count as one of the over.
\end{flushleft}


\begin{flushleft}
- The umpire shall inform the other umpire, the captain of the fielding side, the batsmen and, as soon as practicable,
\end{flushleft}


\begin{flushleft}
the captain of the batting side of the reason for this action.
\end{flushleft}





\begin{flushleft}
25 BATSMAN'S INNINGS
\end{flushleft}


25.1





\begin{flushleft}
Eligibility to act as a batsman
\end{flushleft}





\begin{flushleft}
Only a nominated player may bat and, subject to clause 25.3, may do so even though a substitute fielder has
\end{flushleft}


\begin{flushleft}
previously acted for him.
\end{flushleft}





25.2





\begin{flushleft}
Commencement of a batsman's innings
\end{flushleft}





\begin{flushleft}
The innings of the first two batsmen, and that of any new batsman on the resumption of play after a call of Time, shall
\end{flushleft}


\begin{flushleft}
commence at the call of Play. At any other time, a batsman's innings shall be considered to have commenced when
\end{flushleft}


\begin{flushleft}
that batsman first steps onto the field of play.
\end{flushleft}





25.3





\begin{flushleft}
Restriction on batsman commencing an innings
\end{flushleft}





25.3.1





\begin{flushleft}
If a member of the batting side has unserved Penalty time, (see clause 24.2.7), that player shall not be
\end{flushleft}


\begin{flushleft}
permitted to bat until that Penalty time has been served. However, even if the unserved Penalty time has not
\end{flushleft}


\begin{flushleft}
expired, that player may bat after his side has lost 5 wickets.
\end{flushleft}





25.3.2





\begin{flushleft}
A member of the batting side's Penalty time is served during Playing time. In the event of an unscheduled
\end{flushleft}


\begin{flushleft}
stoppage, the stoppage time after the batsman notifies an umpire in person that he is able to participate
\end{flushleft}


\begin{flushleft}
shall count as Penalty time served.
\end{flushleft}





25.4





\begin{flushleft}
Batsman retiring
\end{flushleft}





25.4.1





\begin{flushleft}
A batsman may retire at any time during his innings when the ball is dead. The umpires, before allowing play
\end{flushleft}


\begin{flushleft}
to proceed, shall be informed of the reason for a batsman retiring.
\end{flushleft}





25.4.2





\begin{flushleft}
If a batsman retires because of illness, injury or any other unavoidable cause, that batsman is entitled to
\end{flushleft}


\begin{flushleft}
resume his innings. If for any reason this does not happen, that batsman is to be recorded as {`}Retired - not
\end{flushleft}


\begin{flushleft}
out'.
\end{flushleft}





25.4.3





\begin{flushleft}
If a batsman retires for any reason other than as in clause 25.4.2, the innings of that batsman may be
\end{flushleft}


\begin{flushleft}
resumed only with the consent of the opposing captain. If for any reason his innings is not resumed, that
\end{flushleft}


\begin{flushleft}
batsman is to be recorded as {`}Retired - out'.
\end{flushleft}





25.4.4





\begin{flushleft}
If after retiring a batsman resumes his innings, subject to the requirements of clauses 25.4.2 and 25.4.3, it
\end{flushleft}


\begin{flushleft}
shall be only at the fall of a wicket or the retirement of another batsman.
\end{flushleft}





25.5





\begin{flushleft}
Runners
\end{flushleft}





\begin{flushleft}
Runners shall not be permitted.
\end{flushleft}





35





\begin{flushleft}
\newpage
26 PRACTICE ON THE FIELD
\end{flushleft}


26.1





\begin{flushleft}
Practice on the pitch or the rest of the square
\end{flushleft}





26.1.1





\begin{flushleft}
There shall not be any practice on the pitch at any time.
\end{flushleft}





26.1.2





\begin{flushleft}
There shall not be any practice on the rest of the square at any time except with the approval of the umpires.
\end{flushleft}


26.1.2.1





\begin{flushleft}
If approved by the umpires, the use of the square for practice on any day of any match will be
\end{flushleft}


\begin{flushleft}
restricted to any netted practice area or bowling strips specifically prepared on the edge of the
\end{flushleft}


\begin{flushleft}
square for that purpose.
\end{flushleft}





26.1.2.2





\begin{flushleft}
Bowling practice on the bowling strips referred to above shall also be permitted during the
\end{flushleft}


\begin{flushleft}
interval (and change of innings if not the interval) unless the umpires consider that, in the
\end{flushleft}


\begin{flushleft}
prevailing conditions of ground and weather, it will be detrimental to the surface of the square.
\end{flushleft}





26.2





\begin{flushleft}
Practice on the outfield
\end{flushleft}





26.2.1





\begin{flushleft}
On any day of the match, all forms of practice are permitted on the outfield
\end{flushleft}


\begin{flushleft}
- before the start of play;
\end{flushleft}


\begin{flushleft}
- after the close of play; and
\end{flushleft}


\begin{flushleft}
- during the interval or between innings
\end{flushleft}


\begin{flushleft}
providing the umpires are satisfied that such practice will not cause significant deterioration in the condition
\end{flushleft}


\begin{flushleft}
of the outfield.
\end{flushleft}





26.2.2





\begin{flushleft}
Between the call of Play and the call of Time, practice shall be permitted on the outfield, providing that all of
\end{flushleft}


\begin{flushleft}
the following conditions are met:
\end{flushleft}


\begin{flushleft}
- only the fielders as defined in paragraph 7 of Appendix A participate in such practice.
\end{flushleft}


\begin{flushleft}
- no ball other than the match ball is used for this practice.
\end{flushleft}


\begin{flushleft}
- no bowling practice takes place in the area between the square and the boundary in a direction parallel to
\end{flushleft}


\begin{flushleft}
the match pitch.
\end{flushleft}


\begin{flushleft}
- the umpires are satisfied that it will not contravene either of clauses 41.3 (The match ball changing its
\end{flushleft}


\begin{flushleft}
condition) or 41.9 (Time wasting by the fielding side).
\end{flushleft}


\begin{flushleft}
Bowling a ball, using a short run up to a player in the outfield is not to be regarded as bowling practice but
\end{flushleft}


\begin{flushleft}
shall be subject to the other conditions in this clause.
\end{flushleft}





26.3





\begin{flushleft}
Trial run-up
\end{flushleft}





\begin{flushleft}
A bowler is permitted to have a trial run-up provided the umpire is satisfied that it will not contravene either of clauses
\end{flushleft}


\begin{flushleft}
41.9 (Time wasting by the fielding side) or 41.12 (Fielder damaging the pitch).
\end{flushleft}





26.4





\begin{flushleft}
Penalties for contravention
\end{flushleft}





\begin{flushleft}
All forms of practice are subject to the provisions of clauses 41.3 (The match ball -- changing its condition), 41.9
\end{flushleft}


\begin{flushleft}
(Time wasting by the fielding side) and 41.12 (Fielder damaging the pitch).
\end{flushleft}


26.4.1





\begin{flushleft}
If there is a contravention of any of the provisions of clause 26.1 or 26.2, the umpire shall
\end{flushleft}


\begin{flushleft}
- warn the player that the practice is not permitted;
\end{flushleft}


\begin{flushleft}
- inform the other umpire and, as soon as practicable, both captains of the reason for this action.
\end{flushleft}


26.4.1.1





\begin{flushleft}
If the contravention is by a batsman at the wicket, the umpire shall inform the other batsman
\end{flushleft}


\begin{flushleft}
and each incoming batsman that the warning has been issued. The warning shall apply to the
\end{flushleft}


\begin{flushleft}
team of that player throughout the match.
\end{flushleft}





36





\newpage
26.4.2





\begin{flushleft}
If during the match there is any further contravention by any player of that team, the umpire shall
\end{flushleft}


\begin{flushleft}
- award 5 Penalty runs to the opposing side;
\end{flushleft}


\begin{flushleft}
- inform the other umpire, the scorers and, as soon as practicable, both captains, and, if the contravention is
\end{flushleft}


\begin{flushleft}
during play, the batsmen at the wicket.
\end{flushleft}





\begin{flushleft}
27 THE WICKET-KEEPER
\end{flushleft}


27.1





\begin{flushleft}
Protective equipment
\end{flushleft}





\begin{flushleft}
The wicket-keeper is the only fielder permitted to wear gloves and external leg guards. If these are worn, they are to
\end{flushleft}


\begin{flushleft}
be regarded as part of his person for the purposes of clause 28.2 (Fielding the ball). If by the wicket-keeper's actions
\end{flushleft}


\begin{flushleft}
and positioning when the ball comes into play it is apparent to the umpires that he will not be able to carry out the
\end{flushleft}


\begin{flushleft}
normal duties of a wicket-keeper, he shall forfeit this right and also the right to be recognised as a wicket-keeper for
\end{flushleft}


\begin{flushleft}
the purposes of clauses 33.2 (A fair catch), 39 (Stumped), 28.1 (Protective equipment), 28.4 (Limitation of on-side
\end{flushleft}


\begin{flushleft}
fielders) and 28.5 (Fielders not to encroach on pitch).
\end{flushleft}





27.2





\begin{flushleft}
Gloves
\end{flushleft}





27.2.1





\begin{flushleft}
If, as permitted under clause 27.1, the wicket-keeper wears gloves, they shall have no webbing between the
\end{flushleft}


\begin{flushleft}
fingers except joining index finger and thumb, where webbing may be inserted as a means of support.
\end{flushleft}





27.2.2





\begin{flushleft}
If used, the webbing shall be a single piece of non-stretch material which, although it may have facing
\end{flushleft}


\begin{flushleft}
material attached, shall have no reinforcements or tucks.
\end{flushleft}





27.2.3





\begin{flushleft}
The top edge of the webbing shall not protrude beyond the straight line joining the top of the index finger to
\end{flushleft}


\begin{flushleft}
the top of the thumb and shall be taut when a hand wearing the glove has the thumb fully extended. See
\end{flushleft}


\begin{flushleft}
paragraph 3 of Appendix B.
\end{flushleft}





27.3





\begin{flushleft}
Position of wicket-keeper
\end{flushleft}





27.3.1





\begin{flushleft}
The wicket-keeper shall remain wholly behind the wicket at the striker's end from the moment the ball comes
\end{flushleft}


\begin{flushleft}
into play until a ball delivered by the bowler
\end{flushleft}


\begin{flushleft}
touches the bat or person of the striker; or
\end{flushleft}


\begin{flushleft}
passes the wicket at the striker's end; or
\end{flushleft}


\begin{flushleft}
the striker attempts a run.
\end{flushleft}





27.3.2





\begin{flushleft}
In the event of the wicket-keeper contravening this clause, the striker's end umpire shall call and signal No
\end{flushleft}


\begin{flushleft}
ball as soon as applicable after the delivery of the ball.
\end{flushleft}





27.4





\begin{flushleft}
Movement by wicket-keeper
\end{flushleft}





27.4.1





\begin{flushleft}
After the ball comes into play and before it reaches the striker, it is unfair if the wicket-keeper significantly
\end{flushleft}


\begin{flushleft}
alters his position in relation to the striker's wicket, except for the following:
\end{flushleft}


27.4.1.1





\begin{flushleft}
movement of a few paces forward for a slower delivery, unless in so doing it brings him within
\end{flushleft}


\begin{flushleft}
reach of the wicket.
\end{flushleft}





27.4.1.2





\begin{flushleft}
lateral movement in response to the direction in which the ball has been delivered.
\end{flushleft}





27.4.1.3





\begin{flushleft}
movement in response to the stroke that the striker is playing or that his actions suggest he
\end{flushleft}


\begin{flushleft}
intends to play. However the provisions of clause 27.3 shall apply.
\end{flushleft}





27.4.2





\begin{flushleft}
In the event of unfair movement by the wicket-keeper, either umpire shall call and signal Dead ball.
\end{flushleft}





27.5





\begin{flushleft}
Restriction on actions of wicket-keeper
\end{flushleft}





\begin{flushleft}
If, in the opinion of either umpire, the wicket-keeper interferes with the striker's right to play the ball and to guard his
\end{flushleft}


\begin{flushleft}
wicket, clause 20.4.2.6 (Umpire calling and signalling Dead ball) shall apply.
\end{flushleft}





37





\begin{flushleft}
\newpage
If, however, either umpire considers that the interference by the wicket-keeper was wilful, then clause 41.1
\end{flushleft}


\begin{flushleft}
(Deliberate attempt to distract striker) shall also apply.
\end{flushleft}





27.6





\begin{flushleft}
Interference with wicket-keeper by striker
\end{flushleft}





\begin{flushleft}
If, in playing at the ball or in the legitimate defence of his wicket, the striker interferes with the wicket-keeper, he shall
\end{flushleft}


\begin{flushleft}
not be out except as provided for in clause 37.3 (Obstructing a ball from being caught).
\end{flushleft}





\begin{flushleft}
28 THE FIELDER
\end{flushleft}


28.1





\begin{flushleft}
Protective equipment
\end{flushleft}





\begin{flushleft}
No fielder other than the wicket-keeper shall be permitted to wear gloves or external leg guards. In addition,
\end{flushleft}


\begin{flushleft}
protection for the hand or fingers may be worn only with the consent of the umpires.
\end{flushleft}





28.2





\begin{flushleft}
Fielding the ball
\end{flushleft}





28.2.1





\begin{flushleft}
A fielder may field the ball with any part of his person (see paragraph 12 of Appendix A), except as in clause
\end{flushleft}


\begin{flushleft}
28.2.1.2. However, he will be deemed to have fielded the ball illegally if, while the ball is in play he wilfully
\end{flushleft}


28.2.1.1





\begin{flushleft}
uses anything other than part of his person to field the ball.
\end{flushleft}





28.2.1.2





\begin{flushleft}
extends his clothing with his hands and uses this to field the ball.
\end{flushleft}





28.2.1.3





\begin{flushleft}
discards a piece of clothing, equipment or any other object which subsequently makes contact
\end{flushleft}


\begin{flushleft}
with the ball.
\end{flushleft}





28.2.2





\begin{flushleft}
It is not illegal fielding if the ball in play makes contact with a piece of clothing, equipment or any other object
\end{flushleft}


\begin{flushleft}
which has accidentally fallen from the fielder's person.
\end{flushleft}





28.2.3





\begin{flushleft}
If a fielder illegally fields the ball, the ball shall immediately become dead and
\end{flushleft}


\begin{flushleft}
- the penalty for a No ball or a Wide shall stand.
\end{flushleft}


\begin{flushleft}
- any runs completed by the batsmen shall be credited to the batting side, together with the run in progress if
\end{flushleft}


\begin{flushleft}
the batsmen had already crossed at the instant of the offence.
\end{flushleft}


\begin{flushleft}
- the ball shall not count as one of the over.
\end{flushleft}


\begin{flushleft}
In addition the umpire shall:
\end{flushleft}


\begin{flushleft}
- award 5 Penalty runs to the batting side.
\end{flushleft}


\begin{flushleft}
- inform the other umpire and the captain of the fielding side of the reason for this action.
\end{flushleft}


\begin{flushleft}
- inform the batsmen and, as soon as practicable, the captain of the batting side of what has occurred.
\end{flushleft}





28.3





\begin{flushleft}
Protective helmets belonging to the fielding side
\end{flushleft}





28.3.1





\begin{flushleft}
Protective helmets, when not in use by fielders, may not be placed on the ground, above the surface except
\end{flushleft}


\begin{flushleft}
behind the wicket-keeper and in line with both sets of stumps.
\end{flushleft}





28.3.2





\begin{flushleft}
If the ball while in play strikes a helmet, placed as described in clause 28.3.1,
\end{flushleft}


28.3.2.1





\begin{flushleft}
the ball shall become dead
\end{flushleft}


\begin{flushleft}
and, subject to clause 28.3.3,
\end{flushleft}





28.3.2.2





\begin{flushleft}
an award of 5 Penalty runs shall be made to the batting side;
\end{flushleft}





28.3.2.3





\begin{flushleft}
any runs completed by the batsmen before the ball strikes the protective helmet shall be scored,
\end{flushleft}


\begin{flushleft}
together with the run in progress if the batsmen had already crossed at the instant of the ball
\end{flushleft}


\begin{flushleft}
striking the protective helmet.
\end{flushleft}





38





\newpage
28.3.3





\begin{flushleft}
If the ball while in play strikes a helmet, placed as described in clause 28.3.1, unless the circumstances of
\end{flushleft}


\begin{flushleft}
clause 23.3 (Leg byes not to be awarded) or clause 34 (Hit the ball twice), apply, the umpire shall:
\end{flushleft}


\begin{flushleft}
- permit the batsmen's runs as in clause 28.3.2.3 to be scored
\end{flushleft}


\begin{flushleft}
- signal No ball or Wide ball to the scorers if applicable
\end{flushleft}


\begin{flushleft}
- award 5 Penalty runs as in clause 28.3.2.2
\end{flushleft}


\begin{flushleft}
- award any other Penalty runs due to the batting side.
\end{flushleft}





28.3.4





\begin{flushleft}
If the ball while in play strikes a helmet, placed as described in clause 28.3.1, and the circumstances of
\end{flushleft}


\begin{flushleft}
clause 23.3 (Leg byes not to be awarded) or clause 34 (Hit the ball twice) apply, the umpire shall:
\end{flushleft}


\begin{flushleft}
- disallow all runs to the batting side
\end{flushleft}


\begin{flushleft}
- return any not out batsman to his original end
\end{flushleft}


\begin{flushleft}
- signal No ball or Wide ball to the scorers if applicable
\end{flushleft}


\begin{flushleft}
- award any 5-run Penalty that is applicable except for Penalty runs under clause 28.3.2.
\end{flushleft}





28.4





\begin{flushleft}
Limitation of on side fielders
\end{flushleft}





28.4.1





\begin{flushleft}
At the instant of delivery, there may not be more than 5 fielders on the leg side.
\end{flushleft}





28.4.2





\begin{flushleft}
At the instant of the bowler's delivery there shall not be more than two fielders, other than the wicket-keeper,
\end{flushleft}


\begin{flushleft}
behind the popping crease on the on side. A fielder will be considered to be behind the popping crease
\end{flushleft}


\begin{flushleft}
unless the whole of his person whether grounded or in the air is in front of this line.
\end{flushleft}





28.4.3





\begin{flushleft}
In the event of infringement of this clause by any fielder, the striker's end umpire shall call and signal No
\end{flushleft}


\begin{flushleft}
ball.
\end{flushleft}





28.5





\begin{flushleft}
Fielders not to encroach on pitch
\end{flushleft}





\begin{flushleft}
While the ball is in play and until the ball has made contact with the striker's bat or person, or has passed the striker's
\end{flushleft}


\begin{flushleft}
bat, no fielder, other than the bowler, may have any part of his person grounded on or extended over the pitch.
\end{flushleft}


\begin{flushleft}
In the event of infringement of this clause by any fielder other than the wicket-keeper, the bowler's end umpire shall
\end{flushleft}


\begin{flushleft}
call and signal No ball as soon as possible after delivery of the ball. Note, however, clause 27.3 (Position of wicketkeeper).
\end{flushleft}





28.6





\begin{flushleft}
Movement by any fielder other than the wicket-keeper
\end{flushleft}





28.6.1





\begin{flushleft}
Any movement by any fielder, excluding the wicket-keeper, after the ball comes into play and before the ball
\end{flushleft}


\begin{flushleft}
reaches the striker, is unfair except for the following:
\end{flushleft}


28.6.1.1





\begin{flushleft}
minor adjustments to stance or position in relation to the striker's wicket.
\end{flushleft}





28.6.1.2





\begin{flushleft}
movement by any fielder, other than a close fielder, towards the striker or the striker's wicket
\end{flushleft}


\begin{flushleft}
that does not significantly alter the position of the fielder.
\end{flushleft}





28.6.1.3





\begin{flushleft}
movement by any fielder in response to the stroke that the striker is playing or that his actions
\end{flushleft}


\begin{flushleft}
suggest he intends to play.
\end{flushleft}





28.6.2





\begin{flushleft}
In all circumstances clause 28.4 (Limitation of on side fielders) shall apply.
\end{flushleft}





28.6.3





\begin{flushleft}
In the event of such unfair movement, either umpire shall call and signal Dead ball.
\end{flushleft}





28.6.4





\begin{flushleft}
Note also the provisions of clause 41.4 (Deliberate attempt to distract striker). See also clause 27.4
\end{flushleft}


\begin{flushleft}
(Movement by wicket-keeper).
\end{flushleft}





39





\newpage
28.7





\begin{flushleft}
Restrictions on the placement of fielders
\end{flushleft}





28.7.1





\begin{flushleft}
In addition to the restrictions contained in clause 28.4 above, further fielding restrictions shall apply to certain
\end{flushleft}


\begin{flushleft}
overs in each innings. The nature of such fielding restrictions and the overs during which they shall apply are
\end{flushleft}


\begin{flushleft}
set out in the following paragraphs.
\end{flushleft}





28.7.2





\begin{flushleft}
Subject to 28.7.6 below these additional fielding restrictions shall apply to the first 6 overs of each innings
\end{flushleft}


\begin{flushleft}
(Powerplay overs).
\end{flushleft}





28.7.3





\begin{flushleft}
Two semi-circles shall be drawn on the field of play. The semi-circles shall have as their centre the middle
\end{flushleft}


\begin{flushleft}
stump at either end of the pitch. The radius of each of the semi-circles shall be 30 yards (27.43 metres). The
\end{flushleft}


\begin{flushleft}
semi-circles shall be linked by two parallel straight lines drawn on the field (see paragraph 2 of Appendix C).
\end{flushleft}


\begin{flushleft}
These fielding restriction areas should be marked by continuous painted white lines or {`}dots' at 5 yard (4.57
\end{flushleft}


\begin{flushleft}
metres) intervals, each {`}dot' to be covered by a white plastic or rubber (but not metal) disc measuring 7
\end{flushleft}


\begin{flushleft}
inches (18 cm) in diameter.
\end{flushleft}





28.7.4





\begin{flushleft}
During the Powerplay overs only two fielders shall be permitted outside this fielding restriction area at the
\end{flushleft}


\begin{flushleft}
instant of delivery.
\end{flushleft}





28.7.5





\begin{flushleft}
During the non Powerplay overs, no more than 5 fielders shall be permitted outside the fielding restriction
\end{flushleft}


\begin{flushleft}
area referred to in clause 28.7.3 above.
\end{flushleft}





28.7.6





\begin{flushleft}
In circumstances when the number of overs of the batting team is reduced, the number of Powerplay overs
\end{flushleft}


\begin{flushleft}
shall be reduced in accordance with the table below. For the sake of clarity, it should be noted that the table
\end{flushleft}


\begin{flushleft}
shall apply to both the 1st and 2nd innings of the match.
\end{flushleft}


\begin{flushleft}
Total overs in innings
\end{flushleft}


5-8


9-11


12-14


15-18


19-20





\begin{flushleft}
Number of overs for which fielding restrictions in clauses 28.7.2 and
\end{flushleft}


\begin{flushleft}
28.7.4 above will apply
\end{flushleft}


2


3


4


5


6





28.7.7





\begin{flushleft}
If an innings is interrupted during an over and if on the resumption of play, due to the reduced number of
\end{flushleft}


\begin{flushleft}
overs of the batting team, the required number of Powerplay overs have already been bowled, the remaining
\end{flushleft}


\begin{flushleft}
deliveries in the over to be completed shall not be subject to the fielding restrictions.
\end{flushleft}





28.7.8





\begin{flushleft}
In the event of an infringement of any of the above fielding restrictions, the square leg umpire shall call and
\end{flushleft}


\begin{flushleft}
signal No ball.
\end{flushleft}





\begin{flushleft}
29 THE WICKET IS DOWN
\end{flushleft}


29.1





\begin{flushleft}
Wicket put down
\end{flushleft}





29.1.1





\begin{flushleft}
The wicket is put down if a bail is completely removed from the top of the stumps, or a stump is struck out of
\end{flushleft}


\begin{flushleft}
the ground,
\end{flushleft}


29.1.1.1





\begin{flushleft}
by the ball,
\end{flushleft}





29.1.1.2





\begin{flushleft}
by the striker's bat if held or by any part of the bat that he is holding,
\end{flushleft}





29.1.1.3





\begin{flushleft}
for the purpose of this clause only, by the striker's bat not in hand, or by any part of the bat
\end{flushleft}


\begin{flushleft}
which has become detached,
\end{flushleft}





29.1.1.4





\begin{flushleft}
by the striker's person or by any part of his clothing or equipment becoming detached from his
\end{flushleft}


\begin{flushleft}
person,
\end{flushleft}





29.1.1.5





\begin{flushleft}
by a fielder with his hand or arm, providing that the ball is held in the hand or hands so used, or
\end{flushleft}


\begin{flushleft}
in the hand of the arm so used.
\end{flushleft}





40





\newpage
29.1.1.6





\begin{flushleft}
The wicket is also put down if a fielder strikes or pulls a stump out of the ground in the same
\end{flushleft}


\begin{flushleft}
manner.
\end{flushleft}





29.1.2





\begin{flushleft}
The disturbance of a bail, whether temporary or not, shall not constitute its complete removal from the top of
\end{flushleft}


\begin{flushleft}
the stumps, but if a bail in falling lodges between two of the stumps this shall be regarded as complete
\end{flushleft}


\begin{flushleft}
removal.
\end{flushleft}





29.2





\begin{flushleft}
One bail off
\end{flushleft}





\begin{flushleft}
If one bail is off, it shall be sufficient for the purpose of putting the wicket down to remove the remaining bail or to
\end{flushleft}


\begin{flushleft}
strike or pull any of the three stumps out of the ground, in any of the ways stated in clause 29.1.
\end{flushleft}





29.3





\begin{flushleft}
Remaking wicket
\end{flushleft}





\begin{flushleft}
If a wicket is broken or put down while the ball is in play, it shall not be remade by an umpire until the ball is dead.
\end{flushleft}


\begin{flushleft}
See clause 20 (Dead ball). Any fielder may, however, while the ball is in play,
\end{flushleft}


\begin{flushleft}
- replace a bail or bails on top of the stumps.
\end{flushleft}


\begin{flushleft}
- put back one or more stumps into the ground where the wicket originally stood.
\end{flushleft}





29.4





\begin{flushleft}
Dispensing with bails
\end{flushleft}





\begin{flushleft}
If the umpires have agreed to dispense with bails in accordance with clause 8.5 (Dispensing with bails), it is for the
\end{flushleft}


\begin{flushleft}
umpire concerned to decide whether or not the wicket has been put down.
\end{flushleft}


29.4.1





\begin{flushleft}
After a decision to play without bails, the wicket has been put down if the umpire concerned is satisfied that
\end{flushleft}


\begin{flushleft}
the wicket has been struck by the ball, by the striker's bat, person or items of his clothing or equipment as
\end{flushleft}


\begin{flushleft}
described in clauses 29.1.1.2, 29.1.1.3 or 29.1.1.4, or by a fielder in the manner described in clause
\end{flushleft}


29.1.1.5.





29.4.2





\begin{flushleft}
If the wicket has already been broken or put down, clause 29.4.1 shall apply to any stump or stumps still in
\end{flushleft}


\begin{flushleft}
the ground. Any fielder may replace a stump or stumps, in accordance with clause 29.3, in order to have an
\end{flushleft}


\begin{flushleft}
opportunity of putting the wicket down.
\end{flushleft}





\begin{flushleft}
30 BATSMAN OUT OF HIS GROUND
\end{flushleft}


30.1





\begin{flushleft}
When out of his ground
\end{flushleft}





30.1.1





\begin{flushleft}
A batsman shall be considered to be out of his ground unless some part of his person or bat is grounded
\end{flushleft}


\begin{flushleft}
behind the popping crease at that end.
\end{flushleft}





30.1.2





\begin{flushleft}
However, a batsman shall not be considered to be out of his ground if, in running or diving towards his
\end{flushleft}


\begin{flushleft}
ground and beyond, and having grounded some part of his person or bat beyond the popping crease, there
\end{flushleft}


\begin{flushleft}
is subsequent loss of contact
\end{flushleft}


\begin{flushleft}
between the ground and any part of his person or bat, or
\end{flushleft}


\begin{flushleft}
between the bat and person,
\end{flushleft}


\begin{flushleft}
provided that the batsman has continued movement in the same direction.
\end{flushleft}





30.2





\begin{flushleft}
Which is a batsman's ground
\end{flushleft}





30.2.1





\begin{flushleft}
If only one batsman is within a ground, it is his ground and will remain so even if he is later joined there by
\end{flushleft}


\begin{flushleft}
the other batsman.
\end{flushleft}





30.2.2





\begin{flushleft}
If both batsmen are in the same ground and one of them subsequently leaves it, the ground belongs to the
\end{flushleft}


\begin{flushleft}
batsman who remains in it.
\end{flushleft}





30.2.3





\begin{flushleft}
If there is no batsman in either ground, then each ground belongs to whichever batsman is nearer to it, or, if
\end{flushleft}


\begin{flushleft}
the batsmen are level, to whichever batsman was nearer to it immediately prior to their drawing level.
\end{flushleft}





41





\newpage
30.2.4





\begin{flushleft}
If a ground belongs to one batsman then the other ground belongs to the other batsman, irrespective of his
\end{flushleft}


\begin{flushleft}
position.
\end{flushleft}





30.3





\begin{flushleft}
Position of non-striker
\end{flushleft}





\begin{flushleft}
The non-striker, when standing at the bowler's end, should be positioned on the opposite side of the wicket to that
\end{flushleft}


\begin{flushleft}
from which the ball is being delivered, unless a request to do otherwise is granted by the umpire.
\end{flushleft}





\begin{flushleft}
31 APPEALS
\end{flushleft}


31.1





\begin{flushleft}
Umpire not to give batsman out without an appeal
\end{flushleft}





\begin{flushleft}
Neither umpire shall give a batsman out, even though he may be out under these Playing Conditions, unless
\end{flushleft}


\begin{flushleft}
appealed to by a fielder. This shall not debar a batsman who is out under these Playing Conditions from leaving the
\end{flushleft}


\begin{flushleft}
wicket without an appeal having been made. Note, however, the provisions of clause 31.7.
\end{flushleft}





31.2





\begin{flushleft}
Batsman dismissed
\end{flushleft}





\begin{flushleft}
A batsman is dismissed if he is
\end{flushleft}


\begin{flushleft}
either given out by an umpire, on appeal
\end{flushleft}


\begin{flushleft}
or out under these Playing Conditions and leaves the wicket as in clause 31.1.
\end{flushleft}





31.3





\begin{flushleft}
Timing of appeals
\end{flushleft}





\begin{flushleft}
For an appeal to be valid, it must be made before the bowler begins his run-up or, if there is no run-up, his bowling
\end{flushleft}


\begin{flushleft}
action to deliver the next ball, and before Time has been called.
\end{flushleft}


\begin{flushleft}
The call of Over does not invalidate an appeal made prior to the start of the following over, provided Time has not
\end{flushleft}


\begin{flushleft}
been called. See clauses 12.2 (Call of Time) and 17.2 (Start of an over).
\end{flushleft}





31.4





\begin{flushleft}
Appeal {``}How's That?''
\end{flushleft}





\begin{flushleft}
An appeal {``}How's That?'' covers all ways of being out.
\end{flushleft}





31.5





\begin{flushleft}
Answering appeals
\end{flushleft}





\begin{flushleft}
The striker's end umpire shall answer all appeals arising out of any of clauses 35 (Hit wicket), 39 (Stumped) or 38
\end{flushleft}


\begin{flushleft}
(Run out) when this occurs at the wicket-keeper's end. The bowler's end umpire shall answer all other appeals.
\end{flushleft}


\begin{flushleft}
When an appeal is made, each umpire shall answer on any matter that falls within his jurisdiction.
\end{flushleft}


\begin{flushleft}
When a batsman has been given Not out, either umpire may answer an appeal, made in accordance with clause
\end{flushleft}


\begin{flushleft}
31.3, if it is on a further matter and is within his jurisdiction.
\end{flushleft}





31.6





\begin{flushleft}
Consultation by umpires
\end{flushleft}





\begin{flushleft}
Each umpire shall answer appeals on matters within his own jurisdiction. If an umpire is doubtful about any point that
\end{flushleft}


\begin{flushleft}
the other umpire may have been in a better position to see, he/she shall consult the latter on this point of fact and
\end{flushleft}


\begin{flushleft}
shall then give the decision. If, after consultation, there is still doubt remaining, the decision shall be Not out.
\end{flushleft}





31.7





\begin{flushleft}
Batsman leaving the wicket under a misapprehension
\end{flushleft}





\begin{flushleft}
An umpire shall intervene if satisfied that a batsman, not having been given out, has left the wicket under a
\end{flushleft}


\begin{flushleft}
misapprehension of being out. The umpire intervening shall call and signal Dead ball to prevent any further action by
\end{flushleft}


\begin{flushleft}
the fielding side and shall recall the batsman.
\end{flushleft}


\begin{flushleft}
A batsman may be recalled at any time up to the instant when the ball comes into play for the next delivery, unless it
\end{flushleft}


\begin{flushleft}
is the final wicket of the innings, in which case it should be up to the instant when the umpires leave the field.
\end{flushleft}





42





\newpage
31.8





\begin{flushleft}
Withdrawal of an appeal
\end{flushleft}





\begin{flushleft}
The captain of the fielding side may withdraw an appeal only after obtaining the consent of the umpire within whose
\end{flushleft}


\begin{flushleft}
jurisdiction the appeal falls. If such consent is given, the umpire concerned shall, if applicable, revoke the decision
\end{flushleft}


\begin{flushleft}
and recall the batsman.
\end{flushleft}


\begin{flushleft}
The withdrawal of an appeal must be before the instant when the ball comes into play for the next delivery or, if the
\end{flushleft}


\begin{flushleft}
innings has been completed, the instant when the umpires leave the field.
\end{flushleft}





\begin{flushleft}
32 BOWLED
\end{flushleft}


32.1





\begin{flushleft}
Out Bowled
\end{flushleft}





32.1.1





\begin{flushleft}
The striker is out Bowled if his wicket is put down by a ball delivered by the bowler, not being a No ball, even
\end{flushleft}


\begin{flushleft}
if it first touches the striker's bat or person.
\end{flushleft}





32.1.2





\begin{flushleft}
However, the striker shall not be out Bowled if before striking the wicket the ball has been in contact with any
\end{flushleft}


\begin{flushleft}
other player or an umpire. The striker will, however, be subject to clauses 37 (Obstructing the field), 38 (Run
\end{flushleft}


\begin{flushleft}
out) and 39 (Stumped).
\end{flushleft}





32.2





\begin{flushleft}
Bowled to take precedence
\end{flushleft}





\begin{flushleft}
The striker is out Bowled if his wicket is put down as in clause 32.1, even though a decision against him for any other
\end{flushleft}


\begin{flushleft}
method of dismissal would be justified.
\end{flushleft}





\begin{flushleft}
33 CAUGHT
\end{flushleft}


33.1





\begin{flushleft}
Out Caught
\end{flushleft}





\begin{flushleft}
The striker is out Caught if a ball delivered by the bowler, not being a No ball, touches his bat without having
\end{flushleft}


\begin{flushleft}
previously been in contact with any fielder, and is subsequently held by a fielder as a fair catch, as described in
\end{flushleft}


\begin{flushleft}
clause 33.2 and 33.3, before it touches the ground.
\end{flushleft}





33.2





\begin{flushleft}
A fair catch
\end{flushleft}





33.2.1





\begin{flushleft}
A catch will be fair only if, in every case
\end{flushleft}


\begin{flushleft}
either the ball, at any time
\end{flushleft}


\begin{flushleft}
or any fielder in contact with the ball,
\end{flushleft}


\begin{flushleft}
is not grounded beyond the boundary before the catch is completed. Note clauses 19.4 (Ball grounded
\end{flushleft}


\begin{flushleft}
beyond the boundary) and 19.5 (Fielder grounded beyond the boundary).
\end{flushleft}





33.2.2





\begin{flushleft}
Furthermore, a catch will be fair if any of the following conditions applies:
\end{flushleft}


33.2.2.1





\begin{flushleft}
the ball is held in the hand or hands of a fielder, even if the hand holding the ball is touching the
\end{flushleft}


\begin{flushleft}
ground, or is hugged to the body, or lodges in the external protective equipment worn by a
\end{flushleft}


\begin{flushleft}
fielder, or lodges accidentally in a fielder's clothing.
\end{flushleft}





33.2.2.2





\begin{flushleft}
a fielder catches the ball after it has been lawfully struck more than once by the striker, but only
\end{flushleft}


\begin{flushleft}
if it has not been grounded since it was first struck. See clause 34 (Hit the ball twice).
\end{flushleft}





33.2.2.3





\begin{flushleft}
a fielder catches the ball after it has touched the wicket, an umpire, another fielder or the other
\end{flushleft}


\begin{flushleft}
batsman.
\end{flushleft}





33.2.2.4





\begin{flushleft}
a fielder catches the ball after it has crossed the boundary in the air, provided that the conditions
\end{flushleft}


\begin{flushleft}
in clause 33.2.1 are met.
\end{flushleft}





33.2.2.5





\begin{flushleft}
the ball is caught off an obstruction within the boundary that is not designated a boundary by the
\end{flushleft}


\begin{flushleft}
umpires.
\end{flushleft}





43





\newpage
33.3





\begin{flushleft}
Making a catch
\end{flushleft}





\begin{flushleft}
The act of making a catch shall start from the time when the ball first comes into contact with a fielder's person and
\end{flushleft}


\begin{flushleft}
shall end when a fielder obtains complete control over both the ball and his own movement.
\end{flushleft}





33.4





\begin{flushleft}
No runs to be scored
\end{flushleft}





\begin{flushleft}
If the striker is dismissed Caught, runs from that delivery completed by the batsmen before the completion of the
\end{flushleft}


\begin{flushleft}
catch shall not be scored but any runs for penalties awarded to either side shall stand. Clause 18.11.1 (Batsman
\end{flushleft}


\begin{flushleft}
returning to original end) shall apply from the instant of the completion of the catch.
\end{flushleft}





33.5





\begin{flushleft}
Caught to take precedence
\end{flushleft}





\begin{flushleft}
If the criteria of clause 33.1 are met and the striker is not out Bowled, then he is out Caught, even though a decision
\end{flushleft}


\begin{flushleft}
against either batsman for another method of dismissal would be justified.
\end{flushleft}





\begin{flushleft}
34 HIT THE BALL TWICE
\end{flushleft}


34.1





\begin{flushleft}
Out Hit the ball twice
\end{flushleft}





34.1.1





\begin{flushleft}
The striker is out Hit the ball twice if, while the ball is in play, it strikes any part of his person or is struck by
\end{flushleft}


\begin{flushleft}
his bat and, before the ball has been touched by a fielder, the striker wilfully strikes it again with his bat or
\end{flushleft}


\begin{flushleft}
person, other than a hand not holding the bat, except for the sole purpose of guarding his wicket. See clause
\end{flushleft}


\begin{flushleft}
34.3 and clause 37 (Obstructing the field).
\end{flushleft}





34.1.2





\begin{flushleft}
For the purpose of this clause {`}struck' or {`}strike' shall include contact with the person of the striker.
\end{flushleft}





34.2





\begin{flushleft}
Not out Hit the ball twice
\end{flushleft}





\begin{flushleft}
The striker will not be out under this clause if he
\end{flushleft}


34.2.1





\begin{flushleft}
strikes the ball a second or subsequent time in order to return the ball to any fielder.
\end{flushleft}


\begin{flushleft}
Note, however, the provisions of clause 37.4 (Returning the ball to a fielder).
\end{flushleft}





34.2.2





\begin{flushleft}
wilfully strikes the ball after it has touched a fielder. Note, however the provisions of clause 37.1 (Out
\end{flushleft}


\begin{flushleft}
Obstructing the field).
\end{flushleft}





34.3





\begin{flushleft}
Ball lawfully struck more than once
\end{flushleft}





\begin{flushleft}
The striker may, solely in order to guard his wicket and before the ball has been touched by a fielder, lawfully strike
\end{flushleft}


\begin{flushleft}
the ball a second or subsequent time with the bat, or with any part of his person other than a hand not holding the
\end{flushleft}


\begin{flushleft}
bat.
\end{flushleft}


\begin{flushleft}
However, the striker may not prevent the ball from being caught by striking the ball more than once in defence of his
\end{flushleft}


\begin{flushleft}
wicket. See clause 37.3 (Obstructing a ball from being caught).
\end{flushleft}





34.4





\begin{flushleft}
Runs permitted from ball lawfully struck more than once
\end{flushleft}





\begin{flushleft}
When the ball is lawfully struck more than once, as permitted in clause 34.3, if the ball does not become dead for any
\end{flushleft}


\begin{flushleft}
reason, the umpire shall call and signal Dead ball as soon as the ball reaches the boundary or at the completion of
\end{flushleft}


\begin{flushleft}
the first run. However, the umpire shall delay the call of Dead ball to allow the opportunity for a catch to be
\end{flushleft}


\begin{flushleft}
completed.
\end{flushleft}


\begin{flushleft}
The umpire shall
\end{flushleft}


\begin{flushleft}
- disallow all runs to the batting side
\end{flushleft}


\begin{flushleft}
- return any not out batsman to his original end
\end{flushleft}


\begin{flushleft}
- signal No ball to the scorers if applicable; and
\end{flushleft}





44





\begin{flushleft}
\newpage
- award any 5-run Penalty that is applicable except for Penalty runs under clause 28.3 (Protective helmets belonging
\end{flushleft}


\begin{flushleft}
to the fielding side).
\end{flushleft}





34.5





\begin{flushleft}
Bowler does not get credit
\end{flushleft}





\begin{flushleft}
The bowler does not get credit for the wicket.
\end{flushleft}





\begin{flushleft}
35 HIT WICKET
\end{flushleft}


35.1





\begin{flushleft}
Out Hit wicket
\end{flushleft}





35.1.1





\begin{flushleft}
The striker is out Hit wicket if, after the bowler has entered the delivery stride and while the ball is in play, his
\end{flushleft}


\begin{flushleft}
wicket is put down by either the striker's bat or person as described in clauses 29.1.1.2 to 29.1.1.4 (Wicket
\end{flushleft}


\begin{flushleft}
put down) in any of the following circumstances:
\end{flushleft}


35.1.1.1





\begin{flushleft}
in the course of any action taken by him in preparing to receive or in receiving a delivery,
\end{flushleft}





35.1.1.2





\begin{flushleft}
in setting off for the first run immediately after playing or playing at the ball,
\end{flushleft}





35.1.1.3





\begin{flushleft}
if no attempt is made to play the ball, in setting off for the first run, providing that in the opinion
\end{flushleft}


\begin{flushleft}
of the umpire this is immediately after the striker has had the opportunity of playing the ball,
\end{flushleft}





35.1.1.4





\begin{flushleft}
in lawfully making a second or further stroke for the purpose of guarding his wicket within the
\end{flushleft}


\begin{flushleft}
provisions of clause 34.3 (Ball lawfully struck more than once).
\end{flushleft}





35.1.2





\begin{flushleft}
If the striker puts his wicket down in any of the ways described in clauses 29.1.1.2 to 29.1.1.4 before the
\end{flushleft}


\begin{flushleft}
bowler has entered the delivery stride, either umpire shall call and signal Dead ball.
\end{flushleft}





35.2





\begin{flushleft}
Not out Hit wicket
\end{flushleft}





\begin{flushleft}
The striker is not out under this clause should his wicket be put down in any of the ways referred to in clause 35.1 if
\end{flushleft}


\begin{flushleft}
any of the following applies:
\end{flushleft}


\begin{flushleft}
- it occurs after the striker has completed any action in receiving the delivery, other than in clauses 35.1.1.2 to
\end{flushleft}


35.1.1.4.


\begin{flushleft}
- it occurs when the striker is in the act of running, other than setting off immediately for the first run.
\end{flushleft}


\begin{flushleft}
- it occurs when the striker is trying to avoid being run out or stumped.
\end{flushleft}


\begin{flushleft}
- it occurs when the striker is trying to avoid a throw in at any time.
\end{flushleft}


\begin{flushleft}
- the bowler after entering the delivery stride does not deliver the ball. In this case either umpire shall immediately call
\end{flushleft}


\begin{flushleft}
and signal Dead ball. See clause 20.4 (Umpire calling and signalling Dead ball).
\end{flushleft}


\begin{flushleft}
- the delivery is a No ball.
\end{flushleft}





\begin{flushleft}
36 LEG BEFORE WICKET
\end{flushleft}


36.1





\begin{flushleft}
Out LBW
\end{flushleft}





\begin{flushleft}
The striker is out LBW if all the circumstances set out in clauses 36.1.1 to 36.1.5 apply.
\end{flushleft}


36.1.1





\begin{flushleft}
The bowler delivers a ball, not being a No ball
\end{flushleft}





36.1.2





\begin{flushleft}
the ball, if it is not intercepted full-pitch, pitches in line between wicket and wicket or on the off side of the
\end{flushleft}


\begin{flushleft}
striker's wicket
\end{flushleft}





36.1.3





\begin{flushleft}
the ball not having previously touched his bat, the striker intercepts the ball, either full-pitch or after pitching,
\end{flushleft}


\begin{flushleft}
with any part of his person
\end{flushleft}





36.1.4





\begin{flushleft}
the point of impact, even if above the level of the bails,
\end{flushleft}


\begin{flushleft}
either is between wicket and wicket
\end{flushleft}





45





\begin{flushleft}
\newpage
or if the striker has made no genuine attempt to play the ball with the bat, is
\end{flushleft}


\begin{flushleft}
between wicket and wicket or outside the line of the off stump.
\end{flushleft}


36.1.5





\begin{flushleft}
but for the interception, the ball would have hit the wicket.
\end{flushleft}





36.2





\begin{flushleft}
Interception of the ball
\end{flushleft}





36.2.1





\begin{flushleft}
In assessing points of impact in clauses 36.1.3, 36.1.4 and 36.1.5, only the first interception is to be
\end{flushleft}


\begin{flushleft}
considered.
\end{flushleft}





36.2.2





\begin{flushleft}
In assessing 36.1.3, if the bowler's end umpire is not satisfied that the ball intercepted the batsman's person
\end{flushleft}


\begin{flushleft}
before it touched the bat, the batsman shall be given Not out.
\end{flushleft}





36.2.3





\begin{flushleft}
In assessing clause 36.1.5, it is to be assumed that the path of the ball before interception would have
\end{flushleft}


\begin{flushleft}
continued after interception, irrespective of whether the ball might have pitched subsequently or not.
\end{flushleft}





36.3





\begin{flushleft}
Off side of wicket
\end{flushleft}





\begin{flushleft}
The off side of the striker's wicket shall be determined by the striker's stance at the moment the ball comes into play
\end{flushleft}


\begin{flushleft}
for that delivery. See paragraph 13 of Appendix A.
\end{flushleft}





\begin{flushleft}
37 OBSTRUCTING THE FIELD
\end{flushleft}


37.1





\begin{flushleft}
Out Obstructing the field
\end{flushleft}





37.1.1





\begin{flushleft}
Either batsman is out Obstructing the field if, except in the circumstances of clause 37.2, and while the ball
\end{flushleft}


\begin{flushleft}
is in play, he wilfully attempts to obstruct or distract the fielding side by word or action. See also clause 34
\end{flushleft}


\begin{flushleft}
(Hit the ball twice).
\end{flushleft}





37.1.2





\begin{flushleft}
The striker is out Obstructing the field if, except in the circumstances of clause 37.2, in the act of receiving a
\end{flushleft}


\begin{flushleft}
ball delivered by the bowler, he wilfully strikes the ball with a hand not holding the bat. This will apply
\end{flushleft}


\begin{flushleft}
whether it is the first strike or a second or subsequent strike. The act of receiving the ball shall extend both
\end{flushleft}


\begin{flushleft}
to playing at the ball and to striking the ball more than once in defence of his wicket.
\end{flushleft}





37.1.3





\begin{flushleft}
This clause will apply whether or not No ball is called.
\end{flushleft}





37.1.4





\begin{flushleft}
For the avoidance of doubt, if an umpire feels that a batsman, in running between the wickets, has
\end{flushleft}


\begin{flushleft}
significantly changed his direction without probable cause and thereby obstructed a fielder's attempt to effect
\end{flushleft}


\begin{flushleft}
a run out, the batsman should, on appeal, be given out, obstructing the field. It shall not be relevant whether
\end{flushleft}


\begin{flushleft}
a run out would have occurred or not.
\end{flushleft}


\begin{flushleft}
If the change of direction involves the batsman crossing the pitch, clause 41.14 shall also apply.
\end{flushleft}


\begin{flushleft}
See also paragraph 2.2 of Appendix D.
\end{flushleft}





37.2





\begin{flushleft}
Not out Obstructing the field
\end{flushleft}





\begin{flushleft}
A batsman shall not be out Obstructing the field if
\end{flushleft}


\begin{flushleft}
obstruction or distraction is accidental, or
\end{flushleft}


\begin{flushleft}
obstruction is in order to avoid injury, or
\end{flushleft}


\begin{flushleft}
in the case of the striker, he makes a second or subsequent strike to guard his wicket lawfully as in clause 34.3 (Ball
\end{flushleft}


\begin{flushleft}
lawfully struck more than once). However, see clause 37.3.
\end{flushleft}





37.3





\begin{flushleft}
Obstructing a ball from being caught
\end{flushleft}





\begin{flushleft}
The striker is out Obstructing the field should wilful obstruction or distraction by either batsman prevent a catch being
\end{flushleft}


\begin{flushleft}
completed. This shall apply even though the obstruction is caused by the striker in lawfully guarding his wicket under
\end{flushleft}


\begin{flushleft}
the provision of clause 34.3 (Ball lawfully struck more than once).
\end{flushleft}





46





\newpage
37.4





\begin{flushleft}
Returning the ball to a fielder
\end{flushleft}





\begin{flushleft}
Either batsman is out Obstructing the field if, at any time while the ball is in play and, without the consent of a fielder,
\end{flushleft}


\begin{flushleft}
he uses the bat or any part of his person to return the ball to any fielder.
\end{flushleft}





37.5





\begin{flushleft}
Runs scored
\end{flushleft}





\begin{flushleft}
When either batsman is dismissed Obstructing the field,
\end{flushleft}


37.5.1





\begin{flushleft}
unless the obstruction prevents a catch from being made, any runs completed by the batsmen before the
\end{flushleft}


\begin{flushleft}
offence shall be scored, together with any runs awarded for penalties to either side. See clauses 18.6 (Runs
\end{flushleft}


\begin{flushleft}
awarded for penalties) and 18.8 (Runs scored when a batsman is dismissed).
\end{flushleft}





37.5.2





\begin{flushleft}
if the obstruction prevents a catch from being made, any runs completed by the batsmen shall not be scored
\end{flushleft}


\begin{flushleft}
but any penalties awarded to either side shall stand.
\end{flushleft}





37.6





\begin{flushleft}
Bowler does not get credit
\end{flushleft}





\begin{flushleft}
The bowler does not get credit for the wicket.
\end{flushleft}





\begin{flushleft}
38 RUN OUT
\end{flushleft}


38.1





\begin{flushleft}
Out Run out
\end{flushleft}





\begin{flushleft}
Either batsman is out Run out, except as in clause 38.2, if, at any time while the ball is in play,
\end{flushleft}


\begin{flushleft}
he is out of his ground
\end{flushleft}


\begin{flushleft}
and his wicket is fairly put down by the action of a fielder
\end{flushleft}


\begin{flushleft}
even though No ball has been called, except in the circumstances of clause 38.2.2.2, and whether or not a run is
\end{flushleft}


\begin{flushleft}
being attempted.
\end{flushleft}





38.2





\begin{flushleft}
Batsman not out Run out
\end{flushleft}





38.2.1





\begin{flushleft}
A batsman is not out Run out in the circumstances of clauses 38.2.1.1 or 38.2.1.2.
\end{flushleft}


38.2.1.1





\begin{flushleft}
He has been within his ground and has subsequently left it to avoid injury, when the wicket is
\end{flushleft}


\begin{flushleft}
put down.
\end{flushleft}


\begin{flushleft}
Note also the provisions of clause 30.1.2 (When out of his ground).
\end{flushleft}





38.2.1.2


38.2.2





\begin{flushleft}
The ball, delivered by the bowler, has not made contact with a fielder, before the wicket is put
\end{flushleft}


\begin{flushleft}
down.
\end{flushleft}





\begin{flushleft}
The striker is not out Run out in any of the circumstances in clauses 38.2.2.1 and 38.2.2.2.
\end{flushleft}


38.2.2.1





\begin{flushleft}
He is out Stumped. See clause 39.1.2 (Out Stumped).
\end{flushleft}





38.2.2.2





\begin{flushleft}
No ball has been called
\end{flushleft}


\begin{flushleft}
and he is out of his ground not attempting a run
\end{flushleft}


\begin{flushleft}
and the wicket is fairly put down by the wicket-keeper without the intervention of another fielder.
\end{flushleft}





38.3





\begin{flushleft}
Which batsman is out
\end{flushleft}





\begin{flushleft}
The batsman out in the circumstances of clause 38.1 is the one whose ground is at the end where the wicket is put
\end{flushleft}


\begin{flushleft}
down. See clause 30.2 (Which is a batsman's ground).
\end{flushleft}





38.4





\begin{flushleft}
Runs scored
\end{flushleft}





\begin{flushleft}
If either batsman is dismissed Run out, the run in progress when the wicket is put down shall not be scored, but any
\end{flushleft}


\begin{flushleft}
runs completed by the batsmen shall stand, together with any runs for penalties awarded to either side. See clauses
\end{flushleft}


\begin{flushleft}
18.6 (Runs awarded for penalties) and 18.8 (Runs scored when a batsman is dismissed).
\end{flushleft}





47





\newpage
38.5





\begin{flushleft}
Bowler does not get credit
\end{flushleft}





\begin{flushleft}
The bowler does not get credit for the wicket.
\end{flushleft}





\begin{flushleft}
39 STUMPED
\end{flushleft}


39.1





\begin{flushleft}
Out Stumped
\end{flushleft}





39.1.1





\begin{flushleft}
The striker is out Stumped, except as in clause 39.3, if
\end{flushleft}


\begin{flushleft}
a ball which is delivered is not called No ball
\end{flushleft}


\begin{flushleft}
and he is out of his ground, other than as in clause 39.3.1
\end{flushleft}


\begin{flushleft}
and he has not attempted a run
\end{flushleft}


\begin{flushleft}
when his wicket is fairly put down by the wicket-keeper without the intervention of another fielder. Note,
\end{flushleft}


\begin{flushleft}
however clause 27.3 (Position of wicket-keeper).
\end{flushleft}





39.1.2





\begin{flushleft}
The striker is out Stumped if all the conditions of clause 39.1.1 are satisfied, even though a decision of Run
\end{flushleft}


\begin{flushleft}
out would be justified.
\end{flushleft}





39.2





\begin{flushleft}
Ball rebounding from wicket-keeper's person
\end{flushleft}





\begin{flushleft}
If the wicket is put down by the ball, it shall be regarded as having been put down by the wicket-keeper if the ball
\end{flushleft}


\begin{flushleft}
rebounds on to the stumps from any part of the wicket-keeper's person or equipment or has been kicked or thrown on
\end{flushleft}


\begin{flushleft}
to the stumps by the wicket-keeper.
\end{flushleft}





39.3





\begin{flushleft}
Not out Stumped
\end{flushleft}





39.3.1





\begin{flushleft}
The striker will not be out Stumped if he has left his ground in order to avoid injury.
\end{flushleft}





39.3.2





\begin{flushleft}
If the striker is not out Stumped he may, except in the circumstances of 38.2.2.2, (Batsman not out Run out),
\end{flushleft}


\begin{flushleft}
be out Run out if the conditions of clause 38.1 (Out Run out) apply.
\end{flushleft}





\begin{flushleft}
40 TIMED OUT
\end{flushleft}


40.1





\begin{flushleft}
Out Timed out
\end{flushleft}





40.1.1





\begin{flushleft}
After the fall of a wicket or the retirement of a batsman, the incoming batsman must, unless Time has been
\end{flushleft}


\begin{flushleft}
called, be in position to take guard or for the other batsman to be ready to receive the next ball within 1
\end{flushleft}


\begin{flushleft}
minute 30 seconds of the dismissal or retirement. If this requirement is not met, the incoming batsman will
\end{flushleft}


\begin{flushleft}
be out, Timed out.
\end{flushleft}





40.1.2





\begin{flushleft}
The incoming batsman is expected to be ready to make his way to the wicket immediately a wicket falls.
\end{flushleft}


\begin{flushleft}
Dugouts shall be provided.
\end{flushleft}





40.1.3





\begin{flushleft}
In the event of an extended delay in which no batsman comes to the wicket, the umpires shall adopt the
\end{flushleft}


\begin{flushleft}
procedure of clause 16.2 (ICC Match Referee awarding a match). For the purposes of that clause the start
\end{flushleft}


\begin{flushleft}
of the action shall be taken as the expiry of the 1 minute 30 seconds referred to above.
\end{flushleft}





40.2





\begin{flushleft}
Bowler does not get credit
\end{flushleft}





\begin{flushleft}
The bowler does not get credit for the wicket.
\end{flushleft}





\begin{flushleft}
41 UNFAIR PLAY
\end{flushleft}


41.1





\begin{flushleft}
Fair and unfair play -- responsibility of captains
\end{flushleft}





\begin{flushleft}
The captains are responsible for ensuring that play is conducted within the Spirit of Cricket, as well as within these
\end{flushleft}


\begin{flushleft}
Playing Conditions.
\end{flushleft}





48





\newpage
41.2





\begin{flushleft}
Fair and unfair play -- responsibility of umpires
\end{flushleft}





\begin{flushleft}
The umpires shall be the sole judges of fair and unfair play. If either umpire considers an action, not covered by these
\end{flushleft}


\begin{flushleft}
Playing Conditions, to be unfair he/she shall intervene without appeal and, if the ball is in play, call and signal Dead
\end{flushleft}


\begin{flushleft}
ball and implement the procedure as set out in clause 41.19. Otherwise umpires shall not interfere with the progress
\end{flushleft}


\begin{flushleft}
of play without appeal except as required to do so by these Playing Conditions.
\end{flushleft}





41.3





\begin{flushleft}
The match ball -- changing its condition
\end{flushleft}





41.3.1





\begin{flushleft}
The umpires shall make frequent and irregular inspections of the ball. In addition, they shall immediately
\end{flushleft}


\begin{flushleft}
inspect the ball if they suspect anyone of attempting to change the condition of the ball, except as permitted
\end{flushleft}


\begin{flushleft}
in clause 41.3.2.
\end{flushleft}





41.3.2





\begin{flushleft}
It is an offence for any player to take any action which changes the condition of the ball.
\end{flushleft}


\begin{flushleft}
Except in carrying out his normal duties, a batsman is not allowed to damage the ball other than, when the
\end{flushleft}


\begin{flushleft}
ball is in play, in striking it with the bat. See also clause 5.5 (Damage to the ball).
\end{flushleft}


\begin{flushleft}
A fielder may, however:
\end{flushleft}


41.3.2.1





\begin{flushleft}
polish the ball on his clothing provided that no artificial substance is used and that such
\end{flushleft}


\begin{flushleft}
polishing wastes no time.
\end{flushleft}





41.3.2.2





\begin{flushleft}
remove mud from the ball under the supervision of an umpire.
\end{flushleft}





41.3.2.3





\begin{flushleft}
dry a wet ball on a piece of cloth that has been approved by the umpires.
\end{flushleft}





41.3.3





\begin{flushleft}
The umpires shall consider the condition of the ball to have been unfairly changed if any action by any
\end{flushleft}


\begin{flushleft}
player does not comply with the conditions in clause 41.3.2.
\end{flushleft}





41.3.4





\begin{flushleft}
If the umpires together agree that the condition of the ball has been unfairly changed by a member or
\end{flushleft}


\begin{flushleft}
members of either side, or that its condition is inconsistent with the use it has received, they shall consider
\end{flushleft}


\begin{flushleft}
that there has been a contravention of this clause and decide together whether they can identify the
\end{flushleft}


\begin{flushleft}
player(s) responsible for such conduct.
\end{flushleft}





41.3.5





\begin{flushleft}
If it is possible to identify the player(s) responsible for changing the condition of the ball, the umpires shall;
\end{flushleft}


41.3.5.1





41.3.5.2





\begin{flushleft}
Change the ball forthwith.
\end{flushleft}


41.3.5.1.1





\begin{flushleft}
If the umpires together agree that the condition of the ball has been unfairly
\end{flushleft}


\begin{flushleft}
changed by a member or members of the fielding side, the batsman at the wicket
\end{flushleft}


\begin{flushleft}
shall choose the replacement ball from a selection of six other balls of various
\end{flushleft}


\begin{flushleft}
degrees of usage (including a new ball) and of the same brand as the ball in use
\end{flushleft}


\begin{flushleft}
prior to the contravention.
\end{flushleft}





41.3.5.1.2





\begin{flushleft}
If the umpires together agree that the condition of the ball has been unfairly
\end{flushleft}


\begin{flushleft}
changed by a member or members of the batting side, the umpires shall select
\end{flushleft}


\begin{flushleft}
and bring into use immediately, a ball which shall have wear comparable to that
\end{flushleft}


\begin{flushleft}
of the previous ball immediately prior to the contravention.
\end{flushleft}





\begin{flushleft}
Additionally, the bowler's end umpire shall
\end{flushleft}


\begin{flushleft}
- award 5 Penalty runs to the opposing side.
\end{flushleft}


\begin{flushleft}
- if appropriate, inform the batsmen at the wicket and the captain of the fielding side that the ball
\end{flushleft}


\begin{flushleft}
has been changed and the reason for their action.
\end{flushleft}


\begin{flushleft}
- inform the captain of the batting side as soon as practicable of what has occurred.
\end{flushleft}


\begin{flushleft}
The umpires shall then report the matter to the ICC Match Referee who shall take such action as
\end{flushleft}


\begin{flushleft}
is considered appropriate against the player(s) concerned.
\end{flushleft}





41.3.6





\begin{flushleft}
If it is not possible to identify the player(s) responsible for changing the condition of the ball, the umpires
\end{flushleft}


\begin{flushleft}
shall;
\end{flushleft}





49





\newpage
41.3.6.1





\begin{flushleft}
Change the ball forthwith. The umpires shall choose the replacement ball for one of similar wear
\end{flushleft}


\begin{flushleft}
and of the same brand as the ball in use prior to the contravention.
\end{flushleft}





41.3.6.2





\begin{flushleft}
The bowler's end umpire shall issue the captain with a first and final warning, and
\end{flushleft}





41.3.6.3





\begin{flushleft}
Advise the captain that should there be any further instances of changing the condition of the
\end{flushleft}


\begin{flushleft}
ball by that team during the remainder of the series, clause 41.3.5.2 above will be adopted, with
\end{flushleft}


\begin{flushleft}
the captain deemed to be the player responsible for the contravention.
\end{flushleft}





41.4





\begin{flushleft}
Deliberate attempt to distract striker
\end{flushleft}





41.4.1





\begin{flushleft}
It is unfair for any fielder deliberately to attempt to distract the striker while he is preparing to receive or
\end{flushleft}


\begin{flushleft}
receiving a delivery.
\end{flushleft}





41.4.2





\begin{flushleft}
If either umpire considers that any action by a fielder is such an attempt, he/she shall immediately call and
\end{flushleft}


\begin{flushleft}
signal Dead ball and inform the other umpire of the reason for the call. The bowler's end umpire shall
\end{flushleft}


\begin{flushleft}
- award 5 Penalty runs to the batting side.
\end{flushleft}


\begin{flushleft}
- inform the captain of the fielding side, the batsmen and, as soon as practicable, the captain of the batting
\end{flushleft}


\begin{flushleft}
side of the reason for the action.
\end{flushleft}


\begin{flushleft}
Neither batsman shall be dismissed from that delivery and the ball shall not count as one of the over.
\end{flushleft}


\begin{flushleft}
The umpires may then report the matter to the ICC Match Referee who shall take such action as is
\end{flushleft}


\begin{flushleft}
considered appropriate against the fielder concerned.
\end{flushleft}





41.5





\begin{flushleft}
Deliberate distraction, deception or obstruction of batsman
\end{flushleft}





41.5.1





\begin{flushleft}
In addition to clause 41.4, it is unfair for any fielder wilfully to attempt, by word or action, to distract, deceive
\end{flushleft}


\begin{flushleft}
or obstruct either batsman after the striker has received the ball.
\end{flushleft}





41.5.2





\begin{flushleft}
It is for either one of the umpires to decide whether any distraction, deception or obstruction is wilful or not.
\end{flushleft}





41.5.3





\begin{flushleft}
If either umpire considers that a fielder has caused or attempted to cause such a distraction, deception or
\end{flushleft}


\begin{flushleft}
obstruction, he/she shall immediately call and signal Dead ball and inform the other umpire of the reason for
\end{flushleft}


\begin{flushleft}
the call.
\end{flushleft}





41.5.4





\begin{flushleft}
Neither batsman shall be dismissed from that delivery.
\end{flushleft}





41.5.5





\begin{flushleft}
If an obstruction involves physical contact, the umpires together shall decide whether or not an offence
\end{flushleft}


\begin{flushleft}
under clause 42 (Players' conduct) has been committed.
\end{flushleft}





41.5.6





41.5.5.1





\begin{flushleft}
If an offence under clause 42 (Players' conduct) has been committed, they shall apply the
\end{flushleft}


\begin{flushleft}
relevant procedures in clause 42 and shall also apply each of clauses 41.5.7 to 41.5.9.
\end{flushleft}





41.5.5.2





\begin{flushleft}
If they consider that there has been no offence under clause 42 (Players' conduct), they shall
\end{flushleft}


\begin{flushleft}
apply each of clauses 41.5.6 to 41.5.10.
\end{flushleft}





\begin{flushleft}
The bowler's end umpire shall;
\end{flushleft}


\begin{flushleft}
- award 5 Penalty runs to the batting side.
\end{flushleft}


\begin{flushleft}
- inform the captain of the fielding side of the reason for this action and as soon as practicable inform the
\end{flushleft}


\begin{flushleft}
captain of the batting side.
\end{flushleft}





41.5.7





\begin{flushleft}
The ball shall not count as one of the over.
\end{flushleft}





41.5.8





\begin{flushleft}
Any runs completed by the batsmen before the offence shall be scored, together with any runs for penalties
\end{flushleft}


\begin{flushleft}
awarded to either side. Additionally, the run in progress shall be scored whether or not the batsmen had
\end{flushleft}


\begin{flushleft}
already crossed at the instant of the offence.
\end{flushleft}





41.5.9





\begin{flushleft}
The batsmen at the wicket shall decide which of them is to face the next delivery.
\end{flushleft}





50





\begin{flushleft}
\newpage
41.5.10 The umpires may then report the matter to the ICC Match Referee who shall take such action as is
\end{flushleft}


\begin{flushleft}
considered appropriate against the fielder concerned.
\end{flushleft}





41.6





\begin{flushleft}
Bowling of dangerous and unfair short pitched deliveries
\end{flushleft}





41.6.1





\begin{flushleft}
Notwithstanding clause 41.6.2, the bowling of short pitched deliveries is dangerous if the bowler's end
\end{flushleft}


\begin{flushleft}
umpire considers that, taking into consideration the skill of the striker, by their speed, length, height and
\end{flushleft}


\begin{flushleft}
direction they are likely to inflict physical injury on him. The fact that the striker is wearing protective
\end{flushleft}


\begin{flushleft}
equipment shall be disregarded.
\end{flushleft}


\begin{flushleft}
In the first instance the umpire decides that the bowling of short pitched deliveries has become dangerous
\end{flushleft}


\begin{flushleft}
under 41.6.1
\end{flushleft}


41.6.1.1





\begin{flushleft}
The umpire shall call and signal No ball, and when the ball is dead, caution the bowler and
\end{flushleft}


\begin{flushleft}
inform the other umpire, the captain of the fielding side and the batsmen of what has occurred.
\end{flushleft}


\begin{flushleft}
This caution shall apply to that bowler throughout the innings.
\end{flushleft}





41.6.1.2





\begin{flushleft}
If there is a second instance, the umpire shall repeat the above procedure and indicate to the
\end{flushleft}


\begin{flushleft}
bowler that this is a final warning, which shall apply to that bowler throughout the innings.
\end{flushleft}





41.6.1.3





\begin{flushleft}
Should there be any further instance by the same bowler in that innings, the umpire shall
\end{flushleft}


\begin{flushleft}
- call and signal No ball
\end{flushleft}


\begin{flushleft}
- when the ball is dead, direct the captain of the fielding side to suspend the bowler immediately
\end{flushleft}


\begin{flushleft}
from bowling
\end{flushleft}


\begin{flushleft}
- inform the other umpire for the reason for this action.
\end{flushleft}


\begin{flushleft}
The bowler thus suspended shall not be allowed to bowl again in that innings.
\end{flushleft}


\begin{flushleft}
If applicable, the over shall be completed by another bowler, who shall neither have bowled any
\end{flushleft}


\begin{flushleft}
part of the previous over, nor be allowed to bowl any part of the next over.
\end{flushleft}


\begin{flushleft}
- The umpire shall report the occurrence to the batsmen and, as soon as practicable, to the
\end{flushleft}


\begin{flushleft}
captain of the batting side.
\end{flushleft}


\begin{flushleft}
The umpires may then report the matter to the ICC Match Referee who shall take such action as
\end{flushleft}


\begin{flushleft}
is considered appropriate against the bowler concerned.
\end{flushleft}





41.6.1.4





\begin{flushleft}
A bowler shall be limited to one fast short-pitched delivery per over.
\end{flushleft}





41.6.1.5





\begin{flushleft}
A fast short-pitched delivery is defined as a ball, which passes or would have passed above the
\end{flushleft}


\begin{flushleft}
shoulder height of the striker standing upright at the popping crease.
\end{flushleft}





41.6.1.6





\begin{flushleft}
The umpire at the bowler's end shall advise the bowler and the batsman on strike when each
\end{flushleft}


\begin{flushleft}
fast short pitched delivery has been bowled.
\end{flushleft}





41.6.1.7





\begin{flushleft}
In addition, a ball that passes above head height of the batsman, standing upright at the
\end{flushleft}


\begin{flushleft}
popping crease, that prevents him from being able to hit it with his bat by means of a normal
\end{flushleft}


\begin{flushleft}
cricket stroke shall be called a Wide. See also clause 22.1.1.2
\end{flushleft}


41.6.1.7.1





\begin{flushleft}
For the avoidance of doubt any fast short pitched delivery that is called a Wide
\end{flushleft}


\begin{flushleft}
under this clause shall also count as one of the allowable short pitched deliveries
\end{flushleft}


\begin{flushleft}
in that over.
\end{flushleft}





41.6.1.8





\begin{flushleft}
In the event of a bowler bowling more than one fast short-pitched delivery in an over as defined
\end{flushleft}


\begin{flushleft}
in clause 41.6.1.5 above, the umpire at the bowler's end shall call and signal No ball on each
\end{flushleft}


\begin{flushleft}
occasion. A differential signal shall be used to signify a fast short pitched delivery. The umpire
\end{flushleft}


\begin{flushleft}
shall call and signal {`}No ball' and then tap the head with the other hand.
\end{flushleft}





41.6.1.9





\begin{flushleft}
If a bowler delivers a second fast short pitched ball in an over, the umpire, after the call of No
\end{flushleft}


\begin{flushleft}
ball and when the ball is dead, shall caution the bowler, inform the other umpire, the captain of
\end{flushleft}


\begin{flushleft}
the fielding side and the batsmen at the wicket of what has occurred. This caution shall apply
\end{flushleft}


\begin{flushleft}
throughout the innings.
\end{flushleft}





51





\newpage
41.6.1.10





\begin{flushleft}
If there is a second instance of the bowler being No balled in the innings for bowling more than
\end{flushleft}


\begin{flushleft}
one fast short pitched delivery in an over, the umpire shall advise the bowler that this is his final
\end{flushleft}


\begin{flushleft}
warning for the innings.
\end{flushleft}





41.6.1.11





\begin{flushleft}
Should there be any further instance by the same bowler in that innings, the umpire shall
\end{flushleft}


\begin{flushleft}
- call and signal No ball
\end{flushleft}


\begin{flushleft}
- when the ball is dead, direct the captain of the fielding side to suspend the bowler immediately
\end{flushleft}


\begin{flushleft}
from bowling
\end{flushleft}


\begin{flushleft}
- inform the other umpire for the reason for this action.
\end{flushleft}


\begin{flushleft}
The bowler thus suspended shall not be allowed to bowl again in that innings.
\end{flushleft}


\begin{flushleft}
If applicable, the over shall be completed by another bowler, who shall neither have bowled any
\end{flushleft}


\begin{flushleft}
part of the previous over, nor be allowed to bowl any part of the next over.
\end{flushleft}


\begin{flushleft}
- The umpire shall report the occurrence to the batsmen and, as soon as practicable, to the
\end{flushleft}


\begin{flushleft}
captain of the batting side.
\end{flushleft}


\begin{flushleft}
The umpires may then report the matter to the ICC Match Referee who shall take such action as
\end{flushleft}


\begin{flushleft}
is considered appropriate against the bowler concerned.
\end{flushleft}





41.6.2





\begin{flushleft}
Should the umpires initiate the caution and warning procedures set out in clauses 41.6.1.3, 41.6.1.9 and
\end{flushleft}


\begin{flushleft}
41.7, such cautions and warnings are not to be cumulative.
\end{flushleft}





41.7





\begin{flushleft}
Bowling of dangerous and unfair non-pitching deliveries
\end{flushleft}





41.7.1





\begin{flushleft}
Any delivery, which passes or would have passed, without pitching, above waist height of the striker
\end{flushleft}


\begin{flushleft}
standing upright at the popping crease, is to be deemed to be unfair, whether or not it is likely to inflict
\end{flushleft}


\begin{flushleft}
physical injury on the striker. If the bowler bowls such a delivery the umpire shall immediately call and signal
\end{flushleft}


\begin{flushleft}
No ball.
\end{flushleft}


\begin{flushleft}
If, in the opinion of the umpire, such a delivery is considered likely to inflict physical injury on the batsman by
\end{flushleft}


\begin{flushleft}
its speed and direction, it shall be considered dangerous. When the ball is dead the umpire shall caution the
\end{flushleft}


\begin{flushleft}
bowler, indicating that this is a first and final warning. The umpire shall also inform the other umpire, the
\end{flushleft}


\begin{flushleft}
captain of the fielding side and the batsmen of what has occurred. This caution shall apply to that bowler
\end{flushleft}


\begin{flushleft}
throughout the innings.
\end{flushleft}





41.7.2





\begin{flushleft}
Should there be any further instance (where a dangerous non-pitching delivery is bowled and is considered
\end{flushleft}


\begin{flushleft}
likely to inflict physical injury on the batsman) by the same bowler in that innings, the umpire shall
\end{flushleft}


\begin{flushleft}
- call and signal No ball
\end{flushleft}


\begin{flushleft}
- when the ball is dead, direct the captain of the fielding side to suspend the bowler immediately from
\end{flushleft}


\begin{flushleft}
bowling
\end{flushleft}


\begin{flushleft}
- inform the other umpire for the reason for this action.
\end{flushleft}


\begin{flushleft}
The bowler thus suspended shall not be allowed to bowl again in that innings.
\end{flushleft}


\begin{flushleft}
If applicable, the over shall be completed by another bowler, who shall neither have bowled any part of the
\end{flushleft}


\begin{flushleft}
previous over, nor be allowed to bowl any part of the next over.
\end{flushleft}


\begin{flushleft}
Additionally the umpire shall
\end{flushleft}


\begin{flushleft}
- report the occurrence to the batsmen and, as soon as practicable, to the captain of the batting side.
\end{flushleft}


\begin{flushleft}
The umpires may then report the matter to the ICC Match Referee who shall take such action as is
\end{flushleft}


\begin{flushleft}
considered appropriate against the bowler concerned.
\end{flushleft}





41.7.3





\begin{flushleft}
The warning sequence in clauses 41.7.1 and 41.7.2 is independent of the warning and action sequence in
\end{flushleft}


\begin{flushleft}
clause 41.6.
\end{flushleft}





52





\newpage
41.7.4





\begin{flushleft}
If the umpire considers that a bowler deliberately bowled a high full-pitched delivery, deemed to be
\end{flushleft}


\begin{flushleft}
dangerous and unfair as defined in clause 41.7.1, then the caution and warning in clause 41.7.1 shall be
\end{flushleft}


\begin{flushleft}
dispensed with. The umpire shall
\end{flushleft}


\begin{flushleft}
- immediately call and signal No ball.
\end{flushleft}


\begin{flushleft}
- when the ball is dead, direct the captain of the fielding side to suspend the bowler immediately from
\end{flushleft}


\begin{flushleft}
bowling and inform the other umpire for the reason for this action.
\end{flushleft}


\begin{flushleft}
The bowler thus suspended shall not be allowed to bowl again in that innings.
\end{flushleft}


\begin{flushleft}
If applicable, the over shall be completed by another bowler, who shall neither have bowled any part of the
\end{flushleft}


\begin{flushleft}
previous over, nor be allowed to bowl any part of the next over.
\end{flushleft}


\begin{flushleft}
- report the occurrence to the batsmen and, as soon as practicable, to the captain of the batting side.
\end{flushleft}


\begin{flushleft}
The umpires together shall report the occurrence to the ICC Match Referee who shall take such action as is
\end{flushleft}


\begin{flushleft}
considered appropriate against the bowler concerned.
\end{flushleft}





41.8





\begin{flushleft}
Bowling of deliberate front-foot No ball
\end{flushleft}





\begin{flushleft}
If the umpire considers that the bowler has delivered a deliberate front-foot No ball, he/she shall
\end{flushleft}


\begin{flushleft}
- immediately call and signal No ball.
\end{flushleft}


\begin{flushleft}
- when the ball is dead, direct the captain of the fielding side to suspend the bowler immediately from bowling
\end{flushleft}


\begin{flushleft}
- inform the other umpire for the reason for this action.
\end{flushleft}


\begin{flushleft}
The bowler thus suspended shall not be allowed to bowl again in that innings.
\end{flushleft}


\begin{flushleft}
If applicable, the over shall be completed by another bowler, who shall neither have bowled any part of the previous
\end{flushleft}


\begin{flushleft}
over, nor be allowed to bowl any part of the next over.
\end{flushleft}


\begin{flushleft}
- report the occurrence to the batsmen and, as soon as practicable, to the captain of the batting side.
\end{flushleft}


\begin{flushleft}
The umpires together shall report the occurrence to the ICC Match Referee who shall take such action as is
\end{flushleft}


\begin{flushleft}
considered appropriate against the bowler concerned.
\end{flushleft}





41.9





\begin{flushleft}
Time wasting by the fielding side
\end{flushleft}





41.9.1





\begin{flushleft}
It is unfair for any fielder to waste time.
\end{flushleft}





41.9.2





\begin{flushleft}
If either umpire considers that the progress of an over is unnecessarily slow, or time is being wasted in any
\end{flushleft}


\begin{flushleft}
other way, by the captain of the fielding side or by any other fielder, at the first instance the umpire
\end{flushleft}


\begin{flushleft}
concerned shall
\end{flushleft}


\begin{flushleft}
- if the ball is in play, call and signal Dead ball.
\end{flushleft}


\begin{flushleft}
- inform the other umpire of what has occurred.
\end{flushleft}


\begin{flushleft}
The bowler's end umpire shall then
\end{flushleft}


\begin{flushleft}
- warn the captain of the fielding side, indicating that this is a first and final warning.
\end{flushleft}


\begin{flushleft}
- inform the batsmen of what has occurred.
\end{flushleft}





41.9.3





\begin{flushleft}
If either umpire considers that there is any further waste of time in that innings by any fielder, the umpire
\end{flushleft}


\begin{flushleft}
concerned shall
\end{flushleft}


\begin{flushleft}
- if the ball is in play, call and signal Dead ball.
\end{flushleft}


\begin{flushleft}
- inform the other umpire of what has occurred.
\end{flushleft}





53





\begin{flushleft}
\newpage
The bowler's end umpire shall then award 5 Penalty runs to the batting side and inform the captain of the
\end{flushleft}


\begin{flushleft}
fielding side of the reason for this action.
\end{flushleft}


\begin{flushleft}
Additionally the umpire shall inform the batsmen and, as soon as is practicable, the captain of the batting
\end{flushleft}


\begin{flushleft}
side of what has occurred.
\end{flushleft}


\begin{flushleft}
If the umpires believe that the act of time wasting was deliberate or repetitive, they may lodge a report under
\end{flushleft}


\begin{flushleft}
the ICC Code of Conduct. In such circumstances the Captain and/or any individual members of the fielding
\end{flushleft}


\begin{flushleft}
team responsible for the time wasting will be charged.
\end{flushleft}





\begin{flushleft}
41.10 Batsman wasting time
\end{flushleft}


\begin{flushleft}
41.10.1 It is unfair for a batsman to waste time. In normal circumstances, the striker should always be ready to take
\end{flushleft}


\begin{flushleft}
strike when the bowler is ready to start his run-up.
\end{flushleft}


\begin{flushleft}
41.10.2 Should either batsman waste time by failing to meet this requirement, or in any other way, the following
\end{flushleft}


\begin{flushleft}
procedure shall be adopted. At the first instance, either before the bowler starts his run-up or when the ball
\end{flushleft}


\begin{flushleft}
becomes dead, as appropriate, the umpire shall
\end{flushleft}


\begin{flushleft}
- warn both batsmen and indicate that this is a first and final warning. This warning shall apply throughout
\end{flushleft}


\begin{flushleft}
the innings. The umpire shall so inform each incoming batsman.
\end{flushleft}


\begin{flushleft}
- inform the other umpire of what has occurred.
\end{flushleft}


\begin{flushleft}
- inform the captain of the fielding side and, as soon as practicable, the captain of the batting side of what
\end{flushleft}


\begin{flushleft}
has occurred.
\end{flushleft}


\begin{flushleft}
41.10.3 If there is any further time wasting by any batsman in that innings, the umpire shall, at the appropriate time
\end{flushleft}


\begin{flushleft}
while the ball is dead
\end{flushleft}


\begin{flushleft}
- award 5 Penalty runs to the fielding side.
\end{flushleft}


\begin{flushleft}
- inform the other umpire of the reason for this action.
\end{flushleft}


\begin{flushleft}
- inform the other batsman, the captain of the fielding side and, as soon as practicable, the captain of the
\end{flushleft}


\begin{flushleft}
batting side of what has occurred.
\end{flushleft}


\begin{flushleft}
If the umpires believe that the act of time wasting was deemed to be deliberate or repetitive, they may lodge
\end{flushleft}


\begin{flushleft}
a report under the ICC Code of Conduct. In such circumstances the batsman concerned will be charged.
\end{flushleft}





\begin{flushleft}
41.11 The protected area
\end{flushleft}


\begin{flushleft}
The protected area is defined as that area of the pitch contained within a rectangle bounded at each end by
\end{flushleft}


\begin{flushleft}
imaginary lines parallel to the popping creases and 5 ft/1.52 m in front of each, and on the sides by imaginary lines,
\end{flushleft}


\begin{flushleft}
one each side of the imaginary line joining the centres of the two middle stumps, each parallel to it and 1 ft/30.48 cm
\end{flushleft}


\begin{flushleft}
from it.
\end{flushleft}





\begin{flushleft}
41.12 Fielder damaging the pitch
\end{flushleft}


\begin{flushleft}
41.12.1 It is unfair to cause deliberate or avoidable damage to the pitch. A fielder will be deemed to be causing
\end{flushleft}


\begin{flushleft}
avoidable damage if either umpire considers that his presence on the pitch is without reasonable cause.
\end{flushleft}


\begin{flushleft}
41.12.2 If a fielder causes avoidable damage to the pitch, other than as in clause 41.13.1, at the first instance the
\end{flushleft}


\begin{flushleft}
umpire seeing the contravention shall, when the ball is dead, inform the other umpire. The bowler's end
\end{flushleft}


\begin{flushleft}
umpire shall then
\end{flushleft}


\begin{flushleft}
- caution the captain of the fielding side and indicate that this is a first and final warning. This warning shall
\end{flushleft}


\begin{flushleft}
apply throughout the innings.
\end{flushleft}


\begin{flushleft}
- inform the batsmen of what has occurred.
\end{flushleft}





54





\begin{flushleft}
\newpage
41.12.3 If, in that innings, there is any further instance of avoidable damage to the pitch, by any fielder, the umpire
\end{flushleft}


\begin{flushleft}
seeing the contravention shall, when the ball is dead, inform the other umpire. The bowler's end umpire shall
\end{flushleft}


\begin{flushleft}
then
\end{flushleft}


\begin{flushleft}
- award 5 Penalty runs to the batting side.
\end{flushleft}


\begin{flushleft}
Additionally the umpire shall
\end{flushleft}


\begin{flushleft}
- inform the fielding captain of the reason for this action.
\end{flushleft}


\begin{flushleft}
- inform the batsmen and, as soon as practicable, the captain of the batting side of what has occurred.
\end{flushleft}


\begin{flushleft}
The umpires together shall report the occurrence to the ICC Match Referee who shall take such action as is
\end{flushleft}


\begin{flushleft}
considered appropriate against the fielder concerned.
\end{flushleft}





\begin{flushleft}
41.13 Bowler running on protected area
\end{flushleft}


\begin{flushleft}
41.13.1 It is unfair for a bowler to enter the protected area in his follow-through without reasonable cause, whether or
\end{flushleft}


\begin{flushleft}
not the ball is delivered.
\end{flushleft}


\begin{flushleft}
41.13.2 If a bowler contravenes this clause, at the first instance and when the ball is dead, the umpire shall
\end{flushleft}


\begin{flushleft}
- caution the bowler and inform the other umpire of what has occurred. This caution shall apply to that
\end{flushleft}


\begin{flushleft}
bowler throughout the innings.
\end{flushleft}


\begin{flushleft}
- inform the captain of the fielding side and the batsmen of what has occurred.
\end{flushleft}


\begin{flushleft}
41.13.3 If, in that innings, the same bowler again contravenes this clause, the umpire shall repeat the above
\end{flushleft}


\begin{flushleft}
procedure indicating that this is a final warning. This warning shall also apply throughout the innings.
\end{flushleft}


\begin{flushleft}
41.13.4 If, in that innings, the same bowler contravenes this clause a third time, when the ball is dead, the umpire
\end{flushleft}


\begin{flushleft}
shall,
\end{flushleft}


\begin{flushleft}
- direct the captain of the fielding side to suspend the bowler immediately from bowling. If applicable, the
\end{flushleft}


\begin{flushleft}
over shall be completed by another bowler, who shall neither have bowled any part of the previous over, nor
\end{flushleft}


\begin{flushleft}
be allowed to bowl any part of the next over. The bowler taken off shall not be allowed to bowl again in that
\end{flushleft}


\begin{flushleft}
innings.
\end{flushleft}


\begin{flushleft}
- inform the other umpire of the reason for this action.
\end{flushleft}


\begin{flushleft}
- inform the batsmen and, as soon as practicable, the captain of the batting side of what has occurred.
\end{flushleft}


\begin{flushleft}
The umpires may then report the matter to the ICC Match Referee who shall take such action as is
\end{flushleft}


\begin{flushleft}
considered appropriate against the bowler concerned.
\end{flushleft}





\begin{flushleft}
41.14 Batsman damaging the pitch
\end{flushleft}


\begin{flushleft}
41.14.1 It is unfair to cause deliberate or avoidable damage to the pitch. If the striker enters the protected area in
\end{flushleft}


\begin{flushleft}
playing or playing at the ball, he must move from it immediately thereafter. A batsman will be deemed to be
\end{flushleft}


\begin{flushleft}
causing avoidable damage if either umpire considers that his presence on the pitch is without reasonable
\end{flushleft}


\begin{flushleft}
cause.
\end{flushleft}


\begin{flushleft}
41.14.2 If either batsman causes deliberate or avoidable damage to the pitch, other than as in clause 41.15, at the
\end{flushleft}


\begin{flushleft}
first instance the umpire seeing the contravention shall, when the ball is dead, inform the other umpire of the
\end{flushleft}


\begin{flushleft}
occurrence. The bowler's end umpire shall then
\end{flushleft}


\begin{flushleft}
- warn both batsmen that the practice is unfair and indicate that this is a first and final warning. This warning
\end{flushleft}


\begin{flushleft}
shall apply throughout the innings. The umpire shall so inform each incoming batsman.
\end{flushleft}


\begin{flushleft}
- inform the captain of the fielding side and, as soon as practicable, the captain of the batting side of what
\end{flushleft}


\begin{flushleft}
has occurred.
\end{flushleft}


\begin{flushleft}
41.14.3 If there is any further instance of avoidable damage to the pitch by any batsman in that innings, the umpire
\end{flushleft}


\begin{flushleft}
seeing the contravention shall, when the ball is dead, inform the other umpire of the occurrence.
\end{flushleft}





55





\begin{flushleft}
\newpage
The bowler's end umpire shall
\end{flushleft}


\begin{flushleft}
- disallow all runs to the batting side
\end{flushleft}


\begin{flushleft}
- return any not out batsman to his original end
\end{flushleft}


\begin{flushleft}
- signal No ball or Wide to the scorers if applicable.
\end{flushleft}


\begin{flushleft}
- award 5 Penalty runs to the fielding side.
\end{flushleft}


\begin{flushleft}
- award any other 5-run Penalty that is applicable except for Penalty runs under clause 28.3 (Protective
\end{flushleft}


\begin{flushleft}
helmets belonging to the fielding side).
\end{flushleft}


\begin{flushleft}
- Inform the captain of the fielding side and, as soon as practicable, the captain of the batting side of the
\end{flushleft}


\begin{flushleft}
reason for this action.
\end{flushleft}


\begin{flushleft}
The umpires together shall report the occurrence to the ICC Match Referee who shall take such action as is
\end{flushleft}


\begin{flushleft}
considered appropriate against the batsman concerned.
\end{flushleft}





\begin{flushleft}
41.15 Striker in protected area
\end{flushleft}


\begin{flushleft}
41.15.1 The striker shall not adopt a stance in the protected area or so close to it that frequent encroachment is
\end{flushleft}


\begin{flushleft}
inevitable.
\end{flushleft}


\begin{flushleft}
The striker may mark a guard on the pitch provided that no mark is unreasonably close to the protected
\end{flushleft}


\begin{flushleft}
area.
\end{flushleft}


\begin{flushleft}
41.15.2 If either umpire considers that the striker is in breach of any of the conditions in clause 41.15.1, if the bowler
\end{flushleft}


\begin{flushleft}
has not entered the delivery stride, he/she shall immediately call Dead ball, otherwise, wait until the ball is
\end{flushleft}


\begin{flushleft}
dead; he/she shall then inform the other umpire of the occurrence.
\end{flushleft}


\begin{flushleft}
The bowler's end umpire shall then
\end{flushleft}


\begin{flushleft}
- warn the striker that the practice is unfair and indicate that this is a first and final warning. This warning
\end{flushleft}


\begin{flushleft}
shall apply throughout the innings. The umpire shall so inform the non-striker and each incoming batsman.
\end{flushleft}


\begin{flushleft}
- inform the captain of the fielding side and, as soon as practicable, the captain of the batting side of what
\end{flushleft}


\begin{flushleft}
has occurred.
\end{flushleft}


\begin{flushleft}
41.15.3 If there is any further breach of any of the conditions in clause 41.15.1 by any batsman in that innings, the
\end{flushleft}


\begin{flushleft}
umpire seeing the contravention shall, if the bowler has not entered his delivery stride, immediately call and
\end{flushleft}


\begin{flushleft}
signal Dead ball, otherwise, he/she shall wait until the ball is dead and then inform the other umpire of the
\end{flushleft}


\begin{flushleft}
occurrence.
\end{flushleft}


\begin{flushleft}
The bowler's end umpire shall
\end{flushleft}


\begin{flushleft}
- disallow all runs to the batting side
\end{flushleft}


\begin{flushleft}
- return any not out batsman to his original end
\end{flushleft}


\begin{flushleft}
- signal No ball or Wide to the scorers if applicable.
\end{flushleft}


\begin{flushleft}
- award 5 Penalty runs to the fielding side.
\end{flushleft}


\begin{flushleft}
- award any other 5-run Penalty that is applicable except for Penalty runs under clause 28.3 (Protective
\end{flushleft}


\begin{flushleft}
helmets belonging to the fielding side).
\end{flushleft}


\begin{flushleft}
- inform the captain of the fielding side and, as soon as practicable, the captain of the batting side of the
\end{flushleft}


\begin{flushleft}
reason for this action.
\end{flushleft}


\begin{flushleft}
The umpires together shall report the occurrence to the ICC Match Referee who shall take such action as is
\end{flushleft}


\begin{flushleft}
considered appropriate against the batsman concerned.
\end{flushleft}





56





\begin{flushleft}
\newpage
41.16 Non-striker leaving his ground early
\end{flushleft}


\begin{flushleft}
If the non-striker is out of his ground from the moment the ball comes into play to the instant when the bowler would
\end{flushleft}


\begin{flushleft}
normally have been expected to release the ball, the bowler is permitted to attempt to run him out. Whether the
\end{flushleft}


\begin{flushleft}
attempt is successful or not, the ball shall not count as one in the over.
\end{flushleft}


\begin{flushleft}
If the bowler fails in an attempt to run out the non-striker, the umpire shall call and signal Dead ball as soon as
\end{flushleft}


\begin{flushleft}
possible.
\end{flushleft}





\begin{flushleft}
41.17 Batsmen stealing a run
\end{flushleft}


\begin{flushleft}
41.17.1 It is unfair for the batsmen to attempt to steal a run during the bowler's run-up.
\end{flushleft}


\begin{flushleft}
Unless the bowler attempts to run out either batsman -- see clauses 41.16 and 21.4 (Bowler throwing
\end{flushleft}


\begin{flushleft}
towards striker's end before delivery) -- the umpire shall
\end{flushleft}


\begin{flushleft}
- call and signal Dead ball as soon as the batsmen cross in such an attempt.
\end{flushleft}


\begin{flushleft}
- inform the other umpire of the reason for this action.
\end{flushleft}


\begin{flushleft}
The bowler's end umpire shall then
\end{flushleft}


\begin{flushleft}
- return the batsmen to their original ends.
\end{flushleft}


\begin{flushleft}
- award 5 Penalty runs to the fielding side.
\end{flushleft}


\begin{flushleft}
- inform the batsmen, the captain of the fielding side and, as soon as practicable, the captain of the batting
\end{flushleft}


\begin{flushleft}
side, of the reason for this action.
\end{flushleft}


\begin{flushleft}
The umpires may then report the matter to the ICC Match Referee who shall take such action as is
\end{flushleft}


\begin{flushleft}
considered appropriate against the batsman concerned.
\end{flushleft}





\begin{flushleft}
41.18 Penalty runs
\end{flushleft}


\begin{flushleft}
41.18.1 When Penalty runs are awarded to either side, when the ball is dead the umpire shall signal the Penalty runs
\end{flushleft}


\begin{flushleft}
to the scorers. See clause 2.13 (Signals).
\end{flushleft}


\begin{flushleft}
41.18.2 Penalty runs shall be awarded in each case where these Playing Conditions require the award, even if a
\end{flushleft}


\begin{flushleft}
result has already been achieved. See clause 16.6 (Winning hit or extras).
\end{flushleft}


\begin{flushleft}
Note, however, that the restrictions on awarding Penalty runs, in clauses 23.3 (Leg byes not to be awarded),
\end{flushleft}


\begin{flushleft}
34.4 (Runs scored from ball lawfully struck more than once) and 28.3 (Protective helmets belonging to the
\end{flushleft}


\begin{flushleft}
fielding side), will apply.
\end{flushleft}


\begin{flushleft}
41.18.3 When 5 Penalty runs are awarded to the batting side under any of clauses 24.4 (Player returning without
\end{flushleft}


\begin{flushleft}
permission), 28.2 (Fielding the ball), or 28.3 (Protective helmets belonging to the fielding side) or under 41.3,
\end{flushleft}


\begin{flushleft}
41.4, 41.5, 41.9 or 41.12, then
\end{flushleft}


\begin{flushleft}
- they shall be scored as Penalty extras and shall be in addition to any other penalties.
\end{flushleft}


\begin{flushleft}
- they are awarded when the ball is dead and shall not be regarded as runs scored from either the
\end{flushleft}


\begin{flushleft}
immediately preceding delivery or the immediately following delivery, and shall be in addition to any runs
\end{flushleft}


\begin{flushleft}
from those deliveries.
\end{flushleft}


\begin{flushleft}
- the batsmen shall not change ends solely by reason of the 5 run penalty.
\end{flushleft}


\begin{flushleft}
41.18.4 When 5 Penalty runs are awarded to the fielding side, under clause 18.5.2 (Deliberate short runs), or under
\end{flushleft}


\begin{flushleft}
41.10, 41.14, 41.15 or 41.17, they shall be added as Penalty extras to that side's total of runs in its most
\end{flushleft}


\begin{flushleft}
recently completed innings. If the fielding side has not completed an innings, the 5 Penalty runs shall be
\end{flushleft}


\begin{flushleft}
added to the score in its next innings.
\end{flushleft}





57





\begin{flushleft}
\newpage
41.19 Unfair actions
\end{flushleft}


\begin{flushleft}
41.19.1 If an umpire considers that any action by a player, not covered in these Playing Conditions, is unfair, he/she
\end{flushleft}


\begin{flushleft}
shall call and signal Dead ball, if appropriate, as soon as it becomes clear that the call will not disadvantage
\end{flushleft}


\begin{flushleft}
the non-offending side, and report the matter to the other umpire.
\end{flushleft}


\begin{flushleft}
The bowler's end umpire shall
\end{flushleft}


41.19.1.1





\begin{flushleft}
If this is a first offence by that side
\end{flushleft}


\begin{flushleft}
- summon the offending player's captain and issue a first and final warning which shall apply to
\end{flushleft}


\begin{flushleft}
all members of the team for the remainder of the match.
\end{flushleft}


\begin{flushleft}
- warn the offending player's captain that any further such offence by any member of his team
\end{flushleft}


\begin{flushleft}
shall result in the award of 5 Penalty runs to the opposing team.
\end{flushleft}





41.19.1.2





\begin{flushleft}
If this is a second or subsequent offence by that side
\end{flushleft}


\begin{flushleft}
- award 5 Penalty runs to the opposing side
\end{flushleft}





41.19.1.3





\begin{flushleft}
The umpires may then report the matter to the ICC Match Referee who shall take such action as
\end{flushleft}


\begin{flushleft}
is considered appropriate against the player concerned.
\end{flushleft}





\begin{flushleft}
42 PLAYERS' CONDUCT
\end{flushleft}


42.1





\begin{flushleft}
Serious misconduct
\end{flushleft}





42.1.1





\begin{flushleft}
The umpires shall act upon any serious misconduct. The relevant offences and the corresponding actions by
\end{flushleft}


\begin{flushleft}
the umpires are identified in clause 42.2.1. These offences correspond with Level 4 offences in the ICC
\end{flushleft}


\begin{flushleft}
Code of Conduct. Level 1 to Level 3 offences continue to be dealt with separately under the ICC Code of
\end{flushleft}


\begin{flushleft}
Conduct.
\end{flushleft}





42.1.2





\begin{flushleft}
If either umpire considers that a player has committed one of these offences at any time during the match,
\end{flushleft}


\begin{flushleft}
the umpire concerned shall call and signal Dead ball. This call may be delayed until the umpire is satisfied
\end{flushleft}


\begin{flushleft}
that it will not disadvantage the non-offending side.
\end{flushleft}





42.1.3





\begin{flushleft}
The umpire concerned shall report the matter to the other umpire and together they shall decide whether an
\end{flushleft}


\begin{flushleft}
offence has been committed. The umpires may also consult with the third umpire and the match referee,
\end{flushleft}


\begin{flushleft}
who may review any audio or video replays to confirm whether an offence has been committed. If so, the
\end{flushleft}


\begin{flushleft}
umpires shall then apply the related sanctions.
\end{flushleft}





42.1.4





\begin{flushleft}
If the offence is committed by a batsman, the umpires shall summon the offending player's captain to the
\end{flushleft}


\begin{flushleft}
field. Solely for the purpose of this clause, the batsmen at the wicket may not deputise for their captain.
\end{flushleft}





42.2





\begin{flushleft}
Level 4 offences and action by umpires
\end{flushleft}





42.2.1





\begin{flushleft}
Any of the following actions by a player shall constitute a Level 4 offence:
\end{flushleft}


\begin{flushleft}
- threatening to assault an umpire
\end{flushleft}


\begin{flushleft}
- making inappropriate and deliberate physical contact with an umpire
\end{flushleft}


\begin{flushleft}
- physically assaulting a player or any other person
\end{flushleft}


\begin{flushleft}
- committing any other act of violence.
\end{flushleft}





42.2.2





\begin{flushleft}
If such an offence is committed, 42.2.2.1 to 42.2.2.5 shall be implemented.
\end{flushleft}


42.2.2.1





\begin{flushleft}
The umpire shall call Time.
\end{flushleft}





42.2.2.2





\begin{flushleft}
Together the umpires shall summon and inform the offending player's captain that an offence at
\end{flushleft}


\begin{flushleft}
this Level has occurred.
\end{flushleft}





58





\newpage
42.2.2.3





42.2.2.4





\begin{flushleft}
The umpires shall instruct the captain to remove the offending player immediately from the field
\end{flushleft}


\begin{flushleft}
of play for the remainder of the match and shall apply the following:
\end{flushleft}


42.2.2.3.1





\begin{flushleft}
If the offending player is a fielder, no substitute shall be allowed for him. He is to
\end{flushleft}


\begin{flushleft}
be recorded as Retired -- out at the commencement of any subsequent innings in
\end{flushleft}


\begin{flushleft}
which his team is the batting side.
\end{flushleft}





42.2.2.3.2





\begin{flushleft}
If a bowler is suspended mid-over, then that over must be completed by a
\end{flushleft}


\begin{flushleft}
different bowler, who shall not have bowled the previous over nor shall be
\end{flushleft}


\begin{flushleft}
permitted to bowl the next over.
\end{flushleft}





42.2.2.3.3





\begin{flushleft}
If the offending player is a batsman he is to be recorded as Retired -- out in the
\end{flushleft}


\begin{flushleft}
current innings, unless he has been dismissed under any of clauses 32 to 39, and
\end{flushleft}


\begin{flushleft}
at the commencement of any subsequent innings in which his team is the batting
\end{flushleft}


\begin{flushleft}
side. If no further batsman is available to bat, the innings is completed.
\end{flushleft}





\begin{flushleft}
As soon as practicable, the umpire shall:
\end{flushleft}


\begin{flushleft}
- award 5 Penalty runs to the opposing team
\end{flushleft}


\begin{flushleft}
- signal the Level 4 penalty to the scorers
\end{flushleft}


\begin{flushleft}
- call Play.
\end{flushleft}





42.2.2.5





\begin{flushleft}
The umpires shall then report the matter to the ICC Match Referee under the ICC Code of
\end{flushleft}


\begin{flushleft}
Conduct.
\end{flushleft}





42.3





\begin{flushleft}
Captain refusing to remove a player from the field
\end{flushleft}





42.3.1





\begin{flushleft}
If a captain refuses to carry out an instruction under 42.2.2.3, the umpires shall invoke clause 16.2 (ICC
\end{flushleft}


\begin{flushleft}
Match Referee awarding a match).
\end{flushleft}





42.3.2





\begin{flushleft}
If both captains refuse to carry out instructions under 42.2.2.3 in respect of the same incident, the umpires
\end{flushleft}


\begin{flushleft}
shall instruct the players to leave the field. The match is not concluded as in clause 16.2 and there shall be
\end{flushleft}


\begin{flushleft}
no result under clause 16.
\end{flushleft}





42.4





\begin{flushleft}
Additional points relating to Level 4 offences
\end{flushleft}





42.4.1





\begin{flushleft}
If a player, while acting as wicket-keeper, commits a Level 4 offence, clause 24.1.2 shall not apply, meaning
\end{flushleft}


\begin{flushleft}
that only a nominated player may keep wicket, even if another fielder becomes injured or ill and is replaced
\end{flushleft}


\begin{flushleft}
by a substitute.
\end{flushleft}





42.4.2





\begin{flushleft}
A nominated player who has a substitute will also suffer the penalty for any Level 4 offence committed by
\end{flushleft}


\begin{flushleft}
the substitute. However, only the substitute will be reported under clause 42.2.2.5.
\end{flushleft}





59





\begin{flushleft}
\newpage
Appendices to ICC Twenty20 International Playing Conditions
\end{flushleft}


\begin{flushleft}
(incorporating the 2017 Code of the MCC Laws of Cricket)
\end{flushleft}


\begin{flushleft}
Effective 1 October 2017
\end{flushleft}





\begin{flushleft}
A.
\end{flushleft}





\begin{flushleft}
Definitions
\end{flushleft}





\begin{flushleft}
B.
\end{flushleft}





\begin{flushleft}
Equipment
\end{flushleft}





\begin{flushleft}
C.
\end{flushleft}





1.





\begin{flushleft}
The bat
\end{flushleft}





2.





\begin{flushleft}
The wickets
\end{flushleft}





3.





\begin{flushleft}
Wicket-keeping gloves
\end{flushleft}





\begin{flushleft}
The venue
\end{flushleft}


1.





\begin{flushleft}
The pitch and the creases
\end{flushleft}





2.





\begin{flushleft}
Advertising on grounds, perimeter boards and sight-screens
\end{flushleft}





3.





\begin{flushleft}
Markings on outfield
\end{flushleft}





\begin{flushleft}
D.
\end{flushleft}





\begin{flushleft}
Decision Review System (DRS) and Third Umpire Protocol
\end{flushleft}





\begin{flushleft}
E.
\end{flushleft}





\begin{flushleft}
Calculations
\end{flushleft}





\begin{flushleft}
F.
\end{flushleft}





\begin{flushleft}
The Super Over
\end{flushleft}





60





\begin{flushleft}
\newpage
Appendix A
\end{flushleft}


\begin{flushleft}
Definitions
\end{flushleft}





1





\begin{flushleft}
The match
\end{flushleft}





1.1





\begin{flushleft}
The game is used in these Playing Conditions as a general term meaning the Game of Cricket.
\end{flushleft}





1.2





\begin{flushleft}
A match is a single Twenty20 International match between two teams, played under these Playing
\end{flushleft}


\begin{flushleft}
Conditions.
\end{flushleft}





1.3





\begin{flushleft}
T20I is an abbreviation for Twenty20 International.
\end{flushleft}





1.4





\begin{flushleft}
A Super Over is a procedure that may be adopted for determining the result of a tied match, as set out in
\end{flushleft}


\begin{flushleft}
Appendix F.
\end{flushleft}





1.5





\begin{flushleft}
The toss is the toss for choice of innings.
\end{flushleft}





1.6





\begin{flushleft}
Before the toss is at any time before the toss on the day of the match.
\end{flushleft}





1.7





\begin{flushleft}
Before the match is at any time before the toss, not restricted to the day of the match.
\end{flushleft}





1.8





\begin{flushleft}
During the match is at any time after the toss until the conclusion of the match, whether play is in progress or
\end{flushleft}


\begin{flushleft}
not.
\end{flushleft}





1.9





\begin{flushleft}
Playing time is any time between the call of Play and the call of Time. See clauses 12.1 (Call of Play) and
\end{flushleft}


\begin{flushleft}
12.2 (Call of Time).
\end{flushleft}





1.10





\begin{flushleft}
Conduct of the match includes any action relevant to the match at any time.
\end{flushleft}





1.11





\begin{flushleft}
Ground Authority is the entity responsible for the selection and preparation of the pitch and other functions
\end{flushleft}


\begin{flushleft}
relating to the hosting and management of the match, including any agents acting on their behalf (including
\end{flushleft}


\begin{flushleft}
but not limited to the curator or other ground staff).
\end{flushleft}





1.12





\begin{flushleft}
Home Board is the ICC member responsible for the home team and the hosting of the match.
\end{flushleft}





1.13





\begin{flushleft}
Visiting Board is the ICC member responsible for the visiting team.
\end{flushleft}





1.14





\begin{flushleft}
The Spirit of Cricket refers to the values of respect and fair play that underpin the game of cricket, as set out
\end{flushleft}


\begin{flushleft}
in the Preamble to these Playing Conditions.
\end{flushleft}





1.15





\begin{flushleft}
The ICC Code of Conduct is the ICC Code of Conduct for Players and Player Support Personnel, as
\end{flushleft}


\begin{flushleft}
amended from time to time.
\end{flushleft}





\begin{flushleft}
2 Implements and equipment
\end{flushleft}


2.1





\begin{flushleft}
Implements used in the match are the bat, the ball, the stumps and bails.
\end{flushleft}





2.2





\begin{flushleft}
External protective equipment is any visible item of apparel worn for protection against external blows.
\end{flushleft}


\begin{flushleft}
For a batsman, items permitted are a protective helmet, external leg guards (batting pads), batting gloves
\end{flushleft}


\begin{flushleft}
and, if visible, forearm guards.
\end{flushleft}


\begin{flushleft}
For a fielder, only a protective helmet is permitted, except in the case of a wicket-keeper, for whom wicketkeeping pads and gloves are also permitted.
\end{flushleft}





2.3





\begin{flushleft}
A protective helmet is headwear made of hard material and designed to protect the head or the face or both,
\end{flushleft}


\begin{flushleft}
which shall (in line with the Clothing and Equipment Regulations) be certified to BS7928:2013. For the
\end{flushleft}


\begin{flushleft}
purposes of interpreting these Playing Conditions, such a description will include faceguards.
\end{flushleft}





2.4





\begin{flushleft}
Equipment -- a batsman's equipment is his/her bat as defined above, together with any external protective
\end{flushleft}


\begin{flushleft}
equipment he is wearing.
\end{flushleft}


\begin{flushleft}
A fielder's equipment is any external protective equipment that he is wearing.
\end{flushleft}





61





\newpage
2.5





2.6





\begin{flushleft}
The bat -- the following are to be considered as part of the bat:
\end{flushleft}


-





\begin{flushleft}
the whole of the bat itself.
\end{flushleft}





-





\begin{flushleft}
the whole of a glove (or gloves) worn on the hand (or hands) holding the bat.
\end{flushleft}





\begin{flushleft}
hands.
\end{flushleft}





\begin{flushleft}
the hand (or hands) holding the bat, if the batsman is not wearing a glove on that hand or on those
\end{flushleft}





\begin{flushleft}
Held in batsman's hand. Contact between a batsman's hand, or glove worn on his/her hand, and any part of
\end{flushleft}


\begin{flushleft}
the bat shall constitute the bat being held in that hand.
\end{flushleft}





\begin{flushleft}
3 The playing area
\end{flushleft}


3.1





\begin{flushleft}
The field of play is the area contained within the boundary.
\end{flushleft}





3.2





\begin{flushleft}
The square is a specially prepared area of the field of play within which the match pitch is situated.
\end{flushleft}





3.3





\begin{flushleft}
The outfield is that part of the field of play between the square and the boundary.
\end{flushleft}





\begin{flushleft}
4 Positioning
\end{flushleft}


4.1





\begin{flushleft}
Behind the popping crease at one end of the pitch is that area of the field of play, including any other
\end{flushleft}


\begin{flushleft}
marking, objects and persons therein, that is on that side of the popping crease that does not include the
\end{flushleft}


\begin{flushleft}
creases at the opposite end of the pitch. Behind, in relation to any other marking, object or person, follows
\end{flushleft}


\begin{flushleft}
the same principle. See the diagram in paragraph 13.
\end{flushleft}





4.2





\begin{flushleft}
In front of the popping crease at one end of the pitch is that area of the field of play, including any other
\end{flushleft}


\begin{flushleft}
marking, objects and persons therein, that is on that side of the popping crease that includes the creases at
\end{flushleft}


\begin{flushleft}
the opposite end of the pitch. In front of, in relation to any other marking, object or person, follows the same
\end{flushleft}


\begin{flushleft}
principle. See the diagram in paragraph 13.
\end{flushleft}





4.3





\begin{flushleft}
The striker's end is the place where the striker stands to receive a delivery from the bowler only insofar as it
\end{flushleft}


\begin{flushleft}
identifies, independently of where the striker may subsequently move, one end of the pitch.
\end{flushleft}





4.4





\begin{flushleft}
The bowler's end is the end from which the bowler delivers the ball. It is the other end of the pitch from the
\end{flushleft}


\begin{flushleft}
striker's end and identifies that end of the pitch that is not the striker's end as described in paragraph 4.3.
\end{flushleft}





4.5





\begin{flushleft}
The wicket-keeper's end is the same as the striker's end as described in paragraph 4.3.
\end{flushleft}





4.6





\begin{flushleft}
In front of the line of the striker's wicket is in the area of the field of play in front of the imaginary line joining
\end{flushleft}


\begin{flushleft}
the fronts of the stumps at the striker's end; this line to be considered extended in both directions to the
\end{flushleft}


\begin{flushleft}
boundary. See paragraph 4.2.
\end{flushleft}





4.7





\begin{flushleft}
Behind the wicket is in the area of the field of play behind the imaginary line joining the backs of the stumps
\end{flushleft}


\begin{flushleft}
at the appropriate end; this line to be considered extended in both directions to the boundary. See paragraph
\end{flushleft}


4.1.





4.8





\begin{flushleft}
Behind the wicket-keeper is behind the wicket at the striker's end, as defined above, but in line with both
\end{flushleft}


\begin{flushleft}
sets of stumps and further from the stumps than the wicket-keeper.
\end{flushleft}





4.9





\begin{flushleft}
Off side/on (leg) side -- see diagram in paragraph 13
\end{flushleft}





4.10





\begin{flushleft}
Inside edge is the edge on the same side as the nearer wicket.
\end{flushleft}





\begin{flushleft}
5 Umpires and decision-making
\end{flushleft}


5.1





\begin{flushleft}
Umpire -- where the description the umpire is used on its own, it always means {`}the bowler's end umpire'
\end{flushleft}


\begin{flushleft}
though this full description is sometimes used for emphasis or clarity. Similarly the umpires always means
\end{flushleft}


\begin{flushleft}
both umpires and the third umpire. An umpire and umpires are generalised terms. Otherwise, a fuller
\end{flushleft}


\begin{flushleft}
description indicates which one of the umpires is specifically intended. Each umpire will be bowler's end
\end{flushleft}


\begin{flushleft}
umpire and striker's end umpire in alternate overs.
\end{flushleft}





62





\newpage
5.2





\begin{flushleft}
Bowler's end umpire is the umpire who is standing at the bowler's end (see paragraph 4.4) for the current
\end{flushleft}


\begin{flushleft}
delivery.
\end{flushleft}





5.3





\begin{flushleft}
Striker's end umpire is the umpire who is standing at the striker's end (see paragraph 4.3), to one side of the
\end{flushleft}


\begin{flushleft}
pitch or the other, depending on his/her choice, for the current delivery.
\end{flushleft}





5.4





\begin{flushleft}
On-field umpires shall mean, collectively, the bowler's end umpire and the striker's end umpire.
\end{flushleft}





5.5





\begin{flushleft}
Third umpire is the umpire who may use television evidence and other available technology in order review a
\end{flushleft}


\begin{flushleft}
decision of the on-field umpires, either by way of an Umpire Review or a Player Review under the protocol set
\end{flushleft}


\begin{flushleft}
out in Appendix D.
\end{flushleft}





5.6





\begin{flushleft}
Umpires together agree applies to decisions which the umpires are to make jointly, independently of the
\end{flushleft}


\begin{flushleft}
players.
\end{flushleft}





5.7





\begin{flushleft}
Decision Review System or DRS is the process covered by the Decision Review System and Third Umpire
\end{flushleft}


\begin{flushleft}
Protocol set out in Appendix D, under which the third umpire may be consulted in relation to a decision of the
\end{flushleft}


\begin{flushleft}
on-field umpires, either by way of an Umpire Review or a Player Review.
\end{flushleft}





5.8





\begin{flushleft}
Player Review is the process set out in Appendix D by which a player may request a review of any decision
\end{flushleft}


\begin{flushleft}
taken by the on-field umpires concerning whether or not a batsman is dismissed (with the exception of {`}Timed
\end{flushleft}


\begin{flushleft}
out').
\end{flushleft}





5.9





\begin{flushleft}
Umpire Review is the process set out in Appendix D by which an on-field umpire has the discretion to refer a
\end{flushleft}


\begin{flushleft}
decision to the third umpire or, under certain circumstances, to consult with the third umpire before making a
\end{flushleft}


\begin{flushleft}
decision.
\end{flushleft}





5.10





\begin{flushleft}
Soft Signal is the visual communication by the bowler's end umpire to the third umpire (accompanied by
\end{flushleft}


\begin{flushleft}
additional information via two-way radio where necessary) of his/her initial on-field decision prior to initiating
\end{flushleft}


\begin{flushleft}
an Umpire Review.
\end{flushleft}





5.11





\begin{flushleft}
Umpire's Call is the concept within the DRS under which the on-field decision of the bowler's end umpire
\end{flushleft}


\begin{flushleft}
shall stand, which shall apply under the specific circumstances set out in paragraphs 3.4.5 and 3.4.6 of
\end{flushleft}


\begin{flushleft}
Appendix D, where the ball-tracking technology indicates a marginal decision in respect of either the Impact
\end{flushleft}


\begin{flushleft}
Zone or the Wicket Zone.
\end{flushleft}





5.12





\begin{flushleft}
The Pitching Zone as used in the DRS is a two dimensional area on the pitch between both sets of stumps
\end{flushleft}


\begin{flushleft}
with its boundaries consisting of the base of both sets of stumps and a line between the outside of the outer
\end{flushleft}


\begin{flushleft}
stumps at each end.
\end{flushleft}





5.13





\begin{flushleft}
The Impact Zone as used in the DRS is a three dimensional space extending between both sets of stumps to
\end{flushleft}


\begin{flushleft}
an indefinite height vertically and with its boundaries consisting of the base of the stumps and the outside of
\end{flushleft}


\begin{flushleft}
the outer stumps at each end.
\end{flushleft}





5.14





\begin{flushleft}
The Wicket Zone as used in the DRS is a two dimensional area with its boundaries consisting of the outside
\end{flushleft}


\begin{flushleft}
of the outer stumps, the base of the stumps, and the lower edge of the bails.
\end{flushleft}





5.15





\begin{flushleft}
A Fair Catch is a catch that has been taken cleanly by the fielder in accordance with clause 33.
\end{flushleft}





5.16





\begin{flushleft}
A Bump Ball is where the ball has made contact with the ground shortly after making contact with the striker's
\end{flushleft}


\begin{flushleft}
bat.
\end{flushleft}





5.17





\begin{flushleft}
The Elite Panel is the group of umpires contracted to the ICC to officiate in international cricket.
\end{flushleft}





5.18





\begin{flushleft}
The International Panel is the group of umpires nominated by the ICC's full members in accordance with
\end{flushleft}


\begin{flushleft}
clause 2.1.3 of the Playing Conditions.
\end{flushleft}





\begin{flushleft}
6 Batsmen
\end{flushleft}


6.1





\begin{flushleft}
Batting side is the side currently batting, whether or not play is in progress.
\end{flushleft}





6.2





\begin{flushleft}
Member of the batting side is one of the players nominated by the captain of the batting side, or any
\end{flushleft}


\begin{flushleft}
authorised replacement for such nominated player.
\end{flushleft}





6.3





\begin{flushleft}
A batsman's ground -- at each end of the pitch, the whole area of the field of play behind the popping crease
\end{flushleft}


\begin{flushleft}
is the ground at that end for a batsman.
\end{flushleft}





63





\newpage
6.4





\begin{flushleft}
Original end is the end where a batsman was when the ball came into play for that delivery.
\end{flushleft}





6.5





\begin{flushleft}
Wicket he has left is the wicket at the end where a batsman was at the start of the run in progress.
\end{flushleft}





6.6





\begin{flushleft}
Guard position is the position and posture adopted by the striker to receive a ball delivered by the bowler
\end{flushleft}





\begin{flushleft}
7 Fielders
\end{flushleft}


7.1





\begin{flushleft}
Fielding side is the side currently fielding, whether or not play is in progress.
\end{flushleft}





7.2





\begin{flushleft}
Member of the fielding side is one of the players nominated by the captain of the fielding side, or any
\end{flushleft}


\begin{flushleft}
authorised replacement or substitute for such nominated player.
\end{flushleft}





7.3





\begin{flushleft}
Fielder is one of the 11 or fewer players who together represent the fielding side on the field of play. This
\end{flushleft}


\begin{flushleft}
definition includes not only both the bowler and the wicket-keeper but also nominated players who are
\end{flushleft}


\begin{flushleft}
legitimately on the field of play, together with players legitimately acting as substitutes for absent nominated
\end{flushleft}


\begin{flushleft}
players. It excludes any nominated player who is absent from the field of play, or who has been absent from
\end{flushleft}


\begin{flushleft}
the field of play and who has not yet obtained the umpire's permission to return.
\end{flushleft}


\begin{flushleft}
A player going briefly outside the boundary in the course of discharging his/her duties as a fielder is not
\end{flushleft}


\begin{flushleft}
absent from the field of play nor, for the purposes of clause 24.2 (Fielder absent or leaving the field of play),
\end{flushleft}


\begin{flushleft}
is he to be regarded as having left the field of play.
\end{flushleft}





\begin{flushleft}
8 Substitutes
\end{flushleft}


8.1





\begin{flushleft}
A Substitute is a player who takes the place of a fielder on the field of play, but does not replace the player
\end{flushleft}


\begin{flushleft}
for whom he substitutes on that side's list of nominated players. A substitute's activities are limited to fielding.
\end{flushleft}





\begin{flushleft}
9 Bowlers
\end{flushleft}


9.1





\begin{flushleft}
Over the wicket / round the wicket -- If, as the bowler runs up between the wicket and the return crease, the
\end{flushleft}


\begin{flushleft}
wicket is on the same side as his bowling arm, he is bowling over the wicket. If the return crease is on the
\end{flushleft}


\begin{flushleft}
same side as his bowling arm, he is bowling round the wicket.
\end{flushleft}





9.2





\begin{flushleft}
Delivery swing is the motion of the bowler's arm during which he normally releases the ball for a delivery.
\end{flushleft}





9.3





\begin{flushleft}
Delivery stride is the stride during which the delivery swing is made, whether the ball is released or not. It
\end{flushleft}


\begin{flushleft}
starts when the bowler's back foot lands for that stride and ends when the front foot lands in the same stride.
\end{flushleft}


\begin{flushleft}
The stride after the delivery stride is completed when the next foot lands, i.e. when the back foot of the
\end{flushleft}


\begin{flushleft}
delivery stride lands again.
\end{flushleft}





9.4





\begin{flushleft}
The Illegal Bowling Regulations are the ICC's regulations governing Illegal Bowling Actions.
\end{flushleft}





9.5





\begin{flushleft}
An Illegal Bowling Action is a bowling action where a bowler's Elbow Extension exceeds 15 degrees,
\end{flushleft}


\begin{flushleft}
measured from the point at which the bowling arm reaches the horizontal until the point at which the ball is
\end{flushleft}


\begin{flushleft}
released (any Elbow Hyperextension shall be discounted for the purposes of determining an Illegal Bowling
\end{flushleft}


\begin{flushleft}
Action).
\end{flushleft}





9.6





\begin{flushleft}
Elbow Extension means the motion that occurs when a bowler's arm moves from a flexed (bent) position at
\end{flushleft}


\begin{flushleft}
the elbow, to a more extended (straight) position (full Elbow Extension occurs when the arm is straight).
\end{flushleft}





9.7





\begin{flushleft}
Elbow Hyperextension is the motion that occurs when a bowler's elbow extends beyond the straight position.
\end{flushleft}





9.8





\begin{flushleft}
The ICC Bowling Action Report Form is the form provided for by Article 3 of the Illegal Bowling Regulations,
\end{flushleft}


\begin{flushleft}
by which an umpire and/or the ICC Match Referee may submit a report relating to a suspected Illegal Bowling
\end{flushleft}


\begin{flushleft}
Action.
\end{flushleft}





\begin{flushleft}
10 The ball
\end{flushleft}


10.1





\begin{flushleft}
The ball is struck/strikes the ball unless specifically defined otherwise, mean {`}the ball is struck by the
\end{flushleft}


\begin{flushleft}
bat'/{`}strikes the ball with the bat'.
\end{flushleft}





64





\newpage
10.2





\begin{flushleft}
Rebounds directly/strikes directly and similar phrases mean {`}without contact with any fielder' but do not
\end{flushleft}


\begin{flushleft}
exclude contact with the ground.
\end{flushleft}





10.3





\begin{flushleft}
Full-pitch describes a ball delivered by the bowler that reaches or passes the striker without having touched
\end{flushleft}


\begin{flushleft}
the ground. Sometimes described as non-pitching.
\end{flushleft}





\begin{flushleft}
11 Runs
\end{flushleft}


11.1





\begin{flushleft}
A run to be disallowed is one that in these Playing Conditions should not have been taken. It is not only to
\end{flushleft}


\begin{flushleft}
be cancelled but the batsmen are to be returned to their original ends.
\end{flushleft}





11.2





\begin{flushleft}
A run not to be scored is one that is not illegal, but is not recognised as a properly executed run. It is not a
\end{flushleft}


\begin{flushleft}
run that has been made, so the question of cancellation does not arise. The loss of the run so attempted is
\end{flushleft}


\begin{flushleft}
not a disallowance and the batsmen will not be returned to their original ends on that account.
\end{flushleft}





\begin{flushleft}
12 The person
\end{flushleft}


12.1





\begin{flushleft}
Person; A player's person is his/her physical person (flesh and blood) together with any clothing or legitimate
\end{flushleft}


\begin{flushleft}
external protective equipment that he is wearing except, in the case of a batsman, his/her bat.
\end{flushleft}


\begin{flushleft}
A hand, whether gloved or not, that is not holding the bat is part of the batsman's person.
\end{flushleft}


\begin{flushleft}
No item of clothing or equipment is part of the player's person unless it is attached to him.
\end{flushleft}


\begin{flushleft}
For a batsman, a glove being held but not worn is part of his/her person.
\end{flushleft}


\begin{flushleft}
For a fielder, an item of clothing or equipment he is holding in his/her hand or hands is not part of his person.
\end{flushleft}





12.2





\begin{flushleft}
Clothing -- anything that a player is wearing, including such items as spectacles or jewellery, that is not
\end{flushleft}


\begin{flushleft}
classed as external protective equipment is classed as clothing, even though he may be wearing some items
\end{flushleft}


\begin{flushleft}
of apparel, which are not visible, for protection. A bat being carried by a batsman does not come within this
\end{flushleft}


\begin{flushleft}
definition of clothing.
\end{flushleft}





12.3





\begin{flushleft}
Hand for batsman or wicket-keeper shall include both the hand itself and the whole of a glove worn on the
\end{flushleft}


\begin{flushleft}
hand.
\end{flushleft}





65





\begin{flushleft}
\newpage
13 Off side / on side; in front of / behind the popping crease.
\end{flushleft}





66





\begin{flushleft}
\newpage
Appendix B
\end{flushleft}


\begin{flushleft}
Equipment
\end{flushleft}





1





\begin{flushleft}
The Bat
\end{flushleft}





1.1





\begin{flushleft}
General guidance
\end{flushleft}





1.1.1





\begin{flushleft}
Measurements - All provisions in paragraphs 1.2 to 1.6 below are subject to the measurements and
\end{flushleft}


\begin{flushleft}
restrictions stated in the Playing Conditions and this Appendix.
\end{flushleft}





1.1.2





\begin{flushleft}
Adhesives -- Throughout, adhesives are permitted only where essential and only in minimal quantity.
\end{flushleft}





1.2





\begin{flushleft}
Specifications for the Handle
\end{flushleft}





1.2.1





\begin{flushleft}
One end of the handle is inserted into a recess in the blade as a means of joining the handle and the blade.
\end{flushleft}


\begin{flushleft}
This lower portion is used purely for joining the blade and the handle together. It is not part of the blade but,
\end{flushleft}


\begin{flushleft}
solely in interpreting paragraphs 1.3 and 1.4 below, references to the blade shall be considered to extend
\end{flushleft}


\begin{flushleft}
also to this lower portion of the handle where relevant.
\end{flushleft}





1.2.2





\begin{flushleft}
The handle may be glued where necessary and bound with twine along the upper portion.
\end{flushleft}


\begin{flushleft}
Providing clause 5.5 is not contravened, the upper portion may be covered with materials solely to provide a
\end{flushleft}


\begin{flushleft}
surface suitable for gripping. Such covering is an addition and is not part of the bat, except in relation to
\end{flushleft}


\begin{flushleft}
clause 5.6. The bottom of this grip should not extend below the point defined in paragraph 1.2.4 below.
\end{flushleft}


\begin{flushleft}
Twine binding and the covering grip may extend beyond the junction of the upper and lower portions of the
\end{flushleft}


\begin{flushleft}
handle, to cover part of the shoulders of the bat as defined in paragraph 1.3.1.
\end{flushleft}


\begin{flushleft}
No material may be placed on or inserted into the lower portion of the handle other than as permitted above
\end{flushleft}


\begin{flushleft}
together with the minimal adhesives or adhesive tape used solely for fixing these items, or for fixing the
\end{flushleft}


\begin{flushleft}
handle to the blade.
\end{flushleft}





1.2.3





\begin{flushleft}
Materials in handle -- As a proportion of the total volume of the handle, materials other than cane, wood or
\end{flushleft}


\begin{flushleft}
twine are restricted to one-tenth. Such materials must not project more than 3.25 in/8.26 cm into the lower
\end{flushleft}


\begin{flushleft}
portion of the handle
\end{flushleft}





1.2.4





\begin{flushleft}
Binding and covering of handle -- The permitted continuation beyond the junction of the upper and lower
\end{flushleft}


\begin{flushleft}
portions of the handle is restricted to a maximum, measured along the length of the handle, of
\end{flushleft}


\begin{flushleft}
2.5 in/6.35 cm in for the twine binding
\end{flushleft}


\begin{flushleft}
2.75 in/6.99 cm for the covering grip.
\end{flushleft}





1.3





\begin{flushleft}
Specifications for the Blade
\end{flushleft}





1.3.1





\begin{flushleft}
The blade has a face, a back, a toe, sides and shoulders
\end{flushleft}


1.3.1.1





\begin{flushleft}
The face of the blade is its main striking surface and shall be flat or have a slight convex curve
\end{flushleft}


\begin{flushleft}
resulting from traditional pressing techniques. The back is the opposite surface.
\end{flushleft}





1.3.1.2





\begin{flushleft}
The shoulders, sides and toe are the remaining surfaces, separating the face and the back.
\end{flushleft}





1.3.1.3





\begin{flushleft}
The shoulders, one on each side of the handle, are along that portion of the blade between the
\end{flushleft}


\begin{flushleft}
first entry point of the handle and the point at which the blade first reaches its full width.
\end{flushleft}





1.3.1.4





\begin{flushleft}
The toe is the surface opposite to the shoulders taken as a pair.
\end{flushleft}





1.3.1.5





\begin{flushleft}
The sides, one each side of the blade, are along the rest of the blade, between the toe and the
\end{flushleft}


\begin{flushleft}
shoulders.
\end{flushleft}





67





\newpage
1.3.2





\begin{flushleft}
No material may be placed on or inserted into the blade other than as permitted in paragraph 1.2.4,
\end{flushleft}


\begin{flushleft}
paragraph 1.3.3, and clause 5.4 together with the minimal adhesives or adhesive tape used solely for fixing
\end{flushleft}


\begin{flushleft}
these items, or for fixing the handle to the blade.
\end{flushleft}





1.3.3





\begin{flushleft}
Covering the blade. Bats shall have no covering on the blade except as permitted in clause 5.4.
\end{flushleft}


\begin{flushleft}
Any materials referred to above, in clause 5.4 and paragraph 1.4 below, are to be considered as part of the
\end{flushleft}


\begin{flushleft}
bat, which must still pass through the gauge as defined in paragraph 1.6.
\end{flushleft}





1.4





\begin{flushleft}
Protection and repair
\end{flushleft}





1.4.1





\begin{flushleft}
The surface of the blade may be treated with non-solid materials to improve resistance to moisture
\end{flushleft}


\begin{flushleft}
penetration and/or mask natural blemishes in the appearance of the wood. Save for the purpose of giving a
\end{flushleft}


\begin{flushleft}
homogeneous appearance by masking natural blemishes, such treatment shall not materially alter the colour
\end{flushleft}


\begin{flushleft}
of the blade.
\end{flushleft}





1.4.2





\begin{flushleft}
Materials can be used for protection and repair as stated in clause 5.4 and are additional to the blade. Note
\end{flushleft}


\begin{flushleft}
however clause 5.6.
\end{flushleft}


\begin{flushleft}
Any such material shall not extend over any part of the back of the blade except in the case of clause 5.4.1
\end{flushleft}


\begin{flushleft}
and then only when it is applied as a continuous wrapping covering the damaged area.
\end{flushleft}


\begin{flushleft}
The repair material shall not extend along the length of the blade more than 0.79 in/2.0 cm in each direction
\end{flushleft}


\begin{flushleft}
beyond the limits of the damaged area. Where used as a continuous binding, any overlapping shall not
\end{flushleft}


\begin{flushleft}
breach the maximum of 0.04 in/0.1 cm in total thickness.
\end{flushleft}


\begin{flushleft}
The use of non-solid material which when dry forms a hard layer more than 0.004 in/0.01 cm in thickness is
\end{flushleft}


\begin{flushleft}
not permitted.
\end{flushleft}





1.4.3





\begin{flushleft}
Permitted coverings, repair material and toe guards, not exceeding their specified thicknesses, may be
\end{flushleft}


\begin{flushleft}
additional to the dimensions above, but the bat must still pass through the gauge as described in paragraph
\end{flushleft}


1.6.





1.5





\begin{flushleft}
Commercial identifications
\end{flushleft}


\begin{flushleft}
Such identifications shall comply with the restrictions set out in the Clothing and Equipment Regulations in
\end{flushleft}


\begin{flushleft}
relation to the size and position of marks and logos.
\end{flushleft}





1.6





\begin{flushleft}
Bat Gauge
\end{flushleft}


\begin{flushleft}
All bats must meet the specifications defined in clause 5.7. They must also, with or without protective
\end{flushleft}


\begin{flushleft}
coverings permitted in clause 5.4, be able to pass through a bat gauge, the dimensions and shape of which
\end{flushleft}


\begin{flushleft}
are shown in the following diagram:
\end{flushleft}





68





\begin{flushleft}
\newpage
2 The wickets
\end{flushleft}





2.1





\begin{flushleft}
Bails
\end{flushleft}


\begin{flushleft}
Overall 4.31 in / 10.95 cm
\end{flushleft}


\begin{flushleft}
a = 1.38 in / 3.50 cm
\end{flushleft}


\begin{flushleft}
b = 2.13 in / 5.40 cm
\end{flushleft}


\begin{flushleft}
c = 0.81 in / 2.06 cm
\end{flushleft}





2.2





\begin{flushleft}
Stumps
\end{flushleft}


\begin{flushleft}
Height (d) = 28 in / 71.1 cm
\end{flushleft}


\begin{flushleft}
Diameter (e) - maximum = 1.5 in / 3.81 cm; minimum = 1.38 in / 3.50 cm
\end{flushleft}





2.3





\begin{flushleft}
Overall
\end{flushleft}


\begin{flushleft}
Width (f) of wicket 9 in / 22.86 cm
\end{flushleft}





\begin{flushleft}
3 Wicket-keeping gloves
\end{flushleft}


3.1





\begin{flushleft}
The images below illustrate the requirements of clause 27.2 in relation to:
\end{flushleft}


$\bullet$





\begin{flushleft}
no webbing between the fingers;
\end{flushleft}





$\bullet$





\begin{flushleft}
a single piece of non-stretch material between finger and thumb as a means of support; and
\end{flushleft}





$\bullet$





\begin{flushleft}
when a hand wearing the glove has the thumb fully extended, the top edge being taut and not
\end{flushleft}


\begin{flushleft}
protruding beyond the straight line joining the top of the index finger to the top of the thumb.
\end{flushleft}





69





\newpage
3.2





\begin{flushleft}
Note also the requirement for wicket-keeping gloves to comply with the Clothing and Equipment Regulations
\end{flushleft}


\begin{flushleft}
in relation to the size and position of marks and logos.
\end{flushleft}





70





\begin{flushleft}
\newpage
Appendix C
\end{flushleft}


\begin{flushleft}
The venue
\end{flushleft}





1





\begin{flushleft}
The pitch and the creases
\end{flushleft}





\begin{flushleft}
2 Restriction on the placement of fielders
\end{flushleft}





71





\begin{flushleft}
\newpage
3 Advertising on grounds, perimeter boards and sightscreens
\end{flushleft}


3.1





\begin{flushleft}
Advertising on grounds
\end{flushleft}





3.1.1





\begin{flushleft}
The logos on outfields are to be positioned as follows:
\end{flushleft}


\begin{flushleft}
(a)
\end{flushleft}





\begin{flushleft}
Behind the stumps -- a minimum of 25.15 yards (23 meters) from the stumps.
\end{flushleft}





\begin{flushleft}
(b)
\end{flushleft}





\begin{flushleft}
Midwicket/cover area -- no advertising to be positioned within 30 yards (27.50 meters) of the centre
\end{flushleft}


\begin{flushleft}
of the pitch being used for the match.
\end{flushleft}





3.1.2





\begin{flushleft}
Note: Advertising closer to the stumps as set out above which is required to meet 3D requirements for
\end{flushleft}


\begin{flushleft}
broadcasters may be permitted, subject to prior ICC approval having been obtained.
\end{flushleft}





3.2





\begin{flushleft}
Perimeter boards
\end{flushleft}





3.2.1





\begin{flushleft}
Advertising on perimeter boards placed in front of the sight-screens is permitted save that the predominant
\end{flushleft}


\begin{flushleft}
colour of such advertising shall be of a contrasting colour to that of the ball.
\end{flushleft}





3.2.2





\begin{flushleft}
Advertising on perimeter boards behind the stumps at both ends shall not contain moving, flashing or
\end{flushleft}


\begin{flushleft}
flickering images and operators should ensure that the images are only changed or moved at a time that will
\end{flushleft}


\begin{flushleft}
not be distracting to the players or the umpires.
\end{flushleft}





3.2.3





\begin{flushleft}
The brightness of any electronic images shall be set at a level so that it is not a distraction to the players or
\end{flushleft}


\begin{flushleft}
umpires.
\end{flushleft}





3.3





\begin{flushleft}
Sight-screens
\end{flushleft}





3.3.1





\begin{flushleft}
Sight-screens shall be provided at both ends of all grounds.
\end{flushleft}





3.3.2





\begin{flushleft}
Advertising shall be permitted on the sight-screen behind the striker, providing it is removed for the
\end{flushleft}


\begin{flushleft}
subsequent over from that end.
\end{flushleft}





3.3.3





\begin{flushleft}
Such advertising shall not contain flashing or flickering images and particular care should be taken by the
\end{flushleft}


\begin{flushleft}
operators that the advertising is not changed at a time which is distracting to the umpire.
\end{flushleft}





\begin{flushleft}
4 Markings on outfield
\end{flushleft}


\begin{flushleft}
With the permission of the Ground Authority, a bowler may use paint to make a small marking on the outfield
\end{flushleft}


\begin{flushleft}
for the purposes of identifying their run-up. Paint used for this purpose shall be any colour other than white.
\end{flushleft}





72





\begin{flushleft}
\newpage
Appendix D
\end{flushleft}


\begin{flushleft}
Decision Review System (DRS) and Third Umpire Protocol
\end{flushleft}





1


1.1





\begin{flushleft}
General
\end{flushleft}


\begin{flushleft}
Minimum requirements for use of DRS and appointment of third umpire
\end{flushleft}





1.1.1





\begin{flushleft}
Save with the express written consent of the ICC General Manager - Cricket, the Home Board shall ensure
\end{flushleft}


\begin{flushleft}
the live television broadcast of all T20I matches played in its country.
\end{flushleft}





1.1.2





\begin{flushleft}
Where matches are broadcast, the camera specification set out below shall be mandatory as a minimum
\end{flushleft}


\begin{flushleft}
requirement.
\end{flushleft}





1.1.3





\begin{flushleft}
Where the camera specification set out above is provided, a third umpire shall be appointed to the match.
\end{flushleft}





1.1.4





\begin{flushleft}
The provisions of paragraphs 1.1.1, 1.1.2, and 1.1.3 above shall not apply for matches between a Full
\end{flushleft}


\begin{flushleft}
Member country and Associate Member countries (whose matches have been granted T20I status) and for
\end{flushleft}


\begin{flushleft}
matches between such Associate Member countries.
\end{flushleft}





1.1.5





\begin{flushleft}
If the minimum requirements for DRS to be used are satisfied, both participating Boards may agree to
\end{flushleft}


\begin{flushleft}
employ the DRS for a T20I match. Otherwise, the third umpire shall be appointed and empowered to use
\end{flushleft}


\begin{flushleft}
broadcast replays to make decisions that are referred to him/her in accordance with paragraph 2 (Umpire
\end{flushleft}


\begin{flushleft}
Reviews).
\end{flushleft}





1.1.6





\begin{flushleft}
The table below summarises the minimum requirements for DRS to be used, and the regulations around the
\end{flushleft}


\begin{flushleft}
appointment of the third umpire:
\end{flushleft}





\begin{flushleft}
Minimum
\end{flushleft}


\begin{flushleft}
Requirement
\end{flushleft}





\begin{flushleft}
Third Umpire (non-DRS)
\end{flushleft}





\begin{flushleft}
DRS
\end{flushleft}





\begin{flushleft}
Cameras Specification detailed in paragraph 1.1.2
\end{flushleft}





\begin{flushleft}
Cameras
\end{flushleft}


-





-





73





\begin{flushleft}
Specification detailed in
\end{flushleft}


\begin{flushleft}
paragraph 1.1.2.
\end{flushleft}


\begin{flushleft}
Technology
\end{flushleft}


\begin{flushleft}
Approved ball-tracking
\end{flushleft}


\begin{flushleft}
technology.
\end{flushleft}


\begin{flushleft}
Approved sound-based
\end{flushleft}


\begin{flushleft}
edge detection
\end{flushleft}





\begin{flushleft}
\newpage
technology.
\end{flushleft}


\begin{flushleft}
Third
\end{flushleft}


\begin{flushleft}
Umpire
\end{flushleft}


\begin{flushleft}
Appointment
\end{flushleft}





\begin{flushleft}
Appointed by Home Board.
\end{flushleft}





\begin{flushleft}
Appointed by the Home Board.
\end{flushleft}





\begin{flushleft}
From the home country.
\end{flushleft}





\begin{flushleft}
From ICC Elite Panel or
\end{flushleft}


\begin{flushleft}
International Panel of umpires.
\end{flushleft}





\begin{flushleft}
From ICC Elite Panel or International Panel of
\end{flushleft}


\begin{flushleft}
umpires.
\end{flushleft}





1.1.7





\begin{flushleft}
Third
\end{flushleft}


\begin{flushleft}
Umpire
\end{flushleft}


\begin{flushleft}
Jurisdiction
\end{flushleft}





\begin{flushleft}
Umpire Reviews only
\end{flushleft}





\begin{flushleft}
Umpire Reviews and Player
\end{flushleft}


\begin{flushleft}
Reviews
\end{flushleft}





\begin{flushleft}
Replays that
\end{flushleft}


\begin{flushleft}
can be used
\end{flushleft}





\begin{flushleft}
The third umpire shall only have access to replays of
\end{flushleft}


\begin{flushleft}
any camera images. Other technology which may be
\end{flushleft}


\begin{flushleft}
in use by the broadcaster for broadcast purposes (for
\end{flushleft}


\begin{flushleft}
example, ball-tracking technology, sound-based edge
\end{flushleft}


\begin{flushleft}
detection technology, and heat-based edge detection
\end{flushleft}


\begin{flushleft}
technology) shall not be used during Umpire Reviews.
\end{flushleft}





\begin{flushleft}
Any replay, stump microphone
\end{flushleft}


\begin{flushleft}
audio or technology detailed in
\end{flushleft}


\begin{flushleft}
paragraph 3.8.1 below.
\end{flushleft}





\begin{flushleft}
ICC
\end{flushleft}


\begin{flushleft}
Technical
\end{flushleft}


\begin{flushleft}
Officer
\end{flushleft}





\begin{flushleft}
Not required.
\end{flushleft}





\begin{flushleft}
The ICC shall appoint an
\end{flushleft}


\begin{flushleft}
independent technology expert
\end{flushleft}


\begin{flushleft}
(ICC Technical Officer) to be
\end{flushleft}


\begin{flushleft}
present at every series in which
\end{flushleft}


\begin{flushleft}
the DRS is used to assist the third
\end{flushleft}


\begin{flushleft}
umpire and to protect the integrity
\end{flushleft}


\begin{flushleft}
of the DRS process.
\end{flushleft}





\begin{flushleft}
The Home Board shall ensure that a separate room is provided for the third umpire and that he/she has
\end{flushleft}


\begin{flushleft}
access to the television equipment and technology (where DRS is used) so as to be in the best position to
\end{flushleft}


\begin{flushleft}
facilitate the referral and/or consultation processes referred to in paragraphs 2 (Umpire Review) and 3
\end{flushleft}


\begin{flushleft}
(Player Review) below.
\end{flushleft}





\begin{flushleft}
2 Umpire Review
\end{flushleft}


\begin{flushleft}
In the circumstances detailed in paragraphs 2.1, 2.2, 2.3 and 2.4 below, the on-field umpire shall have the
\end{flushleft}


\begin{flushleft}
discretion to refer the decision to the third umpire or, in the case of paragraphs 2.2, and 2.4, to consult with
\end{flushleft}


\begin{flushleft}
the third umpire before making the decision.
\end{flushleft}


\begin{flushleft}
Save for requesting the umpire to review his/her decision under paragraph 3 (Player Review) below, players
\end{flushleft}


\begin{flushleft}
may not appeal to the on-field umpires to use the Umpire Review. Breach of this provision may constitute
\end{flushleft}


\begin{flushleft}
dissent and the player may be subject to disciplinary action under the ICC Code of Conduct for Players and
\end{flushleft}


\begin{flushleft}
Player Support Personnel.
\end{flushleft}


2.1





\begin{flushleft}
Run Out, Stumped, Bowled and Hit Wicket Decisions
\end{flushleft}





2.1.1





\begin{flushleft}
The relevant on-field umpire shall be entitled to refer an appeal for run-out, stumped, bowled or hit wicket to
\end{flushleft}


\begin{flushleft}
the third umpire.
\end{flushleft}





2.1.2





\begin{flushleft}
An on-field umpire wishing to refer a decision to the third umpire shall signal to the third umpire by making
\end{flushleft}


\begin{flushleft}
the shape of a TV screen with his/her hands.
\end{flushleft}





2.1.3





\begin{flushleft}
In the case of a referral of a bowled, hit wicket or stumped decision, the third umpire shall first check the
\end{flushleft}


\begin{flushleft}
fairness of the delivery (all modes of No ball except for the bowler using an Illegal Bowling Action, subject to
\end{flushleft}


\begin{flushleft}
the proviso that the third umpire may review whether the bowler has used a prohibited Specific Variation
\end{flushleft}


\begin{flushleft}
under Article 6.2 of the Illegal Bowling Regulations). If the delivery was not a fair delivery the third umpire
\end{flushleft}


\begin{flushleft}
shall indicate that the batsman is Not out and advise the on-field umpire to signal No ball. See also
\end{flushleft}


\begin{flushleft}
paragraph 2.5 below.
\end{flushleft}





74





\newpage
2.1.4





\begin{flushleft}
Additionally, if the third umpire finds the batsman is Out by another mode of dismissal (excluding LBW), or
\end{flushleft}


\begin{flushleft}
Not out by any mode of dismissal (excluding LBW), he/she shall notify the on-field umpire so that the correct
\end{flushleft}


\begin{flushleft}
decision is made.
\end{flushleft}





2.1.5





\begin{flushleft}
If the third umpire decides that the batsman is Out, a red light shall be displayed; if the third umpire decides
\end{flushleft}


\begin{flushleft}
that the batsman is Not out, a green light shall be displayed. Should the third umpire be temporarily unable
\end{flushleft}


\begin{flushleft}
to respond, a white light (where available) shall remain illuminated throughout the period of interruption to
\end{flushleft}


\begin{flushleft}
signify to the on-field umpires that Umpire Reviews are temporarily unavailable, in which case the decision
\end{flushleft}


\begin{flushleft}
shall be taken by the on-field umpire. As an alternative to the red/green light system, the replay screen
\end{flushleft}


\begin{flushleft}
(where available) may be used for the purpose of conveying the third umpire's decision, in line with the ICC
\end{flushleft}


\begin{flushleft}
Big Screen Policy.
\end{flushleft}





2.2





\begin{flushleft}
Caught Decisions, Obstructing the Field
\end{flushleft}





2.2.1





\begin{flushleft}
Where the bowler's end umpire is unable to decide upon a Fair Catch or a Bump Ball, or if, on appeal from
\end{flushleft}


\begin{flushleft}
the fielding side, the batsman obstructed the field, he/she shall first consult with the striker's end umpire.
\end{flushleft}





2.2.2





\begin{flushleft}
Should both on-field umpires require assistance from the third umpire to make a decision, the bowler's end
\end{flushleft}


\begin{flushleft}
umpire shall firstly take a decision on-field after consulting with the striker's end umpire, before consulting by
\end{flushleft}


\begin{flushleft}
two-way radio with the third umpire. Such consultation shall be initiated by the bowler's end umpire to the
\end{flushleft}


\begin{flushleft}
third umpire by making the shape of a TV screen with his/her hands, followed by a Soft Signal of Out or Not
\end{flushleft}


\begin{flushleft}
out made with the hands close to the chest at chest height. If the third umpire advises that the replay
\end{flushleft}


\begin{flushleft}
evidence is inconclusive, the on-field decision communicated at the start of the consultation process shall
\end{flushleft}


\begin{flushleft}
stand.
\end{flushleft}





2.2.3





\begin{flushleft}
The third umpire shall determine whether the batsman has been caught, whether the delivery was a Bump
\end{flushleft}


\begin{flushleft}
Ball, or if the batsman obstructed the field. However, in reviewing the television replay(s), the third umpire
\end{flushleft}


\begin{flushleft}
shall first check the fairness of the delivery for all decisions involving a catch (all modes of No ball except for
\end{flushleft}


\begin{flushleft}
the bowler using an Illegal Bowling Action, subject to the proviso that the third umpire may review whether
\end{flushleft}


\begin{flushleft}
the bowler has used a prohibited Specific Variation under Article 6.2 of the Illegal Bowling Regulations) and
\end{flushleft}


\begin{flushleft}
whether the batsman has hit the ball. If the delivery was not a fair delivery or if it is clear to the third umpire
\end{flushleft}


\begin{flushleft}
that the batsman did not hit the ball he/she shall indicate to the bowler's end umpire that the batsman is Not
\end{flushleft}


\begin{flushleft}
out caught, and in the case of an unfair delivery, advise the bowler's end umpire to signal No ball. See also
\end{flushleft}


\begin{flushleft}
paragraph 2.5 below. Additionally, if it is clear to the third umpire that the batsman is Out by another mode of
\end{flushleft}


\begin{flushleft}
dismissal (excluding LBW), or Not out by any mode of dismissal (excluding LBW), he/she shall notify the
\end{flushleft}


\begin{flushleft}
bowler's end umpire so that the correct decision can be made.
\end{flushleft}





2.2.4





\begin{flushleft}
The third umpire shall communicate his/her decision as set out in paragraph 2.1.5.
\end{flushleft}





2.3





\begin{flushleft}
Boundary Decisions
\end{flushleft}





2.3.1





\begin{flushleft}
The bowler's end umpire shall be entitled to refer to the third umpire for a decision on:
\end{flushleft}


2.3.1.1





\begin{flushleft}
whether a four or six has been scored;
\end{flushleft}





2.3.1.2





\begin{flushleft}
whether a fielder had any part of his/her person in contact with the ball when he touched the
\end{flushleft}


\begin{flushleft}
boundary; or
\end{flushleft}





2.3.1.3





\begin{flushleft}
whether the fielder had any part of his/her person in contact with the ball when he had any part
\end{flushleft}


\begin{flushleft}
of his person grounded beyond the boundary.
\end{flushleft}





2.3.2





\begin{flushleft}
A decision shall be made immediately and cannot be changed thereafter.
\end{flushleft}





2.3.3





\begin{flushleft}
If the television evidence is inconclusive as to whether or not a boundary has been scored, the default
\end{flushleft}


\begin{flushleft}
presumption shall be in favour of no boundary being awarded.
\end{flushleft}





2.3.4





\begin{flushleft}
Where the bowler's end umpire wishes to use the assistance of the third umpire in this circumstance, he/she
\end{flushleft}


\begin{flushleft}
shall communicate with the third umpire by use of a two-way radio and the third umpire shall convey his/her
\end{flushleft}


\begin{flushleft}
decision to the bowler's end umpire by the same method.
\end{flushleft}





2.3.5





\begin{flushleft}
The third umpire may initiate contact with the on-field umpire by two-way radio if TV coverage shows a
\end{flushleft}


\begin{flushleft}
boundary line infringement or incident that appears not to have been acted upon by the on-field umpires.
\end{flushleft}





75





\newpage
2.4





\begin{flushleft}
Batsmen Running to the Same End
\end{flushleft}





2.4.1





\begin{flushleft}
Where both batsmen have run to the same end and the on-field umpires are uncertain over which batsman
\end{flushleft}


\begin{flushleft}
made his/her ground first, the on-field umpires may consult with the third umpire.
\end{flushleft}





2.4.2





\begin{flushleft}
The procedure set out in paragraph 2.3.4 shall apply.
\end{flushleft}





2.5





\begin{flushleft}
No Balls
\end{flushleft}





2.5.1





\begin{flushleft}
If the bowler's end umpire is uncertain as to the fairness of the delivery following a dismissal, either affecting
\end{flushleft}


\begin{flushleft}
the validity of the dismissal or which batsman is dismissed, he/she shall be entitled to request the batsman
\end{flushleft}


\begin{flushleft}
to delay leaving the field and to check the fairness of the delivery with the third umpire. Communication with
\end{flushleft}


\begin{flushleft}
the third umpire shall be by two-way radio.
\end{flushleft}





2.5.2





\begin{flushleft}
The third umpire shall check all modes of No ball except for the bowler using an Illegal Bowling Action
\end{flushleft}


\begin{flushleft}
(subject to the proviso that the third umpire may review whether the bowler has used a prohibited Specific
\end{flushleft}


\begin{flushleft}
Variation under Article 6.2 of the Illegal Bowling Regulations). The third umpire shall apply clause 21.5 when
\end{flushleft}


\begin{flushleft}
deciding whether a No ball should have been called (and must therefore be satisfied that none of the three
\end{flushleft}


\begin{flushleft}
conditions in clause 21.5 have been met before calling a No ball).
\end{flushleft}





2.5.3





\begin{flushleft}
If the delivery was not a fair delivery, the bowler's end umpire shall indicate that the batsman is Not out and
\end{flushleft}


\begin{flushleft}
signal No ball (except in the case of a dismissal for obstructing the field, which may still be effected despite a
\end{flushleft}


\begin{flushleft}
No ball being called, in which case the bowler's end umpire shall indicate that the relevant batsman is Out
\end{flushleft}


\begin{flushleft}
and additionally call a No ball).
\end{flushleft}





2.5.4





\begin{flushleft}
If a No ball is called following the check by the third umpire, the batting side shall benefit from the reversal of
\end{flushleft}


\begin{flushleft}
the dismissal and the one run for the No ball, but shall not benefit from any runs that may subsequently have
\end{flushleft}


\begin{flushleft}
accrued from the delivery had the on-field umpire originally called a No ball. Where the batsmen crossed
\end{flushleft}


\begin{flushleft}
while the ball was in the air before being caught, the batsmen shall remain at the same ends as if the striker
\end{flushleft}


\begin{flushleft}
had been dismissed, but no runs shall be credited to the striker even if one (or more) runs were completed
\end{flushleft}


\begin{flushleft}
prior to the catch being taken.
\end{flushleft}





2.6





\begin{flushleft}
Cameras On or Over the Field of Play
\end{flushleft}





2.6.1





\begin{flushleft}
The on-field umpires shall be entitled to refer to the third umpire for a decision as to whether the ball has at
\end{flushleft}


\begin{flushleft}
any time during the normal course of play come into contact with any part of the camera, its apparatus or its
\end{flushleft}


\begin{flushleft}
cables above the playing area, as contemplated in clause 20.1.3.
\end{flushleft}





2.6.2





\begin{flushleft}
Where an on-field umpire wishes to use the assistance of the third umpire in this circumstance, he/she shall
\end{flushleft}


\begin{flushleft}
communicate with the third umpire by use of a two-way radio and the third umpire shall convey his/her
\end{flushleft}


\begin{flushleft}
decision to the bowler's end umpire by the same method.
\end{flushleft}





2.6.3





\begin{flushleft}
A decision shall be made immediately and cannot be changed thereafter. If the television evidence is
\end{flushleft}


\begin{flushleft}
inconclusive as to whether or not the ball has come into contact with any part of the camera, its apparatus or
\end{flushleft}


\begin{flushleft}
its cables above the playing area, the default presumption shall be in favour of no contact having been
\end{flushleft}


\begin{flushleft}
made.
\end{flushleft}





2.6.4





\begin{flushleft}
The third umpire may initiate contact with the on-field umpire by two-way radio if TV coverage shows the ball
\end{flushleft}


\begin{flushleft}
to have been in contact with any part of the camera or its cables above the playing area as envisaged under
\end{flushleft}


\begin{flushleft}
this paragraph.
\end{flushleft}





\begin{flushleft}
3 Player Review
\end{flushleft}


\begin{flushleft}
The following paragraphs shall operate in addition to and in conjunction with paragraph 2 (Umpire Review).
\end{flushleft}





3.1





\begin{flushleft}
Circumstances in which a Player Review may be requested
\end{flushleft}





3.1.1





\begin{flushleft}
A player may request a review of any decision taken by the on-field umpires concerning whether or not a
\end{flushleft}


\begin{flushleft}
batsman is dismissed, with the exception of {`}Timed Out' (Player Review).
\end{flushleft}





3.1.2





\begin{flushleft}
No other decisions made by the umpires are eligible for a Player Review with the exception of Fair
\end{flushleft}


\begin{flushleft}
Catch/Bump Ball (even after the third umpire has been consulted and the decision communicated).
\end{flushleft}





76





\newpage
3.1.3





\begin{flushleft}
Only the batsman involved in a dismissal may request a Player Review of an Out decision and only the
\end{flushleft}


\begin{flushleft}
captain (or acting captain) of the fielding team may request a Player Review of a Not out decision.
\end{flushleft}





3.1.4





\begin{flushleft}
A decision concerning whether or not a batsman is dismissed that could have been the subject of a Umpire
\end{flushleft}


\begin{flushleft}
Review under paragraph 2 is eligible for a Player Review as soon as it is clear that the on-field umpire has
\end{flushleft}


\begin{flushleft}
chosen not to initiate the Umpire Review.
\end{flushleft}





3.2





\begin{flushleft}
The manner of requesting the Player Review
\end{flushleft}





3.2.1





\begin{flushleft}
The request shall be made by the player making a {`}T' sign with both forearms at head height.
\end{flushleft}





3.2.2





\begin{flushleft}
The total time elapsed between the ball becoming dead and the review request being made shall be no
\end{flushleft}


\begin{flushleft}
more than 15 seconds. The only exception permitted shall be when an Umpire Review for Fair Catch or
\end{flushleft}


\begin{flushleft}
Bump Ball (as permitted in paragraph 2.2 above) is required to answer an appeal for a caught decision, in
\end{flushleft}


\begin{flushleft}
which case either team is able to request a Player Review of that caught decision within 15 seconds of the
\end{flushleft}


\begin{flushleft}
decision being communicated. The bowler's end umpire shall provide the relevant player with a prompt after
\end{flushleft}


\begin{flushleft}
10 seconds if the request has not been made at that time and the player shall request the review
\end{flushleft}


\begin{flushleft}
immediately thereafter. If the on-field umpires believe that a request has not been made within the 15
\end{flushleft}


\begin{flushleft}
second time limit, they shall decline the request for a Player Review.
\end{flushleft}





3.2.3





\begin{flushleft}
The captain may consult with the bowler and other fielders, and the two batsmen may consult with each
\end{flushleft}


\begin{flushleft}
other prior to deciding whether to request a Player Review. Under no circumstances is any player permitted
\end{flushleft}


\begin{flushleft}
to query an umpire about any aspect of a decision before deciding on whether or not to request a Player
\end{flushleft}


\begin{flushleft}
Review. If the on-field umpires believe that the captain or either batsman has received direct or indirect input
\end{flushleft}


\begin{flushleft}
emanating other than from the players on the field, then they may at their discretion decline the request for a
\end{flushleft}


\begin{flushleft}
Player Review. In particular, signals from the dressing room must not be given.
\end{flushleft}





3.2.4





\begin{flushleft}
No replays, either at normal speed or slow motion, shall be shown on a big screen to spectators until the 15
\end{flushleft}


\begin{flushleft}
second time limit allowed for requesting a Player Review has elapsed. The only exception to this provision is
\end{flushleft}


\begin{flushleft}
where a Player Review of a caught decision is requested after the Umpire Review of a Fair Catch or Bump
\end{flushleft}


\begin{flushleft}
Ball has concluded, as detailed in paragraph 3.2.2 above (due to the fact that replays may have been shown
\end{flushleft}


\begin{flushleft}
on the big screen during that Umpire Review process).
\end{flushleft}





3.2.5





\begin{flushleft}
Where either on-field umpire initiates an Umpire Review, this does not preclude a player seeking a Player
\end{flushleft}


\begin{flushleft}
Review of a separate incident from the same delivery. The request for a Player Review may be made after
\end{flushleft}


\begin{flushleft}
the Umpire Review, provided the request is still within the 15 second time limit described in paragraph 3.2.2
\end{flushleft}


\begin{flushleft}
above. (See paragraphs 3.9.2 and 3.9.3 below for the process for addressing both an Umpire and Player
\end{flushleft}


\begin{flushleft}
Review).
\end{flushleft}





3.2.6





\begin{flushleft}
A request for a Player Review cannot be withdrawn once it has been made.
\end{flushleft}





3.3





\begin{flushleft}
The process of consultation
\end{flushleft}





3.3.1





\begin{flushleft}
On receipt of an eligible and timely request for a Player Review, the relevant on-field umpire shall make the
\end{flushleft}


\begin{flushleft}
sign of a shape of a TV screen with his/her hands in the normal way.
\end{flushleft}





3.3.2





\begin{flushleft}
The relevant on-field umpire shall initiate communication with the third umpire by confirming;
\end{flushleft}





3.3.3





3.3.2.1





\begin{flushleft}
That a Player Review has been requested,
\end{flushleft}





3.3.2.2





\begin{flushleft}
The mode of dismissal for which the relevant on-field umpire adjudicated the appeal,
\end{flushleft}





3.3.2.3





\begin{flushleft}
The decision that has been made (Out or Not out), and;
\end{flushleft}





3.3.2.4





\begin{flushleft}
For LBW appeals, where relevant, if the bowler's end umpire believed that the striker made no
\end{flushleft}


\begin{flushleft}
genuine attempt to play the ball with the bat (the default presumption of the third umpire in the
\end{flushleft}


\begin{flushleft}
absence of any information on this point from the bowler's end umpire shall be that a genuine
\end{flushleft}


\begin{flushleft}
attempt to play the ball with the bat was made).
\end{flushleft}





\begin{flushleft}
A two-way consultation process shall begin to investigate whether there is anything that the third umpire can
\end{flushleft}


\begin{flushleft}
see or hear which would indicate that the on-field umpire should change his/her original decision.
\end{flushleft}





77





\newpage
3.3.4





\begin{flushleft}
The third umpire shall not withhold any factual information which may help in the decision making process.
\end{flushleft}


\begin{flushleft}
In particular, in reviewing a dismissal, if the third umpire believes that the batsman may instead be Out by
\end{flushleft}


\begin{flushleft}
any other mode of dismissal, he/she shall advise the on-field umpire accordingly. The process of
\end{flushleft}


\begin{flushleft}
consultation described in this paragraph in respect of such other mode of dismissal shall then be conducted
\end{flushleft}


\begin{flushleft}
as if the batsman has been given Not out.
\end{flushleft}





3.3.5





\begin{flushleft}
The third umpire shall initially check all modes of No ball except for the bowler using an Illegal Bowling
\end{flushleft}


\begin{flushleft}
Action (subject to the proviso that the third umpire may review whether the bowler has used a prohibited
\end{flushleft}


\begin{flushleft}
Specific Variation under Article 6.2 of the Illegal Bowling Regulations), where appropriate advising the onfield umpire accordingly.
\end{flushleft}





3.3.6





\begin{flushleft}
If despite the available technology, the third umpire is unable to decide with a high degree of confidence
\end{flushleft}


\begin{flushleft}
whether the original on-field decision should be changed, then he/she shall report that the replays are
\end{flushleft}


\begin{flushleft}
{`}inconclusive', and that the on-field decision shall stand. The third umpire shall not give answers conveying
\end{flushleft}


\begin{flushleft}
likelihoods or probabilities.
\end{flushleft}





3.3.7





\begin{flushleft}
In circumstances where the television technology (all or parts thereof) is not available to the third umpire or
\end{flushleft}


\begin{flushleft}
fails for whatever reason, the third umpire shall advise the on-field umpire of this fact but still provide any
\end{flushleft}


\begin{flushleft}
relevant factual information that may be ascertained from the available television replays and other
\end{flushleft}


\begin{flushleft}
technology.
\end{flushleft}





3.3.8





\begin{flushleft}
The on-field umpire shall then make his/her decision based on the information provided by the third umpire,
\end{flushleft}


\begin{flushleft}
any other factual information offered by the third umpire and his/her recollection and opinion of the original
\end{flushleft}


\begin{flushleft}
incident.
\end{flushleft}





3.3.9





\begin{flushleft}
The on-field umpire shall reverse his/her decision if the nature of the supplementary information received
\end{flushleft}


\begin{flushleft}
from the third umpire leads him/her to conclude that his/her original decision was incorrect.
\end{flushleft}





3.4





\begin{flushleft}
Review of LBW Decisions
\end{flushleft}





3.4.1





\begin{flushleft}
In assessing whether a batsman is Out LBW in accordance with clause 36, the third umpire shall first judge
\end{flushleft}


\begin{flushleft}
whether the delivery is fair (as set out in clause 36.1.1), and second, whether or not the ball has touched the
\end{flushleft}


\begin{flushleft}
bat before being intercepted by any part of the striker's person (as set out in clause 36.1.3).
\end{flushleft}





3.4.2





\begin{flushleft}
If the batsman is still eligible to be Out, the ball-tracking technology shall then present three pieces of
\end{flushleft}


\begin{flushleft}
information to the third umpire relating to the path of the ball:
\end{flushleft}


3.4.2.1





\begin{flushleft}
The point of pitching (where applicable) (PITCHING)
\end{flushleft}





3.4.2.2





\begin{flushleft}
The position of the ball at the point of first interception (IMPACT)
\end{flushleft}





3.4.2.3





\begin{flushleft}
Whether the ball would have hit the wicket (WICKET)
\end{flushleft}





3.4.3





\begin{flushleft}
This Decision Review System (DRS) and Third Umpire Protocol includes a category of Umpire's Call, which
\end{flushleft}


\begin{flushleft}
shall be the conclusion reported where the technology indicates a marginal decision in respect of either the
\end{flushleft}


\begin{flushleft}
point of first interception or whether the ball would have hit the stumps.
\end{flushleft}





3.4.4





\begin{flushleft}
PITCHING
\end{flushleft}


3.4.4.1





\begin{flushleft}
The interpretation of {``}pitches in line between wicket and wicket'' in clause 36.1.2 shall refer to
\end{flushleft}


\begin{flushleft}
the position of the centre of the ball at the point of pitching, in relation to the Pitching Zone.
\end{flushleft}





3.4.4.2





\begin{flushleft}
The Pitching Zone is defined as a two dimensional area on the pitch between both sets of
\end{flushleft}


\begin{flushleft}
stumps with its boundaries consisting of the base of both sets of stumps and a line between the
\end{flushleft}


\begin{flushleft}
outside of the outer stumps at each end.
\end{flushleft}





3.4.4.3





\begin{flushleft}
Where applicable, the ball-tracking technology shall report that the ball pitched in one of the
\end{flushleft}


\begin{flushleft}
following three areas in relation to the Pitching Zone:
\end{flushleft}





\begin{flushleft}
In Line
\end{flushleft}





\begin{flushleft}
The centre of the ball was inside the Pitching Zone
\end{flushleft}





\begin{flushleft}
Outside Off
\end{flushleft}





\begin{flushleft}
The centre of the ball was outside, and to the off side of, the Pitching Zone
\end{flushleft}





78





\begin{flushleft}
\newpage
Outside Leg
\end{flushleft}





3.4.4.4





3.4.5





3.4.6





\begin{flushleft}
The centre of the ball was outside, and to the leg side of, the Pitching Zone
\end{flushleft}





\begin{flushleft}
Subject to the satisfaction of the other elements of clause 36.1, the batsman can be Out if the
\end{flushleft}


\begin{flushleft}
ball-tracking technology reports that the ball pitched Outside Off or In Line, but the batsman
\end{flushleft}


\begin{flushleft}
shall be Not out if the ball pitched Outside Leg.
\end{flushleft}





\begin{flushleft}
IMPACT
\end{flushleft}


3.4.5.1





\begin{flushleft}
The interpretation of {``}the (first) point of impact, even if in above the level of the bails, is between
\end{flushleft}


\begin{flushleft}
wicket and wicket'' in clause 36.1.4 shall refer to position of the ball at the point of first
\end{flushleft}


\begin{flushleft}
interception, in relation to the Impact Zone.
\end{flushleft}





3.4.5.2





\begin{flushleft}
The Impact Zone is defined as a three dimensional space extending between both wickets to an
\end{flushleft}


\begin{flushleft}
indefinite height and with its boundaries consisting of a line between the outside of the outer
\end{flushleft}


\begin{flushleft}
stumps at each end.
\end{flushleft}





3.4.5.3





\begin{flushleft}
The ball-tracking technology shall report that the point of first interception was in one of the
\end{flushleft}


\begin{flushleft}
following categories in relation to the Impact Zone:
\end{flushleft}


\begin{flushleft}
In Line
\end{flushleft}





\begin{flushleft}
The centre of the ball was inside the Impact Zone
\end{flushleft}





\begin{flushleft}
Umpire's Call
\end{flushleft}





\begin{flushleft}
Some part of the ball was inside the Impact Zone, but the centre of the
\end{flushleft}


\begin{flushleft}
ball was outside the Impact Zone, with the further sub-category of
\end{flushleft}


\begin{flushleft}
{`}Umpire's Call (off side)' where the centre of the ball was to the off side of
\end{flushleft}


\begin{flushleft}
the Impact Zone and the bowler's end umpire communicates to the third
\end{flushleft}


\begin{flushleft}
umpire that no genuine attempt to play the ball was made by the
\end{flushleft}


\begin{flushleft}
batsman.
\end{flushleft}





\begin{flushleft}
Outside
\end{flushleft}





\begin{flushleft}
No part of the ball was inside the Impact Zone, with the further subcategories of {`}Outside (off)' and {`}Outside (leg)' to indicate the location of
\end{flushleft}


\begin{flushleft}
the point of first interception in relation to the Impact Zone when the
\end{flushleft}


\begin{flushleft}
bowler's end umpire communicates to the third umpire that no genuine
\end{flushleft}


\begin{flushleft}
attempt to play the ball was made by the batsman.
\end{flushleft}





3.4.5.4





\begin{flushleft}
Where a Not out decision is being reviewed, and it is judged that the batsman has made a
\end{flushleft}


\begin{flushleft}
genuine attempt to play the ball, the ball-tracking technology must report that the point of first
\end{flushleft}


\begin{flushleft}
interception was In Line for the batsman to be eligible to be given Out, otherwise the batsman
\end{flushleft}


\begin{flushleft}
shall remain Not out.
\end{flushleft}





3.4.5.5





\begin{flushleft}
Where a Not out decision is being reviewed, and it is judged that the batsman has made no
\end{flushleft}


\begin{flushleft}
genuine attempt to play the ball, the ball-tracking technology must report that the point of impact
\end{flushleft}


\begin{flushleft}
was In Line, or Umpire's Call (off side), or Outside (off) for the batsman to be eligible to be given
\end{flushleft}


\begin{flushleft}
Out, otherwise the batsman shall remain Not out.
\end{flushleft}





3.4.5.6





\begin{flushleft}
Where an Out decision is being reviewed, and it is judged that the batsman has made a genuine
\end{flushleft}


\begin{flushleft}
attempt to play the ball, the ball-tracking technology must report that the point of first
\end{flushleft}


\begin{flushleft}
interception was Outside for the decision to be reversed to Not out, otherwise the batsman shall
\end{flushleft}


\begin{flushleft}
remain eligible to be given Out.
\end{flushleft}





3.4.5.7





\begin{flushleft}
Where an Out decision is being reviewed, and it is judged that the batsman has made no
\end{flushleft}


\begin{flushleft}
genuine attempt to play the ball, the ball-tracking technology must report that the point of first
\end{flushleft}


\begin{flushleft}
interception was Outside (leg) for the decision to be reversed to Not out, otherwise the batsman
\end{flushleft}


\begin{flushleft}
shall remain eligible to be given Out.
\end{flushleft}





\begin{flushleft}
WICKET
\end{flushleft}


3.4.6.1





\begin{flushleft}
The interpretation of whether {``}the ball would have hit the wicket'' in clause 36.1.5 shall refer to
\end{flushleft}


\begin{flushleft}
the position of the ball as it either hits or passes the wicket, in relation to the Wicket Zone.
\end{flushleft}





3.4.6.2





\begin{flushleft}
The Wicket Zone is defined as a two dimensional area whose boundaries are the outside of the
\end{flushleft}


\begin{flushleft}
outer stumps, the base of the stumps and the bottom of the bails.
\end{flushleft}





79





\newpage
3.4.6.3





3.4.6.4





\begin{flushleft}
The ball-tracking technology shall report whether the ball would have hit the wicket with
\end{flushleft}


\begin{flushleft}
reference to the following three categories:
\end{flushleft}


\begin{flushleft}
Hitting
\end{flushleft}





\begin{flushleft}
The ball was hitting the wicket, and the centre of the ball was inside the
\end{flushleft}


\begin{flushleft}
Wicket Zone
\end{flushleft}





\begin{flushleft}
Umpire's Call
\end{flushleft}





\begin{flushleft}
The ball was hitting the wicket, but the centre of the ball was not inside
\end{flushleft}


\begin{flushleft}
the Wicket Zone
\end{flushleft}





\begin{flushleft}
Missing
\end{flushleft}





\begin{flushleft}
The ball was missing the wicket
\end{flushleft}





\begin{flushleft}
Where a Not out decision is being reviewed, the ball-tracking technology must report that the
\end{flushleft}


\begin{flushleft}
ball was Hitting for the batsman to be eligible to be given Out, otherwise the batsman shall
\end{flushleft}


\begin{flushleft}
remain Not out.
\end{flushleft}


\begin{flushleft}
However, where the evidence shows that the ball was Hitting, the point of first interception was
\end{flushleft}


\begin{flushleft}
In Line, and the ball pitched In Line or Outside Off, but that:
\end{flushleft}


$\bullet$





\begin{flushleft}
The point of first interception was 300cm or more from the stumps; or
\end{flushleft}





$\bullet$


\begin{flushleft}
The point of first interception was more than 250cm but less than 300cm from the stumps
\end{flushleft}


\begin{flushleft}
and the distance between the point of pitching and the point of first interception was less than
\end{flushleft}


\begin{flushleft}
40cm,
\end{flushleft}


\begin{flushleft}
the on-field decision shall stand (that is, Not out).
\end{flushleft}


3.4.6.5





\begin{flushleft}
Where an Out decision is being reviewed, the ball-tracking technology must report that the ball
\end{flushleft}


\begin{flushleft}
was Missing for the on-field decision to be reversed to Not out, otherwise the batsman shall
\end{flushleft}


\begin{flushleft}
remain eligible to be given Out.
\end{flushleft}





3.4.7





\begin{flushleft}
When the ball strikes the batsman on the full, and the evidence provided by the ball-tracking technology
\end{flushleft}


\begin{flushleft}
indicates that the ball would have pitched before striking or passing the wicket, there will be no information
\end{flushleft}


\begin{flushleft}
available from that delivery that will allow the ball-tracking technology to accurately predict the height of the
\end{flushleft}


\begin{flushleft}
ball after pitching.
\end{flushleft}





3.4.8





\begin{flushleft}
With regard to determining whether the ball would have hit the wicket under these circumstances, the balltracking technology shall project the line of the ball in accordance with clause 36.2.3 (it is to be assumed
\end{flushleft}


\begin{flushleft}
that the path of the ball before interception would have continued after interception, irrespective of whether
\end{flushleft}


\begin{flushleft}
the ball might have pitched subsequently or not), and display the simulated path of the ball from directly
\end{flushleft}


\begin{flushleft}
above the wicket.
\end{flushleft}





3.4.9





\begin{flushleft}
The third umpire shall advise the bowler's end umpire only on the point of first interception and whether the
\end{flushleft}


\begin{flushleft}
ball would have hit the stumps (in line with the process set out in paragraph 3.4 above), but shall make no
\end{flushleft}


\begin{flushleft}
comment on the predicted height of the ball after pitching, which shall remain a judgment of the bowler's end
\end{flushleft}


\begin{flushleft}
umpire.
\end{flushleft}





3.5





\begin{flushleft}
The process for communicating the final decision
\end{flushleft}





3.5.1





\begin{flushleft}
For Player Reviews concerning potential dismissals, the relevant on-field umpire shall indicate Out by raising
\end{flushleft}


\begin{flushleft}
his/her finger above his/her head in a normal yet prominent manner or indicate Not out by the call of {`}not out'
\end{flushleft}


\begin{flushleft}
and by crossing his/her hands in a horizontal position side to side in front and above his/her waist three
\end{flushleft}


\begin{flushleft}
times. Where the decision is a reversal of the on-field umpire's previous decision, he/she shall make the
\end{flushleft}


\begin{flushleft}
{`}revoke last signal' indication immediately prior to the above.
\end{flushleft}





3.5.2





\begin{flushleft}
If the mode of dismissal is not obvious or not the same as that on which the original decision was based,
\end{flushleft}


\begin{flushleft}
then the umpire shall advise the scorers via the third umpire.
\end{flushleft}





3.6





\begin{flushleft}
Number of Player Review requests permitted
\end{flushleft}





3.6.1





\begin{flushleft}
In each innings, each team shall be allowed to make a maximum of one Player Review request that is
\end{flushleft}


\begin{flushleft}
categorised as {`}Unsuccessful' (as set out in paragraph 3.6.3 below).
\end{flushleft}





80





\newpage
3.6.2





\begin{flushleft}
Where a request for a Player Review results in the original on-field decision being reversed, then the Player
\end{flushleft}


\begin{flushleft}
Review shall be categorised as {`}Successful' and shall not count towards the innings limit.
\end{flushleft}





3.6.3





\begin{flushleft}
Where a request for a Player Review results in the original on-field decision remaining unchanged (other
\end{flushleft}


\begin{flushleft}
than in the circumstances set out in paragraphs 3.6.4, 3.6.6 or 3.6.8), the Player Review shall be
\end{flushleft}


\begin{flushleft}
categorised as 'Unsuccessful'.
\end{flushleft}





3.6.4





\begin{flushleft}
Where a request for a Player Review of an LBW decision results in the on-field decision remaining
\end{flushleft}


\begin{flushleft}
unchanged solely on the basis of an Umpire's Call, the Player Review shall be categorised as {`}Unchanged --
\end{flushleft}


\begin{flushleft}
Umpire's Call'. A Player Review categorised as {`}Unchanged -- Umpire's Call' shall not count towards the
\end{flushleft}


\begin{flushleft}
innings limit set out in paragraph 3.6.1.
\end{flushleft}





3.6.5





\begin{flushleft}
Where, following a request for a Player Review, the original on-field decision of Out is unchanged, but for a
\end{flushleft}


\begin{flushleft}
different mode of dismissal from the original on-field decision, then the Player Review shall still be
\end{flushleft}


\begin{flushleft}
categorised as 'Unsuccessful'.
\end{flushleft}





3.6.6





\begin{flushleft}
Where, following a request for a Player Review, the original on-field decision of Not out is unchanged on
\end{flushleft}


\begin{flushleft}
account of the delivery being a No ball (for any reason), thereby not requiring any further evaluation, the
\end{flushleft}


\begin{flushleft}
Player Review shall not be counted as {`}Unsuccessful' and accordingly shall not count towards the innings
\end{flushleft}


\begin{flushleft}
limit set out in paragraph 3.6.1.
\end{flushleft}





3.6.7





\begin{flushleft}
Where a Player Review and an Umpire Review are requested from the same delivery and the decision of
\end{flushleft}


\begin{flushleft}
the third umpire from the Umpire Review renders the Player Review unnecessary (see paragraphs 3.9.2 and
\end{flushleft}


\begin{flushleft}
3.9.3), the Player Review request shall be disregarded and accordingly shall not count towards the innings
\end{flushleft}


\begin{flushleft}
limit set out in paragraph 3.6.1.
\end{flushleft}





3.6.8





\begin{flushleft}
A Player Review categorised as {`}Unsuccessful' may be reinstated by the ICC Match Referee at his/her sole
\end{flushleft}


\begin{flushleft}
discretion (if appropriate after consultation with the ICC Technical Official and/or the television broadcast
\end{flushleft}


\begin{flushleft}
director) if the Player Review could not properly be concluded due to a failure of the technology. Any such
\end{flushleft}


\begin{flushleft}
decision shall be final and shall be taken as soon as possible, being communicated to both teams once all
\end{flushleft}


\begin{flushleft}
the relevant facts have been ascertained by the ICC Match Referee. A Player Review categorised as
\end{flushleft}


\begin{flushleft}
{`}Unsuccessful' shall not be reinstated if, despite any technical failures, the correct decision could still have
\end{flushleft}


\begin{flushleft}
been made using the other available technology. Similarly, a Player Review categorised as {`}Unsuccessful'
\end{flushleft}


\begin{flushleft}
shall not be reinstated where the technology worked as intended, but the evidence gleaned from its use was
\end{flushleft}


\begin{flushleft}
inconclusive.
\end{flushleft}





3.6.9





\begin{flushleft}
The third umpire shall be responsible for counting the number Player Reviews categorised as {`}Unsuccessful'
\end{flushleft}


\begin{flushleft}
and shall advise the on-field umpires once either team has exhausted their allowance for the innings.
\end{flushleft}





3.6.10





\begin{flushleft}
The scoreboard shall display, for the innings in progress, the number of Player Reviews remaining available
\end{flushleft}


\begin{flushleft}
to each team.
\end{flushleft}


\begin{flushleft}
Category of Player Review
\end{flushleft}





\begin{flushleft}
Outcome of Player Review
\end{flushleft}





\begin{flushleft}
Consequence of Player Review
\end{flushleft}





\begin{flushleft}
Successful (paragraph 3.6.2)
\end{flushleft}





\begin{flushleft}
On-field decision reversed
\end{flushleft}





\begin{flushleft}
Does not count towards innings
\end{flushleft}


\begin{flushleft}
limit set out in paragraph 3.6.1
\end{flushleft}





\begin{flushleft}
Unsuccessful (paragraphs 3.6.3
\end{flushleft}


\begin{flushleft}
and 3.6.5)
\end{flushleft}





\begin{flushleft}
On-field decision unchanged
\end{flushleft}





\begin{flushleft}
Counts towards innings limit set
\end{flushleft}


\begin{flushleft}
out in paragraph 3.6.1
\end{flushleft}





\begin{flushleft}
Unchanged -- Umpire's Call
\end{flushleft}


\begin{flushleft}
(paragraph 3.6.4)
\end{flushleft}





\begin{flushleft}
On-field decision unchanged
\end{flushleft}





\begin{flushleft}
Does not count towards innings
\end{flushleft}


\begin{flushleft}
limit set out in paragraph 3.6.1
\end{flushleft}





\begin{flushleft}
No ball -- no evaluation required
\end{flushleft}


\begin{flushleft}
(paragraph 3.6.6)
\end{flushleft}





\begin{flushleft}
On-field decision unchanged
\end{flushleft}





\begin{flushleft}
Does not count towards innings
\end{flushleft}


\begin{flushleft}
limit set out in paragraph 3.6.1
\end{flushleft}





\begin{flushleft}
Failure of technology (paragraph
\end{flushleft}


3.6.8)





\begin{flushleft}
On-field decision unchanged
\end{flushleft}





\begin{flushleft}
Does not count towards innings
\end{flushleft}


\begin{flushleft}
limit set out in paragraph 3.6.1
\end{flushleft}





81





\newpage
3.7





\begin{flushleft}
Dead ball
\end{flushleft}





3.7.1





\begin{flushleft}
If following a Player Review request, an original decision of Out is changed to Not out, then the ball is still
\end{flushleft}


\begin{flushleft}
deemed to have become dead when the original decision was made (as per clause 20.1.1.3). The batting
\end{flushleft}


\begin{flushleft}
side, while benefiting from the reversal of the dismissal, shall not benefit from any runs that may
\end{flushleft}


\begin{flushleft}
subsequently have accrued from the delivery had the on-field umpire originally made a Not out decision,
\end{flushleft}


\begin{flushleft}
other than any No ball penalty that could arise under paragraph 3.3.5 above.
\end{flushleft}





3.7.2





\begin{flushleft}
If an original decision of Not out is changed to Out, the ball shall retrospectively be deemed to have become
\end{flushleft}


\begin{flushleft}
dead from the moment of the dismissal event. All subsequent events, including any runs scored, shall be
\end{flushleft}


\begin{flushleft}
ignored.
\end{flushleft}





3.8





\begin{flushleft}
Use of technology
\end{flushleft}





3.8.1





\begin{flushleft}
The following technology may be used by the third umpire during a Player Review:
\end{flushleft}


3.8.1.1





\begin{flushleft}
Replays, at any speed, from any available broadcast camera
\end{flushleft}





3.8.1.2





\begin{flushleft}
Sound from the stump microphones with the replays at normal speed and slow motion
\end{flushleft}





3.8.1.3





\begin{flushleft}
Approved ball-tracking technology:
\end{flushleft}





3.8.1.4





3.8.1.5





$\bullet$





\begin{flushleft}
HawkEye (HawkEye Innovations), or;
\end{flushleft}





$\bullet$





\begin{flushleft}
VirtualEye (ARL)
\end{flushleft}





\begin{flushleft}
Approved sound-based edge detection technology:
\end{flushleft}


$\bullet$





\begin{flushleft}
Real-Time Snickometer (BBG Sports), or;
\end{flushleft}





$\bullet$





\begin{flushleft}
UltraEdge (HawkEye Innovations)
\end{flushleft}





\begin{flushleft}
Approved heat-based edge detection technology:
\end{flushleft}


$\bullet$





3.8.1.6





\begin{flushleft}
Hot Spot cameras (BBG Sports)
\end{flushleft}





\begin{flushleft}
LED Wickets (using the lights to indicate if the wicket is broken, as set out in paragraph 4.2):
\end{flushleft}


$\bullet$





\begin{flushleft}
Zing Bails and Stumps
\end{flushleft}





3.8.2





\begin{flushleft}
In addition, other forms of technology may be used subject to the ICC being satisfied that the required
\end{flushleft}


\begin{flushleft}
standards of accuracy and time efficiency can be met.
\end{flushleft}





3.8.3





\begin{flushleft}
Where practical usage or further testing indicates that any of the above forms of technology cannot reliably
\end{flushleft}


\begin{flushleft}
provide accurate and timely information, then it may be removed prior to or during a match. The final
\end{flushleft}


\begin{flushleft}
decision regarding the technology to be used in a given match shall be taken by the ICC Match Referee in
\end{flushleft}


\begin{flushleft}
consultation with the ICC Technical Official, ICC management and the competing teams' governing bodies.
\end{flushleft}





3.9





\begin{flushleft}
Combining Umpire Review with Player Review
\end{flushleft}





3.9.1





\begin{flushleft}
If an Umpire Review (under paragraph 2) and a request for a Player Review (under paragraph 3) are made
\end{flushleft}


\begin{flushleft}
following the same delivery but relating to separate modes of dismissal, the following process shall apply.
\end{flushleft}





3.9.2





\begin{flushleft}
The Umpire Review shall be carried out prior to the Player Review if all of the following conditions apply:
\end{flushleft}





3.9.3





3.9.2.1





\begin{flushleft}
The Player Review has been requested by the fielding side
\end{flushleft}





3.9.2.2





\begin{flushleft}
The Umpire Review and the Player Review both relate to the dismissal of the same batsman
\end{flushleft}





3.9.2.3





\begin{flushleft}
If the batsman is out, the number of runs scored from the delivery would be the same for both
\end{flushleft}


\begin{flushleft}
modes of dismissal
\end{flushleft}





3.9.2.4





\begin{flushleft}
If the batsman is out, the batsman on strike for the next delivery would be the same for both
\end{flushleft}


\begin{flushleft}
modes of dismissal.
\end{flushleft}





\begin{flushleft}
If the Umpire Review leads the third umpire to make a decision of Out, then this shall be displayed in the
\end{flushleft}


\begin{flushleft}
usual manner and the Player Review shall not be undertaken. If the Umpire Review results in a Not out
\end{flushleft}





82





\begin{flushleft}
\newpage
decision, then the third umpire shall make no public decision but shall proceed to address the request for a
\end{flushleft}


\begin{flushleft}
Player Review.
\end{flushleft}


3.9.4





\begin{flushleft}
For illustration, following an LBW appeal which is given Not out by the bowler's end umpire, the striker sets
\end{flushleft}


\begin{flushleft}
off for a run, is sent back and there is an appeal for his/her run out. The players request that the LBW
\end{flushleft}


\begin{flushleft}
decision is reviewed and the umpires request that the run out be reviewed. The four criteria above are
\end{flushleft}


\begin{flushleft}
satisfied, so the run out referral is determined first. Should the appeal for run out be Out, then there is no
\end{flushleft}


\begin{flushleft}
requirement for the LBW review to take place.
\end{flushleft}





3.9.5





\begin{flushleft}
In all other circumstances, the incidents shall be addressed in chronological order. If the conclusion from the
\end{flushleft}


\begin{flushleft}
first incident is that a batsman is dismissed, then the ball would be deemed to have become dead at that
\end{flushleft}


\begin{flushleft}
point, rendering investigation of the second incident unnecessary.
\end{flushleft}





4





\begin{flushleft}
Interpretation of Playing Conditions
\end{flushleft}





4.1





\begin{flushleft}
When using a replay to determine the moment at which the wicket has been put down (as per clause 29.1),
\end{flushleft}


\begin{flushleft}
the third umpire shall deem this to be the first frame in which one of the bails is shown (or can be deduced)
\end{flushleft}


\begin{flushleft}
to have lost all contact with the top of the stumps and subsequent frames show the bail permanently
\end{flushleft}


\begin{flushleft}
removed from the top of the stumps.
\end{flushleft}





4.2





\begin{flushleft}
Where LED Wickets are used (as provided for in paragraph 3.8.1.6) the moment at which the wicket has
\end{flushleft}


\begin{flushleft}
been put down (as per clause 29.1) shall be deemed to be the first frame in which the LED lights are
\end{flushleft}


\begin{flushleft}
illuminated and subsequent frames show the bail permanently removed from the top of the stumps.
\end{flushleft}





83





\begin{flushleft}
\newpage
Appendix E
\end{flushleft}


\begin{flushleft}
Calculations
\end{flushleft}


\begin{flushleft}
Table 1: Calculation sheet for use when a delay or interruptions occur in the First Innings
\end{flushleft}


\begin{flushleft}
Time
\end{flushleft}


\begin{flushleft}
Net playing time available at start of the match
\end{flushleft}





\begin{flushleft}
170 minutes (A)
\end{flushleft}





\begin{flushleft}
Time innings in progress
\end{flushleft}





\begin{flushleft}
\_\_\_\_\_\_\_\_\_\_\_ (B)
\end{flushleft}





\begin{flushleft}
Playing time lost
\end{flushleft}





\begin{flushleft}
\_\_\_\_\_\_\_\_\_\_\_ (C)
\end{flushleft}





\begin{flushleft}
Extra time available
\end{flushleft}





\begin{flushleft}
\_\_\_\_\_\_\_\_\_\_\_ (D)
\end{flushleft}





\begin{flushleft}
Time made up from reduced interval
\end{flushleft}





\begin{flushleft}
\_\_\_\_\_\_\_\_\_\_\_ (E)
\end{flushleft}





\begin{flushleft}
Effective playing time lost [C -- (D + E)]
\end{flushleft}





\begin{flushleft}
\_\_\_\_\_\_\_\_\_\_\_ (F)
\end{flushleft}





\begin{flushleft}
Remaining playing time available (A - F)
\end{flushleft}





\begin{flushleft}
\_\_\_\_\_\_\_\_\_\_\_ (G)
\end{flushleft}





\begin{flushleft}
G divided by 4.25 (to 2 decimal places)
\end{flushleft}





\begin{flushleft}
\_\_\_\_\_\_\_\_\_\_\_ (H)
\end{flushleft}





\begin{flushleft}
Max overs per team [H/2] (round up fractions)
\end{flushleft}





\begin{flushleft}
\_\_\_\_\_\_\_\_\_\_\_ (I)
\end{flushleft}





\begin{flushleft}
Maximum overs per bowler [I / 5]
\end{flushleft}





\_\_\_\_\_\_\_\_\_\_\_





\begin{flushleft}
Number of Powerplay overs
\end{flushleft}





\_\_\_\_\_\_\_\_\_\_\_





\begin{flushleft}
Rescheduled Playing Hours
\end{flushleft}


\begin{flushleft}
First session to commence or recommence
\end{flushleft}





\begin{flushleft}
\_\_\_\_\_\_\_\_\_\_\_ (J)
\end{flushleft}





\begin{flushleft}
Length of innings [I x 4.25] (round up fractions)
\end{flushleft}





\begin{flushleft}
\_\_\_\_\_\_\_\_\_\_\_ (K)
\end{flushleft}





\begin{flushleft}
Rescheduled first innings cessation time [J + (K -- B)]
\end{flushleft}





\begin{flushleft}
\_\_\_\_\_\_\_\_\_\_\_ (L)
\end{flushleft}





\begin{flushleft}
Length of interval
\end{flushleft}





\begin{flushleft}
\_\_\_\_\_\_\_\_\_\_\_ (M)
\end{flushleft}





\begin{flushleft}
Second innings commencement time [L + M]
\end{flushleft}





\begin{flushleft}
\_\_\_\_\_\_\_\_\_\_\_ (N)
\end{flushleft}





\begin{flushleft}
Rescheduled second innings cessation time [N + K]
\end{flushleft}





\begin{flushleft}
\_\_\_\_\_\_\_\_\_\_\_ *(O)
\end{flushleft}





\begin{flushleft}
* Ensure that the match is not finishing earlier than the original or rescheduled cessation time by applying clause
\end{flushleft}


\begin{flushleft}
13.7.2. If so, add at least one over to each team and recalculate (I) to (O) above to prevent this from happening.
\end{flushleft}





84





\begin{flushleft}
\newpage
Table 2: Calculation sheet to check whether an interruption during the First Innings should terminate the
\end{flushleft}


\begin{flushleft}
innings
\end{flushleft}


\begin{flushleft}
Proposed re-start time
\end{flushleft}





\begin{flushleft}
\_\_\_\_\_\_\_\_\_\_\_ (P)
\end{flushleft}





\begin{flushleft}
Rescheduled cut-off time allowing for full use of any extra time provision
\end{flushleft}





\begin{flushleft}
\_\_\_\_\_\_\_\_\_\_\_ (Q)
\end{flushleft}





\begin{flushleft}
Minutes between P and Q
\end{flushleft}





\begin{flushleft}
\_\_\_\_\_\_\_\_\_\_\_ (R)
\end{flushleft}





\begin{flushleft}
Potential overs to be bowled [R / 4.25] (round up fractions)
\end{flushleft}





\begin{flushleft}
\_\_\_\_\_\_\_\_\_\_\_ (S)
\end{flushleft}





\begin{flushleft}
Number of complete overs faced to date in first innings
\end{flushleft}





\begin{flushleft}
\_\_\_\_\_\_\_\_\_\_\_ (T)
\end{flushleft}





\begin{flushleft}
If S is greater than T then revert to Table 1
\end{flushleft}


\begin{flushleft}
If S is less than or equal to T then the first innings is terminated - go to Table 3
\end{flushleft}





\begin{flushleft}
Table 3: Calculation sheet for the start of the Second Innings
\end{flushleft}


\begin{flushleft}
Maximum overs to be bowled:
\end{flushleft}


\begin{flushleft}
(If first innings was terminated, S from Table 2)
\end{flushleft}





\begin{flushleft}
\_\_\_\_\_\_\_\_\_\_\_ (A)
\end{flushleft}





\begin{flushleft}
Scheduled length of innings: [A x 4.25] (round up fractions)
\end{flushleft}





\begin{flushleft}
\_\_\_\_\_\_\_\_\_\_\_ (B)
\end{flushleft}





\begin{flushleft}
Start time
\end{flushleft}





\begin{flushleft}
\_\_\_\_\_\_\_\_\_\_\_ (C)
\end{flushleft}





\begin{flushleft}
Scheduled cessation time [C + B]
\end{flushleft}





\begin{flushleft}
\_\_\_\_\_\_\_\_\_\_\_ (D)
\end{flushleft}





\begin{flushleft}
Overs per bowler and Fielding Restrictions
\end{flushleft}


\begin{flushleft}
Maximum overs per bowler [A / 5]
\end{flushleft}





\begin{flushleft}
\_\_\_\_\_\_\_\_\_\_\_ overs
\end{flushleft}





\begin{flushleft}
Number of Powerplay overs
\end{flushleft}





\begin{flushleft}
\_\_\_\_\_\_\_\_\_\_\_ overs
\end{flushleft}





\begin{flushleft}
Table 4: Calculation sheet for use when interruption occurs after the start of the Second Innings
\end{flushleft}


\begin{flushleft}
Time
\end{flushleft}


\begin{flushleft}
Time at start of innings
\end{flushleft}





\begin{flushleft}
\_\_\_\_\_\_\_\_\_\_\_ (A)
\end{flushleft}





\begin{flushleft}
Time at start of interruption
\end{flushleft}





\begin{flushleft}
\_\_\_\_\_\_\_\_\_\_\_ (B)
\end{flushleft}





\begin{flushleft}
Time innings in progress
\end{flushleft}





\begin{flushleft}
\_\_\_\_\_\_\_\_\_\_\_ (C)
\end{flushleft}





\begin{flushleft}
Restart time
\end{flushleft}





\begin{flushleft}
\_\_\_\_\_\_\_\_\_\_\_ (D)
\end{flushleft}





\begin{flushleft}
Length of interruption [D -- B]
\end{flushleft}





\begin{flushleft}
\_\_\_\_\_\_\_\_\_\_\_ (E)
\end{flushleft}





\begin{flushleft}
Additional time available:
\end{flushleft}


\begin{flushleft}
\_\_\_\_\_\_\_\_\_\_\_ (F)
\end{flushleft}


\begin{flushleft}
(Any unused provision for {`}Extra Time' or for earlier than scheduled start of second innings)
\end{flushleft}


\begin{flushleft}
Total playing time lost [E -- F]
\end{flushleft}





\begin{flushleft}
\_\_\_\_\_\_\_\_\_\_\_ (G)
\end{flushleft}





85





\begin{flushleft}
\newpage
Overs
\end{flushleft}


\begin{flushleft}
Maximum overs at start of innings
\end{flushleft}





\begin{flushleft}
\_\_\_\_\_\_\_\_\_\_\_ (H)
\end{flushleft}





\begin{flushleft}
Overs lost [G / 4.25] (rounded down)
\end{flushleft}





\begin{flushleft}
\_\_\_\_\_\_\_\_\_\_\_ (I)
\end{flushleft}





\begin{flushleft}
Adjusted maximum length of innings [H -- I]
\end{flushleft}





\begin{flushleft}
\_\_\_\_\_\_\_\_\_\_\_ (J)
\end{flushleft}





\begin{flushleft}
Rescheduled length of innings [J x 4.25 rounded up]
\end{flushleft}





\begin{flushleft}
\_\_\_\_\_\_\_\_\_\_\_ (K)
\end{flushleft}





\begin{flushleft}
Amended cessation time of innings [D + (K -- C)]
\end{flushleft}





\begin{flushleft}
\_\_\_\_\_\_\_\_\_\_\_ (L)
\end{flushleft}





\begin{flushleft}
Overs per bowler and Fielding Restrictions
\end{flushleft}


\begin{flushleft}
Maximum overs per bowler [J / 5]
\end{flushleft}





\begin{flushleft}
\_\_\_\_\_\_\_\_\_\_\_ overs
\end{flushleft}





\begin{flushleft}
Number of Powerplay overs
\end{flushleft}





\begin{flushleft}
\_\_\_\_\_\_\_\_\_\_\_ overs
\end{flushleft}





86





\begin{flushleft}
\newpage
Appendix F
\end{flushleft}


\begin{flushleft}
Procedure for the Super Over
\end{flushleft}


\begin{flushleft}
The following procedure shall apply should the provision for a Super Over be adopted in any match.
\end{flushleft}


1.





\begin{flushleft}
Subject to weather conditions the Super Over will take place on the scheduled day of the match at a time to
\end{flushleft}


\begin{flushleft}
be determined by the ICC Match Referee. In normal circumstances it shall commence 10 minutes after the
\end{flushleft}


\begin{flushleft}
conclusion of the match.
\end{flushleft}





2.





\begin{flushleft}
The amount of extra time allocated to the Super Over is the greater of (a) the extra time allocated to the
\end{flushleft}


\begin{flushleft}
original match less the amount of extra time actually utilised and (b) the gap between the actual end of the
\end{flushleft}


\begin{flushleft}
match and the time the original match would have been scheduled to finish had the whole of the extra time
\end{flushleft}


\begin{flushleft}
provision been utilised. Should play be delayed prior to or during the Super Over once the playing time lost
\end{flushleft}


\begin{flushleft}
exceeds the extra time allocated, the Super Over shall be abandoned. See paragraph 16 below.
\end{flushleft}





3.





\begin{flushleft}
The Super Over shall take place on the pitch allocated for the match (the designated pitch) unless otherwise
\end{flushleft}


\begin{flushleft}
determined by the umpires in consultation with the Ground Authority and the ICC Match Referee.
\end{flushleft}





4.





\begin{flushleft}
The umpires shall stand at the same end as that in which they finished the match.
\end{flushleft}





5.





\begin{flushleft}
In both innings of the Super Over, the fielding side shall choose from which end to bowl.
\end{flushleft}





6.





\begin{flushleft}
Only nominated players in the match may participate in the Super Over. Should any player (including the
\end{flushleft}


\begin{flushleft}
batsmen and bowler) be unable to continue to participate in the Super Over due to injury, illness or other
\end{flushleft}


\begin{flushleft}
wholly acceptable reasons, the relevant Playing Conditions as they apply in the match shall also apply in the
\end{flushleft}


\begin{flushleft}
Super Over.
\end{flushleft}





7.





\begin{flushleft}
Any penalty time being served in the match shall be carried forward to the Super Over.
\end{flushleft}





8.





\begin{flushleft}
Each team's over is played with the same fielding restrictions as apply for the last over in a match played
\end{flushleft}


\begin{flushleft}
under the ICC Twenty20 International Playing Conditions.
\end{flushleft}





9.





\begin{flushleft}
The team batting second in the match shall bat first in the Super Over.
\end{flushleft}





\begin{flushleft}
10. The captain of the fielding team (or his/her nominee) shall select the ball with which the fielding team shall
\end{flushleft}


\begin{flushleft}
bowl their over in the Super Over from the box of spare balls provided by the umpires (which shall include
\end{flushleft}


\begin{flushleft}
the balls used in the match, but no new balls). The team fielding first in the Super Over shall have first
\end{flushleft}


\begin{flushleft}
choice of ball. The team fielding second may choose to use the same ball as chosen by the team bowling
\end{flushleft}


\begin{flushleft}
first. If the ball needs to be changed, the Playing Conditions shall apply.
\end{flushleft}


\begin{flushleft}
11. The loss of two wickets in the over ends the team's one over innings.
\end{flushleft}


\begin{flushleft}
12. Each team shall be allowed to make one unsuccessful Player Review in each innings of the Super Over.
\end{flushleft}


\begin{flushleft}
This entitlement shall apply irrespective of the number of unsuccessful Player Review requests made during
\end{flushleft}


\begin{flushleft}
the match itself.
\end{flushleft}


\begin{flushleft}
13. In the event of the teams having the same score after the Super Over has been completed, if the original
\end{flushleft}


\begin{flushleft}
match was a tie under the Duckworth/Lewis/Stern method, paragraph 15 below shall apply. Otherwise, the
\end{flushleft}


\begin{flushleft}
team whose batsmen hit the most number of boundaries combined from its two innings in both the match
\end{flushleft}


\begin{flushleft}
and the Super Over shall be the winner.
\end{flushleft}


\begin{flushleft}
14. If the number of boundaries hit by both teams is equal, the team whose batsmen scored more boundaries
\end{flushleft}


\begin{flushleft}
during its innings in the main match (ignoring the Super Over) shall be the winner.
\end{flushleft}


\begin{flushleft}
15. If still equal, a count-back from the final ball of the Super Over shall be conducted. The team with the higher
\end{flushleft}


\begin{flushleft}
scoring delivery shall be the winner. If a team loses two wickets during its over, then any unbowled
\end{flushleft}


\begin{flushleft}
deliveries will be counted as dot balls. Note that for this purpose, the runs scored from a delivery is defined
\end{flushleft}


\begin{flushleft}
as the total team runs scored since the completion of the previous legitimate ball, i.e including any runs
\end{flushleft}


\begin{flushleft}
resulting from Wides, No balls or penalty runs.
\end{flushleft}





87





\begin{flushleft}
\newpage
Example:
\end{flushleft}


\begin{flushleft}
Runs scored from:
\end{flushleft}





\begin{flushleft}
Team 1
\end{flushleft}





\begin{flushleft}
Team 2
\end{flushleft}





\begin{flushleft}
Ball 6
\end{flushleft}





1





1





\begin{flushleft}
Ball 5
\end{flushleft}





4





4





\begin{flushleft}
Ball 4
\end{flushleft}





2





1





\begin{flushleft}
Ball 3
\end{flushleft}





6





2





\begin{flushleft}
Ball 2
\end{flushleft}





0





1





\begin{flushleft}
Ball 1
\end{flushleft}





2





6





\begin{flushleft}
In this example both teams scored an equal number of runs from the 6th and 5th ball of their innings. However
\end{flushleft}


\begin{flushleft}
team 1 scored 2 runs from its 4th ball while team 2 scored a single so team 1 is the winner.
\end{flushleft}


\begin{flushleft}
16. Paragraph 2 examples:
\end{flushleft}


\begin{flushleft}
Scheduled finish 5.00, 30 minutes extra time available, so scheduled finish time if the whole of the extra time
\end{flushleft}


\begin{flushleft}
provision is utilised is 5.30.
\end{flushleft}


\begin{flushleft}
a)
\end{flushleft}





\begin{flushleft}
No extra time is utilised in the original match which overruns ten minutes and finishes at 5.10. The
\end{flushleft}


\begin{flushleft}
Super Over is scheduled to start at 5.20 with 30 minutes extra time available. It starts on time but is
\end{flushleft}


\begin{flushleft}
interrupted at 5.25. Play must resume by 5.55 otherwise the Super Over is abandoned.
\end{flushleft}





\begin{flushleft}
b)
\end{flushleft}





\begin{flushleft}
20 minutes of extra time was utilised, with the match scheduled to finish at 5.20, but it actually finishes
\end{flushleft}


\begin{flushleft}
at 5.10. Therefore the extra time allocated to the Super Over is the greater of a) 10 minutes (30 minutes
\end{flushleft}


\begin{flushleft}
extra time less 20 already utilised) and b) 20 minutes (the gap from the actual finish time of 5.10 and the
\end{flushleft}


\begin{flushleft}
scheduled finish had the full extra time been utilised of 5.30). The Super Over was due to start at 5.20,
\end{flushleft}


\begin{flushleft}
but is delayed by rain. It must therefore start by 5.40 or the Super Over is abandoned.
\end{flushleft}





\begin{flushleft}
c)
\end{flushleft}





\begin{flushleft}
The match finishes at 5.40 (having started 30 minutes late and overrun by 10 minutes). There is no
\end{flushleft}


\begin{flushleft}
extra time allocated to the Super Over which should start at 5.50. Any delay or interruption after 5.50
\end{flushleft}


\begin{flushleft}
means the Super Over is abandoned.
\end{flushleft}





88





\newpage



\end{document}
