%%%%%%%%%%%%  Generated using docx2latex.com  %%%%%%%%%%%%%%

%%%%%%%%%%%%  v2.0.0-beta  %%%%%%%%%%%%%%

\documentclass[12pt]{article}
\usepackage{amsmath}
\usepackage{latexsym}
\usepackage{amsfonts}
\usepackage[normalem]{ulem}
\usepackage{array}
\usepackage{amssymb}
\usepackage{graphicx}
\usepackage{subfig}
\usepackage{wrapfig}
\usepackage{wasysym}
\usepackage{enumitem}
\usepackage{adjustbox}
\usepackage{ragged2e}
\usepackage[svgnames,table]{xcolor}
\usepackage{tikz}
\usepackage{longtable}
\usepackage{changepage}
\usepackage{setspace}
\usepackage{hhline}
\usepackage{multicol}
\usepackage{tabto}
\usepackage{float}
\usepackage{multirow}
\usepackage{makecell}
\usepackage{fancyhdr}
\usepackage[toc,page]{appendix}
\usepackage[paperheight=11.0in,paperwidth=8.5in,left=1.0in,right=1.0in,top=1.0in,bottom=0.11in,headheight=1in]{geometry}
\usepackage[utf8]{inputenc}
\usepackage[T1]{fontenc}
\usepackage[hidelinks]{hyperref}
\usetikzlibrary{shapes.symbols,shapes.geometric,shadows,arrows.meta}
\tikzset{>={Latex[width=1.5mm,length=2mm]}}
\usepackage{flowchart}\TabPositions{0.5in,1.0in,1.5in,2.0in,2.5in,3.0in,3.5in,4.0in,4.5in,5.0in,5.5in,6.0in,}

\urlstyle{same}


 %%%%%%%%%%%%  Set Depths for Sections  %%%%%%%%%%%%%%

% 1) Section
% 1.1) SubSection
% 1.1.1) SubSubSection
% 1.1.1.1) Paragraph
% 1.1.1.1.1) Subparagraph


\setcounter{tocdepth}{5}
\setcounter{secnumdepth}{5}


 %%%%%%%%%%%%  Set Depths for Nested Lists created by \begin{enumerate}  %%%%%%%%%%%%%%


\setlistdepth{9}
\renewlist{enumerate}{enumerate}{9}
	\setlist[enumerate,1]{label=\arabic*)}
	\setlist[enumerate,2]{label=\alph*)}
	\setlist[enumerate,3]{label=(\roman*)}
	\setlist[enumerate,4]{label=(\arabic*)}
	\setlist[enumerate,5]{label=(\Alph*)}
	\setlist[enumerate,6]{label=(\Roman*)}
	\setlist[enumerate,7]{label=\arabic*}
	\setlist[enumerate,8]{label=\alph*}
	\setlist[enumerate,9]{label=\roman*}

\renewlist{itemize}{itemize}{9}
	\setlist[itemize]{label=$\cdot$}
	\setlist[itemize,1]{label=\textbullet}
	\setlist[itemize,2]{label=$\circ$}
	\setlist[itemize,3]{label=$\ast$}
	\setlist[itemize,4]{label=$\dagger$}
	\setlist[itemize,5]{label=$\triangleright$}
	\setlist[itemize,6]{label=$\bigstar$}
	\setlist[itemize,7]{label=$\blacklozenge$}
	\setlist[itemize,8]{label=$\prime$}

\setlength{\topsep}{0pt}\setlength{\parindent}{0pt}
\renewcommand{\arraystretch}{1.3}


%%%%%%%%%%%%%%%%%%%% Document code starts here %%%%%%%%%%%%%%%%%%%%



\begin{document}

\vspace{\baselineskip}

\vspace{\baselineskip}

\vspace{\baselineskip}

\vspace{\baselineskip}

\vspace{\baselineskip}

\vspace{\baselineskip}

\vspace{\baselineskip}

\vspace{\baselineskip}

\vspace{\baselineskip}

\vspace{\baselineskip}

\vspace{\baselineskip}

\vspace{\baselineskip}

\vspace{\baselineskip}

\vspace{\baselineskip}

\vspace{\baselineskip}

\vspace{\baselineskip}

\vspace{\baselineskip}

\vspace{\baselineskip}

\vspace{\baselineskip}

\vspace{\baselineskip}

\vspace{\baselineskip}

\vspace{\baselineskip}

\vspace{\baselineskip}

\vspace{\baselineskip}

\vspace{\baselineskip}

\vspace{\baselineskip}

\vspace{\baselineskip}

\vspace{\baselineskip}

\vspace{\baselineskip}

\vspace{\baselineskip}

\vspace{\baselineskip}

\vspace{\baselineskip}

\vspace{\baselineskip}
\begin{Center}
{\fontsize{15pt}{18.0pt}\selectfont \textbf{ICC Men’s Twenty20 International Playing Conditions}\par}
\end{Center}\par


\vspace{\baselineskip}

\vspace{\baselineskip}
\begin{Center}
{\fontsize{11pt}{13.2pt}\selectfont \textbf{(incorporating the 2017 Code of the MCC Laws of Cricket)}\par}
\end{Center}\par


\vspace{\baselineskip}
\begin{Center}
\textbf{Effective 28\textsuperscript{th} September 2017}
\end{Center}\par


\vspace{\baselineskip}
\begin{adjustwidth}{2.92in}{0.0in}
{\fontsize{11pt}{13.2pt}\selectfont \textbf{Contents}\par}\par

\end{adjustwidth}


\vspace{\baselineskip}
{\fontsize{9pt}{10.8pt}\selectfont 1 \tabto{0.29in}  \tabto{6.42in} \par}\par


\vspace{\baselineskip}
{\fontsize{9pt}{10.8pt}\selectfont 2 \tabto{0.29in}  \tabto{6.42in} \par}\par


\vspace{\baselineskip}
{\fontsize{9pt}{10.8pt}\selectfont 3 \tabto{0.29in}  \tabto{6.42in} \par}\par


\vspace{\baselineskip}
{\fontsize{9pt}{10.8pt}\selectfont 4 \tabto{0.29in}  \tabto{6.42in} \par}\par


\vspace{\baselineskip}
{\fontsize{9pt}{10.8pt}\selectfont 5 \tabto{0.29in}  \tabto{6.42in} \par}\par


\vspace{\baselineskip}
{\fontsize{9pt}{10.8pt}\selectfont 6 \tabto{0.29in}  \tabto{6.42in} \par}\par


\vspace{\baselineskip}
{\fontsize{9pt}{10.8pt}\selectfont 7 \tabto{0.29in}  \tabto{6.35in} \par}\par


\vspace{\baselineskip}
{\fontsize{9pt}{10.8pt}\selectfont 8 \tabto{0.29in}  \tabto{6.35in} \par}\par


\vspace{\baselineskip}
{\fontsize{9pt}{10.8pt}\selectfont 9 \tabto{0.29in}  \tabto{6.35in} \par}\par


\vspace{\baselineskip}
{\fontsize{9pt}{10.8pt}\selectfont 10 \tabto{0.29in}  \tabto{6.35in} \par}\par


\vspace{\baselineskip}
{\fontsize{9pt}{10.8pt}\selectfont 11 \tabto{0.29in}  \tabto{6.35in} \par}\par


\vspace{\baselineskip}
{\fontsize{9pt}{10.8pt}\selectfont 12 \tabto{0.29in}  \tabto{6.35in} \par}\par


\vspace{\baselineskip}
{\fontsize{9pt}{10.8pt}\selectfont 13 \tabto{0.29in}  \tabto{6.35in} \par}\par


\vspace{\baselineskip}
{\fontsize{9pt}{10.8pt}\selectfont 14 \tabto{0.29in}  \tabto{6.35in} \par}\par


\vspace{\baselineskip}
{\fontsize{9pt}{10.8pt}\selectfont 15 \tabto{0.29in}  \tabto{6.35in} \par}\par


\vspace{\baselineskip}
{\fontsize{9pt}{10.8pt}\selectfont 16 \tabto{0.29in}  \tabto{6.35in} \par}\par


\vspace{\baselineskip}
{\fontsize{9pt}{10.8pt}\selectfont 17 \tabto{0.29in}  \tabto{6.35in} \par}\par


\vspace{\baselineskip}
{\fontsize{9pt}{10.8pt}\selectfont 18 \tabto{0.29in}  \tabto{6.35in} \par}\par


\vspace{\baselineskip}
{\fontsize{9pt}{10.8pt}\selectfont 19 \tabto{0.29in}  \tabto{6.35in} \par}\par


\vspace{\baselineskip}
{\fontsize{9pt}{10.8pt}\selectfont 20 \tabto{0.29in}  \tabto{6.35in} \par}\par


\vspace{\baselineskip}
{\fontsize{9pt}{10.8pt}\selectfont 21 \tabto{0.29in}  \tabto{6.35in} \par}\par


\vspace{\baselineskip}
{\fontsize{9pt}{10.8pt}\selectfont 22 \tabto{0.29in}  \tabto{6.35in} \par}\par


\vspace{\baselineskip}
{\fontsize{9pt}{10.8pt}\selectfont 23 \tabto{0.29in}  \tabto{6.35in} \par}\par


\vspace{\baselineskip}
{\fontsize{9pt}{10.8pt}\selectfont 24 \tabto{0.29in}  \tabto{6.35in} \par}\par


\vspace{\baselineskip}
{\fontsize{9pt}{10.8pt}\selectfont 25 \tabto{0.29in}  \tabto{6.35in} \par}\par


\vspace{\baselineskip}
{\fontsize{9pt}{10.8pt}\selectfont 26 \tabto{0.29in}  \tabto{6.35in} \par}\par


\vspace{\baselineskip}
{\fontsize{9pt}{10.8pt}\selectfont 27 \tabto{0.29in}  \tabto{6.35in} \par}\par


\vspace{\baselineskip}
{\fontsize{9pt}{10.8pt}\selectfont 28 \tabto{0.29in}  \tabto{6.35in} \par}\par


\vspace{\baselineskip}
{\fontsize{9pt}{10.8pt}\selectfont 29 \tabto{0.29in}  \tabto{6.35in} \par}\par


\vspace{\baselineskip}
{\fontsize{9pt}{10.8pt}\selectfont 30 \tabto{0.29in}  \tabto{6.35in} \par}\par


\vspace{\baselineskip}
{\fontsize{9pt}{10.8pt}\selectfont 31 \tabto{0.29in}  \tabto{6.35in} \par}\par


\vspace{\baselineskip}
{\fontsize{9pt}{10.8pt}\selectfont 32 \tabto{0.29in}  \tabto{6.35in} \par}\par


\vspace{\baselineskip}
{\fontsize{9pt}{10.8pt}\selectfont 33 \tabto{0.29in}  \tabto{6.35in} \par}\par


\vspace{\baselineskip}
{\fontsize{9pt}{10.8pt}\selectfont 34 \tabto{0.29in}  \tabto{6.35in} \par}\par


\vspace{\baselineskip}
{\fontsize{9pt}{10.8pt}\selectfont 35 \tabto{0.29in}  \tabto{6.35in} \par}\par


\vspace{\baselineskip}
{\fontsize{9pt}{10.8pt}\selectfont 36 \tabto{0.29in}  \tabto{6.35in} \par}\par


\vspace{\baselineskip}
{\fontsize{9pt}{10.8pt}\selectfont 37 \tabto{0.29in}  \tabto{6.35in} \par}\par


\vspace{\baselineskip}
{\fontsize{9pt}{10.8pt}\selectfont 38 \tabto{0.29in}  \tabto{6.35in} \par}\par


\vspace{\baselineskip}
{\fontsize{9pt}{10.8pt}\selectfont 39 \tabto{0.29in}  \tabto{6.35in} \par}\par


\vspace{\baselineskip}

\vspace{\baselineskip}

\vspace{\baselineskip}
\begin{Center}
{\fontsize{9pt}{10.8pt}\selectfont i\par}
\end{Center}\par


\vspace{\baselineskip}
{\fontsize{9pt}{10.8pt}\selectfont 40 \tabto{0.29in}  \tabto{6.35in} \par}\par


\vspace{\baselineskip}
{\fontsize{9pt}{10.8pt}\selectfont 41 \tabto{0.29in}  \tabto{6.35in} \par}\par


\vspace{\baselineskip}
{\fontsize{9pt}{10.8pt}\selectfont 42 \tabto{0.29in}  \tabto{6.35in} \par}\par


\vspace{\baselineskip}
 \tabto{6.35in} \par


\vspace{\baselineskip}
{\fontsize{9pt}{10.8pt}\selectfont 1 \tabto{0.29in}  \tabto{6.35in} \par}\par


\vspace{\baselineskip}
{\fontsize{9pt}{10.8pt}\selectfont 2 \tabto{0.29in}  \tabto{6.35in} \par}\par


\vspace{\baselineskip}
{\fontsize{9pt}{10.8pt}\selectfont 3 \tabto{0.29in}  \tabto{6.35in} \par}\par


\vspace{\baselineskip}
{\fontsize{9pt}{10.8pt}\selectfont 4 \tabto{0.29in}  \tabto{6.35in} \par}\par


\vspace{\baselineskip}
{\fontsize{9pt}{10.8pt}\selectfont 5 \tabto{0.29in}  \tabto{6.35in} \par}\par


\vspace{\baselineskip}
{\fontsize{9pt}{10.8pt}\selectfont 6 \tabto{0.29in}  \tabto{6.35in} \par}\par


\vspace{\baselineskip}
{\fontsize{9pt}{10.8pt}\selectfont 7 \tabto{0.29in}  \tabto{6.35in} \par}\par


\vspace{\baselineskip}
{\fontsize{9pt}{10.8pt}\selectfont 8 \tabto{0.29in}  \tabto{6.35in} \par}\par


\vspace{\baselineskip}
{\fontsize{9pt}{10.8pt}\selectfont 9 \tabto{0.29in}  \tabto{6.35in} \par}\par


\vspace{\baselineskip}
{\fontsize{9pt}{10.8pt}\selectfont 10 \tabto{0.29in}  \tabto{6.35in} \par}\par


\vspace{\baselineskip}
{\fontsize{9pt}{10.8pt}\selectfont 11 \tabto{0.29in}  \tabto{6.35in} \par}\par


\vspace{\baselineskip}
{\fontsize{9pt}{10.8pt}\selectfont 12 \tabto{0.29in}  \tabto{6.35in} \par}\par


\vspace{\baselineskip}
{\fontsize{9pt}{10.8pt}\selectfont 13 \tabto{0.29in}  \tabto{6.35in} \par}\par


\vspace{\baselineskip}
 \tabto{6.35in} \par


\vspace{\baselineskip}
{\fontsize{9pt}{10.8pt}\selectfont 1 \tabto{0.29in}  \tabto{6.35in} \par}\par


\vspace{\baselineskip}
{\fontsize{9pt}{10.8pt}\selectfont 2 \tabto{0.29in}  \tabto{6.35in} \par}\par


\vspace{\baselineskip}
{\fontsize{9pt}{10.8pt}\selectfont 3 \tabto{0.29in}  \tabto{6.35in} \par}\par


\vspace{\baselineskip}
 \tabto{6.35in} \par


\vspace{\baselineskip}
{\fontsize{9pt}{10.8pt}\selectfont 1 \tabto{0.29in}  \tabto{6.35in} \par}\par


\vspace{\baselineskip}
{\fontsize{9pt}{10.8pt}\selectfont 2 \tabto{0.29in}  \tabto{6.35in} \par}\par


\vspace{\baselineskip}
{\fontsize{9pt}{10.8pt}\selectfont 3 \tabto{0.29in}  \tabto{6.35in} \par}\par


\vspace{\baselineskip}
{\fontsize{9pt}{10.8pt}\selectfont 4 \tabto{0.29in}  \tabto{6.35in} \par}\par


\vspace{\baselineskip}
 \tabto{6.35in} \par


\vspace{\baselineskip}
{\fontsize{9pt}{10.8pt}\selectfont 1 \tabto{0.29in}  \tabto{6.35in} \par}\par


\vspace{\baselineskip}
{\fontsize{9pt}{10.8pt}\selectfont 2 \tabto{0.29in}  \tabto{6.35in} \par}\par


\vspace{\baselineskip}
{\fontsize{9pt}{10.8pt}\selectfont 3 \tabto{0.29in}  \tabto{6.35in} \par}\par


\vspace{\baselineskip}
{\fontsize{9pt}{10.8pt}\selectfont 4 \tabto{0.29in}  \tabto{6.35in} \par}\par


\vspace{\baselineskip}
 \tabto{6.35in} \par


\vspace{\baselineskip}
 \tabto{6.35in} \par


\vspace{\baselineskip}

\vspace{\baselineskip}

\vspace{\baselineskip}

\vspace{\baselineskip}

\vspace{\baselineskip}

\vspace{\baselineskip}

\vspace{\baselineskip}

\vspace{\baselineskip}

\vspace{\baselineskip}

\vspace{\baselineskip}

\vspace{\baselineskip}

\vspace{\baselineskip}

\vspace{\baselineskip}

\vspace{\baselineskip}

\vspace{\baselineskip}
\begin{Center}
{\fontsize{9pt}{10.8pt}\selectfont ii\par}
\end{Center}\par


\vspace{\baselineskip}
\begin{Center}
{\fontsize{16pt}{19.2pt}\selectfont \textbf{ICC Men’s Twenty20 International}\par}
\end{Center}\par


\vspace{\baselineskip}
\begin{Center}
{\fontsize{16pt}{19.2pt}\selectfont \textbf{Playing Conditions}\par}
\end{Center}\par


\vspace{\baselineskip}
\begin{adjustwidth}{0.97in}{0.0in}
\textbf{(incorporating the 2017 Code of the MCC Laws of Cricket)}\par

\end{adjustwidth}


\vspace{\baselineskip}
{\fontsize{11pt}{13.2pt}\selectfont \textbf{Preamble - The Spirit of Cricket}\par}\par


\vspace{\baselineskip}
\begin{adjustwidth}{0.0in}{0.01in}
{\fontsize{9pt}{10.8pt}\selectfont Cricket owes much of its appeal and enjoyment to the fact that it should be played not only according to the Laws (which are incorporated within these Playing Conditions), but also within the Spirit of Cricket.\par}\par

\end{adjustwidth}


\vspace{\baselineskip}
{\fontsize{9pt}{10.8pt}\selectfont The major responsibility for ensuring fair play rests with the captains, but extends to all players, umpires and, especially in junior cricket, teachers, coaches and parents.\par}\par


\vspace{\baselineskip}
{\fontsize{9pt}{10.8pt}\selectfont Respect is central to the Spirit of Cricket.\par}\par


\vspace{\baselineskip}
{\fontsize{9pt}{10.8pt}\selectfont Respect your captain, team-mates, opponents and the authority of the umpires.\par}\par


\vspace{\baselineskip}
{\fontsize{9pt}{10.8pt}\selectfont Play hard and play fair.\par}\par


\vspace{\baselineskip}
{\fontsize{9pt}{10.8pt}\selectfont Accept the umpire’s decision.\par}\par


\vspace{\baselineskip}
{\fontsize{9pt}{10.8pt}\selectfont Create a positive atmosphere by your own conduct, and encourage others to do likewise.\par}\par


\vspace{\baselineskip}
{\fontsize{9pt}{10.8pt}\selectfont Show self-discipline, even when things go against you.\par}\par


\vspace{\baselineskip}
{\fontsize{9pt}{10.8pt}\selectfont Congratulate the opposition on their successes, and enjoy those of your own team.\par}\par


\vspace{\baselineskip}
{\fontsize{9pt}{10.8pt}\selectfont Thank the officials and your opposition at the end of the match, whatever the result.\par}\par


\vspace{\baselineskip}
{\fontsize{9pt}{10.8pt}\selectfont Cricket is an exciting game that encourages leadership, friendship and teamwork, which brings together people from different nationalities, cultures and religions, especially when played within the Spirit of Cricket.\par}\par


\vspace{\baselineskip}
{\fontsize{16pt}{19.2pt}\selectfont \textbf{1 \tabto{0.29in} }{\fontsize{15pt}{18.0pt}\selectfont \textbf{THE PLAYERS}\par}\par}\par


\vspace{\baselineskip}
{\fontsize{11pt}{13.2pt}\selectfont \textbf{1.1 \tabto{0.47in} Number of players}\par}\par


\vspace{\baselineskip}
{\fontsize{9pt}{10.8pt}\selectfont A match is played between two sides, each of eleven players, one of whom shall be captain.\par}\par


\vspace{\baselineskip}
{\fontsize{11pt}{13.2pt}\selectfont \textbf{1.2 \tabto{0.47in} }{\fontsize{10pt}{12.0pt}\selectfont \textbf{Nomination and replacement of players}\par}\par}\par


\vspace{\baselineskip}
\begin{adjustwidth}{0.5in}{0.12in}
{\fontsize{9pt}{10.8pt}\selectfont 1.2.1 \tabto{0.49in} Each captain shall nominate 11 players plus a maximum of 4 substitute fielders in writing to the ICC Match Referee before the toss. No player (member of the playing eleven) may be changed after the nomination without the consent of the opposing captain.\par}\par

\end{adjustwidth}


\vspace{\baselineskip}
\begin{adjustwidth}{0.5in}{0.19in}
{\fontsize{9pt}{10.8pt}\selectfont 1.2.2 \tabto{0.49in} Only those nominated as substitute fielders shall be entitled to act as substitute fielders during the match, unless the ICC Match Referee, in exceptional circumstances, allows subsequent additions.\par}\par

\end{adjustwidth}


\vspace{\baselineskip}
\begin{adjustwidth}{0.0in}{0.47in}
\begin{Center}
{\fontsize{9pt}{10.8pt}\selectfont 1.2.3\ \ \  All those nominated including those nominated as substitute fielders, must be eligible to play for that particular team and by such nomination the nominees shall warrant that they are so eligible.\par}
\end{Center}\par

\end{adjustwidth}


\vspace{\baselineskip}
\begin{adjustwidth}{0.5in}{0.04in}
{\fontsize{9pt}{10.8pt}\selectfont 1.2.4 \tabto{0.49in} In addition, by their nomination, the nominees shall be deemed to have agreed to abide by all the applicable ICC Regulations pertaining to international cricket and in particular, the Clothing and Equipment Regulations, the Code of Conduct for Players and Player Support Personnel (hereafter referred to as the ICC Code of Conduct), the Anti-Racism Code for Players and Player Support Personnel, the Anti-Doping Code and the Anti-Corruption Code.\par}\par

\end{adjustwidth}


\vspace{\baselineskip}
\begin{adjustwidth}{0.5in}{0.11in}
{\fontsize{9pt}{10.8pt}\selectfont 1.2.5 \tabto{0.49in} A player or player support personnel who has been suspended from participating in a match shall not, from the toss of the coin and for the remainder of the match thereafter:\par}\par

\end{adjustwidth}


\vspace{\baselineskip}
\begin{adjustwidth}{0.49in}{0.0in}
{\fontsize{9pt}{10.8pt}\selectfont 1.2.5.1 \tabto{1.17in} {\fontsize{8pt}{9.6pt}\selectfont Be nominated as, or carry out any of the duties or responsibilities of a substitute fielder, or\par}\par}\par

\end{adjustwidth}


\vspace{\baselineskip}

\vspace{\baselineskip}

\vspace{\baselineskip}

\vspace{\baselineskip}
\begin{adjustwidth}{0.0in}{-0.01in}
\begin{Center}
{\fontsize{8pt}{9.6pt}\selectfont 1\par}
\end{Center}\par

\end{adjustwidth}


\vspace{\baselineskip}

\vspace{\baselineskip}
\begin{adjustwidth}{1.18in}{0.03in}
{\fontsize{9pt}{10.8pt}\selectfont 1.2.5.2 \tabto{1.17in} Enter any part of the playing area (which shall include the field of play and the area between the boundary and the perimeter boards) at any time, including any scheduled or unscheduled breaks in play.\par}\par

\end{adjustwidth}


\vspace{\baselineskip}
\begin{adjustwidth}{0.5in}{0.07in}
{\fontsize{9pt}{10.8pt}\selectfont A player who has been suspended from participating in a match shall be permitted from the toss of the coin and for the remainder of the match thereafter be permitted to enter the players’ dressing room provided that the players’ dressing room (or any part thereof) for the match is not within the playing area described in clause above (for example, the player is not permitted to enter the on-field ‘dug-out’).\par}\par

\end{adjustwidth}


\vspace{\baselineskip}
{\fontsize{11pt}{13.2pt}\selectfont \textbf{1.3 \tabto{0.47in} Captain}\par}\par


\vspace{\baselineskip}
{\fontsize{9pt}{10.8pt}\selectfont 1.3.1 \tabto{0.49in} {\fontsize{8pt}{9.6pt}\selectfont If at any time the captain is not available, a deputy shall act for him.\par}\par}\par


\vspace{\baselineskip}
\begin{adjustwidth}{0.5in}{0.15in}
{\fontsize{9pt}{10.8pt}\selectfont 1.3.2 \tabto{0.49in} If a captain is not available to nominate the players, then any person associated with that team may act as his deputy to do so. See clause \par}\par

\end{adjustwidth}


\vspace{\baselineskip}
\begin{adjustwidth}{0.5in}{0.03in}
{\fontsize{9pt}{10.8pt}\selectfont 1.3.3 \tabto{0.49in} At any time after the nomination of the players, only a nominated player can act as deputy in discharging the duties and responsibilities of the captain as stated in these Playing Conditions, including at the toss. See clause (The toss).\par}\par

\end{adjustwidth}


\vspace{\baselineskip}
{\fontsize{9pt}{10.8pt}\selectfont 1.3.4 \tabto{0.49in} Each Member Board must nominate its ‘T20I Team Captain’ to the ICC when appointed.\par}\par


\vspace{\baselineskip}
\begin{adjustwidth}{0.5in}{0.22in}
{\fontsize{9pt}{10.8pt}\selectfont 1.3.5 \tabto{0.49in} {\fontsize{8pt}{9.6pt}\selectfont If the T20I Team Captain’ is not participating in a series, the relevant Home Board must nominate a replacement ‘T20I Team Captain’ for the series. The Home Board shall advise the series Match Referee.\par}\par}\par

\end{adjustwidth}


\vspace{\baselineskip}
\begin{adjustwidth}{0.5in}{0.19in}
{\fontsize{9pt}{10.8pt}\selectfont 1.3.6 \tabto{0.49in} If the ‘T20I Team Captain’ plays in a match without being the nominated captain for that match, he will be deemed to be the captain should any penalties be applied for over rate breaches under the ICC Code of Conduct.\par}\par

\end{adjustwidth}


\vspace{\baselineskip}
{\fontsize{11pt}{13.2pt}\selectfont \textbf{1.4 \tabto{0.47in} Responsibility of captains}\par}\par


\vspace{\baselineskip}
\begin{adjustwidth}{0.0in}{0.29in}
{\fontsize{9pt}{10.8pt}\selectfont The captains are responsible at all times for ensuring that play is conducted within the Spirit of Cricket as well as within these Playing Conditions.\par}\par

\end{adjustwidth}


\vspace{\baselineskip}
{\fontsize{16pt}{19.2pt}\selectfont \textbf{2 \tabto{0.29in} }{\fontsize{15pt}{18.0pt}\selectfont \textbf{THE UMPIRES}\par}\par}\par


\vspace{\baselineskip}
{\fontsize{11pt}{13.2pt}\selectfont \textbf{2.1 \tabto{0.47in} Appointment and attendance}\par}\par


\vspace{\baselineskip}
\begin{adjustwidth}{0.0in}{0.03in}
{\fontsize{9pt}{10.8pt}\selectfont The following rules for the selection and appointment of T20I umpires shall be followed as far as it is practicable to do so:\par}\par

\end{adjustwidth}


\vspace{\baselineskip}
\begin{adjustwidth}{0.5in}{0.14in}
{\fontsize{9pt}{10.8pt}\selectfont 2.1.1 \tabto{0.49in} The umpires shall control the game as required by these Playing Conditions, with absolute impartiality and shall be present at the ground at least two hours before the scheduled start of play,\par}\par

\end{adjustwidth}


\vspace{\baselineskip}
{\fontsize{9pt}{10.8pt}\selectfont 2.1.2 \tabto{0.49in} {\fontsize{8pt}{9.6pt}\selectfont The ICC shall establish an ‘Elite Panel’ of umpires who shall be contracted to the ICC.\par}\par}\par


\vspace{\baselineskip}
\begin{adjustwidth}{0.5in}{0.08in}
{\fontsize{9pt}{10.8pt}\selectfont 2.1.3 \tabto{0.49in} Each Full Member shall nominate from its panel of first class umpires up to four umpires to an ‘International Panel’\par}\par

\end{adjustwidth}


\vspace{\baselineskip}
\begin{adjustwidth}{0.5in}{0.18in}
{\fontsize{9pt}{10.8pt}\selectfont 2.1.4 \tabto{0.49in} The Home Board shall appoint both umpires to stand in each T20I match. Such umpires shall be selected from its umpires on the Elite or International Panel.\par}\par

\end{adjustwidth}


\vspace{\baselineskip}
\begin{adjustwidth}{0.5in}{0.1in}
{\fontsize{9pt}{10.8pt}\selectfont 2.1.5 \tabto{0.49in} In all T20I matches, the third umpire will be appointed by the Home Board and shall act as the emergency on-field umpire and officiate in regard to TV replays. Such appointment shall be made from the ‘Elite Panel’ or the ‘International Panel’.\par}\par

\end{adjustwidth}


\vspace{\baselineskip}
\begin{adjustwidth}{1.18in}{0.4in}
{\fontsize{9pt}{10.8pt}\selectfont 2.1.5.1 \tabto{1.17in} The playing conditions governing the use of the DRS and the third umpire are included in Appendix D.\par}\par

\end{adjustwidth}


\vspace{\baselineskip}
\begin{adjustwidth}{0.5in}{0.03in}
\begin{justify}
{\fontsize{9pt}{10.8pt}\selectfont 2.1.6 \tabto{0.49in} The Home Board shall also appoint a fourth umpire for each T20I match from its panel of first class umpires. The fourth umpire shall act as the emergency third umpire. In ‘DRS’ T20I matches the fourth umpire shall be appointed from the $``$International Panel$"$ \par}
\end{justify}\par

\end{adjustwidth}


\vspace{\baselineskip}

\vspace{\baselineskip}

\vspace{\baselineskip}

\vspace{\baselineskip}
\begin{adjustwidth}{0.0in}{-0.01in}
\begin{Center}
{\fontsize{8pt}{9.6pt}\selectfont 2\par}
\end{Center}\par

\end{adjustwidth}


\vspace{\baselineskip}
{\fontsize{9pt}{10.8pt}\selectfont 2.1.7 \tabto{0.49in} The ICC shall appoint the match referee for all matches (ICC Match Referee).\par}\par


\vspace{\baselineskip}
{\fontsize{9pt}{10.8pt}\selectfont 2.1.8 \tabto{0.49in} {\fontsize{8pt}{9.6pt}\selectfont The ICC Match Referee shall not be from the same country as the participating teams.\par}\par}\par


\vspace{\baselineskip}
{\fontsize{9pt}{10.8pt}\selectfont 2.1.9 \tabto{0.49in} Neither team will have a right of objection to the appointment of any umpire or match referee.\par}\par


\vspace{\baselineskip}
{\fontsize{11pt}{13.2pt}\selectfont \textbf{2.2 \tabto{0.47in} Change of umpire}\par}\par


\vspace{\baselineskip}
\begin{adjustwidth}{0.0in}{0.14in}
{\fontsize{9pt}{10.8pt}\selectfont An umpire shall not be changed during the match, other than in exceptional circumstances, unless he/she is injured or ill.\par}\par

\end{adjustwidth}


\vspace{\baselineskip}
{\fontsize{11pt}{13.2pt}\selectfont \textbf{2.3 \tabto{0.47in} Consultation with Home Board}\par}\par


\vspace{\baselineskip}
{\fontsize{9pt}{10.8pt}\selectfont Before the match the umpires shall consult with the Home Board to determine;\par}\par


\vspace{\baselineskip}
{\fontsize{9pt}{10.8pt}\selectfont 2.3.1 \tabto{0.49in} the balls to be used during the match. See clause (The ball).\par}\par


\vspace{\baselineskip}
{\fontsize{9pt}{10.8pt}\selectfont 2.3.2 \tabto{0.49in} {\fontsize{8pt}{9.6pt}\selectfont the hours of play and the times and durations of any agreed intervals.\par}\par}\par


\vspace{\baselineskip}
{\fontsize{9pt}{10.8pt}\selectfont 2.3.3 \tabto{0.49in} which clock or watch and back-up time piece is to be used during the match.\par}\par


\vspace{\baselineskip}
{\fontsize{9pt}{10.8pt}\selectfont 2.3.4 \tabto{0.49in} {\fontsize{8pt}{9.6pt}\selectfont the boundary of the field of play. See clause (Boundaries).\par}\par}\par


\vspace{\baselineskip}
{\fontsize{9pt}{10.8pt}\selectfont 2.3.5 \tabto{0.49in} {\fontsize{8pt}{9.6pt}\selectfont the use of covers. See clause (Covering the pitch).\par}\par}\par


\vspace{\baselineskip}
{\fontsize{9pt}{10.8pt}\selectfont 2.3.6 \tabto{0.49in} {\fontsize{8pt}{9.6pt}\selectfont any special conditions of play affecting the conduct of the match.\par}\par}\par


\vspace{\baselineskip}
{\fontsize{9pt}{10.8pt}\selectfont inform the scorers of agreements in and \par}\par


\vspace{\baselineskip}
{\fontsize{11pt}{13.2pt}\selectfont \textbf{2.4 \tabto{0.47in} The wickets, creases and boundaries}\par}\par


\vspace{\baselineskip}
{\fontsize{9pt}{10.8pt}\selectfont Before the toss and during the match, the umpires shall satisfy themselves that\par}\par


\vspace{\baselineskip}
{\fontsize{9pt}{10.8pt}\selectfont 2.4.1 \tabto{0.49in} the wickets are properly pitched. See clause (The wickets)\par}\par


\vspace{\baselineskip}
{\fontsize{9pt}{10.8pt}\selectfont 2.4.2 \tabto{0.49in} the creases are correctly marked. See clause (The creases).\par}\par


\vspace{\baselineskip}
\begin{adjustwidth}{0.5in}{0.11in}
{\fontsize{9pt}{10.8pt}\selectfont 2.4.3 \tabto{0.49in} {\fontsize{8pt}{9.6pt}\selectfont the boundary of the field of play complies with the requirements of clauses (Determining the boundary of the field of play), (Identifying and marking the boundary) and (Restoring the boundary).\par}\par}\par

\end{adjustwidth}


\vspace{\baselineskip}
{\fontsize{11pt}{13.2pt}\selectfont \textbf{2.5 \tabto{0.47in} Conduct of the match, implements and equipment}\par}\par


\vspace{\baselineskip}
{\fontsize{9pt}{10.8pt}\selectfont Before the toss and during the match, the umpires shall satisfy themselves that\par}\par


\vspace{\baselineskip}
{\fontsize{9pt}{10.8pt}\selectfont 2.5.1 \tabto{0.49in} the conduct of the match is strictly in accordance with these Playing Conditions.\par}\par


\vspace{\baselineskip}
{\fontsize{9pt}{10.8pt}\selectfont 2.5.2 \tabto{0.49in} {\fontsize{8pt}{9.6pt}\selectfont the implements used in the match conform to the following\par}\par}\par


\vspace{\baselineskip}
\begin{adjustwidth}{0.49in}{0.0in}
{\fontsize{9pt}{10.8pt}\selectfont 2.5.2.1 \tabto{1.17in} clause (The ball).\par}\par

\end{adjustwidth}


\vspace{\baselineskip}
\begin{adjustwidth}{0.49in}{0.0in}
{\fontsize{9pt}{10.8pt}\selectfont 2.5.2.2 \tabto{1.17in} externally visible requirements of clause (The bat) and paragraph of Appendix B.\par}\par

\end{adjustwidth}


\vspace{\baselineskip}
\begin{adjustwidth}{0.49in}{0.0in}
{\fontsize{9pt}{10.8pt}\selectfont 2.5.2.3 \tabto{1.17in} {\fontsize{8pt}{9.6pt}\selectfont either clauses (Size of stumps) and (The bails).\par}\par}\par

\end{adjustwidth}


\vspace{\baselineskip}
\begin{adjustwidth}{0.5in}{0.39in}
{\fontsize{9pt}{10.8pt}\selectfont 2.5.3 \tabto{0.49in} no player uses equipment other than that permitted. See paragraph of Appendix A. Note particularly therein the interpretation of ‘protective helmet’.\par}\par

\end{adjustwidth}


\vspace{\baselineskip}
{\fontsize{9pt}{10.8pt}\selectfont 2.5.4 \tabto{0.49in} the wicket-keeper’s gloves comply with the requirements of clause (Gloves).\par}\par


\vspace{\baselineskip}
{\fontsize{11pt}{13.2pt}\selectfont \textbf{2.6 \tabto{0.47in} Fair and unfair play}\par}\par


\vspace{\baselineskip}
{\fontsize{9pt}{10.8pt}\selectfont The umpires shall be the sole judges of fair and unfair play.\par}\par


\vspace{\baselineskip}
{\fontsize{11pt}{13.2pt}\selectfont \textbf{2.7 \tabto{0.47in} Fitness for play}\par}\par


\vspace{\baselineskip}
\begin{adjustwidth}{0.0in}{0.49in}
\begin{FlushRight}
{\fontsize{9pt}{10.8pt}\selectfont 2.7.1\ \ \  It is solely for the umpires together to decide whether either conditions of ground, weather or light or exceptional circumstances mean that it would be dangerous or unreasonable for play to take place.\par}
\end{FlushRight}\par

\end{adjustwidth}


\vspace{\baselineskip}

\vspace{\baselineskip}

\vspace{\baselineskip}
\begin{adjustwidth}{0.0in}{-0.01in}
\begin{Center}
{\fontsize{8pt}{9.6pt}\selectfont 3\par}
\end{Center}\par

\end{adjustwidth}


\vspace{\baselineskip}
\begin{adjustwidth}{0.5in}{0.0in}
{\fontsize{9pt}{10.8pt}\selectfont Conditions shall not be regarded as either dangerous or unreasonable merely because they are not ideal.\par}\par

\end{adjustwidth}


\vspace{\baselineskip}
\begin{adjustwidth}{0.5in}{0.39in}
{\fontsize{9pt}{10.8pt}\selectfont The fact that the grass and the ball are wet does not warrant the ground conditions being regarded as unreasonable or dangerous.\par}\par

\end{adjustwidth}


\vspace{\baselineskip}
\begin{adjustwidth}{0.5in}{0.08in}
{\fontsize{9pt}{10.8pt}\selectfont 2.7.2 \tabto{0.49in} Conditions shall be regarded as dangerous if there is actual and foreseeable risk to the safety of any player or umpire.\par}\par

\end{adjustwidth}


\vspace{\baselineskip}
\begin{adjustwidth}{0.5in}{0.12in}
{\fontsize{9pt}{10.8pt}\selectfont 2.7.3 \tabto{0.49in} Conditions shall be regarded as unreasonable if, although posing no risk to safety, it would not be sensible for play to proceed.\par}\par

\end{adjustwidth}


\vspace{\baselineskip}
\begin{adjustwidth}{0.5in}{0.21in}
{\fontsize{9pt}{10.8pt}\selectfont 2.7.4 \tabto{0.49in} If the umpires consider the ground is so wet or slippery as to deprive the bowler of a reasonable foothold, the fielders of the power of free movement, or the batsmen of the ability to play their strokes or to run between the wickets, then these conditions shall be regarded as so bad that it would be dangerous and unreasonable for play to take place.\par}\par

\end{adjustwidth}


\vspace{\baselineskip}
{\fontsize{11pt}{13.2pt}\selectfont \textbf{2.8 \tabto{0.47in} Suspension of play in dangerous or unreasonable circumstances}\par}\par


\vspace{\baselineskip}
{\fontsize{9pt}{10.8pt}\selectfont 2.8.1 \tabto{0.49in} {\fontsize{8pt}{9.6pt}\selectfont All references to ground include the pitch. See clause (Area of pitch).\par}\par}\par


\vspace{\baselineskip}
\begin{adjustwidth}{0.5in}{0.14in}
{\fontsize{9pt}{10.8pt}\selectfont 2.8.2 \tabto{0.49in} If at any time the umpires together agree that the conditions of ground, weather or light, or any other circumstances are dangerous or unreasonable, they shall immediately suspend play, or not allow play to start or to recommence. The decision as to whether conditions are so bad as to warrant such action is one for the umpires alone to make, following consultation with the ICC Match Referee.\par}\par

\end{adjustwidth}


\vspace{\baselineskip}
\begin{adjustwidth}{0.5in}{0.19in}
{\fontsize{9pt}{10.8pt}\selectfont 2.8.3 \tabto{0.49in} If circumstances are warranted, the umpires shall stop play and instruct the Ground Authority to take whatever action they can and use whatever equipment is necessary to remove as much dew as possible from the outfield when conditions become unreasonable or dangerous. The umpires may also instruct the ground staff to take such action during scheduled and unscheduled breaks in play.\par}\par

\end{adjustwidth}


\vspace{\baselineskip}
\begin{adjustwidth}{0.5in}{0.06in}
{\fontsize{9pt}{10.8pt}\selectfont 2.8.4 \tabto{0.49in} The umpires shall disregard any shadow on the pitch from the stadium or from any permanent object on the ground.\par}\par

\end{adjustwidth}


\vspace{\baselineskip}
{\fontsize{9pt}{10.8pt}\selectfont 2.8.5 \tabto{0.49in} {\fontsize{8pt}{9.6pt}\selectfont Light Meters\par}\par}\par


\vspace{\baselineskip}
\begin{adjustwidth}{0.5in}{0.17in}
{\fontsize{9pt}{10.8pt}\selectfont It is the responsibility of the ICC to supply light meters to the match officials to be used in accordance with these playing conditions.\par}\par

\end{adjustwidth}


\vspace{\baselineskip}
\begin{adjustwidth}{0.49in}{0.0in}
{\fontsize{9pt}{10.8pt}\selectfont 2.8.5.1 \tabto{1.17in} All light meters shall be uniformly calibrated.\par}\par

\end{adjustwidth}


\vspace{\baselineskip}
\begin{adjustwidth}{1.18in}{0.06in}
{\fontsize{9pt}{10.8pt}\selectfont 2.8.5.2 \tabto{1.17in} The umpires shall be entitled to use light meter readings as a guideline for determining whether the light is fit for play in accordance with the criteria set out in clause above.\par}\par

\end{adjustwidth}


\vspace{\baselineskip}
\begin{adjustwidth}{0.49in}{0.0in}
{\fontsize{9pt}{10.8pt}\selectfont 2.8.5.3 \tabto{1.17in} {\fontsize{8pt}{9.6pt}\selectfont Light meter readings may accordingly be used by the umpires:\par}\par}\par

\end{adjustwidth}


\vspace{\baselineskip}
\begin{adjustwidth}{1.97in}{0.75in}
{\fontsize{9pt}{10.8pt}\selectfont 2.8.5.3.1 \tabto{1.96in} To determine whether there has been at any stage a deterioration or improvement in the light.\par}\par

\end{adjustwidth}


\vspace{\baselineskip}
\begin{adjustwidth}{1.18in}{0.0in}
{\fontsize{9pt}{10.8pt}\selectfont 2.8.5.3.2 \tabto{1.96in} {\fontsize{8pt}{9.6pt}\selectfont As benchmarks for the remainder of a match.\par}\par}\par

\end{adjustwidth}


\vspace{\baselineskip}
{\fontsize{9pt}{10.8pt}\selectfont 2.8.6 \tabto{0.49in} Use of artificial lights\par}\par


\vspace{\baselineskip}
\begin{adjustwidth}{0.5in}{0.07in}
{\fontsize{9pt}{10.8pt}\selectfont If in the opinion of the umpires, natural light is deteriorating to an unfit level, they shall authorize the Ground Authority to use the available artificial lighting so that the match can commence or continue in acceptable conditions.\par}\par

\end{adjustwidth}


\vspace{\baselineskip}
\begin{adjustwidth}{0.5in}{0.14in}
{\fontsize{9pt}{10.8pt}\selectfont In the event of power failure or lights malfunction, the provisions relating to the delay or interruption of play due to bad weather or light shall apply.\par}\par

\end{adjustwidth}


\vspace{\baselineskip}
\begin{adjustwidth}{0.5in}{0.03in}
{\fontsize{9pt}{10.8pt}\selectfont 2.8.7 \tabto{0.49in} When there is a suspension of play it is the responsibility of the umpires to monitor conditions. They shall make inspections as often as appropriate, unaccompanied by any players or officials. Immediately the umpires together agree that the conditions are no longer dangerous or unreasonable they shall call upon the players to resume play.\par}\par

\end{adjustwidth}


\vspace{\baselineskip}

\vspace{\baselineskip}

\vspace{\baselineskip}

\vspace{\baselineskip}

\vspace{\baselineskip}
\begin{adjustwidth}{0.0in}{-0.01in}
\begin{Center}
{\fontsize{8pt}{9.6pt}\selectfont 4\par}
\end{Center}\par

\end{adjustwidth}


\vspace{\baselineskip}

\vspace{\baselineskip}
\begin{adjustwidth}{0.5in}{0.04in}
{\fontsize{9pt}{10.8pt}\selectfont 2.8.8 \tabto{0.49in} The safety of all persons within the ground is of paramount importance to the ICC. In the event that of any threatening circumstance, whether actual or perceived (including for example weather, pitch invasions, act of God, etc.), then the umpires, on the advice of the ICC Match Referee, should suspend play and all players and officials should immediately be asked to leave the field of play in a safe and orderly manner and to relocate to a secure and safe area (depending on each particular threat) pending the satisfactory passing or resolution of such threat or risk to the reasonable satisfaction of the umpires, ICC Match Referee, the head of the relevant Ground Authority, the head of ground security and/or the police as the circumstances may require.\par}\par

\end{adjustwidth}


\vspace{\baselineskip}
\begin{adjustwidth}{0.5in}{0.01in}
{\fontsize{9pt}{10.8pt}\selectfont 2.8.9 \tabto{0.49in} Where play is suspended under clause above the decision to abandon or resume play shall be the responsibility of the ICC Match Referee who shall act only after consultation with the head of ground security and the police.\par}\par

\end{adjustwidth}


\vspace{\baselineskip}
{\fontsize{11pt}{13.2pt}\selectfont \textbf{2.9 \tabto{0.47in} Position of umpires}\par}\par


\vspace{\baselineskip}
{\fontsize{9pt}{10.8pt}\selectfont The umpires shall stand where they can best see any act upon which their decision may be required.\par}\par


\vspace{\baselineskip}
\begin{adjustwidth}{0.0in}{0.15in}
{\fontsize{9pt}{10.8pt}\selectfont Subject to this over-riding consideration, the bowler’s end umpire shall stand in a position so as not to interfere with either the bowler’s run-up or the striker’s view.\par}\par

\end{adjustwidth}


\vspace{\baselineskip}
\begin{adjustwidth}{0.0in}{0.01in}
{\fontsize{9pt}{10.8pt}\selectfont The striker’s end umpire may elect to stand on the off side instead of the on side of the pitch, provided he/she informs the captain of the fielding side, the striker and the other umpire.\par}\par

\end{adjustwidth}


\vspace{\baselineskip}
{\fontsize{11pt}{13.2pt}\selectfont \textbf{2.10 \tabto{0.47in} Umpires changing ends}\par}\par


\vspace{\baselineskip}
{\fontsize{9pt}{10.8pt}\selectfont Shall not apply.\par}\par


\vspace{\baselineskip}
{\fontsize{11pt}{13.2pt}\selectfont \textbf{2.11 \tabto{0.47in} Disagreement and dispute}\par}\par


\vspace{\baselineskip}
\begin{adjustwidth}{0.0in}{0.24in}
{\fontsize{9pt}{10.8pt}\selectfont Where there is disagreement or dispute about any matter, the umpires together shall make the final decision. See also clause (Consultation by umpires).\par}\par

\end{adjustwidth}


\vspace{\baselineskip}
{\fontsize{11pt}{13.2pt}\selectfont \textbf{2.12 \tabto{0.47in} Umpire’s decision}\par}\par


\vspace{\baselineskip}
\begin{adjustwidth}{0.0in}{0.19in}
{\fontsize{9pt}{10.8pt}\selectfont An umpire may alter any decision provided that such alteration is made promptly. This apart, an umpire’s decision, once made, is final.\par}\par

\end{adjustwidth}


\vspace{\baselineskip}
{\fontsize{11pt}{13.2pt}\selectfont \textbf{2.13 \tabto{0.47in} Signals}\par}\par


\vspace{\baselineskip}
{\fontsize{9pt}{10.8pt}\selectfont 2.13.1 \tabto{0.49in} {\fontsize{8pt}{9.6pt}\selectfont The following code of signals shall be used by umpires.\par}\par}\par


\vspace{\baselineskip}
\begin{adjustwidth}{0.49in}{0.0in}
{\fontsize{9pt}{10.8pt}\selectfont 2.13.1.1 \tabto{1.17in} Signals made while the ball is in play\par}\par

\end{adjustwidth}


\vspace{\baselineskip}
\begin{adjustwidth}{1.0in}{0.0in}
{\fontsize{9pt}{10.8pt}\selectfont No ball - by extending one arm horizontally.\par}\par

\end{adjustwidth}


\vspace{\baselineskip}
\begin{adjustwidth}{1.0in}{0.0in}
{\fontsize{9pt}{10.8pt}\selectfont Out - by raising an index finger above the head. (If not out, the umpire shall call Not out.)\par}\par

\end{adjustwidth}


\vspace{\baselineskip}
\begin{adjustwidth}{1.0in}{0.0in}
{\fontsize{9pt}{10.8pt}\selectfont Wide - by extending both arms horizontally.\par}\par

\end{adjustwidth}


\vspace{\baselineskip}
\begin{adjustwidth}{1.0in}{0.0in}
{\fontsize{9pt}{10.8pt}\selectfont Dead ball - by crossing and re-crossing the wrists below the waist.\par}\par

\end{adjustwidth}


\vspace{\baselineskip}
\begin{adjustwidth}{1.18in}{0.1in}
{\fontsize{9pt}{10.8pt}\selectfont 2.13.1.2 \tabto{1.17in} When the ball is dead, the bowler’s end umpire shall repeat the signals in clause with the exception of the signal for Out, to the scorers.\par}\par

\end{adjustwidth}


\vspace{\baselineskip}
\begin{adjustwidth}{0.49in}{0.0in}
{\fontsize{9pt}{10.8pt}\selectfont 2.13.1.3 \tabto{1.17in} The signals listed below shall be made to the scorers only when the ball is dead.\par}\par

\end{adjustwidth}


\vspace{\baselineskip}
\begin{adjustwidth}{1.0in}{0.0in}
{\fontsize{9pt}{10.8pt}\selectfont Boundary 4 - by waving an arm from side to side finishing with the arm across the chest\par}\par

\end{adjustwidth}


\vspace{\baselineskip}
\begin{adjustwidth}{1.0in}{0.0in}
{\fontsize{9pt}{10.8pt}\selectfont Boundary 6 - by raising both arms above the head.\par}\par

\end{adjustwidth}


\vspace{\baselineskip}
\begin{adjustwidth}{1.0in}{0.0in}
{\fontsize{9pt}{10.8pt}\selectfont Bye - by raising an open hand above the head.\par}\par

\end{adjustwidth}


\vspace{\baselineskip}
\begin{adjustwidth}{1.0in}{0.42in}
{\fontsize{9pt}{10.8pt}\selectfont Five Penalty runs awarded to the batting side - by repeated tapping of one shoulder with the opposite hand.\par}\par

\end{adjustwidth}


\vspace{\baselineskip}
\begin{adjustwidth}{1.0in}{0.0in}
{\fontsize{9pt}{10.8pt}\selectfont Five Penalty runs awarded to the fielding side - by placing one hand on the opposite shoulder.\par}\par

\end{adjustwidth}


\vspace{\baselineskip}

\vspace{\baselineskip}

\vspace{\baselineskip}

\vspace{\baselineskip}
\begin{adjustwidth}{0.0in}{-0.01in}
\begin{Center}
{\fontsize{8pt}{9.6pt}\selectfont 5\par}
\end{Center}\par

\end{adjustwidth}


\vspace{\baselineskip}
\begin{adjustwidth}{1.0in}{0.0in}
{\fontsize{9pt}{10.8pt}\selectfont Leg bye - by touching a raised knee with the hand.\par}\par

\end{adjustwidth}


\vspace{\baselineskip}
\begin{adjustwidth}{1.0in}{0.0in}
{\fontsize{9pt}{10.8pt}\selectfont Revoke last signal - by touching both shoulders, each with the opposite hand.\par}\par

\end{adjustwidth}


\vspace{\baselineskip}
\begin{adjustwidth}{1.0in}{0.35in}
{\fontsize{9pt}{10.8pt}\selectfont Short run - by bending one arm upwards and touching the nearer shoulder with the tips of the fingers.\par}\par

\end{adjustwidth}


\vspace{\baselineskip}
\begin{adjustwidth}{1.0in}{0.19in}
{\fontsize{9pt}{10.8pt}\selectfont Free Hit – after signaling the No ball, the bowler’s end umpire extends one arm straight upwards and moves it in a circular motion.\par}\par

\end{adjustwidth}


\vspace{\baselineskip}
\begin{adjustwidth}{1.0in}{0.0in}
{\fontsize{9pt}{10.8pt}\selectfont Powerplay Over – by rotating his arm in a large circle.\par}\par

\end{adjustwidth}


\vspace{\baselineskip}
\begin{adjustwidth}{1.0in}{0.07in}
{\fontsize{9pt}{10.8pt}\selectfont The following signal is for Level 4 player conduct offences. The signal has two parts, both of which should be acknowledged separately by the scorers.\par}\par

\end{adjustwidth}


\vspace{\baselineskip}
\begin{adjustwidth}{1.0in}{0.0in}
{\fontsize{9pt}{10.8pt}\selectfont Level 4 conduct \tabto{2.49in} Part 1 - by putting one arm out to the side of the body and repeatedly\par}\par

\end{adjustwidth}


\vspace{\baselineskip}
\begin{adjustwidth}{2.5in}{0.0in}
{\fontsize{9pt}{10.8pt}\selectfont raising it and lowering it.\par}\par

\end{adjustwidth}


\vspace{\baselineskip}
\begin{adjustwidth}{2.5in}{0.0in}
{\fontsize{9pt}{10.8pt}\selectfont Part 2 - by raising an index finger, held at shoulder height, to the side of\par}\par

\end{adjustwidth}


\vspace{\baselineskip}
\begin{adjustwidth}{2.5in}{0.0in}
{\fontsize{9pt}{10.8pt}\selectfont the body.\par}\par

\end{adjustwidth}


\vspace{\baselineskip}
\begin{adjustwidth}{1.18in}{0.06in}
{\fontsize{9pt}{10.8pt}\selectfont 2.13.1.4 \tabto{1.17in} All the signals in clause are to be made by the bowler’s end umpire except that for Short run, which is to be signalled by the umpire at the end where short running occurs. However, the bowler’s end umpire shall be responsible both for the final signal of Short run to the scorers and, if more than one run is short, for informing them as to the number of runs to be recorded.\par}\par

\end{adjustwidth}


\vspace{\baselineskip}
\begin{adjustwidth}{0.5in}{0.06in}
{\fontsize{9pt}{10.8pt}\selectfont 2.13.2 \tabto{0.49in} The umpire shall wait until each signal to the scorers has been separately acknowledged by a scorer before allowing play to proceed.\par}\par

\end{adjustwidth}


\vspace{\baselineskip}
\begin{adjustwidth}{0.5in}{0.0in}
{\fontsize{9pt}{10.8pt}\selectfont If several signals are to be used, they should be given in the order that the events occurred.\par}\par

\end{adjustwidth}


\vspace{\baselineskip}
{\fontsize{11pt}{13.2pt}\selectfont \textbf{2.14 \tabto{0.47in} Informing the umpires}\par}\par


\vspace{\baselineskip}
\begin{adjustwidth}{0.0in}{0.03in}
{\fontsize{9pt}{10.8pt}\selectfont Wherever the umpires are to receive information from captains or other players under these Playing Conditions, it will be sufficient for one umpire to be so informed and for him/her to inform the other umpire.\par}\par

\end{adjustwidth}


\vspace{\baselineskip}
{\fontsize{11pt}{13.2pt}\selectfont \textbf{2.15 \tabto{0.47in} Correctness of scores}\par}\par


\vspace{\baselineskip}
\begin{adjustwidth}{0.0in}{0.11in}
{\fontsize{9pt}{10.8pt}\selectfont Consultation between umpires and scorers on doubtful points is essential. The umpires shall, throughout the match, satisfy themselves as to the correctness of the number of runs scored, the wickets that have fallen and, where appropriate, the number of overs bowled.\par}\par

\end{adjustwidth}


\vspace{\baselineskip}
\begin{adjustwidth}{0.0in}{0.08in}
{\fontsize{9pt}{10.8pt}\selectfont The umpires shall ensure that they are able to contact the scorers at any time during the match and at its conclusion to address any issues relating to the correctness of scores.\par}\par

\end{adjustwidth}


\vspace{\baselineskip}
{\fontsize{16pt}{19.2pt}\selectfont \textbf{3 \tabto{0.29in} }{\fontsize{15pt}{18.0pt}\selectfont \textbf{THE SCORERS}\par}\par}\par


\vspace{\baselineskip}
{\fontsize{11pt}{13.2pt}\selectfont \textbf{3.1 \tabto{0.47in} }{\fontsize{10pt}{12.0pt}\selectfont \textbf{Appointment of scorers}\par}\par}\par


\vspace{\baselineskip}
\begin{adjustwidth}{0.0in}{0.12in}
{\fontsize{9pt}{10.8pt}\selectfont Two scorers shall be appointed to record all runs scored, all wickets taken and, where appropriate, number of overs bowled.\par}\par

\end{adjustwidth}


\vspace{\baselineskip}
{\fontsize{11pt}{13.2pt}\selectfont \textbf{3.2 \tabto{0.47in} Correctness of scores}\par}\par


\vspace{\baselineskip}
\begin{adjustwidth}{0.0in}{0.11in}
{\fontsize{9pt}{10.8pt}\selectfont The scorers shall frequently check to ensure that their records agree and consult with the umpires if necessary. See clause (Correctness of scores).\par}\par

\end{adjustwidth}


\vspace{\baselineskip}
{\fontsize{11pt}{13.2pt}\selectfont \textbf{3.3 \tabto{0.47in} Acknowledging signals}\par}\par


\vspace{\baselineskip}
\begin{adjustwidth}{0.0in}{0.01in}
{\fontsize{9pt}{10.8pt}\selectfont The scorers shall accept all instructions and signals given to them by the umpires and shall immediately acknowledge each separate signal.\par}\par

\end{adjustwidth}


\vspace{\baselineskip}

\vspace{\baselineskip}

\vspace{\baselineskip}

\vspace{\baselineskip}

\vspace{\baselineskip}
\begin{adjustwidth}{0.0in}{-0.01in}
\begin{Center}
{\fontsize{8pt}{9.6pt}\selectfont 6\par}
\end{Center}\par

\end{adjustwidth}


\vspace{\baselineskip}
{\fontsize{16pt}{19.2pt}\selectfont \textbf{4 \tabto{0.29in} }{\fontsize{15pt}{18.0pt}\selectfont \textbf{THE BALL}\par}\par}\par


\vspace{\baselineskip}
{\fontsize{11pt}{13.2pt}\selectfont \textbf{4.1 \tabto{0.47in} Weight and size}\par}\par


\vspace{\baselineskip}
\begin{adjustwidth}{0.0in}{0.42in}
{\fontsize{9pt}{10.8pt}\selectfont The ball, when new, shall weigh not less than 5.5 ounces/155.9 g, nor more than 5.75 ounces/163 g, and shall measure not less than 8.81 in/22.4 cm, nor more than 9 in/22.9 cm in circumference.\par}\par

\end{adjustwidth}


\vspace{\baselineskip}
{\fontsize{11pt}{13.2pt}\selectfont \textbf{4.2 \tabto{0.47in} Approval and control of balls}\par}\par


\vspace{\baselineskip}
\begin{adjustwidth}{0.5in}{0.04in}
{\fontsize{9pt}{10.8pt}\selectfont 4.2.1 \tabto{0.49in} The Home Board shall provide white cricket balls of an approved standard for T20I cricket and spare used balls for changing during a match, which shall also be of the same brand. Note: The Home Board shall be required to advise the Visiting Board of the brand of ball to be used in the match(es) at least 30 days prior to the start of the match(es).\par}\par

\end{adjustwidth}


\vspace{\baselineskip}
\begin{adjustwidth}{0.5in}{0.4in}
{\fontsize{9pt}{10.8pt}\selectfont 4.2.2 \tabto{0.49in} The fielding captain or his nominee may select the ball with which he wishes to bowl from the supply provided by the Home Board. The fourth umpire shall take a box containing at least 6 new balls to the dressing room and supervise the selection of the ball.\par}\par

\end{adjustwidth}


\vspace{\baselineskip}
\begin{adjustwidth}{0.5in}{0.14in}
{\fontsize{9pt}{10.8pt}\selectfont 4.2.3 \tabto{0.49in} The umpires shall retain possession of the match ball(s) throughout the duration of the match when play is not actually taking place.\par}\par

\end{adjustwidth}


\vspace{\baselineskip}
\begin{adjustwidth}{0.5in}{0.47in}
{\fontsize{9pt}{10.8pt}\selectfont 4.2.4 \tabto{0.49in} During play umpires shall periodically and irregularly inspect the condition of the ball and shall retain possession of it at the fall of a wicket or any other disruption in play.\par}\par

\end{adjustwidth}


\vspace{\baselineskip}
{\fontsize{11pt}{13.2pt}\selectfont \textbf{4.3 \tabto{0.47in} New ball}\par}\par


\vspace{\baselineskip}
{\fontsize{9pt}{10.8pt}\selectfont 4.3.1 \tabto{0.49in} One new ball shall be used at the start of each innings.\par}\par


\vspace{\baselineskip}
{\fontsize{11pt}{13.2pt}\selectfont \textbf{4.4 \tabto{0.47in} Ball lost or becoming unfit for play}\par}\par


\vspace{\baselineskip}
\begin{adjustwidth}{0.0in}{0.04in}
{\fontsize{9pt}{10.8pt}\selectfont If, during play, the ball cannot be found or recovered or the umpires agree that it has become unfit for play through normal use, the umpires shall replace it with a ball which has had wear comparable with that which the previous ball had received before the need for its replacement. When the ball is replaced, the umpire shall inform the batsmen and the fielding captain.\par}\par

\end{adjustwidth}


\vspace{\baselineskip}
{\fontsize{16pt}{19.2pt}\selectfont \textbf{5 \tabto{0.29in} }{\fontsize{15pt}{18.0pt}\selectfont \textbf{THE BAT}\par}\par}\par


\vspace{\baselineskip}
{\fontsize{11pt}{13.2pt}\selectfont \textbf{5.1 \tabto{0.47in} }{\fontsize{10pt}{12.0pt}\selectfont \textbf{The bat}\par}\par}\par


\vspace{\baselineskip}
{\fontsize{9pt}{10.8pt}\selectfont 5.1.1 \tabto{0.49in} {\fontsize{8pt}{9.6pt}\selectfont The bat consists of two parts, a handle and a blade.\par}\par}\par


\vspace{\baselineskip}
\begin{adjustwidth}{0.5in}{0.0in}
{\fontsize{9pt}{10.8pt}\selectfont 5.1.2 \tabto{0.49in} The basic requirements and measurements of the bat are set out in this clause with detailed specifications in paragraph of Appendix B.\par}\par

\end{adjustwidth}


\vspace{\baselineskip}
{\fontsize{11pt}{13.2pt}\selectfont \textbf{5.2 \tabto{0.47in} The handle}\par}\par


\vspace{\baselineskip}
{\fontsize{9pt}{10.8pt}\selectfont 5.2.1 \tabto{0.49in} The handle is to be made principally of cane and/or wood.\par}\par


\vspace{\baselineskip}
\begin{adjustwidth}{0.5in}{0.06in}
{\fontsize{9pt}{10.8pt}\selectfont 5.2.2 \tabto{0.49in} The part of the handle that is wholly outside the blade is defined to be the upper portion of the handle. It is a straight shaft for holding the bat.\par}\par

\end{adjustwidth}


\vspace{\baselineskip}
{\fontsize{9pt}{10.8pt}\selectfont 5.2.3 \tabto{0.49in} {\fontsize{8pt}{9.6pt}\selectfont The upper portion of the handle may be covered with a grip as defined in paragraph of Appendix B.\par}\par}\par


\vspace{\baselineskip}
{\fontsize{11pt}{13.2pt}\selectfont \textbf{5.3 \tabto{0.47in} The blade}\par}\par


\vspace{\baselineskip}
{\fontsize{9pt}{10.8pt}\selectfont 5.3.1 \tabto{0.49in} {\fontsize{8pt}{9.6pt}\selectfont The blade comprises the whole of the bat apart from the handle as defined in clause and in paragraph\par}\par}\par


\vspace{\baselineskip}
\begin{adjustwidth}{0.5in}{0.0in}
{\fontsize{9pt}{10.8pt}\selectfont 1.3 of Appendix B.\par}\par

\end{adjustwidth}


\vspace{\baselineskip}
{\fontsize{9pt}{10.8pt}\selectfont 5.3.2 \tabto{0.49in} The blade shall consist solely of wood.\par}\par


\vspace{\baselineskip}
{\fontsize{11pt}{13.2pt}\selectfont \textbf{5.4 \tabto{0.47in} Protection and repair}\par}\par


\vspace{\baselineskip}
{\fontsize{9pt}{10.8pt}\selectfont Subject to the specifications in paragraph of Appendix B. and providing clause is not contravened,\par}\par


\vspace{\baselineskip}

\vspace{\baselineskip}

\vspace{\baselineskip}
\begin{adjustwidth}{0.0in}{-0.01in}
\begin{Center}
{\fontsize{8pt}{9.6pt}\selectfont 7\par}
\end{Center}\par

\end{adjustwidth}


\vspace{\baselineskip}
{\fontsize{9pt}{10.8pt}\selectfont 5.4.1 \tabto{0.49in} solely for the purposes of\par}\par


\vspace{\baselineskip}
\begin{adjustwidth}{1.0in}{0.0in}
{\fontsize{9pt}{10.8pt}\selectfont either \tabto{1.49in} protection from surface damage to the face, sides and shoulders of the blade\par}\par

\end{adjustwidth}


\vspace{\baselineskip}
\begin{adjustwidth}{1.0in}{0.0in}
{\fontsize{9pt}{10.8pt}\selectfont or \tabto{1.49in} repair to the blade after surface damage,\par}\par

\end{adjustwidth}


\vspace{\baselineskip}
\begin{adjustwidth}{0.5in}{0.11in}
{\fontsize{9pt}{10.8pt}\selectfont material that is not rigid, either at the time of its application to the blade or subsequently, may be placed on these surfaces.\par}\par

\end{adjustwidth}


\vspace{\baselineskip}
{\fontsize{9pt}{10.8pt}\selectfont 5.4.2 \tabto{0.49in} {\fontsize{8pt}{9.6pt}\selectfont for repair of the blade after damage other than surface damage\par}\par}\par


\vspace{\baselineskip}
\begin{adjustwidth}{0.49in}{0.0in}
{\fontsize{9pt}{10.8pt}\selectfont 5.4.2.1 \tabto{1.17in} solid material may be inserted into the blade.\par}\par

\end{adjustwidth}


\vspace{\baselineskip}
\begin{adjustwidth}{0.49in}{0.0in}
{\fontsize{9pt}{10.8pt}\selectfont 5.4.2.2 \tabto{1.17in} {\fontsize{8pt}{9.6pt}\selectfont The only material permitted for any insertion is wood with minimal essential adhesives.\par}\par}\par

\end{adjustwidth}


\vspace{\baselineskip}
\begin{adjustwidth}{0.5in}{0.06in}
{\fontsize{9pt}{10.8pt}\selectfont 5.4.3 \tabto{0.49in} to prevent damage to the toe, material may be placed on that part of the blade but shall not extend over any part of the face, back or sides of the blade.\par}\par

\end{adjustwidth}


\vspace{\baselineskip}
{\fontsize{11pt}{13.2pt}\selectfont \textbf{5.5 \tabto{0.47in} Damage to the ball}\par}\par


\vspace{\baselineskip}
\begin{adjustwidth}{0.5in}{0.28in}
{\fontsize{9pt}{10.8pt}\selectfont 5.5.1 \tabto{0.49in} {\fontsize{8pt}{9.6pt}\selectfont For any part of the bat, covered or uncovered, the hardness of the constituent materials and the surface texture thereof shall not be such that either or both could cause unacceptable damage to the ball.\par}\par}\par

\end{adjustwidth}


\vspace{\baselineskip}
\begin{adjustwidth}{0.5in}{0.28in}
{\fontsize{9pt}{10.8pt}\selectfont 5.5.2 \tabto{0.49in} Any material placed on any part of the bat, for whatever purpose, shall similarly not be such that it could cause unacceptable damage to the ball.\par}\par

\end{adjustwidth}


\vspace{\baselineskip}
\begin{adjustwidth}{0.5in}{0.21in}
{\fontsize{9pt}{10.8pt}\selectfont 5.5.3 \tabto{0.49in} For the purpose of this clause, unacceptable damage is any change that is greater than normal wear and tear caused by the ball striking the uncovered wooden surface of the blade.\par}\par

\end{adjustwidth}


\vspace{\baselineskip}
{\fontsize{11pt}{13.2pt}\selectfont \textbf{5.6 \tabto{0.47in} Contact with the ball}\par}\par


\vspace{\baselineskip}
{\fontsize{9pt}{10.8pt}\selectfont In these clauses,\par}\par


\vspace{\baselineskip}
\begin{adjustwidth}{0.5in}{0.38in}
{\fontsize{9pt}{10.8pt}\selectfont 5.6.1 \tabto{0.49in} reference to the bat shall imply that the bat is held in the batsman’s hand or a glove worn on his hand, unless stated otherwise.\par}\par

\end{adjustwidth}


\vspace{\baselineskip}
{\fontsize{9pt}{10.8pt}\selectfont 5.6.2 \tabto{0.49in} contact between the ball and any of 5.6.2.1 to 5.6.2.4\par}\par


\vspace{\baselineskip}
\begin{adjustwidth}{0.49in}{0.0in}
{\fontsize{9pt}{10.8pt}\selectfont 5.6.2.1 \tabto{1.17in} the bat itself\par}\par

\end{adjustwidth}


\vspace{\baselineskip}
\begin{adjustwidth}{0.49in}{0.0in}
{\fontsize{9pt}{10.8pt}\selectfont 5.6.2.2 \tabto{1.17in} {\fontsize{8pt}{9.6pt}\selectfont the batsman’s hand holding the bat\par}\par}\par

\end{adjustwidth}


\vspace{\baselineskip}
\begin{adjustwidth}{0.49in}{0.0in}
{\fontsize{9pt}{10.8pt}\selectfont 5.6.2.3 \tabto{1.17in} {\fontsize{8pt}{9.6pt}\selectfont any part of a glove worn on the batsman’s hand holding the bat\par}\par}\par

\end{adjustwidth}


\vspace{\baselineskip}
\begin{adjustwidth}{0.49in}{0.0in}
{\fontsize{9pt}{10.8pt}\selectfont 5.6.2.4 \tabto{1.17in} {\fontsize{8pt}{9.6pt}\selectfont any additional materials permitted under 5.4\par}\par}\par

\end{adjustwidth}


\vspace{\baselineskip}
\begin{adjustwidth}{0.5in}{0.0in}
{\fontsize{9pt}{10.8pt}\selectfont shall be regarded as the ball striking or touching the bat or being struck by the bat.\par}\par

\end{adjustwidth}


\vspace{\baselineskip}
{\fontsize{11pt}{13.2pt}\selectfont \textbf{5.7 \tabto{0.47in} Bat size limits}\par}\par


\vspace{\baselineskip}
\begin{adjustwidth}{0.5in}{0.28in}
{\fontsize{9pt}{10.8pt}\selectfont 5.7.1 \tabto{0.49in} The overall length of the bat, when the lower portion of the handle is inserted, shall not be more than 38 in/96.52 cm.\par}\par

\end{adjustwidth}


\vspace{\baselineskip}
{\fontsize{9pt}{10.8pt}\selectfont 5.7.2 \tabto{0.49in} The blade of the bat shall not exceed the following dimensions:\par}\par


\vspace{\baselineskip}
\begin{adjustwidth}{0.5in}{0.0in}
{\fontsize{9pt}{10.8pt}\selectfont Width: 4.25in / 10.8 cm\par}\par

\end{adjustwidth}


\vspace{\baselineskip}
\begin{adjustwidth}{0.5in}{0.0in}
{\fontsize{9pt}{10.8pt}\selectfont Depth: 2.64in / 6.7 cm\par}\par

\end{adjustwidth}


\vspace{\baselineskip}
\begin{adjustwidth}{0.5in}{0.0in}
{\fontsize{9pt}{10.8pt}\selectfont Edges: 1.56in / 4.0cm.\par}\par

\end{adjustwidth}


\vspace{\baselineskip}
\begin{adjustwidth}{0.5in}{0.1in}
{\fontsize{9pt}{10.8pt}\selectfont Furthermore, it should also be able to pass through a bat gauge as described in paragraph 1.6 of Appendix B.\par}\par

\end{adjustwidth}


\vspace{\baselineskip}
{\fontsize{9pt}{10.8pt}\selectfont 5.7.3 \tabto{0.49in} {\fontsize{8pt}{9.6pt}\selectfont The handle shall not exceed 52$\%$  of the overall length of the bat.\par}\par}\par


\vspace{\baselineskip}
{\fontsize{9pt}{10.8pt}\selectfont 5.7.4 \tabto{0.49in} {\fontsize{8pt}{9.6pt}\selectfont The material permitted for covering the blade in clause shall not exceed 0.04 in/0.1 cm in thickness.\par}\par}\par


\vspace{\baselineskip}

\vspace{\baselineskip}

\vspace{\baselineskip}
\begin{adjustwidth}{0.0in}{-0.01in}
\begin{Center}
{\fontsize{8pt}{9.6pt}\selectfont 8\par}
\end{Center}\par

\end{adjustwidth}


\vspace{\baselineskip}
{\fontsize{9pt}{10.8pt}\selectfont 5.7.5 \tabto{0.49in} The maximum permitted thickness of protective material placed on the toe of the blade is 0.12 in/0.3 cm.\par}\par


\vspace{\baselineskip}
{\fontsize{11pt}{13.2pt}\selectfont \textbf{5.8 \tabto{0.47in} Categories of bat}\par}\par


\vspace{\baselineskip}
{\fontsize{9pt}{10.8pt}\selectfont 5.8.1 \tabto{0.49in} Type A bats conform to clauses to inclusive.\par}\par


\vspace{\baselineskip}
{\fontsize{9pt}{10.8pt}\selectfont 5.8.2 \tabto{0.49in} {\fontsize{8pt}{9.6pt}\selectfont Only Type A bats may be used in T20I matches.\par}\par}\par


\vspace{\baselineskip}
{\fontsize{16pt}{19.2pt}\selectfont \textbf{6 \tabto{0.29in} }{\fontsize{15pt}{18.0pt}\selectfont \textbf{THE PITCH}\par}\par}\par


\vspace{\baselineskip}
{\fontsize{11pt}{13.2pt}\selectfont \textbf{6.1 \tabto{0.47in} Area of pitch}\par}\par


\vspace{\baselineskip}
\begin{adjustwidth}{0.0in}{0.08in}
{\fontsize{9pt}{10.8pt}\selectfont The pitch is a rectangular area of the ground 22 yards/20.12 m in length and 10 ft/3.05 m in width. It is bounded at either end by the bowling creases and on either side by imaginary lines, one each side of the imaginary line joining the centres of the two middle stumps, each parallel to it and 5 ft/1.52 m from it. If the pitch is next to an artificial pitch which is closer than 5 ft/1.52 m from the middle stumps, the pitch on that side will extend only to the junction of the two surfaces. See clauses (Description, width and pitching) and (The bowling crease).\par}\par

\end{adjustwidth}


\vspace{\baselineskip}
{\fontsize{11pt}{13.2pt}\selectfont \textbf{6.2 \tabto{0.47in} }{\fontsize{10pt}{12.0pt}\selectfont \textbf{Fitness of pitch for play}\par}\par}\par


\vspace{\baselineskip}
\begin{adjustwidth}{0.0in}{0.25in}
{\fontsize{9pt}{10.8pt}\selectfont The umpires shall be the sole judges of the fitness of the pitch for play. See clauses (Fitness for play) and  (Suspension of play in dangerous or unreasonable conditions).\par}\par

\end{adjustwidth}


\vspace{\baselineskip}
{\fontsize{11pt}{13.2pt}\selectfont \textbf{6.3 \tabto{0.47in} Selection and preparation}\par}\par


\vspace{\baselineskip}
\begin{adjustwidth}{0.0in}{0.07in}
{\fontsize{9pt}{10.8pt}\selectfont Before the match, the Ground Authority shall be responsible for the selection and preparation of the pitch. During the match, the umpires shall control its use and maintenance.\par}\par

\end{adjustwidth}


\vspace{\baselineskip}
\begin{adjustwidth}{0.5in}{0.12in}
{\fontsize{9pt}{10.8pt}\selectfont 6.3.1 \tabto{0.49in} The Ground Authority shall ensure that during the period prior to the start of play and during intervals, the pitch area shall be roped off so as to prevent unauthorised access. (The pitch area shall include an area at least 2 metres beyond the rectangle made by the crease markings at both ends of the pitch).\par}\par

\end{adjustwidth}


\vspace{\baselineskip}
\begin{adjustwidth}{0.5in}{0.08in}
\begin{justify}
{\fontsize{9pt}{10.8pt}\selectfont 6.3.2 \tabto{0.49in} The fourth umpire shall ensure that, prior to the start of play and during any intervals, only authorised staff, the ICC match officials, players, team coaches and authorised television personnel shall be allowed access to the pitch area. Such access shall be subject to the following limitations:\par}
\end{justify}\par

\end{adjustwidth}


\vspace{\baselineskip}
\begin{adjustwidth}{1.18in}{0.39in}
{\fontsize{9pt}{10.8pt}\selectfont 6.3.2.1 \tabto{1.17in} Only captains and team coaches may walk on the actual playing surface of the pitch area (outside of the crease markings).\par}\par

\end{adjustwidth}


\vspace{\baselineskip}
\begin{adjustwidth}{1.18in}{0.11in}
{\fontsize{9pt}{10.8pt}\selectfont 6.3.2.2 \tabto{1.17in} Access to the pitch area by television personnel shall be restricted to one camera crew (including one or two television commentators) of the official licensed television broadcaster(s) (but not news crews).\par}\par

\end{adjustwidth}


\vspace{\baselineskip}
\begin{adjustwidth}{0.49in}{0.0in}
{\fontsize{9pt}{10.8pt}\selectfont 6.3.2.3 \tabto{1.17in} {\fontsize{8pt}{9.6pt}\selectfont No spiked footwear shall be permitted.\par}\par}\par

\end{adjustwidth}


\vspace{\baselineskip}
\begin{adjustwidth}{1.18in}{0.12in}
{\fontsize{9pt}{10.8pt}\selectfont 6.3.2.4 \tabto{1.17in} No one shall be permitted to bounce a ball on the pitch, strike it with a bat or cause damage to the pitch in any other way.\par}\par

\end{adjustwidth}


\vspace{\baselineskip}
\begin{adjustwidth}{0.49in}{0.0in}
{\fontsize{9pt}{10.8pt}\selectfont 6.3.2.5 \tabto{1.17in} {\fontsize{8pt}{9.6pt}\selectfont Access shall not interfere with pitch preparation.\par}\par}\par

\end{adjustwidth}


\vspace{\baselineskip}
{\fontsize{9pt}{10.8pt}\selectfont 6.3.3 \tabto{0.49in} {\fontsize{8pt}{9.6pt}\selectfont In the event of any dispute, the ICC Match Referee will rule and his ruling will be final.\par}\par}\par


\vspace{\baselineskip}
{\fontsize{11pt}{13.2pt}\selectfont \textbf{6.4 \tabto{0.47in} Changing the pitch}\par}\par


\vspace{\baselineskip}
\begin{adjustwidth}{0.5in}{0.18in}
{\fontsize{9pt}{10.8pt}\selectfont 6.4.1 \tabto{0.49in} If the on-field umpires decide that it is dangerous or unreasonable for play to continue on the match pitch, they shall stop play and immediately advise the ICC Match Referee.\par}\par

\end{adjustwidth}


\vspace{\baselineskip}
{\fontsize{9pt}{10.8pt}\selectfont 6.4.2 \tabto{0.49in} {\fontsize{8pt}{9.6pt}\selectfont The on-field umpires and the ICC Match Referee shall then consult with both captains.\par}\par}\par


\vspace{\baselineskip}
{\fontsize{9pt}{10.8pt}\selectfont 6.4.3 \tabto{0.49in} If the captains agree to continue, play shall resume.\par}\par


\vspace{\baselineskip}
{\fontsize{9pt}{10.8pt}\selectfont 6.4.4 \tabto{0.49in} {\fontsize{8pt}{9.6pt}\selectfont If the decision is not to resume play, the on-field umpires together with the ICC Match Referee shall consider\par}\par}\par


\vspace{\baselineskip}
\begin{adjustwidth}{0.0in}{0.03in}
\begin{Center}
{\fontsize{9pt}{10.8pt}\selectfont whether the existing pitch can be repaired and the match resumed from the point it was stopped. In\par}
\end{Center}\par

\end{adjustwidth}


\vspace{\baselineskip}

\vspace{\baselineskip}

\vspace{\baselineskip}

\vspace{\baselineskip}

\vspace{\baselineskip}
\begin{adjustwidth}{0.0in}{-0.01in}
\begin{Center}
{\fontsize{8pt}{9.6pt}\selectfont 9\par}
\end{Center}\par

\end{adjustwidth}


\vspace{\baselineskip}

\vspace{\baselineskip}
\begin{adjustwidth}{0.5in}{0.03in}
{\fontsize{8pt}{9.6pt}\selectfont considering whether to authorise such repairs, the ICC Match Referee must consider whether this would place either side at an unfair advantage, given the play that had already taken place on the dangerous pitch.\par}\par

\end{adjustwidth}


\vspace{\baselineskip}
\begin{adjustwidth}{0.5in}{0.36in}
{\fontsize{9pt}{10.8pt}\selectfont 6.4.5 \tabto{0.49in} If the decision is that the existing pitch cannot be repaired, then the match is to be abandoned with the following consequences:\par}\par

\end{adjustwidth}


\vspace{\baselineskip}
\begin{adjustwidth}{1.18in}{0.01in}
{\fontsize{9pt}{10.8pt}\selectfont 6.4.5.1 \tabto{1.17in} In the event of the required number of overs to constitute a match having been completed at the time the match is abandoned, the result shall be determined according to the provisions of clause \par}\par

\end{adjustwidth}


\vspace{\baselineskip}
\begin{adjustwidth}{1.18in}{0.12in}
{\fontsize{9pt}{10.8pt}\selectfont 6.4.5.2 \tabto{1.17in} In the event of the required number of overs to constitute a match not having been completed, the match will be abandoned as a no result.\par}\par

\end{adjustwidth}


\vspace{\baselineskip}
\begin{adjustwidth}{0.5in}{0.01in}
{\fontsize{9pt}{10.8pt}\selectfont 6.4.6 \tabto{0.49in} {\fontsize{8pt}{9.6pt}\selectfont If the abandonment occurs on the day of the match, the ICC Match Referee shall consult with the Home Board with the objective of finding a way for a new match (including a new nomination of teams and toss) to commence on the same date and venue. Such a match may be played either on the repaired pitch or on another pitch, subject to the ICC Match Referee and the relevant Ground Authority both being satisfied that the new pitch will be of the required T20I standard. The playing time lost between the scheduled start time of the original match and the actual start time of the new match will be covered by the provisions of clause \par}\par}\par

\end{adjustwidth}


\vspace{\baselineskip}
\begin{adjustwidth}{0.0in}{0.24in}
\begin{FlushRight}
{\fontsize{9pt}{10.8pt}\selectfont 6.4.7\ \ \  If it is not possible to start a new match on the scheduled day of the match, the relevant officials from the participating Boards shall agree on whether the match can be replayed within the existing tour schedule.\par}
\end{FlushRight}\par

\end{adjustwidth}


\vspace{\baselineskip}
\begin{adjustwidth}{0.5in}{0.1in}
{\fontsize{9pt}{10.8pt}\selectfont 6.4.8 \tabto{0.49in} Throughout the above decision making processes, the ICC Match Referee shall keep informed both captains and the head of the Ground Authority. The head of the Ground Authority shall ensure that suitable and prompt public announcements are made.\par}\par

\end{adjustwidth}


\vspace{\baselineskip}
{\fontsize{11pt}{13.2pt}\selectfont \textbf{6.5 \tabto{0.47in} Non-turf pitches}\par}\par


\vspace{\baselineskip}
\begin{adjustwidth}{0.0in}{0.22in}
{\fontsize{9pt}{10.8pt}\selectfont All T20I matches shall be played on natural turf pitches. The use of PVA and other adhesives in the preparation of pitches is not permitted.\par}\par

\end{adjustwidth}


\vspace{\baselineskip}
{\fontsize{16pt}{19.2pt}\selectfont \textbf{7 \tabto{0.29in} }{\fontsize{15pt}{18.0pt}\selectfont \textbf{THE CREASES}\par}\par}\par


\vspace{\baselineskip}
{\fontsize{11pt}{13.2pt}\selectfont \textbf{7.1 \tabto{0.47in} The creases}\par}\par


\vspace{\baselineskip}
\begin{adjustwidth}{0.0in}{0.21in}
{\fontsize{9pt}{10.8pt}\selectfont The positions of a bowling crease, a popping crease and two return creases shall be marked by white lines, as set out in clauses and at each end of the pitch. See paragraph of Appendix C.\par}\par

\end{adjustwidth}


\vspace{\baselineskip}
{\fontsize{11pt}{13.2pt}\selectfont \textbf{7.2 \tabto{0.47in} The bowling crease}\par}\par


\vspace{\baselineskip}
\begin{adjustwidth}{0.0in}{0.21in}
{\fontsize{9pt}{10.8pt}\selectfont The bowling crease, which is the back edge of the crease marking, is the line that marks the end of the pitch, as in clause (Area of pitch). It shall be 8 ft 8 in/2.64 m in length.\par}\par

\end{adjustwidth}


\vspace{\baselineskip}
{\fontsize{11pt}{13.2pt}\selectfont \textbf{7.3 \tabto{0.47in} The popping crease}\par}\par


\vspace{\baselineskip}
\begin{adjustwidth}{0.0in}{0.11in}
{\fontsize{9pt}{10.8pt}\selectfont The popping crease, which is the back edge of the crease marking, shall be in front of and parallel to the bowling crease and shall be 4 ft/1.22 m from it. The popping crease shall be marked to a minimum of 15 yards/13.71 m on either side of the imaginary line joining the centres of the two middle stumps and shall be considered to be unlimited in length.\par}\par

\end{adjustwidth}


\vspace{\baselineskip}
{\fontsize{11pt}{13.2pt}\selectfont \textbf{7.4 \tabto{0.47in} The return creases}\par}\par


\vspace{\baselineskip}
\begin{adjustwidth}{0.0in}{0.06in}
{\fontsize{9pt}{10.8pt}\selectfont The return creases, which are the inside edges of the crease markings, shall be at right angles to the popping crease at a distance of 4 ft 4 in/1.32 m either side of the imaginary line joining the centres of the two middle stumps. Each return crease shall be marked from the popping crease to a minimum of 8 ft/2.44 m behind it and shall be considered to be unlimited in length.\par}\par

\end{adjustwidth}


\vspace{\baselineskip}
{\fontsize{11pt}{13.2pt}\selectfont \textbf{7.5 \tabto{0.47in} Additional Crease Markings}\par}\par


\vspace{\baselineskip}
\begin{adjustwidth}{0.0in}{0.15in}
{\fontsize{9pt}{10.8pt}\selectfont As a guideline to the umpires for the calling of Wides on the offside, the crease markings detailed in paragraph 1 of Appendix C shall be marked in white at each end of the pitch.\par}\par

\end{adjustwidth}


\vspace{\baselineskip}

\vspace{\baselineskip}

\vspace{\baselineskip}

\vspace{\baselineskip}
\begin{Center}
{\fontsize{8pt}{9.6pt}\selectfont 10\par}
\end{Center}\par


\vspace{\baselineskip}
{\fontsize{16pt}{19.2pt}\selectfont \textbf{8 \tabto{0.29in} }{\fontsize{15pt}{18.0pt}\selectfont \textbf{THE WICKETS}\par}\par}\par


\vspace{\baselineskip}
{\fontsize{11pt}{13.2pt}\selectfont \textbf{8.1 \tabto{0.47in} Description, width and pitching}\par}\par


\vspace{\baselineskip}
\begin{adjustwidth}{0.0in}{0.14in}
{\fontsize{9pt}{10.8pt}\selectfont Two sets of wickets shall be pitched opposite and parallel to each other in the centres of the bowling creases. Each set shall be 9 in/22.86 cm wide and shall consist of three wooden stumps with two wooden bails on top. See paragraph of Appendix B.\par}\par

\end{adjustwidth}


\vspace{\baselineskip}
{\fontsize{11pt}{13.2pt}\selectfont \textbf{8.2 \tabto{0.47in} Size of stumps}\par}\par


\vspace{\baselineskip}
\begin{adjustwidth}{0.0in}{0.04in}
{\fontsize{9pt}{10.8pt}\selectfont The tops of the stumps shall be 28 in/71.12 cm above the playing surface and shall be dome shaped except for the bail grooves. The portion of a stump above the playing surface shall be cylindrical apart from the domed top, with circular section of diameter not less than 1.38 in/3.50 cm nor more than 1.5 in/3.81 cm. See paragraph of Appendix B.\par}\par

\end{adjustwidth}


\vspace{\baselineskip}
\begin{adjustwidth}{0.0in}{0.24in}
{\fontsize{9pt}{10.8pt}\selectfont For televised matches the Home Board may provide a slightly larger cylindrical stump to accommodate the stump camera. When the larger stump is used, all three stumps must be exactly the same size.\par}\par

\end{adjustwidth}


\vspace{\baselineskip}
{\fontsize{11pt}{13.2pt}\selectfont \textbf{8.3 \tabto{0.47in} The bails}\par}\par


\vspace{\baselineskip}
{\fontsize{9pt}{10.8pt}\selectfont 8.3.1 \tabto{0.49in} The bails, when in position on top of the stumps,\par}\par


\vspace{\baselineskip}
\begin{itemize}
	\item {\fontsize{9pt}{10.8pt}\selectfont shall not project more than 0.5 in/1.27 cm above them.\par}\par


\vspace{\baselineskip}
	\item {\fontsize{9pt}{10.8pt}\selectfont shall fit between the stumps without forcing them out of the vertical.\par}
\end{itemize}\par


\vspace{\baselineskip}
{\fontsize{9pt}{10.8pt}\selectfont 8.3.2 \tabto{0.49in} {\fontsize{8pt}{9.6pt}\selectfont Each bail shall conform to the following specifications (see paragraph of Appendix B).\par}\par}\par


\vspace{\baselineskip}
\begin{adjustwidth}{0.5in}{0.0in}
{\fontsize{9pt}{10.8pt}\selectfont Overall length 4.31 in/10.95 cm\par}\par

\end{adjustwidth}


\vspace{\baselineskip}
\begin{adjustwidth}{0.5in}{0.0in}
{\fontsize{9pt}{10.8pt}\selectfont Length of barrel 2.13 in /5.40 cm\par}\par

\end{adjustwidth}


\vspace{\baselineskip}
\begin{adjustwidth}{0.5in}{0.0in}
{\fontsize{9pt}{10.8pt}\selectfont Longer spigot 1.38 in/3.50 cm\par}\par

\end{adjustwidth}


\vspace{\baselineskip}
\begin{adjustwidth}{0.5in}{0.0in}
{\fontsize{9pt}{10.8pt}\selectfont Shorter spigot 0.81 in/2.06 cm.\par}\par

\end{adjustwidth}


\vspace{\baselineskip}
{\fontsize{9pt}{10.8pt}\selectfont 8.3.3 \tabto{0.49in} {\fontsize{8pt}{9.6pt}\selectfont The two spigots and the barrel shall have the same centre line.\par}\par}\par


\vspace{\baselineskip}
\begin{adjustwidth}{0.5in}{0.07in}
{\fontsize{9pt}{10.8pt}\selectfont 8.3.4 \tabto{0.49in} Devices aimed at protecting player safety by limiting the distance that a bail can travel off the stumps will be allowed, subject to the approval of the Home Board and the ICC.\par}\par

\end{adjustwidth}


\vspace{\baselineskip}
{\fontsize{11pt}{13.2pt}\selectfont \textbf{8.4 \tabto{0.47in} Dispensing with bails}\par}\par


\vspace{\baselineskip}
\begin{adjustwidth}{0.0in}{0.06in}
{\fontsize{8pt}{9.6pt}\selectfont The umpires may agree to dispense with the use of bails, if necessary. If they so agree then no bails shall be used at either end. The use of bails shall be resumed as soon as conditions permit. See clause (Dispensing with bails).\par}\par

\end{adjustwidth}


\vspace{\baselineskip}
{\fontsize{11pt}{13.2pt}\selectfont \textbf{8.5 \tabto{0.47in} LED Wickets}\par}\par


\vspace{\baselineskip}
{\fontsize{9pt}{10.8pt}\selectfont The use of approved LED Wickets is permitted. Refer also to paragraphs and of Appendix D.\par}\par


\vspace{\baselineskip}
\begin{enumerate}[label*={\fontsize{16pt}{16pt}\selectfont \textbf{\arabic*.}}]
	\item {\fontsize{16pt}{19.2pt}\selectfont \textbf{PREPARATION AND MAINTENANCE OF THE PLAYING AREA}\par}
\end{enumerate}\par


\vspace{\baselineskip}
{\fontsize{11pt}{13.2pt}\selectfont \textbf{9.1 \tabto{0.47in} Rolling}\par}\par


\vspace{\baselineskip}
{\fontsize{9pt}{10.8pt}\selectfont The pitch shall not be rolled during the match except as permitted in clauses and \par}\par


\vspace{\baselineskip}
{\fontsize{9pt}{10.8pt}\selectfont 9.1.1 \tabto{0.49in} {\fontsize{8pt}{9.6pt}\selectfont Frequency and duration of rolling\par}\par}\par


\vspace{\baselineskip}
\begin{adjustwidth}{0.5in}{0.01in}
{\fontsize{9pt}{10.8pt}\selectfont During the match the pitch may be rolled at the request of the captain of the side batting second, for a period of not more than 7 minutes, before the start of the second innings.\par}\par

\end{adjustwidth}


\vspace{\baselineskip}

\vspace{\baselineskip}

\vspace{\baselineskip}

\vspace{\baselineskip}

\vspace{\baselineskip}
\begin{Center}
{\fontsize{8pt}{9.6pt}\selectfont 11\par}
\end{Center}\par


\vspace{\baselineskip}
{\fontsize{9pt}{10.8pt}\selectfont 9.1.2 \tabto{0.49in} Rolling after a delayed start\par}\par


\vspace{\baselineskip}
\begin{adjustwidth}{0.5in}{0.1in}
{\fontsize{9pt}{10.8pt}\selectfont In addition to the rolling permitted above, if, after the toss and before the first innings of the match, the start is delayed, the captain of the batting side may request that the pitch be rolled for not more than 7 minutes. However, if the umpires together agree that the delay has had no significant effect on the state of the pitch, they shall refuse such request for rolling of the pitch.\par}\par

\end{adjustwidth}


\vspace{\baselineskip}
{\fontsize{9pt}{10.8pt}\selectfont 9.1.3 \tabto{0.49in} {\fontsize{8pt}{9.6pt}\selectfont Choice of rollers\par}\par}\par


\vspace{\baselineskip}
\begin{adjustwidth}{0.5in}{0.0in}
{\fontsize{8pt}{9.6pt}\selectfont If there is more than one roller available the captain of the batting side shall choose which one is to be used.\par}\par

\end{adjustwidth}


\vspace{\baselineskip}
{\fontsize{9pt}{10.8pt}\selectfont The following shall apply in addition to clause \par}\par


\vspace{\baselineskip}
\begin{adjustwidth}{0.5in}{0.07in}
\begin{justify}
{\fontsize{9pt}{10.8pt}\selectfont 9.1.4 \tabto{0.49in} Prior to the scheduled time for the toss, the artificial drying of the pitch and outfield shall be at the discretion of the Ground Authority. Thereafter and throughout the match the drying of the outfield may be undertaken at any time by the Ground Authority, but the drying of the affected area of the pitch shall be carried out only on the instructions and under the supervision of the umpires. The umpires shall be empowered to have the pitch dried without reference to the captains at any time they are of the opinion that it is unfit for play.\par}
\end{justify}\par

\end{adjustwidth}


\vspace{\baselineskip}
\begin{adjustwidth}{0.5in}{0.15in}
{\fontsize{9pt}{10.8pt}\selectfont 9.1.5 \tabto{0.49in} The umpires may instruct the Ground Authority to use any available equipment, including any roller for the purpose of drying the pitch and making it fit for play.\par}\par

\end{adjustwidth}


\vspace{\baselineskip}
{\fontsize{9pt}{10.8pt}\selectfont 9.1.6 \tabto{0.49in} {\fontsize{8pt}{9.6pt}\selectfont An absorbent roller may be used to remove water from the covers including the cover on the match pitch.\par}\par}\par


\vspace{\baselineskip}
{\fontsize{11pt}{13.2pt}\selectfont \textbf{9.2 \tabto{0.47in} Clearing debris from the pitch}\par}\par


\vspace{\baselineskip}
{\fontsize{9pt}{10.8pt}\selectfont 9.2.1 \tabto{0.49in} The pitch shall be cleared of any debris\par}\par


\vspace{\baselineskip}
\begin{adjustwidth}{0.49in}{0.0in}
{\fontsize{9pt}{10.8pt}\selectfont 9.2.1.1 \tabto{1.17in} between innings. This shall precede rolling if any is to take place.\par}\par

\end{adjustwidth}


\vspace{\baselineskip}
\begin{adjustwidth}{0.5in}{0.08in}
\begin{justify}
{\fontsize{9pt}{10.8pt}\selectfont 9.2.2 \tabto{0.49in} The clearance of debris in clause shall be done by sweeping, except where the umpires consider that this may be detrimental to the surface of the pitch. In this case the debris must be cleared from that area by hand, without sweeping.\par}
\end{justify}\par

\end{adjustwidth}


\vspace{\baselineskip}
\begin{adjustwidth}{0.5in}{0.08in}
\begin{justify}
{\fontsize{9pt}{10.8pt}\selectfont 9.2.3 \tabto{0.49in} In addition to clause debris may be cleared from the pitch by hand, without sweeping, before mowing and whenever either umpire considers it necessary.\par}
\end{justify}\par

\end{adjustwidth}


\vspace{\baselineskip}
{\fontsize{11pt}{13.2pt}\selectfont \textbf{9.3 \tabto{0.47in} Mowing}\par}\par


\vspace{\baselineskip}
{\fontsize{9pt}{10.8pt}\selectfont 9.3.1 \tabto{0.49in} {\fontsize{8pt}{9.6pt}\selectfont Responsibility for mowing\par}\par}\par


\vspace{\baselineskip}
\begin{adjustwidth}{1.18in}{0.44in}
{\fontsize{9pt}{10.8pt}\selectfont 9.3.1.1 \tabto{1.17in} All mowings which are carried out before the match shall be the sole responsibility of the Ground Authority.\par}\par

\end{adjustwidth}


\vspace{\baselineskip}
{\fontsize{11pt}{13.2pt}\selectfont \textbf{9.4 \tabto{0.47in} Watering the pitch}\par}\par


\vspace{\baselineskip}
{\fontsize{9pt}{10.8pt}\selectfont The pitch shall not be watered during the match.\par}\par


\vspace{\baselineskip}
{\fontsize{11pt}{13.2pt}\selectfont \textbf{9.5 \tabto{0.47in} Re-marking creases}\par}\par


\vspace{\baselineskip}
{\fontsize{9pt}{10.8pt}\selectfont Creases shall be re-marked whenever either umpire considers it necessary.\par}\par


\vspace{\baselineskip}
{\fontsize{11pt}{13.2pt}\selectfont \textbf{9.6 \tabto{0.47in} Maintenance of footholes}\par}\par


\vspace{\baselineskip}
\begin{adjustwidth}{0.0in}{0.33in}
{\fontsize{9pt}{10.8pt}\selectfont The umpires shall ensure that the holes made by the bowlers and batsmen are cleaned out and dried whenever necessary to facilitate play.\par}\par

\end{adjustwidth}


\vspace{\baselineskip}
\begin{adjustwidth}{0.0in}{0.18in}
{\fontsize{9pt}{10.8pt}\selectfont The umpires shall allow, if necessary, the returfing of footholes made by the bowlers in their delivery strides, or the use of quick-setting fillings for the same purpose.\par}\par

\end{adjustwidth}


\vspace{\baselineskip}
\begin{adjustwidth}{0.0in}{0.24in}
{\fontsize{9pt}{10.8pt}\selectfont In addition, the umpires shall see that wherever possible and whenever it is considered necessary, action is taken during all intervals in play to do whatever is practicable to improve the bowler’s footholes.\par}\par

\end{adjustwidth}


\vspace{\baselineskip}

\vspace{\baselineskip}

\vspace{\baselineskip}

\vspace{\baselineskip}

\vspace{\baselineskip}

\vspace{\baselineskip}
\begin{Center}
{\fontsize{8pt}{9.6pt}\selectfont 12\par}
\end{Center}\par


\vspace{\baselineskip}
{\fontsize{11pt}{13.2pt}\selectfont \textbf{9.7 \tabto{0.47in} Securing of footholds and maintenance of pitch}\par}\par


\vspace{\baselineskip}
\begin{adjustwidth}{0.0in}{0.07in}
{\fontsize{9pt}{10.8pt}\selectfont During play, umpires shall allow the players to secure their footholds by the use of sawdust provided that no damage to the pitch is caused and that clause (Unfair play) is not contravened.\par}\par

\end{adjustwidth}


\vspace{\baselineskip}
{\fontsize{11pt}{13.2pt}\selectfont \textbf{9.8 \tabto{0.47in} Protection and preparation of adjacent pitches during matches}\par}\par


\vspace{\baselineskip}
\begin{adjustwidth}{0.0in}{0.1in}
{\fontsize{9pt}{10.8pt}\selectfont The protection (by way of an appropriate cover) and preparation of pitches which are adjacent to the match pitch will be permitted during the match subject to the following:\par}\par

\end{adjustwidth}


\vspace{\baselineskip}
\begin{adjustwidth}{0.5in}{0.29in}
{\fontsize{9pt}{10.8pt}\selectfont 9.8.1 \tabto{0.49in} Such measures will only be possible if requested by the Ground Authority and approved by the umpires before the start of the match.\par}\par

\end{adjustwidth}


\vspace{\baselineskip}
\begin{adjustwidth}{0.5in}{0.08in}
{\fontsize{9pt}{10.8pt}\selectfont 9.8.2 \tabto{0.49in} Approval should only be granted where such measures are unavoidable and will not compromise the safety of the players or their ability to execute their actions with complete freedom.\par}\par

\end{adjustwidth}


\vspace{\baselineskip}
{\fontsize{9pt}{10.8pt}\selectfont 9.8.3 \tabto{0.49in} {\fontsize{8pt}{9.6pt}\selectfont The preparation work shall be carried out under the supervision of the fourth umpire.\par}\par}\par


\vspace{\baselineskip}
\begin{adjustwidth}{0.5in}{0.21in}
{\fontsize{9pt}{10.8pt}\selectfont 9.8.4 \tabto{0.49in} The consent of the captains is not required but the umpires shall advise both captains and the ICC Match Referee before the start of the match on what has been agreed.\par}\par

\end{adjustwidth}


\vspace{\baselineskip}
{\fontsize{16pt}{19.2pt}\selectfont \textbf{10 COVERING THE PITCH}\par}\par


\vspace{\baselineskip}
{\fontsize{11pt}{13.2pt}\selectfont \textbf{10.1 \tabto{0.47in} Before the match}\par}\par


\vspace{\baselineskip}
\begin{adjustwidth}{0.0in}{0.38in}
{\fontsize{9pt}{10.8pt}\selectfont The use of covers before the match is the responsibility of the Ground Authority and may include full covering if required.\par}\par

\end{adjustwidth}


\vspace{\baselineskip}
{\fontsize{9pt}{10.8pt}\selectfont The pitch shall be entirely protected against rain up to the commencement of play.\par}\par


\vspace{\baselineskip}
\begin{adjustwidth}{0.0in}{0.01in}
{\fontsize{9pt}{10.8pt}\selectfont However, the Ground Authority shall grant suitable facility to the captains to inspect the pitch before the nomination of their players and to the umpires to discharge their duties as laid down in clauses (The umpires), (The pitch),  (The creases), (The wickets), and (Preparation and maintenance of the playing area).\par}\par

\end{adjustwidth}


\vspace{\baselineskip}
{\fontsize{11pt}{13.2pt}\selectfont \textbf{10.2 \tabto{0.47in} During the match}\par}\par


\vspace{\baselineskip}
\begin{adjustwidth}{0.0in}{0.11in}
{\fontsize{9pt}{10.8pt}\selectfont The pitch shall be entirely protected against rain up to the commencement of play, and for the duration of the period of the match.\par}\par

\end{adjustwidth}


\vspace{\baselineskip}
\begin{adjustwidth}{0.0in}{0.08in}
{\fontsize{9pt}{10.8pt}\selectfont The covers must totally protect the pitch and also the pitch surroundings, to a minimum of 5 metres either side of the pitch, and any worn or soft areas in the outfield.\par}\par

\end{adjustwidth}


\vspace{\baselineskip}
\begin{adjustwidth}{0.0in}{0.14in}
{\fontsize{9pt}{10.8pt}\selectfont The bowlers’ run-ups shall be covered during inclement weather, in order to keep them dry, to a distance of at least 10 x 10 metres.\par}\par

\end{adjustwidth}


\vspace{\baselineskip}
{\fontsize{11pt}{13.2pt}\selectfont \textbf{10.3 \tabto{0.47in} Removal of covers}\par}\par


\vspace{\baselineskip}
\begin{adjustwidth}{0.0in}{0.11in}
{\fontsize{9pt}{10.8pt}\selectfont All covers (including $``$hessian$"$  or $``$scrim$"$  covers used to protect the pitch against the sun) shall be removed not later than 2 ½ hours before the scheduled start of play provided it is not raining at the time, but the pitch will be covered again if rain falls prior to the commencement of play.\par}\par

\end{adjustwidth}


\vspace{\baselineskip}
{\fontsize{16pt}{19.2pt}\selectfont \textbf{11 INTERVALS}\par}\par


\vspace{\baselineskip}
{\fontsize{11pt}{13.2pt}\selectfont \textbf{11.1 \tabto{0.47in} An interval}\par}\par


\vspace{\baselineskip}
{\fontsize{9pt}{10.8pt}\selectfont 11.1.1 \tabto{0.49in} {\fontsize{8pt}{9.6pt}\selectfont The following shall be classed as intervals.\par}\par}\par


\vspace{\baselineskip}
\begin{itemize}
	\item {\fontsize{9pt}{10.8pt}\selectfont Intervals between innings.\par}\par


\vspace{\baselineskip}
	\item {\fontsize{9pt}{10.8pt}\selectfont Any other agreed interval.\par}
\end{itemize}\par


\vspace{\baselineskip}
{\fontsize{9pt}{10.8pt}\selectfont 11.1.2 \tabto{0.49in} {\fontsize{8pt}{9.6pt}\selectfont Only these intervals shall be considered as scheduled breaks for the purposes of clause \par}\par}\par


\vspace{\baselineskip}

\vspace{\baselineskip}

\vspace{\baselineskip}

\vspace{\baselineskip}
\begin{Center}
{\fontsize{8pt}{9.6pt}\selectfont 13\par}
\end{Center}\par


\vspace{\baselineskip}
{\fontsize{11pt}{13.2pt}\selectfont \textbf{11.2 \tabto{0.47in} Duration of interval}\par}\par


\vspace{\baselineskip}
\begin{adjustwidth}{0.5in}{0.1in}
{\fontsize{9pt}{10.8pt}\selectfont 11.2.1 \tabto{0.49in} There shall be a 20 minute interval between innings, taken from the call of Time before the interval until the call of Play on resumption after the interval.\par}\par

\end{adjustwidth}


\vspace{\baselineskip}
{\fontsize{11pt}{13.2pt}\selectfont \textbf{11.3 \tabto{0.47in} Allowance for interval between innings}\par}\par


\vspace{\baselineskip}
\begin{adjustwidth}{0.49in}{0.0in}
{\fontsize{9pt}{10.8pt}\selectfont Law 11.3 of the Laws of Cricket shall not apply.\par}\par

\end{adjustwidth}


\vspace{\baselineskip}
{\fontsize{11pt}{13.2pt}\selectfont \textbf{11.4 \tabto{0.47in} Changing agreed times of intervals}\par}\par


\vspace{\baselineskip}
\begin{adjustwidth}{0.5in}{0.04in}
{\fontsize{9pt}{10.8pt}\selectfont 11.4.1 \tabto{0.49in} If the innings of the team batting first is completed prior to the scheduled time for the interval, the interval shall take place immediately and the innings of the team batting second will commence correspondingly earlier. In circumstances where the side bowling first has not completed the allotted number of overs by the scheduled or re-scheduled cessation time for the first innings, the umpires shall reduce the length of the interval by the amount of time that the first innings over-ran. The minimum time for the interval will be 10 minutes.\par}\par

\end{adjustwidth}


\vspace{\baselineskip}
\begin{adjustwidth}{0.5in}{0.15in}
\begin{justify}
{\fontsize{9pt}{10.8pt}\selectfont 11.4.2 \tabto{0.49in} However, following a lengthy delay or interruption prior to the completion of the innings of the team batting first, the Match Referee may, at his discretion, reduce the interval between innings from 20 minutes to not less than 10 minutes.\par}
\end{justify}\par

\end{adjustwidth}


\vspace{\baselineskip}
\begin{adjustwidth}{0.5in}{0.01in}
{\fontsize{9pt}{10.8pt}\selectfont 11.4.3 \tabto{0.49in} Such discretion should only be exercised after determining the adjusted overs per side based on a 20 minute interval. If having exercised this discretion, the rescheduled finishing time for the match is earlier than the latest possible finishing time, then these minutes should be deducted from the length of any interruption during the second innings before determining the overs remaining.\par}\par

\end{adjustwidth}


\vspace{\baselineskip}
{\fontsize{11pt}{13.2pt}\selectfont \textbf{11.5 \tabto{0.47in} Intervals for drinks}\par}\par


\vspace{\baselineskip}
{\fontsize{9pt}{10.8pt}\selectfont 11.5.1 \tabto{0.49in} {\fontsize{8pt}{9.6pt}\selectfont No drinks intervals shall be permitted.\par}\par}\par


\vspace{\baselineskip}
\begin{adjustwidth}{0.5in}{0.06in}
\begin{justify}
{\fontsize{9pt}{10.8pt}\selectfont 11.5.2 \tabto{0.49in} An individual player may be given a drink either on the boundary edge or at the fall of a wicket, on the field, provided that no playing time is wasted. No other drinks shall be taken onto the field without the permission of the umpires. Any player taking drinks onto the field shall be dressed in proper cricket attire (subject to the wearing of bibs – refer to the note in clause .\par}
\end{justify}\par

\end{adjustwidth}


\vspace{\baselineskip}
{\fontsize{11pt}{13.2pt}\selectfont \textbf{11.6 \tabto{0.47in} Scorers to be informed}\par}\par


\vspace{\baselineskip}
\begin{adjustwidth}{0.0in}{0.01in}
{\fontsize{9pt}{10.8pt}\selectfont The umpires shall ensure that the scorers are informed of all agreements about hours of play and intervals and of any changes made thereto as permitted under this clause.\par}\par

\end{adjustwidth}


\vspace{\baselineskip}
{\fontsize{16pt}{19.2pt}\selectfont \textbf{12 START OF PLAY; CESSATION OF PLAY}\par}\par


\vspace{\baselineskip}
{\fontsize{11pt}{13.2pt}\selectfont \textbf{12.1 \tabto{0.47in} Call of Play}\par}\par


\vspace{\baselineskip}
\begin{adjustwidth}{0.0in}{0.35in}
{\fontsize{9pt}{10.8pt}\selectfont The bowler’s end umpire shall call Play before the first ball of the match and on the resumption of play after any interval or interruption.\par}\par

\end{adjustwidth}


\vspace{\baselineskip}
{\fontsize{11pt}{13.2pt}\selectfont \textbf{12.2 \tabto{0.47in} Call of Time}\par}\par


\vspace{\baselineskip}
\begin{adjustwidth}{0.0in}{0.22in}
{\fontsize{9pt}{10.8pt}\selectfont The bowler’s end umpire shall call Time, when the ball is dead, at the end of any session of play or as required by these Playing Conditions. See also clause (Call of Over or Time).\par}\par

\end{adjustwidth}


\vspace{\baselineskip}
{\fontsize{11pt}{13.2pt}\selectfont \textbf{12.3 \tabto{0.47in} Removal of bails}\par}\par


\vspace{\baselineskip}
{\fontsize{9pt}{10.8pt}\selectfont After the call of Time, the bails shall be removed from both wickets.\par}\par


\vspace{\baselineskip}
{\fontsize{11pt}{13.2pt}\selectfont \textbf{12.4 \tabto{0.47in} Starting a new over}\par}\par


\vspace{\baselineskip}
\begin{adjustwidth}{0.0in}{0.15in}
{\fontsize{9pt}{10.8pt}\selectfont Another over shall always be started at any time during the match, unless an interval is to be taken in the circumstances set out in clause if the umpire, walking at normal pace, has arrived at the position behind the stumps at the bowler’s end before the time agreed for the next interval has been reached.\par}\par

\end{adjustwidth}


\vspace{\baselineskip}

\vspace{\baselineskip}

\vspace{\baselineskip}

\vspace{\baselineskip}
\begin{Center}
{\fontsize{8pt}{9.6pt}\selectfont 14\par}
\end{Center}\par


\vspace{\baselineskip}
{\fontsize{11pt}{13.2pt}\selectfont \textbf{12.5 \tabto{0.47in} Completion of an over}\par}\par


\vspace{\baselineskip}
{\fontsize{9pt}{10.8pt}\selectfont Other than at the end of the match,\par}\par


\vspace{\baselineskip}
\begin{adjustwidth}{0.5in}{0.04in}
{\fontsize{9pt}{10.8pt}\selectfont 12.5.1 \tabto{0.49in} if the agreed time for an interval is reached during an over, the over shall be completed before the interval is taken, except as provided for in clause \par}\par

\end{adjustwidth}


\vspace{\baselineskip}
\begin{adjustwidth}{0.5in}{0.24in}
{\fontsize{9pt}{10.8pt}\selectfont 12.5.2 \tabto{0.49in} when less than 3 minutes remains before the time agreed for the next interval, the interval shall be taken immediately if\par}\par

\end{adjustwidth}


\vspace{\baselineskip}
\begin{adjustwidth}{0.5in}{0.0in}
{\fontsize{9pt}{10.8pt}\selectfont either a batsman is dismissed or retires, or\par}\par

\end{adjustwidth}


\vspace{\baselineskip}
\begin{adjustwidth}{0.5in}{0.0in}
{\fontsize{9pt}{10.8pt}\selectfont the players have occasion to leave the field,\par}\par

\end{adjustwidth}


\vspace{\baselineskip}
\begin{adjustwidth}{0.5in}{0.28in}
{\fontsize{9pt}{10.8pt}\selectfont whether this occurs during an over or at the end of an over. Except at the end of an innings, if an over is thus interrupted it shall be completed on the resumption of play.\par}\par

\end{adjustwidth}


\vspace{\baselineskip}
{\fontsize{11pt}{13.2pt}\selectfont \textbf{12.6 \tabto{0.47in} Conclusion of match}\par}\par


\vspace{\baselineskip}
{\fontsize{9pt}{10.8pt}\selectfont 12.6.1 \tabto{0.49in} {\fontsize{8pt}{9.6pt}\selectfont The match is concluded\par}\par}\par


\vspace{\baselineskip}
\begin{adjustwidth}{0.49in}{0.0in}
{\fontsize{9pt}{10.8pt}\selectfont 12.6.1.1 \tabto{1.17in} {\fontsize{8pt}{9.6pt}\selectfont as soon as a result as defined in clauses to 16.5 (The result) is reached.\par}\par}\par

\end{adjustwidth}


\vspace{\baselineskip}
\begin{adjustwidth}{0.49in}{0.0in}
{\fontsize{9pt}{10.8pt}\selectfont 12.6.1.2 \tabto{1.17in} {\fontsize{8pt}{9.6pt}\selectfont as soon as the prescribed number of overs have been completed\par}\par}\par

\end{adjustwidth}


\vspace{\baselineskip}
\begin{adjustwidth}{0.5in}{0.11in}
{\fontsize{9pt}{10.8pt}\selectfont 12.6.2 \tabto{0.49in} The match is concluded if, without a conclusion having been reached under the players leave the field for adverse conditions of ground, weather or light, or in exceptional circumstances, and no further play is possible.\par}\par

\end{adjustwidth}


\vspace{\baselineskip}
{\fontsize{11pt}{13.2pt}\selectfont \textbf{12.7 \tabto{0.47in} Hours of Play; Minimum Overs Requirement}\par}\par


\vspace{\baselineskip}
\begin{adjustwidth}{0.5in}{0.53in}
{\fontsize{9pt}{10.8pt}\selectfont 12.7.1 \tabto{0.49in} To be determined by the Home Board subject to there being 2 sessions of 1 hour 25 minutes each, separated by a 20 minute interval between innings.\par}\par

\end{adjustwidth}


\vspace{\baselineskip}
{\fontsize{11pt}{13.2pt}\selectfont \textbf{12.8 \tabto{0.47in} Minimum Over Rates}\par}\par


\vspace{\baselineskip}
{\fontsize{9pt}{10.8pt}\selectfont 12.8.1 \tabto{0.49in} The minimum over rate to be achieved in T20I Matches shall be 14.11 overs per hour.\par}\par


\vspace{\baselineskip}
{\fontsize{9pt}{10.8pt}\selectfont 12.8.2 \tabto{0.49in} {\fontsize{8pt}{9.6pt}\selectfont The actual over rate shall be calculated at the end of each innings by the umpires.\par}\par}\par


\vspace{\baselineskip}
{\fontsize{9pt}{10.8pt}\selectfont 12.8.3 \tabto{0.49in} In calculating the actual over rate for the match, allowances shall be given as follows:\par}\par


\vspace{\baselineskip}
\begin{adjustwidth}{1.18in}{0.15in}
{\fontsize{9pt}{10.8pt}\selectfont 12.8.3.1 \tabto{1.17in} The time lost as a result of treatment given to a player by an authorised medical personnel on the field of play;\par}\par

\end{adjustwidth}


\vspace{\baselineskip}
\begin{adjustwidth}{1.18in}{0.31in}
{\fontsize{9pt}{10.8pt}\selectfont 12.8.3.2 \tabto{1.17in} The time lost as a result of a player being required to leave the field as a result of a serious injury;\par}\par

\end{adjustwidth}


\vspace{\baselineskip}
\begin{adjustwidth}{0.49in}{0.0in}
{\fontsize{9pt}{10.8pt}\selectfont 12.8.3.3 \tabto{1.17in} {\fontsize{8pt}{9.6pt}\selectfont The time taken for all third umpire referrals and consultations and any umpire or player reviews;\par}\par}\par

\end{adjustwidth}


\vspace{\baselineskip}
\begin{adjustwidth}{0.49in}{0.0in}
{\fontsize{9pt}{10.8pt}\selectfont 12.8.3.4 \tabto{1.17in} The time lost as a result of time wasting by the batting side; and\par}\par

\end{adjustwidth}


\vspace{\baselineskip}
\begin{adjustwidth}{0.49in}{0.0in}
{\fontsize{9pt}{10.8pt}\selectfont 12.8.3.5 \tabto{1.17in} The time lost due to all other circumstances that are beyond the control of the fielding side.\par}\par

\end{adjustwidth}


\vspace{\baselineskip}
\begin{adjustwidth}{0.5in}{0.24in}
{\fontsize{9pt}{10.8pt}\selectfont 12.8.4 \tabto{0.49in} In the event of any time allowances being granted to the fielding team under clause above (time wasting by batting team), then such time shall be deducted from the allowances granted to such batting team in the determination of its over rate.\par}\par

\end{adjustwidth}


\vspace{\baselineskip}
{\fontsize{9pt}{10.8pt}\selectfont 12.8.5 \tabto{0.49in} {\fontsize{8pt}{9.6pt}\selectfont In addition to the allowances as provided for above,\par}\par}\par


\vspace{\baselineskip}
\begin{adjustwidth}{1.18in}{0.24in}
{\fontsize{9pt}{10.8pt}\selectfont 12.8.5.1 \tabto{1.17in} in the case of an innings that has been reduced due to any delay or interruption in play, an additional allowance of 1 minute for every full 3 overs by which the innings is reduced will be granted.\par}\par

\end{adjustwidth}


\vspace{\baselineskip}
\begin{adjustwidth}{1.18in}{0.22in}
{\fontsize{9pt}{10.8pt}\selectfont 12.8.5.2 \tabto{1.17in} an additional allowance of 1 minute will be given for each of the 6th, 7th, 8th and 9th wickets taken during an innings.\par}\par

\end{adjustwidth}


\vspace{\baselineskip}

\vspace{\baselineskip}

\vspace{\baselineskip}

\vspace{\baselineskip}
\begin{Center}
{\fontsize{8pt}{9.6pt}\selectfont 15\par}
\end{Center}\par


\vspace{\baselineskip}

\vspace{\baselineskip}
\begin{adjustwidth}{0.5in}{0.03in}
{\fontsize{9pt}{10.8pt}\selectfont 12.8.6 \tabto{0.49in} If a batting team is bowled out within the time determined for that innings pursuant to these playing conditions (taking into account all of the time allowances set out above), the fielding side shall be deemed to have complied with the required minimum over rate.\par}\par

\end{adjustwidth}


\vspace{\baselineskip}
\begin{adjustwidth}{0.5in}{0.1in}
{\fontsize{9pt}{10.8pt}\selectfont 12.8.7 \tabto{0.49in} The current over rate of the fielding team (+/- overs compared to the minimum rate required), to be advised by the 3rd umpire every 30 minutes as a minimum, shall be displayed on a scoreboard or replay screen.\par}\par

\end{adjustwidth}


\vspace{\baselineskip}
{\fontsize{16pt}{19.2pt}\selectfont \textbf{13 INNINGS}\par}\par


\vspace{\baselineskip}
{\fontsize{11pt}{13.2pt}\selectfont \textbf{13.1 \tabto{0.47in} Number of innings}\par}\par


\vspace{\baselineskip}
{\fontsize{9pt}{10.8pt}\selectfont 13.1.1 \tabto{0.49in} A match shall be one innings for each side.\par}\par


\vspace{\baselineskip}
{\fontsize{11pt}{13.2pt}\selectfont \textbf{13.2 \tabto{0.47in} Alternate innings}\par}\par


\vspace{\baselineskip}
{\fontsize{9pt}{10.8pt}\selectfont Each side shall take their innings alternately.\par}\par


\vspace{\baselineskip}
{\fontsize{11pt}{13.2pt}\selectfont \textbf{13.3 \tabto{0.47in} Completed innings}\par}\par


\vspace{\baselineskip}
\begin{adjustwidth}{0.0in}{2.14in}
{\fontsize{9pt}{10.8pt}\selectfont A side’s innings is to be considered as completed if any of the following applies 13.3.1 the side is all out.\par}\par

\end{adjustwidth}


\vspace{\baselineskip}
\begin{adjustwidth}{0.5in}{0.5in}
{\fontsize{9pt}{10.8pt}\selectfont 13.3.2 \tabto{0.49in} at the fall of a wicket or the retirement of a batsman, further balls remain to be bowled but no further batsman is available to come in.\par}\par

\end{adjustwidth}


\vspace{\baselineskip}
{\fontsize{9pt}{10.8pt}\selectfont 13.3.3 \tabto{0.49in} {\fontsize{8pt}{9.6pt}\selectfont the prescribed number of overs have been bowled to the batting side.\par}\par}\par


\vspace{\baselineskip}
{\fontsize{11pt}{13.2pt}\selectfont \textbf{13.4 \tabto{0.47in} The toss}\par}\par


\vspace{\baselineskip}
\begin{adjustwidth}{0.0in}{0.11in}
{\fontsize{9pt}{10.8pt}\selectfont The captains shall toss a coin for the choice of innings, on the field of play and under the supervision of the ICC Match Referee, not earlier than 30 minutes, nor later than 15 minutes before the scheduled or any rescheduled time for the start of play. Note, however, the provisions of clause (Captain).\par}\par

\end{adjustwidth}


\vspace{\baselineskip}
{\fontsize{11pt}{13.2pt}\selectfont \textbf{13.5 \tabto{0.47in} Decision to be notified}\par}\par


\vspace{\baselineskip}
\begin{adjustwidth}{0.0in}{0.14in}
{\fontsize{8pt}{9.6pt}\selectfont As soon as the toss is completed, the captain of the side winning the toss shall decide whether to bat or to field and shall notify the opposing captain and the umpires of this decision. Once notified, the decision cannot be changed.\par}\par

\end{adjustwidth}


\vspace{\baselineskip}
{\fontsize{11pt}{13.2pt}\selectfont \textbf{13.6 \tabto{0.47in} Duration of Match}\par}\par


\vspace{\baselineskip}
\begin{adjustwidth}{0.5in}{0.24in}
{\fontsize{9pt}{10.8pt}\selectfont 13.6.1 \tabto{0.49in} All matches will consist of one innings per side, each innings being limited to a maximum of 20 overs. All matches shall be of one day’s scheduled duration.\par}\par

\end{adjustwidth}


\vspace{\baselineskip}
{\fontsize{11pt}{13.2pt}\selectfont \textbf{13.7 \tabto{0.47in} Length of Innings}\par}\par


\vspace{\baselineskip}
{\fontsize{9pt}{10.8pt}\selectfont 13.7.1 \tabto{0.49in} Uninterrupted Matches.\par}\par


\vspace{\baselineskip}
\begin{adjustwidth}{0.49in}{0.0in}
{\fontsize{9pt}{10.8pt}\selectfont 13.7.1.1 \tabto{1.17in} {\fontsize{8pt}{9.6pt}\selectfont Each team shall bat for 20 overs unless all out earlier.\par}\par}\par

\end{adjustwidth}


\vspace{\baselineskip}
\begin{adjustwidth}{1.18in}{0.14in}
{\fontsize{9pt}{10.8pt}\selectfont 13.7.1.2 \tabto{1.17in} If the team fielding first fails to bowl the required number of overs by the scheduled time for cessation of the first innings, play shall continue until the required number of overs has been bowled. The interval shall not be extended and the second session shall commence at the scheduled time. The team batting second shall receive its full quota of 20 overs irrespective of the number of overs it bowled in the scheduled time for the cessation of the first innings.\par}\par

\end{adjustwidth}


\vspace{\baselineskip}
\begin{adjustwidth}{1.18in}{0.32in}
{\fontsize{9pt}{10.8pt}\selectfont 13.7.1.3 \tabto{1.17in} If the team batting first is dismissed in less than 20 overs, the team batting second shall be entitled to bat for 20 overs.\par}\par

\end{adjustwidth}


\vspace{\baselineskip}
\begin{adjustwidth}{1.18in}{0.11in}
{\fontsize{9pt}{10.8pt}\selectfont 13.7.1.4 \tabto{1.17in} If the team fielding second fails to bowl 20 overs by the scheduled cessation time, the hours of play shall be extended until the required number of overs has been bowled or a result is achieved.\par}\par

\end{adjustwidth}


\vspace{\baselineskip}
\begin{adjustwidth}{0.49in}{0.0in}
{\fontsize{9pt}{10.8pt}\selectfont 13.7.1.5 \tabto{1.17in} Penalties shall apply for slow over rates (refer to the ICC Code of Conduct).\par}\par

\end{adjustwidth}


\vspace{\baselineskip}

\vspace{\baselineskip}

\vspace{\baselineskip}

\vspace{\baselineskip}
\begin{Center}
{\fontsize{8pt}{9.6pt}\selectfont 16\par}
\end{Center}\par


\vspace{\baselineskip}
{\fontsize{9pt}{10.8pt}\selectfont 13.7.2 \tabto{0.49in} {\fontsize{8pt}{9.6pt}\selectfont Delayed or Interrupted Matches\par}\par}\par


\vspace{\baselineskip}
\begin{adjustwidth}{0.49in}{0.0in}
{\fontsize{9pt}{10.8pt}\selectfont 13.7.2.1 \tabto{1.17in} Delay or Interruption to the Innings of the Team Batting First (see paragraph 1 of Appendix E)\par}\par

\end{adjustwidth}


\vspace{\baselineskip}
\begin{adjustwidth}{1.97in}{0.03in}
\begin{justify}
{\fontsize{9pt}{10.8pt}\selectfont 13.7.2.1.1 \tabto{1.96in} When playing time has been lost the revised number of overs to be bowled in the match shall be based on a rate of 14.11 overs per hour in the total remaining time available for play.\par}
\end{justify}\par

\end{adjustwidth}


\vspace{\baselineskip}
\begin{adjustwidth}{1.97in}{0.08in}
{\fontsize{9pt}{10.8pt}\selectfont 13.7.2.1.2 \tabto{1.96in} The revision of the number of overs should ensure, whenever possible, that both teams have the opportunity of batting for the same number of overs. The team batting second shall not bat for a greater number of overs than the first team unless the latter completed its innings in less than its allocated overs. To constitute a match, a minimum of 5 overs have to be bowled to the side batting second, subject to a result not being achieved earlier.\par}\par

\end{adjustwidth}


\vspace{\baselineskip}
\begin{adjustwidth}{1.97in}{0.06in}
\begin{justify}
{\fontsize{9pt}{10.8pt}\selectfont 13.7.2.1.3 \tabto{1.96in} As soon as the total minutes of playing time remaining is less than the completed overs faced by Team 1 multiplied by 4.25, then the first innings is terminated and the provisions of below take effect.\par}
\end{justify}\par

\end{adjustwidth}


\vspace{\baselineskip}
\begin{adjustwidth}{1.97in}{0.08in}
{\fontsize{9pt}{10.8pt}\selectfont 13.7.2.1.4 \tabto{1.96in} A fixed time will be specified for the commencement of the interval, and also the close of play for the match, by applying a rate of 14.11 overs per hour. When calculating the length of playing time available for the match, or the length of either innings, the timing and duration of all relative delays, extensions in playing hours, interruptions in play, and intervals will be taken into consideration. This calculation must not cause the match to finish earlier than the original or rescheduled time for cessation of play on the final scheduled day for play. If required the original time shall be extended to allow for one extra over for each team.\par}\par

\end{adjustwidth}


\vspace{\baselineskip}
\begin{adjustwidth}{1.97in}{0.1in}
\begin{justify}
{\fontsize{9pt}{10.8pt}\selectfont 13.7.2.1.5 \tabto{1.96in} If the team fielding first fails to bowl the revised number of overs by the specified time, play shall continue until the required number of overs have been bowled or the innings is completed.\par}
\end{justify}\par

\end{adjustwidth}


\vspace{\baselineskip}
\begin{adjustwidth}{1.18in}{0.0in}
{\fontsize{9pt}{10.8pt}\selectfont 13.7.2.1.6 \tabto{1.96in} Penalties shall apply for slow over rates (refer to the ICC Code of Conduct).\par}\par

\end{adjustwidth}


\vspace{\baselineskip}
\begin{adjustwidth}{1.18in}{0.14in}
{\fontsize{9pt}{10.8pt}\selectfont 13.7.2.2 \tabto{1.17in} Delay or Interruption to the innings of the Team Batting Second (see paragraph 2 of Appendix E)\par}\par

\end{adjustwidth}


\vspace{\baselineskip}

\vspace{\baselineskip}

\vspace{\baselineskip}

\vspace{\baselineskip}

\vspace{\baselineskip}

\vspace{\baselineskip}

\vspace{\baselineskip}

\vspace{\baselineskip}

\vspace{\baselineskip}

\vspace{\baselineskip}

\vspace{\baselineskip}

\vspace{\baselineskip}

\vspace{\baselineskip}

\vspace{\baselineskip}

\vspace{\baselineskip}

\vspace{\baselineskip}

\vspace{\baselineskip}

\vspace{\baselineskip}

\vspace{\baselineskip}

\vspace{\baselineskip}

\vspace{\baselineskip}

\vspace{\baselineskip}

\vspace{\baselineskip}

\vspace{\baselineskip}

\vspace{\baselineskip}

\vspace{\baselineskip}

\vspace{\baselineskip}

\vspace{\baselineskip}

\vspace{\baselineskip}
\begin{Center}
{\fontsize{8pt}{9.6pt}\selectfont 17\par}
\end{Center}\par


\vspace{\baselineskip}

\vspace{\baselineskip}
\begin{adjustwidth}{1.97in}{0.03in}
{\fontsize{9pt}{10.8pt}\selectfont 13.7.2.2.1 \tabto{1.96in} {\fontsize{8pt}{9.6pt}\selectfont When playing time has been lost and, as a result, it is not possible for the team batting second to have the opportunity of receiving its allocated, or revised allocation of overs in the playing time available, the number of overs shall be reduced at a rate of 14.11 overs per hour in respect of the lost playing time. Should the calculations result in a fraction of an over the fraction shall be ignored.\par}\par}\par

\end{adjustwidth}


\vspace{\baselineskip}
\begin{adjustwidth}{1.97in}{0.06in}
{\fontsize{9pt}{10.8pt}\selectfont 13.7.2.2.2 \tabto{1.96in} In addition, should the innings of the team batting first have been completed prior to the scheduled, or re-scheduled time for the commencement of the interval, then any calculation relating to the revision of overs shall not be effective until an amount of time equivalent to that by which the second innings started early has elapsed.\par}\par

\end{adjustwidth}


\vspace{\baselineskip}
\begin{adjustwidth}{0.0in}{-0.82in}
\begin{Center}
{\fontsize{9pt}{10.8pt}\selectfont 13.7.2.2.3 \tabto{0.22in} {\fontsize{8pt}{9.6pt}\selectfont To constitute a match, a minimum of 5 overs have to be bowled to the team\par}\par}
\end{Center}\par

\end{adjustwidth}


\vspace{\baselineskip}
\begin{adjustwidth}{0.0in}{-0.76in}
\begin{Center}
{\fontsize{9pt}{10.8pt}\selectfont batting second subject to a result not being achieved earlier.\par}
\end{Center}\par

\end{adjustwidth}


\vspace{\baselineskip}
\begin{adjustwidth}{1.97in}{0.07in}
{\fontsize{9pt}{10.8pt}\selectfont 13.7.2.2.4 \tabto{1.96in} The team batting second shall not bat for a greater number of overs than the first team unless the latter completed its innings in less than its allocated overs.\par}\par

\end{adjustwidth}


\vspace{\baselineskip}
\begin{adjustwidth}{1.97in}{0.0in}
{\fontsize{9pt}{10.8pt}\selectfont 13.7.2.2.5 \tabto{1.96in} A fixed time will be specified for the close of play by applying a rate of 14.11 overs per hour. The timing and duration of all relative delays, extensions in playing hours and interruptions in play will be taken into consideration in specifying this time.\par}\par

\end{adjustwidth}


\vspace{\baselineskip}
\begin{adjustwidth}{1.97in}{0.08in}
\begin{justify}
{\fontsize{9pt}{10.8pt}\selectfont 13.7.2.2.6 \tabto{1.96in} If the team fielding second fails to bowl the revised overs by the scheduled or re-scheduled close of play, the hours of play shall be extended until the overs have been bowled or a result achieved.\par}
\end{justify}\par

\end{adjustwidth}


\vspace{\baselineskip}
\begin{adjustwidth}{1.18in}{0.0in}
{\fontsize{9pt}{10.8pt}\selectfont 13.7.2.2.7 \tabto{1.96in} Penalties shall apply for slow over rates (refer to the ICC Code of Conduct).\par}\par

\end{adjustwidth}


\vspace{\baselineskip}
{\fontsize{11pt}{13.2pt}\selectfont \textbf{13.8 \tabto{0.47in} Extra Time}\par}\par


\vspace{\baselineskip}
\begin{adjustwidth}{0.0in}{0.04in}
{\fontsize{9pt}{10.8pt}\selectfont The participating countries may agree to provide for extra time where the start of play is delayed or play is suspended. For clarity, the changeover period (maximum 10 mins) for a Super Over after the main match is not to be taken into account when applying any permitted extra time available.\par}\par

\end{adjustwidth}


\vspace{\baselineskip}
{\fontsize{11pt}{13.2pt}\selectfont \textbf{13.9 \tabto{0.47in} Number of Overs per Bowler}\par}\par


\vspace{\baselineskip}
{\fontsize{9pt}{10.8pt}\selectfont 13.9.1 \tabto{0.49in} {\fontsize{8pt}{9.6pt}\selectfont No bowler shall bowl more than 4 overs in an innings.\par}\par}\par


\vspace{\baselineskip}
\begin{adjustwidth}{0.5in}{0.4in}
{\fontsize{9pt}{10.8pt}\selectfont 13.9.2 \tabto{0.49in} In a delayed or interrupted match where the overs are reduced for both teams or for the team bowling second;\par}\par

\end{adjustwidth}


\vspace{\baselineskip}
\begin{adjustwidth}{1.18in}{0.01in}
{\fontsize{9pt}{10.8pt}\selectfont 13.9.2.1 \tabto{1.17in} for innings of rescheduled length of at least 10 overs, no bowler may bowl more than one-fifth of the total overs allowed. Where the total overs is not divisible by 5, one additional over shall be allowed to the maximum number per bowler necessary to make up the balance.\par}\par

\end{adjustwidth}


\vspace{\baselineskip}
\begin{adjustwidth}{1.18in}{0.07in}
{\fontsize{9pt}{10.8pt}\selectfont 13.9.2.2 \tabto{1.17in} for innings of rescheduled length of between 5 and 9 overs, no bowler may bowl more than two overs.\par}\par

\end{adjustwidth}


\vspace{\baselineskip}
\begin{adjustwidth}{0.5in}{0.1in}
{\fontsize{9pt}{10.8pt}\selectfont 13.9.3 \tabto{0.49in} In the event of a bowler breaking down and being unable to complete an over, the remaining balls will be allowed by another bowler. Such part of an over will count as a full over only in so far as each bowler’s limit is concerned.\par}\par

\end{adjustwidth}


\vspace{\baselineskip}
\begin{adjustwidth}{0.5in}{0.38in}
{\fontsize{9pt}{10.8pt}\selectfont 13.9.4 \tabto{0.49in} The scoreboard shall show the total number of overs bowled and the number of overs bowled by each bowler.\par}\par

\end{adjustwidth}


\vspace{\baselineskip}
{\fontsize{16pt}{19.2pt}\selectfont \textbf{14 THE FOLLOW-ON}\par}\par


\vspace{\baselineskip}
{\fontsize{9pt}{10.8pt}\selectfont Shall not apply.\par}\par


\vspace{\baselineskip}

\vspace{\baselineskip}

\vspace{\baselineskip}

\vspace{\baselineskip}

\vspace{\baselineskip}

\vspace{\baselineskip}

\vspace{\baselineskip}
\begin{Center}
{\fontsize{8pt}{9.6pt}\selectfont 18\par}
\end{Center}\par


\vspace{\baselineskip}
{\fontsize{16pt}{19.2pt}\selectfont \textbf{15 DECLARATION AND FORFEITURE}\par}\par


\vspace{\baselineskip}
{\fontsize{9pt}{10.8pt}\selectfont Shall not apply.\par}\par


\vspace{\baselineskip}
{\fontsize{16pt}{19.2pt}\selectfont \textbf{16 THE RESULT}\par}\par


\vspace{\baselineskip}
{\fontsize{11pt}{13.2pt}\selectfont \textbf{16.1 \tabto{0.47in} A Win}\par}\par


\vspace{\baselineskip}
\begin{adjustwidth}{0.5in}{0.1in}
\begin{justify}
{\fontsize{9pt}{10.8pt}\selectfont 16.1.1 \tabto{0.49in} The side which has scored in its one innings a total of runs in excess of that scored by the opposing side in its one completed innings shall win the match. See clause (Completed innings). Note also clause  (Winning hit or extras).\par}
\end{justify}\par

\end{adjustwidth}


\vspace{\baselineskip}
\begin{adjustwidth}{0.5in}{0.04in}
{\fontsize{9pt}{10.8pt}\selectfont 16.1.2 \tabto{0.49in} Save for circumstances where a match is awarded to a team as a consequence of the opposing team’s refusal to play (Clause a result can be achieved only if both teams have had the opportunity of batting for at least 5 overs, unless one team has been all out in less than 5 overs or unless the team batting second scores enough runs to win in less than 5 overs.\par}\par

\end{adjustwidth}


\vspace{\baselineskip}
\begin{adjustwidth}{0.5in}{0.18in}
{\fontsize{9pt}{10.8pt}\selectfont 16.1.3 \tabto{0.49in} Save for circumstances where a match is awarded to a team as a consequence of the opposing team’s refusal to play (Clause all matches in which both teams have not had an opportunity of batting for a minimum of 5 overs, shall be declared a No Result.\par}\par

\end{adjustwidth}


\vspace{\baselineskip}
{\fontsize{11pt}{13.2pt}\selectfont \textbf{16.2 \tabto{0.47in} ICC Match Referee awarding a match}\par}\par


\vspace{\baselineskip}
{\fontsize{9pt}{10.8pt}\selectfont 16.2.1 \tabto{0.49in} {\fontsize{8pt}{9.6pt}\selectfont A match shall be lost by a side which either\par}\par}\par


\vspace{\baselineskip}
\begin{adjustwidth}{0.49in}{0.0in}
{\fontsize{9pt}{10.8pt}\selectfont 16.2.1.1 \tabto{1.17in} concedes defeat or\par}\par

\end{adjustwidth}


\vspace{\baselineskip}
\begin{adjustwidth}{1.18in}{0.06in}
{\fontsize{9pt}{10.8pt}\selectfont 16.2.1.2 \tabto{1.17in} in the opinion of the ICC Match Referee refuses to play and the ICC Match Referee shall award the match to the other side.\par}\par

\end{adjustwidth}


\vspace{\baselineskip}
\begin{adjustwidth}{0.5in}{0.03in}
{\fontsize{9pt}{10.8pt}\selectfont 16.2.2 \tabto{0.49in} If an umpire considers that an action by any player or players might constitute a refusal by either side to play then the umpires together shall inform the ICC Match Referee of this fact. The ICC Match Referee shall together with the umpires ascertain the cause of the action. If the ICC Match Referee, after due consultation with the umpires, then decides that this action does constitute a refusal to play by one side, he/she shall so inform the captain of that side. If the captain persists in the action the ICC Match Referee shall award the match in accordance with clause above.\par}\par

\end{adjustwidth}


\vspace{\baselineskip}
\begin{adjustwidth}{0.5in}{0.03in}
{\fontsize{9pt}{10.8pt}\selectfont 16.2.3 \tabto{0.49in} If action as in clause above takes place after play has started and does not constitute a refusal to play the delay or interruption in play shall be dealt with in the same manner as provided for in clauses  (Delayed and Interrupted Matches) and (Changing agreed times for intervals) above.\par}\par

\end{adjustwidth}


\vspace{\baselineskip}
\begin{adjustwidth}{0.5in}{0.31in}
{\fontsize{9pt}{10.8pt}\selectfont \textbf{Note: }In addition to the consequences of any refusal to play prescribed under this clause, any such\textbf{ }refusal, whether temporary or final, may result in disciplinary action being taken against the captain and team responsible under the ICC Code of Conduct.\par}\par

\end{adjustwidth}


\vspace{\baselineskip}
{\fontsize{11pt}{13.2pt}\selectfont \textbf{16.3 \tabto{0.47in} All other matches – A Tie or No Result}\par}\par


\vspace{\baselineskip}
{\fontsize{9pt}{10.8pt}\selectfont 16.3.1 \tabto{0.49in} {\fontsize{8pt}{9.6pt}\selectfont A Tie\par}\par}\par


\vspace{\baselineskip}
\begin{adjustwidth}{0.5in}{0.0in}
{\fontsize{9pt}{10.8pt}\selectfont The result of a match shall be a Tie when all innings have been completed and the scores are equal.\par}\par

\end{adjustwidth}


\vspace{\baselineskip}
\begin{adjustwidth}{0.5in}{0.11in}
\begin{justify}
{\fontsize{9pt}{10.8pt}\selectfont If the scores are equal, the result shall be a tie and no account shall be taken of the number of wickets that have fallen. In the event of a tied match the teams shall compete in a Super Over to determine the winner. Refer to Appendix F.\par}
\end{justify}\par

\end{adjustwidth}


\vspace{\baselineskip}
{\fontsize{9pt}{10.8pt}\selectfont 16.3.2 \tabto{0.49in} No Result\par}\par


\vspace{\baselineskip}
\begin{adjustwidth}{0.5in}{0.0in}
{\fontsize{9pt}{10.8pt}\selectfont See 16.1.3 above.\par}\par

\end{adjustwidth}


\vspace{\baselineskip}
{\fontsize{11pt}{13.2pt}\selectfont \textbf{16.4 \tabto{0.47in} Prematurely Terminated Matches - Calculation of the Target Score}\par}\par


\vspace{\baselineskip}
{\fontsize{9pt}{10.8pt}\selectfont 16.4.1 \tabto{0.49in} {\fontsize{8pt}{9.6pt}\selectfont Interrupted Matches - Calculation of the Target Score\par}\par}\par


\vspace{\baselineskip}

\vspace{\baselineskip}

\vspace{\baselineskip}

\vspace{\baselineskip}
\begin{Center}
{\fontsize{8pt}{9.6pt}\selectfont 19\par}
\end{Center}\par


\vspace{\baselineskip}

\vspace{\baselineskip}
\begin{adjustwidth}{1.18in}{0.0in}
{\fontsize{9pt}{10.8pt}\selectfont 16.4.1.1 \tabto{1.17in} If, due to suspension of play after the start of the match, the number of overs in the innings of either team has to be revised to a lesser number than originally allotted (minimum of 5 overs), then a revised target score (to win) should be set for the number of overs which the team batting second will have the opportunity of facing. This revised target is to be calculated using the current Duckworth/Lewis/Stern method. The target set will always be a whole number and one run less will constitute a Tie. (Refer Duckworth/Lewis/Stern Regulations)\par}\par

\end{adjustwidth}


\vspace{\baselineskip}
{\fontsize{9pt}{10.8pt}\selectfont 16.4.2 \tabto{0.49in} {\fontsize{8pt}{9.6pt}\selectfont Prematurely Terminated Matches\par}\par}\par


\vspace{\baselineskip}
\begin{adjustwidth}{1.18in}{0.06in}
{\fontsize{9pt}{10.8pt}\selectfont 16.4.2.1 \tabto{1.17in} If the innings of the side batting second is suspended (with at least 5 overs bowled) and it is not possible for the match to be resumed, the match will be decided by comparison with the DLS ‘Par Score’ determined at the instant of the suspension by the Duckworth/Lewis/Stern method (refer Duckworth/Lewis/Stern Regulations). If the score is equal to the par score, the match is a Tie. Otherwise the result is a victory, or defeat, by the margin of runs by which the score exceeds, or falls short of, the Par Score.\par}\par

\end{adjustwidth}


\vspace{\baselineskip}
{\fontsize{11pt}{13.2pt}\selectfont \textbf{16.5 \tabto{0.47in} Winning hit or extras}\par}\par


\vspace{\baselineskip}
\begin{adjustwidth}{0.5in}{0.03in}
{\fontsize{9pt}{10.8pt}\selectfont 16.5.1 \tabto{0.49in} As soon as a result is reached as defined in clauses or the match is at an end. Nothing that happens thereafter, except as in clause (Penalty runs), shall be regarded as part of it. Note also clause \par}\par

\end{adjustwidth}


\vspace{\baselineskip}
\begin{adjustwidth}{0.5in}{0.04in}
{\fontsize{9pt}{10.8pt}\selectfont 16.5.2 \tabto{0.49in} The side batting last will have scored enough runs to win only if its total of runs is sufficient without including any runs completed by the batsmen before the completion of a catch, or the obstruction of a catch, from which the striker could be dismissed.\par}\par

\end{adjustwidth}


\vspace{\baselineskip}
\begin{adjustwidth}{0.5in}{0.0in}
{\fontsize{9pt}{10.8pt}\selectfont 16.5.3 \tabto{0.49in} If a boundary is scored before the batsmen have completed sufficient runs to win the match, the whole of the boundary allowance shall be credited to the side’s total and, in the case of a hit by the bat, to the striker’s score.\par}\par

\end{adjustwidth}


\vspace{\baselineskip}
{\fontsize{11pt}{13.2pt}\selectfont \textbf{16.6 \tabto{0.47in} Statement of result}\par}\par


\vspace{\baselineskip}
{\fontsize{9pt}{10.8pt}\selectfont If the side batting last wins the match without losing all its wickets, the result shall be stated as a win by the number of wickets still then to fall.\par}\par


\vspace{\baselineskip}
\begin{adjustwidth}{0.0in}{0.19in}
\begin{justify}
{\fontsize{9pt}{10.8pt}\selectfont If, without having scored a total of runs in excess of the total scored by the opposing side, the side batting last has lost all its wickets, but as the result of an award of 5 Penalty runs its total of runs is then sufficient to win, the result shall be stated as a win to that side by Penalty runs.\par}
\end{justify}\par

\end{adjustwidth}


\vspace{\baselineskip}
{\fontsize{9pt}{10.8pt}\selectfont If the side fielding last wins the match, the result shall be stated as a win by runs.\par}\par


\vspace{\baselineskip}
\begin{adjustwidth}{0.0in}{0.6in}
{\fontsize{9pt}{10.8pt}\selectfont If the match is decided by one side conceding defeat or refusing to play, the result shall be stated as Match Conceded or Match Awarded, as the case may be.\par}\par

\end{adjustwidth}


\vspace{\baselineskip}
{\fontsize{11pt}{13.2pt}\selectfont \textbf{16.7 \tabto{0.47in} Correctness of result}\par}\par


\vspace{\baselineskip}
\begin{adjustwidth}{0.0in}{0.62in}
{\fontsize{9pt}{10.8pt}\selectfont Any decision as to the correctness of the scores shall be the responsibility of the umpires. See clause  (Correctness of scores).\par}\par

\end{adjustwidth}


\vspace{\baselineskip}
{\fontsize{11pt}{13.2pt}\selectfont \textbf{16.8 \tabto{0.47in} Mistakes in scoring}\par}\par


\vspace{\baselineskip}
\begin{adjustwidth}{0.0in}{0.08in}
{\fontsize{9pt}{10.8pt}\selectfont If, after the players and umpires have left the field in the belief that the match has been concluded, the umpires discover that a mistake in scoring has occurred which affects the result then, subject to clause they shall adopt the following procedure.\par}\par

\end{adjustwidth}


\vspace{\baselineskip}
{\fontsize{9pt}{10.8pt}\selectfont 16.8.1 \tabto{0.49in} {\fontsize{8pt}{9.6pt}\selectfont If, when the players leave the field, the side batting last has not completed its innings and,\par}\par}\par


\vspace{\baselineskip}
\begin{adjustwidth}{0.5in}{0.0in}
{\fontsize{9pt}{10.8pt}\selectfont either the number of overs to be bowled in that innings has not been completed, or\par}\par

\end{adjustwidth}


\vspace{\baselineskip}
\begin{adjustwidth}{0.5in}{0.0in}
{\fontsize{9pt}{10.8pt}\selectfont the end of the innings has not been reached\par}\par

\end{adjustwidth}


\vspace{\baselineskip}
\begin{adjustwidth}{0.5in}{0.0in}
{\fontsize{9pt}{10.8pt}\selectfont then, unless one side concedes defeat, the umpires shall order play to resume.\par}\par

\end{adjustwidth}


\vspace{\baselineskip}

\vspace{\baselineskip}

\vspace{\baselineskip}

\vspace{\baselineskip}

\vspace{\baselineskip}
\begin{Center}
{\fontsize{8pt}{9.6pt}\selectfont 20\par}
\end{Center}\par


\vspace{\baselineskip}

\vspace{\baselineskip}
\begin{adjustwidth}{0.5in}{0.01in}
{\fontsize{9pt}{10.8pt}\selectfont Unless a result is reached sooner, play will then continue, if conditions permit, until the prescribed number of overs has been completed. The number of overs shall be taken as they were at the call of Time for the supposed conclusion of the match. No account shall be taken of the time between that moment and the resumption of play.\par}\par

\end{adjustwidth}


\vspace{\baselineskip}
\begin{adjustwidth}{0.5in}{0.04in}
{\fontsize{9pt}{10.8pt}\selectfont 16.8.2 \tabto{0.49in} If, at this call of Time, the overs have been completed and no Playing time remains, or if the side batting last has completed its innings, the umpires shall immediately inform both captains of the necessary corrections to the scores and to the result.\par}\par

\end{adjustwidth}


\vspace{\baselineskip}
{\fontsize{11pt}{13.2pt}\selectfont \textbf{16.9 \tabto{0.47in} Result not to be changed}\par}\par


\vspace{\baselineskip}
\begin{adjustwidth}{0.0in}{0.19in}
{\fontsize{9pt}{10.8pt}\selectfont Once the umpires have agreed with the scorers the correctness of the scores at the conclusion of the match – see clauses (Correctness of scores) and (Correctness of scores) – the result cannot thereafter be changed.\par}\par

\end{adjustwidth}


\vspace{\baselineskip}
{\fontsize{11pt}{13.2pt}\selectfont \textbf{16.10\  Points}\par}\par


\vspace{\baselineskip}
{\fontsize{9pt}{10.8pt}\selectfont A points system shall not apply.\par}\par


\vspace{\baselineskip}
{\fontsize{16pt}{19.2pt}\selectfont \textbf{17 THE OVER}\par}\par


\vspace{\baselineskip}
{\fontsize{11pt}{13.2pt}\selectfont \textbf{17.1 \tabto{0.47in} Number of balls}\par}\par


\vspace{\baselineskip}
{\fontsize{9pt}{10.8pt}\selectfont The ball shall be bowled from each end alternately in overs of 6 balls.\par}\par


\vspace{\baselineskip}
{\fontsize{11pt}{13.2pt}\selectfont \textbf{17.2 \tabto{0.47in} Start of an over}\par}\par


\vspace{\baselineskip}
\begin{adjustwidth}{0.0in}{0.08in}
{\fontsize{9pt}{10.8pt}\selectfont An over has started when the bowler starts his run-up or, if there is no run-up, starts his action for the first delivery of that over.\par}\par

\end{adjustwidth}


\vspace{\baselineskip}
{\fontsize{11pt}{13.2pt}\selectfont \textbf{17.3 \tabto{0.47in} Validity of balls}\par}\par


\vspace{\baselineskip}
\begin{adjustwidth}{0.5in}{0.07in}
{\fontsize{9pt}{10.8pt}\selectfont 17.3.1 \tabto{0.49in} A ball shall not count as one of the 6 balls of the over unless it is delivered, even though, as in clause  (Non-striker leaving his ground early) a batsman may be dismissed or some other incident occurs without the ball having been delivered.\par}\par

\end{adjustwidth}


\vspace{\baselineskip}
{\fontsize{9pt}{10.8pt}\selectfont 17.3.2 \tabto{0.49in} {\fontsize{8pt}{9.6pt}\selectfont A ball delivered by the bowler shall not count as one of the 6 balls of the over\par}\par}\par


\vspace{\baselineskip}
\begin{adjustwidth}{1.18in}{0.07in}
{\fontsize{9pt}{10.8pt}\selectfont 17.3.2.1 \tabto{1.17in} if it is called dead, or is to be considered dead, before the striker has had an opportunity to play it. See clause (Dead ball; ball counting as one of over).\par}\par

\end{adjustwidth}


\vspace{\baselineskip}
\begin{adjustwidth}{1.18in}{0.19in}
{\fontsize{9pt}{10.8pt}\selectfont 17.3.2.2 \tabto{1.17in} if it is called dead in the circumstances of clause Note also the special provisions of clause (Umpire calling and signaling Dead ball).\par}\par

\end{adjustwidth}


\vspace{\baselineskip}
\begin{adjustwidth}{0.49in}{0.0in}
{\fontsize{9pt}{10.8pt}\selectfont 17.3.2.3 \tabto{1.17in} if it is a No ball. See clause (No ball).\par}\par

\end{adjustwidth}


\vspace{\baselineskip}
\begin{adjustwidth}{0.49in}{0.0in}
{\fontsize{9pt}{10.8pt}\selectfont 17.3.2.4 \tabto{1.17in} {\fontsize{8pt}{9.6pt}\selectfont if it is a Wide. See clause (Wide ball).\par}\par}\par

\end{adjustwidth}


\vspace{\baselineskip}
\begin{adjustwidth}{1.18in}{0.01in}
{\fontsize{9pt}{10.8pt}\selectfont 17.3.2.5 \tabto{1.17in} when any of clauses (Player returning without permission), (Fielding the ball),  (Deliberate attempt to distract striker), or (Deliberate distraction, deception or obstruction of batsman) is applied.\par}\par

\end{adjustwidth}


\vspace{\baselineskip}
\begin{adjustwidth}{0.5in}{0.19in}
{\fontsize{9pt}{10.8pt}\selectfont 17.3.3 \tabto{0.49in} Any deliveries other than those listed in clause and shall be known as valid balls. Only valid balls shall count towards the 6 balls of the over.\par}\par

\end{adjustwidth}


\vspace{\baselineskip}
{\fontsize{11pt}{13.2pt}\selectfont \textbf{17.4 \tabto{0.47in} Call of Over}\par}\par


\vspace{\baselineskip}
\begin{adjustwidth}{0.0in}{0.06in}
{\fontsize{9pt}{10.8pt}\selectfont When 6 valid balls have been bowled and when the ball becomes dead, the umpire shall call Over before leaving the wicket. See also clause (Call of Over or Time).\par}\par

\end{adjustwidth}


\vspace{\baselineskip}
{\fontsize{11pt}{13.2pt}\selectfont \textbf{17.5 \tabto{0.47in} Umpire miscounting}\par}\par


\vspace{\baselineskip}
{\fontsize{9pt}{10.8pt}\selectfont 17.5.1 \tabto{0.49in} {\fontsize{8pt}{9.6pt}\selectfont If the umpire miscounts the number of valid balls, the over as counted by the umpire shall stand.\par}\par}\par


\vspace{\baselineskip}

\vspace{\baselineskip}

\vspace{\baselineskip}

\vspace{\baselineskip}

\vspace{\baselineskip}

\vspace{\baselineskip}
\begin{Center}
{\fontsize{8pt}{9.6pt}\selectfont 21\par}
\end{Center}\par


\vspace{\baselineskip}

\vspace{\baselineskip}
\begin{adjustwidth}{0.5in}{0.21in}
{\fontsize{9pt}{10.8pt}\selectfont 17.5.2 \tabto{0.49in} If, having miscounted, the umpire allows an over to continue after 6 valid balls have been bowled, he/she may subsequently call Over when the ball becomes dead after any delivery, even if that delivery is not a valid ball.\par}\par

\end{adjustwidth}


\vspace{\baselineskip}
\begin{adjustwidth}{0.5in}{0.03in}
{\fontsize{9pt}{10.8pt}\selectfont 17.5.3 \tabto{0.49in} Whenever possible, the third umpire shall liaise with the scorers and if possible inform the on-field umpires if the over has been miscounted.\par}\par

\end{adjustwidth}


\vspace{\baselineskip}
{\fontsize{11pt}{13.2pt}\selectfont \textbf{17.6 \tabto{0.47in} Bowler changing ends}\par}\par


\vspace{\baselineskip}
\begin{adjustwidth}{0.0in}{0.19in}
{\fontsize{9pt}{10.8pt}\selectfont A bowler shall be allowed to change ends as often as desired, provided he does not bowl two overs consecutively, nor bowl parts of each of two consecutive overs, in the same innings.\par}\par

\end{adjustwidth}


\vspace{\baselineskip}
{\fontsize{11pt}{13.2pt}\selectfont \textbf{17.7 \tabto{0.47in} }{\fontsize{10pt}{12.0pt}\selectfont \textbf{Finishing an over}\par}\par}\par


\vspace{\baselineskip}
\begin{adjustwidth}{0.5in}{0.5in}
{\fontsize{9pt}{10.8pt}\selectfont 17.7.1 \tabto{0.49in} Other than at the end of an innings, a bowler shall finish an over in progress unless incapacitated or suspended under these Playing Conditions.\par}\par

\end{adjustwidth}


\vspace{\baselineskip}
\begin{adjustwidth}{0.5in}{0.18in}
{\fontsize{9pt}{10.8pt}\selectfont 17.7.2 \tabto{0.49in} If for any reason, other than the end of an innings, an over is left uncompleted at the start of an interval or interruption, it shall be completed on resumption of play.\par}\par

\end{adjustwidth}


\vspace{\baselineskip}
{\fontsize{11pt}{13.2pt}\selectfont \textbf{17.8 \tabto{0.47in} Bowler incapacitated or suspended during an over}\par}\par


\vspace{\baselineskip}
\begin{adjustwidth}{0.0in}{0.19in}
{\fontsize{9pt}{10.8pt}\selectfont If for any reason a bowler is incapacitated while running up to deliver the first ball of an over, or is incapacitated or suspended during an over, the umpire shall call and signal Dead ball. Another bowler shall complete the over from the same end, provided that he does not bowl two overs consecutively, nor bowl parts of each of two consecutive overs, in that innings.\par}\par

\end{adjustwidth}


\vspace{\baselineskip}
{\fontsize{16pt}{19.2pt}\selectfont \textbf{18 SCORING RUNS}\par}\par


\vspace{\baselineskip}
{\fontsize{11pt}{13.2pt}\selectfont \textbf{18.1 \tabto{0.47in} A run}\par}\par


\vspace{\baselineskip}
{\fontsize{9pt}{10.8pt}\selectfont The score shall be reckoned by runs. A run is scored\par}\par


\vspace{\baselineskip}
\begin{adjustwidth}{0.5in}{0.0in}
{\fontsize{9pt}{10.8pt}\selectfont 18.1.1 \tabto{0.49in} so often as the batsmen, at any time while the ball is in play, have crossed and made good their ground from end to end.\par}\par

\end{adjustwidth}


\vspace{\baselineskip}
{\fontsize{9pt}{10.8pt}\selectfont 18.1.2 \tabto{0.49in} {\fontsize{8pt}{9.6pt}\selectfont when a boundary is scored. See clause (Boundaries).\par}\par}\par


\vspace{\baselineskip}
{\fontsize{9pt}{10.8pt}\selectfont 18.1.3 \tabto{0.49in} {\fontsize{8pt}{9.6pt}\selectfont when Penalty runs are awarded. See clause18.6.\par}\par}\par


\vspace{\baselineskip}
{\fontsize{11pt}{13.2pt}\selectfont \textbf{18.2 \tabto{0.47in} Runs disallowed}\par}\par


\vspace{\baselineskip}
\begin{adjustwidth}{0.0in}{0.19in}
{\fontsize{9pt}{10.8pt}\selectfont Wherever in these Playing Conditions provision is made for the scoring of runs or awarding of penalties, such runs and penalties will be subject to any provisions that may be applicable for the disallowance of runs or for the non-award of penalties.\par}\par

\end{adjustwidth}


\vspace{\baselineskip}
\begin{adjustwidth}{0.0in}{0.17in}
{\fontsize{9pt}{10.8pt}\selectfont When runs are disallowed, the one run penalty for No ball or Wide shall stand and 5 run penalties shall be allowed, except for Penalty runs under clause (Protective helmets belonging to the fielding side).\par}\par

\end{adjustwidth}


\vspace{\baselineskip}
{\fontsize{11pt}{13.2pt}\selectfont \textbf{18.3 \tabto{0.47in} Short runs}\par}\par


\vspace{\baselineskip}
{\fontsize{9pt}{10.8pt}\selectfont 18.3.1 \tabto{0.49in} {\fontsize{8pt}{9.6pt}\selectfont A run is short if a batsman fails to make good his ground in turning for a further run.\par}\par}\par


\vspace{\baselineskip}
\begin{adjustwidth}{0.5in}{0.14in}
{\fontsize{9pt}{10.8pt}\selectfont 18.3.2 \tabto{0.49in} Although a short run shortens the succeeding one, the latter if completed shall not be regarded as short. A striker setting off for the first run from in front of the popping crease may do so also without penalty.\par}\par

\end{adjustwidth}


\vspace{\baselineskip}
{\fontsize{11pt}{13.2pt}\selectfont \textbf{18.4 \tabto{0.47in} Unintentional short runs}\par}\par


\vspace{\baselineskip}
{\fontsize{9pt}{10.8pt}\selectfont Except in the circumstances of clause \par}\par


\vspace{\baselineskip}
\begin{adjustwidth}{0.5in}{0.11in}
{\fontsize{9pt}{10.8pt}\selectfont 18.4.1 \tabto{0.49in} if either batsman runs a short run, the umpire concerned shall, unless a boundary is scored, call and signal Short run as soon as the ball becomes dead and that run shall not be scored.\par}\par

\end{adjustwidth}


\vspace{\baselineskip}

\vspace{\baselineskip}

\vspace{\baselineskip}

\vspace{\baselineskip}

\vspace{\baselineskip}
\begin{Center}
{\fontsize{8pt}{9.6pt}\selectfont 22\par}
\end{Center}\par


\vspace{\baselineskip}

\vspace{\baselineskip}
\begin{adjustwidth}{0.5in}{0.28in}
{\fontsize{9pt}{10.8pt}\selectfont 18.4.2 \tabto{0.49in} if, after either or both batsmen run short, a boundary is scored the umpire concerned shall disregard the short running and shall not call or signal Short run.\par}\par

\end{adjustwidth}


\vspace{\baselineskip}
{\fontsize{9pt}{10.8pt}\selectfont 18.4.3 \tabto{0.49in} {\fontsize{8pt}{9.6pt}\selectfont if both batsmen run short in one and the same run, this shall be regarded as only one short run.\par}\par}\par


\vspace{\baselineskip}
\begin{adjustwidth}{0.5in}{0.11in}
{\fontsize{9pt}{10.8pt}\selectfont 18.4.4 \tabto{0.49in} if more than one run is short then, subject to clauses and all runs called as short shall not be scored.\par}\par

\end{adjustwidth}


\vspace{\baselineskip}
\begin{adjustwidth}{0.5in}{0.01in}
{\fontsize{9pt}{10.8pt}\selectfont 18.4.5 \tabto{0.49in} if there has been more than one short run, the umpire shall inform the scorers as to the number of runs to be recorded.\par}\par

\end{adjustwidth}


\vspace{\baselineskip}
{\fontsize{11pt}{13.2pt}\selectfont \textbf{18.5 \tabto{0.47in} Deliberate short runs}\par}\par


\vspace{\baselineskip}
\begin{adjustwidth}{0.5in}{0.18in}
\begin{justify}
{\fontsize{9pt}{10.8pt}\selectfont 18.5.1 \tabto{0.49in} If either umpire considers that one or both batsmen deliberately ran short at that umpire’s end, the umpire concerned shall, when the ball is dead, call and signal Short run and inform the other umpire of what has occurred and apply clause \par}
\end{justify}\par

\end{adjustwidth}


\vspace{\baselineskip}
{\fontsize{9pt}{10.8pt}\selectfont 18.5.2 \tabto{0.49in} {\fontsize{8pt}{9.6pt}\selectfont The bowler’s end umpire shall\par}\par}\par


\vspace{\baselineskip}
\begin{itemize}
	\item {\fontsize{9pt}{10.8pt}\selectfont disallow all runs to the batting side\par}\par


\vspace{\baselineskip}
	\item {\fontsize{9pt}{10.8pt}\selectfont return any not out batsman to his original end\par}\par


\vspace{\baselineskip}
	\item {\fontsize{9pt}{10.8pt}\selectfont signal No ball or Wide to the scorers, if applicable\par}\par


\vspace{\baselineskip}
	\item {\fontsize{9pt}{10.8pt}\selectfont award 5 Penalty runs to the fielding side\par}\par


\vspace{\baselineskip}
	\item {\fontsize{9pt}{10.8pt}\selectfont award any other 5-run Penalty that is applicable except for Penalty runs under clause (Protective helmets belonging to the fielding side)\par}\par


\vspace{\baselineskip}
	\item {\fontsize{9pt}{10.8pt}\selectfont inform the scorers as to the number of runs to be recorded, and\par}\par


\vspace{\baselineskip}
	\item {\fontsize{9pt}{10.8pt}\selectfont inform the captain of the fielding side and, as soon as practicable, the captain of the batting side of the reason for this action.\par}
\end{itemize}\par


\vspace{\baselineskip}
{\fontsize{11pt}{13.2pt}\selectfont \textbf{18.6 \tabto{0.47in} Runs awarded for penalties}\par}\par


\vspace{\baselineskip}
\begin{adjustwidth}{0.0in}{0.12in}
{\fontsize{9pt}{10.8pt}\selectfont Runs shall be awarded for penalties under clause (Deliberate short runs), (Player returning without permission), (Penalties for contravention), (No ball), (Wide ball), 28.2(Fielding the ball), (Protective helmets belonging to the fielding side) (Unfair play) and (Players’ conduct). Note, however, the restrictions on the award of Penalty runs in clauses (Leg byes not to be awarded), (Protective helmets belonging to the fielding side) and (Hit the ball twice).\par}\par

\end{adjustwidth}


\vspace{\baselineskip}
{\fontsize{11pt}{13.2pt}\selectfont \textbf{18.7 \tabto{0.47in} Runs scored for boundaries}\par}\par


\vspace{\baselineskip}
{\fontsize{9pt}{10.8pt}\selectfont Runs shall be scored for boundary allowances under clause (Boundaries).\par}\par


\vspace{\baselineskip}
{\fontsize{11pt}{13.2pt}\selectfont \textbf{18.8 \tabto{0.47in} Runs scored when a batsman is dismissed}\par}\par


\vspace{\baselineskip}
{\fontsize{9pt}{10.8pt}\selectfont When a batsman is dismissed, any runs for penalties awarded to either side shall stand.\par}\par


\vspace{\baselineskip}
{\fontsize{9pt}{10.8pt}\selectfont No other runs shall be credited to the batting side, except as follows.\par}\par


\vspace{\baselineskip}
\begin{adjustwidth}{0.5in}{0.12in}
{\fontsize{9pt}{10.8pt}\selectfont 18.8.1 \tabto{0.49in} If a batsman is dismissed Obstructing the field, the batting side shall also score any runs completed before the offence.\par}\par

\end{adjustwidth}


\vspace{\baselineskip}
\begin{adjustwidth}{0.5in}{0.0in}
{\fontsize{9pt}{10.8pt}\selectfont If, however, the obstruction prevented a catch being made, no runs other than penalties shall be scored.\par}\par

\end{adjustwidth}


\vspace{\baselineskip}
\begin{adjustwidth}{0.5in}{0.21in}
{\fontsize{9pt}{10.8pt}\selectfont 18.8.2 \tabto{0.49in} If a batsman is dismissed Run out, the batting side shall also score any runs completed before the wicket was put down.\par}\par

\end{adjustwidth}


\vspace{\baselineskip}

\vspace{\baselineskip}

\vspace{\baselineskip}

\vspace{\baselineskip}

\vspace{\baselineskip}

\vspace{\baselineskip}

\vspace{\baselineskip}

\vspace{\baselineskip}
\begin{Center}
{\fontsize{8pt}{9.6pt}\selectfont 23\par}
\end{Center}\par


\vspace{\baselineskip}
{\fontsize{11pt}{13.2pt}\selectfont \textbf{18.9 \tabto{0.47in} Runs scored when the ball becomes dead other than at the fall of a wicket}\par}\par


\vspace{\baselineskip}
\begin{adjustwidth}{0.0in}{0.14in}
{\fontsize{9pt}{10.8pt}\selectfont When the ball becomes dead for any reason other than the fall of a wicket, or is called dead by an umpire, unless there is specific provision otherwise in these Playing Conditions, any runs for penalties awarded to either side shall be scored. Note however the provisions of clauses (Leg byes not to be awarded) and 28.3 (Protective helmets belonging to the fielding side).\par}\par

\end{adjustwidth}


\vspace{\baselineskip}
{\fontsize{9pt}{10.8pt}\selectfont Additionally the batting side shall be credited with all runs completed by the batsmen before the incident or call of Dead ball and the run in progress if the batsmen had already crossed at the instant of the incident or call of Dead ball. Note specifically, however, the provisions of clause (Deliberate distraction, deception or obstruction of batsman).\par}\par


\vspace{\baselineskip}
{\fontsize{11pt}{13.2pt}\selectfont \textbf{18.10\  Crediting of runs scored}\par}\par


\vspace{\baselineskip}
{\fontsize{9pt}{10.8pt}\selectfont Unless stated otherwise in these Playing Conditions,\par}\par


\vspace{\baselineskip}
\begin{adjustwidth}{0.5in}{0.21in}
{\fontsize{9pt}{10.8pt}\selectfont 18.10.1 if the ball is struck by the bat, all runs scored by the batting side shall be credited to the striker, except for the following:\par}\par

\end{adjustwidth}


\vspace{\baselineskip}
\begin{itemize}
	\item {\fontsize{9pt}{10.8pt}\selectfont an award of 5 Penalty runs, which shall be scored as Penalty runs\par}\par


\vspace{\baselineskip}
	\item {\fontsize{9pt}{10.8pt}\selectfont the one run penalty for a No ball, which shall be scored as a No balls extra.\par}
\end{itemize}\par


\vspace{\baselineskip}
\begin{adjustwidth}{0.5in}{0.17in}
{\fontsize{9pt}{10.8pt}\selectfont 18.10.2 if the ball is not struck by the bat, runs shall be scored as Penalty runs, Byes, Leg byes, No ball extras or Wides as the case may be. If Byes or Leg byes accrue from a No ball, only the one run penalty for No ball shall be scored as such, and the remainder as Byes or Leg byes as appropriate.\par}\par

\end{adjustwidth}


\vspace{\baselineskip}
{\fontsize{9pt}{10.8pt}\selectfont 18.10.3\  the bowler shall be debited with:\par}\par


\vspace{\baselineskip}
\begin{itemize}
	\item {\fontsize{9pt}{10.8pt}\selectfont all runs scored by the striker\par}\par


\vspace{\baselineskip}
	\item {\fontsize{9pt}{10.8pt}\selectfont all runs scored as No ball extras\par}\par


\vspace{\baselineskip}
	\item {\fontsize{9pt}{10.8pt}\selectfont all runs scored as Wides.\par}
\end{itemize}\par


\vspace{\baselineskip}
{\fontsize{11pt}{13.2pt}\selectfont \textbf{18.11\  Batsman returning to original end}\par}\par


\vspace{\baselineskip}
\begin{adjustwidth}{0.5in}{0.32in}
{\fontsize{9pt}{10.8pt}\selectfont 18.11.1 When the striker is dismissed in any of the circumstances in clauses to the not out batsman shall return to his original end.\par}\par

\end{adjustwidth}


\vspace{\baselineskip}
\begin{adjustwidth}{0.49in}{0.0in}
{\fontsize{9pt}{10.8pt}\selectfont 18.11.1.1 \tabto{1.17in} {\fontsize{8pt}{9.6pt}\selectfont Bowled.\par}\par}\par

\end{adjustwidth}


\vspace{\baselineskip}
\begin{adjustwidth}{0.49in}{0.0in}
{\fontsize{9pt}{10.8pt}\selectfont 18.11.1.2 \tabto{1.17in} Stumped.\par}\par

\end{adjustwidth}


\vspace{\baselineskip}
\begin{adjustwidth}{0.49in}{0.0in}
{\fontsize{9pt}{10.8pt}\selectfont 18.11.1.3 \tabto{1.17in} {\fontsize{8pt}{9.6pt}\selectfont Hit the ball twice.\par}\par}\par

\end{adjustwidth}


\vspace{\baselineskip}
\begin{adjustwidth}{0.49in}{0.0in}
{\fontsize{9pt}{10.8pt}\selectfont 18.11.1.4 \tabto{1.17in} {\fontsize{8pt}{9.6pt}\selectfont LBW.\par}\par}\par

\end{adjustwidth}


\vspace{\baselineskip}
\begin{adjustwidth}{0.49in}{0.0in}
{\fontsize{9pt}{10.8pt}\selectfont 18.11.1.5 \tabto{1.17in} {\fontsize{8pt}{9.6pt}\selectfont Hit wicket.\par}\par}\par

\end{adjustwidth}


\vspace{\baselineskip}
{\fontsize{9pt}{10.8pt}\selectfont 18.11.2\  The batsmen shall return to their original ends in any of the cases of clauses to \par}\par


\vspace{\baselineskip}
\begin{adjustwidth}{0.49in}{0.0in}
{\fontsize{9pt}{10.8pt}\selectfont 18.11.2.1 \tabto{1.17in} {\fontsize{8pt}{9.6pt}\selectfont A boundary is scored.\par}\par}\par

\end{adjustwidth}


\vspace{\baselineskip}
\begin{adjustwidth}{0.49in}{0.0in}
{\fontsize{9pt}{10.8pt}\selectfont 18.11.2.2 \tabto{1.17in} Runs are disallowed for any reason.\par}\par

\end{adjustwidth}


\vspace{\baselineskip}
\begin{adjustwidth}{1.18in}{0.29in}
{\fontsize{9pt}{10.8pt}\selectfont 18.11.2.3 \tabto{1.17in} A decision by the batsmen at the wicket to do so, under clause (Deliberate distraction, deception or obstruction of batsman).\par}\par

\end{adjustwidth}


\vspace{\baselineskip}
{\fontsize{11pt}{13.2pt}\selectfont \textbf{18.12\  Batsman returning to wicket he has left}\par}\par


\vspace{\baselineskip}
\begin{adjustwidth}{0.5in}{0.07in}
{\fontsize{9pt}{10.8pt}\selectfont 18.12.1 When a batsman is dismissed in any of the ways in clauses to the not out batsman shall return to the wicket he has left but only if the batsmen had not already crossed at the instant of the incident causing the dismissal. If runs are to be disallowed, however, the not out batsman shall return to his original end.\par}\par

\end{adjustwidth}


\vspace{\baselineskip}

\vspace{\baselineskip}

\vspace{\baselineskip}

\vspace{\baselineskip}

\vspace{\baselineskip}
\begin{Center}
{\fontsize{8pt}{9.6pt}\selectfont 24\par}
\end{Center}\par


\vspace{\baselineskip}
\begin{adjustwidth}{0.49in}{0.0in}
{\fontsize{9pt}{10.8pt}\selectfont 18.12.1.1 \tabto{1.17in} {\fontsize{8pt}{9.6pt}\selectfont Caught\par}\par}\par

\end{adjustwidth}


\vspace{\baselineskip}
\begin{adjustwidth}{0.49in}{0.0in}
{\fontsize{9pt}{10.8pt}\selectfont 18.12.1.2 \tabto{1.17in} {\fontsize{8pt}{9.6pt}\selectfont Obstructing the field\par}\par}\par

\end{adjustwidth}


\vspace{\baselineskip}
\begin{adjustwidth}{0.49in}{0.0in}
{\fontsize{9pt}{10.8pt}\selectfont 18.12.1.3 \tabto{1.17in} {\fontsize{8pt}{9.6pt}\selectfont Run out.\par}\par}\par

\end{adjustwidth}


\vspace{\baselineskip}
\begin{adjustwidth}{0.5in}{0.11in}
{\fontsize{9pt}{10.8pt}\selectfont 18.12.2 If, while a run is in progress, the ball becomes dead for any reason other than the dismissal of a batsman, the batsmen shall return to the wickets they had left, but only if they had not already crossed in running when the ball became dead. If, however, any of the circumstances of clauses to apply, the batsmen shall return to their original ends.\par}\par

\end{adjustwidth}


\vspace{\baselineskip}
{\fontsize{16pt}{19.2pt}\selectfont \textbf{19 BOUNDARIES}\par}\par


\vspace{\baselineskip}
{\fontsize{11pt}{13.2pt}\selectfont \textbf{19.1 \tabto{0.47in} }{\fontsize{10pt}{12.0pt}\selectfont \textbf{Determining the boundary of the field of play}\par}\par}\par


\vspace{\baselineskip}
\begin{adjustwidth}{0.5in}{0.24in}
{\fontsize{9pt}{10.8pt}\selectfont 19.1.1 \tabto{0.49in} Before the toss, the umpires shall determine the boundary of the field of play, which shall be fixed for the duration of the match. See clause (Consultation with Home Board).\par}\par

\end{adjustwidth}


\vspace{\baselineskip}
\begin{adjustwidth}{0.5in}{0.12in}
{\fontsize{9pt}{10.8pt}\selectfont 19.1.2 \tabto{0.49in} The boundary shall be determined such that no part of any sight-screen, will, at any stage of the match, be within the field of play.\par}\par

\end{adjustwidth}


\vspace{\baselineskip}
\begin{adjustwidth}{0.5in}{0.12in}
{\fontsize{9pt}{10.8pt}\selectfont 19.1.3 \tabto{0.49in} The aim shall be to maximize the size of the playing area at each venue. With respect to the size of the boundaries, no boundary shall be longer than 90 yards (82.29 meters), and no boundary should be shorter than 65 yards (59.43 metres) from the centre of the pitch to be used.\par}\par

\end{adjustwidth}


\vspace{\baselineskip}
\begin{adjustwidth}{0.5in}{0.07in}
{\fontsize{9pt}{10.8pt}\selectfont 19.1.4 \tabto{0.49in} Any ground which has previously been approved to host international cricket which is unable to conform to the minimum boundary dimension shall be exempt. In such cases the boundary shall be positioned so as to maximize the size of the playing area.\par}\par

\end{adjustwidth}


\vspace{\baselineskip}
{\fontsize{11pt}{13.2pt}\selectfont \textbf{19.2 \tabto{0.47in} Identifying and marking the boundary}\par}\par


\vspace{\baselineskip}
\begin{adjustwidth}{0.5in}{0.03in}
{\fontsize{9pt}{10.8pt}\selectfont 19.2.1 \tabto{0.49in} All boundaries must be designated by a rope, or similar object of a minimum standard as authorised by the ICC from time to time. The rope should be positioned a required minimum distance (3 yards (2.74 metres) minimum) inside the perimeter fencing or advertising signs, or from any solid object located between the rope and the fence/signs. For grounds with a large playing area, the maximum length of boundary should be used before applying the minimum 3 yards (2.74 metres) between the boundary and the fence.\par}\par

\end{adjustwidth}


\vspace{\baselineskip}
\begin{adjustwidth}{0.5in}{0.15in}
{\fontsize{9pt}{10.8pt}\selectfont 19.2.2 \tabto{0.49in} If the boundary is marked by means of an object that is in contact with the ground the boundary will be the edge of the grounded part of the object which is nearest the pitch.\par}\par

\end{adjustwidth}


\vspace{\baselineskip}
\begin{adjustwidth}{0.5in}{0.0in}
{\fontsize{9pt}{10.8pt}\selectfont 19.2.3 \tabto{0.49in} An obstacle within the field of play shall not be regarded as a boundary unless so determined by the umpires before the toss. See clause (Consultation with Home Board).\par}\par

\end{adjustwidth}


\vspace{\baselineskip}
\begin{adjustwidth}{0.5in}{0.03in}
{\fontsize{9pt}{10.8pt}\selectfont 19.2.4 \tabto{0.49in} {\fontsize{8pt}{9.6pt}\selectfont If an unauthorized person enters the playing arena and handles the ball, the umpire at the bowler’s end shall be the sole judge of whether the boundary allowance should be scored or the ball be treated as still in play or called dead ball if a batsman is liable to be out as a result of the unauthorized person handling the ball.\par}\par}\par

\end{adjustwidth}


\vspace{\baselineskip}
{\fontsize{11pt}{13.2pt}\selectfont \textbf{19.3 \tabto{0.47in} }{\fontsize{10pt}{12.0pt}\selectfont \textbf{Restoring the boundary}\par}\par}\par


\vspace{\baselineskip}
{\fontsize{9pt}{10.8pt}\selectfont If a solid object used to mark the boundary is disturbed for any reason, then:\par}\par


\vspace{\baselineskip}
{\fontsize{9pt}{10.8pt}\selectfont 19.3.1 \tabto{0.49in} the boundary shall be considered to be in its original position.\par}\par


\vspace{\baselineskip}
\begin{adjustwidth}{0.5in}{0.11in}
{\fontsize{9pt}{10.8pt}\selectfont 19.3.2 \tabto{0.49in} the object shall be returned to its original position as soon as is practicable; if play is taking place, this shall be as soon as the ball is dead.\par}\par

\end{adjustwidth}


\vspace{\baselineskip}
\begin{adjustwidth}{0.5in}{0.06in}
{\fontsize{9pt}{10.8pt}\selectfont 19.3.3 \tabto{0.49in} if some part of a fence or other marker has come within the field of play, that part shall be removed from the field of play as soon as is practicable; if play is taking place, this shall be as soon as the ball is dead.\par}\par

\end{adjustwidth}


\vspace{\baselineskip}
{\fontsize{11pt}{13.2pt}\selectfont \textbf{19.4 \tabto{0.47in} Ball grounded beyond the boundary}\par}\par


\vspace{\baselineskip}
{\fontsize{9pt}{10.8pt}\selectfont 19.4.1 \tabto{0.49in} {\fontsize{8pt}{9.6pt}\selectfont The ball in play is grounded beyond the boundary if it touches\par}\par}\par


\vspace{\baselineskip}

\vspace{\baselineskip}

\vspace{\baselineskip}

\vspace{\baselineskip}

\vspace{\baselineskip}
\begin{Center}
{\fontsize{8pt}{9.6pt}\selectfont 25\par}
\end{Center}\par


\vspace{\baselineskip}
\begin{itemize}
	\item {\fontsize{9pt}{10.8pt}\selectfont the boundary or any part of an object used to mark the boundary;\par}\par


\vspace{\baselineskip}
	\item {\fontsize{9pt}{10.8pt}\selectfont the ground beyond the boundary;\par}\par


\vspace{\baselineskip}
	\item {\fontsize{9pt}{10.8pt}\selectfont any object that is grounded beyond the boundary.\par}
\end{itemize}\par


\vspace{\baselineskip}
{\fontsize{9pt}{10.8pt}\selectfont 19.4.2 \tabto{0.49in} The ball in play is to be regarded as being grounded beyond the boundary if\par}\par


\vspace{\baselineskip}
\begin{itemize}
	\item {\fontsize{9pt}{10.8pt}\selectfont a fielder, grounded beyond the boundary as in clause touches the ball;\par}\par


\vspace{\baselineskip}
	\item {\fontsize{9pt}{10.8pt}\selectfont a fielder, after catching the ball within the boundary, becomes grounded beyond the boundary while in contact with the ball, before completing the catch.\par}
\end{itemize}\par


\vspace{\baselineskip}
{\fontsize{11pt}{13.2pt}\selectfont \textbf{19.5 \tabto{0.47in} Fielder grounded beyond the boundary}\par}\par


\vspace{\baselineskip}
{\fontsize{9pt}{10.8pt}\selectfont 19.5.1 \tabto{0.49in} A fielder is grounded beyond the boundary if some part of his person is in contact with any of the following:\par}\par


\vspace{\baselineskip}
\begin{itemize}
	\item {\fontsize{9pt}{10.8pt}\selectfont the boundary or any part of an object used to mark the boundary;\par}\par


\vspace{\baselineskip}
	\item {\fontsize{9pt}{10.8pt}\selectfont the ground beyond the boundary;\par}\par


\vspace{\baselineskip}
	\item {\fontsize{9pt}{10.8pt}\selectfont any object that is in contact with the ground beyond the boundary;\par}\par


\vspace{\baselineskip}
	\item {\fontsize{9pt}{10.8pt}\selectfont another fielder who is grounded beyond the boundary.\par}
\end{itemize}\par


\vspace{\baselineskip}
\begin{adjustwidth}{0.5in}{0.07in}
{\fontsize{9pt}{10.8pt}\selectfont 19.5.2 \tabto{0.49in} A fielder who is not in contact with the ground is considered to be grounded beyond the boundary if his final contact with the ground, before his first contact with the ball after it has been delivered by the bowler, was not entirely within the boundary.\par}\par

\end{adjustwidth}


\vspace{\baselineskip}
{\fontsize{11pt}{13.2pt}\selectfont \textbf{19.6 \tabto{0.47in} Boundary allowances}\par}\par


\vspace{\baselineskip}
{\fontsize{9pt}{10.8pt}\selectfont 19.6.1 \tabto{0.49in} {\fontsize{8pt}{9.6pt}\selectfont 6 runs shall be allowed for a boundary 6; and 4 runs for a boundary 4. See also clause \par}\par}\par


\vspace{\baselineskip}
{\fontsize{11pt}{13.2pt}\selectfont \textbf{19.7 \tabto{0.47in} Runs scored from boundaries}\par}\par


\vspace{\baselineskip}
\begin{adjustwidth}{0.5in}{0.0in}
{\fontsize{9pt}{10.8pt}\selectfont 19.7.1 \tabto{0.49in} A boundary 6 will be scored if and only if the ball has been struck by the bat and is first grounded beyond the boundary without having been in contact with the ground within the field of play. This shall apply even if the ball has previously touched a fielder.\par}\par

\end{adjustwidth}


\vspace{\baselineskip}
{\fontsize{9pt}{10.8pt}\selectfont 19.7.2 \tabto{0.49in} {\fontsize{8pt}{9.6pt}\selectfont A boundary 4 will be scored when a ball that is grounded beyond the boundary\par}\par}\par


\vspace{\baselineskip}
\begin{itemize}
	\item {\fontsize{9pt}{10.8pt}\selectfont whether struck by the bat or not, was first grounded within the boundary, or\par}\par


\vspace{\baselineskip}
	\item {\fontsize{9pt}{10.8pt}\selectfont has not been struck by the bat.\par}
\end{itemize}\par


\vspace{\baselineskip}
\begin{adjustwidth}{0.5in}{0.12in}
{\fontsize{9pt}{10.8pt}\selectfont 19.7.3 \tabto{0.49in} When a boundary is scored, the batting side, except in the circumstances of clause shall be awarded whichever is the greater of\par}\par

\end{adjustwidth}


\vspace{\baselineskip}
\begin{adjustwidth}{0.49in}{0.0in}
{\fontsize{9pt}{10.8pt}\selectfont 19.7.3.1 \tabto{1.17in} {\fontsize{8pt}{9.6pt}\selectfont the allowance for the boundary\par}\par}\par

\end{adjustwidth}


\vspace{\baselineskip}
\begin{adjustwidth}{1.18in}{0.01in}
{\fontsize{9pt}{10.8pt}\selectfont 19.7.3.2 \tabto{1.17in} the runs completed by the batsmen together with the run in progress if they had already crossed at the instant the boundary is scored.\par}\par

\end{adjustwidth}


\vspace{\baselineskip}
\begin{adjustwidth}{0.5in}{0.01in}
{\fontsize{9pt}{10.8pt}\selectfont 19.7.4 \tabto{0.49in} When the runs in clause exceed the boundary allowance they shall replace the boundary allowance for the purposes of clause \par}\par

\end{adjustwidth}


\vspace{\baselineskip}
{\fontsize{9pt}{10.8pt}\selectfont 19.7.5 \tabto{0.49in} The scoring of Penalty runs by either side is not affected by the scoring of a boundary.\par}\par


\vspace{\baselineskip}
{\fontsize{11pt}{13.2pt}\selectfont \textbf{19.8 \tabto{0.47in} Overthrow or wilful act of fielder}\par}\par


\vspace{\baselineskip}
{\fontsize{9pt}{10.8pt}\selectfont If the boundary results from an overthrow or from the wilful act of a fielder, the runs scored shall be\par}\par


\vspace{\baselineskip}
{\fontsize{9pt}{10.8pt}\selectfont any runs for penalties awarded to either side\par}\par


\vspace{\baselineskip}
{\fontsize{9pt}{10.8pt}\selectfont and the allowance for the boundary\par}\par


\vspace{\baselineskip}

\vspace{\baselineskip}

\vspace{\baselineskip}

\vspace{\baselineskip}
\begin{Center}
{\fontsize{8pt}{9.6pt}\selectfont 26\par}
\end{Center}\par


\vspace{\baselineskip}

\vspace{\baselineskip}
\begin{adjustwidth}{0.0in}{0.03in}
{\fontsize{9pt}{10.8pt}\selectfont and the runs completed by the batsmen, together with the run in progress if they had already crossed at the instant of the throw or act.\par}\par

\end{adjustwidth}


\vspace{\baselineskip}
{\fontsize{9pt}{10.8pt}\selectfont Clause (Batsman returning to wicket he has left) shall apply as from the instant of the throw or act.\par}\par


\vspace{\baselineskip}
{\fontsize{16pt}{19.2pt}\selectfont \textbf{20 DEAD BALL}\par}\par


\vspace{\baselineskip}
{\fontsize{11pt}{13.2pt}\selectfont \textbf{20.1 \tabto{0.47in} Ball is dead}\par}\par


\vspace{\baselineskip}
{\fontsize{9pt}{10.8pt}\selectfont 20.1.1 \tabto{0.49in} The ball becomes dead when\par}\par


\vspace{\baselineskip}
\begin{adjustwidth}{0.49in}{0.0in}
{\fontsize{9pt}{10.8pt}\selectfont 20.1.1.1 \tabto{1.17in} it is finally settled in the hands of the wicket-keeper or of the bowler.\par}\par

\end{adjustwidth}


\vspace{\baselineskip}
\begin{adjustwidth}{0.49in}{0.0in}
{\fontsize{9pt}{10.8pt}\selectfont 20.1.1.2 \tabto{1.17in} {\fontsize{8pt}{9.6pt}\selectfont a boundary is scored. See clause (Runs scored from boundaries).\par}\par}\par

\end{adjustwidth}


\vspace{\baselineskip}
\begin{adjustwidth}{1.18in}{0.31in}
{\fontsize{9pt}{10.8pt}\selectfont 20.1.1.3 \tabto{1.17in} a batsman is dismissed. The ball will be deemed to be dead from the instant of the incident causing the dismissal.\par}\par

\end{adjustwidth}


\vspace{\baselineskip}
\begin{adjustwidth}{1.18in}{0.03in}
{\fontsize{9pt}{10.8pt}\selectfont 20.1.1.4 \tabto{1.17in} whether played or not it becomes trapped between the bat and person of a batsman or between items of his clothing or equipment.\par}\par

\end{adjustwidth}


\vspace{\baselineskip}
\begin{adjustwidth}{1.18in}{0.15in}
{\fontsize{9pt}{10.8pt}\selectfont 20.1.1.5 \tabto{1.17in} whether played or not it lodges in the clothing or equipment of a batsman or the clothing of an umpire.\par}\par

\end{adjustwidth}


\vspace{\baselineskip}
\begin{adjustwidth}{1.18in}{0.53in}
{\fontsize{9pt}{10.8pt}\selectfont 20.1.1.6 \tabto{1.17in} {\fontsize{8pt}{9.6pt}\selectfont there is an award of Penalty runs under either of clauses (Player returning without permission) or (Fielding the ball). The ball shall not count as one of the over.\par}\par}\par

\end{adjustwidth}


\vspace{\baselineskip}
\begin{adjustwidth}{0.49in}{0.0in}
{\fontsize{9pt}{10.8pt}\selectfont 20.1.1.7 \tabto{1.17in} there is a contravention of clause (Protective helmets belonging to the fielding side).\par}\par

\end{adjustwidth}


\vspace{\baselineskip}
\begin{adjustwidth}{0.49in}{0.0in}
{\fontsize{9pt}{10.8pt}\selectfont 20.1.1.8 \tabto{1.17in} {\fontsize{8pt}{9.6pt}\selectfont the match is concluded in any of the ways stated in clause (Conclusion of match).\par}\par}\par

\end{adjustwidth}


\vspace{\baselineskip}
\begin{adjustwidth}{0.5in}{0.07in}
{\fontsize{9pt}{10.8pt}\selectfont 20.1.2 \tabto{0.49in} The ball shall be considered to be dead when it is clear to the bowler’s end umpire that the fielding side and both batsmen at the wicket have ceased to regard it as in play.\par}\par

\end{adjustwidth}


\vspace{\baselineskip}
\begin{adjustwidth}{0.5in}{0.01in}
{\fontsize{9pt}{10.8pt}\selectfont 20.1.3 \tabto{0.49in} In a match where cameras are being used on or over the field of play (e.g. Spidercam), should a ball that has been hit by the batsman make contact, while still in play, with the camera, its apparatus or its cable, either umpire shall call and signal ‘dead ball’. The ball shall not count as one of the over and no runs shall be scored. If the delivery was called a No ball it shall count and the No ball penalty shall be applied. No other runs (including penalty runs) apart from the No ball penalty shall be scored.\par}\par

\end{adjustwidth}


\vspace{\baselineskip}
\begin{adjustwidth}{0.5in}{0.07in}
{\fontsize{9pt}{10.8pt}\selectfont 20.1.4 \tabto{0.49in} Should a ball thrown by a fielder make contact with a camera on or over the field of play, its apparatus or its cable, either umpire shall call and signal dead ball. Unless this was already a No ball or Wide, the ball shall count as one of the over. All runs scored to that point shall count, plus the run in progress if the batsmen have already crossed.\par}\par

\end{adjustwidth}


\vspace{\baselineskip}
{\fontsize{9pt}{10.8pt}\selectfont 20.1.5 \tabto{0.49in} {\fontsize{8pt}{9.6pt}\selectfont Refer also to paragraph of Appendix D.\par}\par}\par


\vspace{\baselineskip}
{\fontsize{11pt}{13.2pt}\selectfont \textbf{20.2 \tabto{0.47in} Ball finally settled}\par}\par


\vspace{\baselineskip}
{\fontsize{9pt}{10.8pt}\selectfont Whether the ball is finally settled or not is a matter for the umpire alone to decide.\par}\par


\vspace{\baselineskip}
{\fontsize{11pt}{13.2pt}\selectfont \textbf{20.3 \tabto{0.47in} Call of Over or Time}\par}\par


\vspace{\baselineskip}
\begin{adjustwidth}{0.0in}{0.14in}
{\fontsize{9pt}{10.8pt}\selectfont Neither the call of Over (see clause nor the call of Time (see clause is to be made until the ball is dead, either under clauses or \par}\par

\end{adjustwidth}


\vspace{\baselineskip}
{\fontsize{11pt}{13.2pt}\selectfont \textbf{20.4 \tabto{0.47in} Umpire calling and signalling Dead ball}\par}\par


\vspace{\baselineskip}
\begin{adjustwidth}{0.5in}{0.03in}
{\fontsize{9pt}{10.8pt}\selectfont 20.4.1 \tabto{0.49in} When the ball has become dead under clause the bowler’s end umpire may call and signal Dead ball if it is necessary to inform the players.\par}\par

\end{adjustwidth}


\vspace{\baselineskip}
{\fontsize{9pt}{10.8pt}\selectfont 20.4.2 \tabto{0.49in} Either umpire shall call and signal Dead ball when\par}\par


\vspace{\baselineskip}
\begin{adjustwidth}{0.49in}{0.0in}
{\fontsize{9pt}{10.8pt}\selectfont 20.4.2.1 \tabto{1.17in} {\fontsize{8pt}{9.6pt}\selectfont intervening in a case of unfair play.\par}\par}\par

\end{adjustwidth}


\vspace{\baselineskip}

\vspace{\baselineskip}

\vspace{\baselineskip}

\vspace{\baselineskip}
\begin{Center}
{\fontsize{8pt}{9.6pt}\selectfont 27\par}
\end{Center}\par


\vspace{\baselineskip}
\begin{adjustwidth}{0.49in}{0.0in}
{\fontsize{9pt}{10.8pt}\selectfont 20.4.2.2 \tabto{1.17in} {\fontsize{8pt}{9.6pt}\selectfont a possibly serious injury to a player or umpire occurs.\par}\par}\par

\end{adjustwidth}


\vspace{\baselineskip}
\begin{adjustwidth}{0.49in}{0.0in}
{\fontsize{9pt}{10.8pt}\selectfont 20.4.2.3 \tabto{1.17in} leaving his/her normal position for consultation.\par}\par

\end{adjustwidth}


\vspace{\baselineskip}
\begin{adjustwidth}{1.18in}{0.4in}
{\fontsize{9pt}{10.8pt}\selectfont 20.4.2.4 \tabto{1.17in} one or both bails fall from the striker’s wicket before the striker has had the opportunity of playing the ball.\par}\par

\end{adjustwidth}


\vspace{\baselineskip}
\begin{adjustwidth}{1.18in}{0.11in}
{\fontsize{9pt}{10.8pt}\selectfont 20.4.2.5 \tabto{1.17in} the striker is not ready for the delivery of the ball and, if the ball is delivered, makes no attempt to play it. Provided the umpire is satisfied that the striker had adequate reason for not being ready, the ball shall not count as one of the over.\par}\par

\end{adjustwidth}


\vspace{\baselineskip}
\begin{adjustwidth}{1.18in}{0.17in}
{\fontsize{9pt}{10.8pt}\selectfont 20.4.2.6 \tabto{1.17in} the striker is distracted by any noise or movement or in any other way while preparing to receive, or receiving a delivery. This shall apply whether the source of the distraction is within the match or outside it. Note also clause The ball shall not count as one of the over.\par}\par

\end{adjustwidth}


\vspace{\baselineskip}
\begin{adjustwidth}{1.18in}{0.1in}
\begin{justify}
{\fontsize{9pt}{10.8pt}\selectfont 20.4.2.7 \tabto{1.17in} there is an instance of a deliberate attempt to distract under either of clauses (Deliberate attempt to distract striker) or (Deliberate distraction, deception or obstruction of batsman). The ball shall not count as one of the over.\par}
\end{justify}\par

\end{adjustwidth}


\vspace{\baselineskip}
\begin{adjustwidth}{0.49in}{0.0in}
{\fontsize{9pt}{10.8pt}\selectfont 20.4.2.8 \tabto{1.17in} the bowler drops the ball accidentally before delivery.\par}\par

\end{adjustwidth}


\vspace{\baselineskip}
\begin{adjustwidth}{1.18in}{0.22in}
{\fontsize{9pt}{10.8pt}\selectfont 20.4.2.9 \tabto{1.17in} the ball does not leave the bowler’s hand for any reason other than an attempt to run out the non-striker under clause (Non-striker leaving his ground early).\par}\par

\end{adjustwidth}


\vspace{\baselineskip}
\begin{adjustwidth}{0.49in}{0.0in}
{\fontsize{9pt}{10.8pt}\selectfont 20.4.2.10 \tabto{1.17in} satisfied that the ball in play cannot be recovered.\par}\par

\end{adjustwidth}


\vspace{\baselineskip}
\begin{adjustwidth}{0.49in}{0.0in}
{\fontsize{9pt}{10.8pt}\selectfont 20.4.2.11 \tabto{1.17in} required to do so under any of the Playing Conditions not included above.\par}\par

\end{adjustwidth}


\vspace{\baselineskip}
{\fontsize{11pt}{13.2pt}\selectfont \textbf{20.5 \tabto{0.47in} Ball ceases to be dead}\par}\par


\vspace{\baselineskip}
\begin{adjustwidth}{0.0in}{0.18in}
{\fontsize{9pt}{10.8pt}\selectfont The ball ceases to be dead – that is, it comes into play – when the bowler starts his run-up or, if there is no run-up, starts his bowling action.\par}\par

\end{adjustwidth}


\vspace{\baselineskip}
{\fontsize{11pt}{13.2pt}\selectfont \textbf{20.6 \tabto{0.47in} Dead ball; ball counting as one of over}\par}\par


\vspace{\baselineskip}
\begin{adjustwidth}{0.5in}{0.35in}
{\fontsize{9pt}{10.8pt}\selectfont 20.6.1 \tabto{0.49in} When a ball which has been delivered is called dead or is to be considered dead then, other than as in clause \par}\par

\end{adjustwidth}


\vspace{\baselineskip}
\begin{adjustwidth}{0.49in}{0.0in}
{\fontsize{9pt}{10.8pt}\selectfont 20.6.1.1 \tabto{1.17in} {\fontsize{8pt}{9.6pt}\selectfont it will not count in the over if the striker has not had an opportunity to play it.\par}\par}\par

\end{adjustwidth}


\vspace{\baselineskip}
\begin{adjustwidth}{1.18in}{0.0in}
{\fontsize{9pt}{10.8pt}\selectfont 20.6.1.2 \tabto{1.17in} unless No ball or Wide ball has been called, it will be a valid ball if the striker has had an opportunity to play it, except in the circumstances of clauses and ( Player returning without permission), (Fielding the ball), (Deliberate attempt to distract striker) and (Deliberate distraction, deception or obstruction of batsman).\par}\par

\end{adjustwidth}


\vspace{\baselineskip}
\begin{adjustwidth}{0.5in}{0.11in}
{\fontsize{9pt}{10.8pt}\selectfont 20.6.2 \tabto{0.49in} In clause the ball will not count in the over only if both conditions of not attempting to play the ball and having an adequate reason for not being ready are met. Otherwise the delivery will be a valid ball.\par}\par

\end{adjustwidth}


\vspace{\baselineskip}
{\fontsize{16pt}{19.2pt}\selectfont \textbf{21 NO BALL}\par}\par


\vspace{\baselineskip}
{\fontsize{11pt}{13.2pt}\selectfont \textbf{21.1 \tabto{0.47in} Mode of delivery}\par}\par


\vspace{\baselineskip}
\begin{adjustwidth}{0.5in}{0.03in}
{\fontsize{9pt}{10.8pt}\selectfont 21.1.1 \tabto{0.49in} The umpire shall ascertain whether the bowler intends to bowl right handed or left handed, over or round the wicket, and shall so inform the striker.\par}\par

\end{adjustwidth}


\vspace{\baselineskip}
\begin{adjustwidth}{0.5in}{0.1in}
{\fontsize{9pt}{10.8pt}\selectfont It is unfair if the bowler fails to notify the umpire of a change in his mode of delivery. In this case the umpire shall call and signal No ball.\par}\par

\end{adjustwidth}


\vspace{\baselineskip}
{\fontsize{9pt}{10.8pt}\selectfont 21.1.2 \tabto{0.49in} {\fontsize{8pt}{9.6pt}\selectfont Underarm bowling shall not be permitted.\par}\par}\par


\vspace{\baselineskip}
{\fontsize{11pt}{13.2pt}\selectfont \textbf{21.2 \tabto{0.47in} Fair delivery – the arm}\par}\par


\vspace{\baselineskip}
{\fontsize{9pt}{10.8pt}\selectfont For a delivery to be fair in respect of the arm the ball must not be delivered with an Illegal Bowling Action.\par}\par


\vspace{\baselineskip}

\vspace{\baselineskip}

\vspace{\baselineskip}

\vspace{\baselineskip}

\vspace{\baselineskip}

\vspace{\baselineskip}
\begin{Center}
{\fontsize{8pt}{9.6pt}\selectfont 28\par}
\end{Center}\par


\vspace{\baselineskip}

\vspace{\baselineskip}
\begin{adjustwidth}{0.0in}{0.12in}
{\fontsize{9pt}{10.8pt}\selectfont An Illegal Bowling Action is defined as a bowling action where a bowler’s Elbow Extension exceeds 15 degrees, measured from the point at which the bowling arm reaches the horizontal until the point at which the ball is released (any Elbow Hyperextension shall be discounted for the purposes of determining an Illegal Bowling Action).\par}\par

\end{adjustwidth}


\vspace{\baselineskip}
\begin{adjustwidth}{0.0in}{0.15in}
{\fontsize{9pt}{10.8pt}\selectfont Should either umpire or the ICC Match Referee suspect that a bowler has used an Illegal Bowling Action, they shall complete the ICC Bowling Action Report Form at the conclusion of the match, as set out in the Illegal Bowling Regulations.\par}\par

\end{adjustwidth}


\vspace{\baselineskip}
{\fontsize{11pt}{13.2pt}\selectfont \textbf{21.3 \tabto{0.47in} Ball thrown or delivered underarm – action by umpires}\par}\par


\vspace{\baselineskip}
\begin{adjustwidth}{0.5in}{0.0in}
{\fontsize{9pt}{10.8pt}\selectfont 21.3.1 \tabto{0.49in} If, in the opinion of either umpire, the ball has been thrown (where such mode of delivery does not correspond to the bowler’s normal bowling action) or delivered underarm, he/she shall call and signal No ball and, when the ball is dead, inform the other umpire of the reason for the call.\par}\par

\end{adjustwidth}


\vspace{\baselineskip}
\begin{adjustwidth}{0.5in}{0.0in}
{\fontsize{9pt}{10.8pt}\selectfont The bowler’s end umpire shall then\par}\par

\end{adjustwidth}


\vspace{\baselineskip}
\begin{itemize}
	\item {\fontsize{9pt}{10.8pt}\selectfont warn the bowler, indicating that this is a first and final warning. This warning shall apply to that bowler throughout the innings.\par}\par


\vspace{\baselineskip}
	\item {\fontsize{9pt}{10.8pt}\selectfont inform the captain of the fielding side of the reason for this action.\par}\par


\vspace{\baselineskip}
	\item {\fontsize{9pt}{10.8pt}\selectfont inform the batsmen at the wicket of what has occurred.\par}
\end{itemize}\par


\vspace{\baselineskip}
\begin{adjustwidth}{0.5in}{0.08in}
{\fontsize{9pt}{10.8pt}\selectfont 21.3.2 \tabto{0.49in} {\fontsize{8pt}{9.6pt}\selectfont If either umpire considers that, in that innings, a further delivery by the same bowler is thrown (where such mode of delivery does not correspond to the bowler’s normal bowling action) or delivered underarm, he/she shall call and signal No ball and when the ball is dead inform the other umpire of the reason for the call.\par}\par}\par

\end{adjustwidth}


\vspace{\baselineskip}
\begin{adjustwidth}{0.5in}{0.0in}
{\fontsize{9pt}{10.8pt}\selectfont The bowler’s end umpire shall then\par}\par

\end{adjustwidth}


\vspace{\baselineskip}
\begin{itemize}
	\item {\fontsize{9pt}{10.8pt}\selectfont direct the captain of the fielding side to suspend the bowler immediately from bowling. The over shall, if applicable, be completed by another bowler, who shall neither have bowled the previous over or part thereof nor be allowed to bowl any part of the next over. The bowler thus suspended shall not bowl again in that innings.\par}\par


\vspace{\baselineskip}
	\item {\fontsize{9pt}{10.8pt}\selectfont inform the batsmen at the wicket and, as soon as practicable, the captain of the batting side of the reason for this action.\par}
\end{itemize}\par


\vspace{\baselineskip}
\begin{adjustwidth}{0.5in}{0.32in}
{\fontsize{9pt}{10.8pt}\selectfont 21.3.3 \tabto{0.49in} The umpires together shall report the occurrence as soon as possible after the match to the ICC Match Referee, who shall take such action as is considered appropriate against the bowler concerned.\par}\par

\end{adjustwidth}


\vspace{\baselineskip}
{\fontsize{11pt}{13.2pt}\selectfont \textbf{21.4 \tabto{0.47in} Bowler throwing towards striker’s end before delivery}\par}\par


\vspace{\baselineskip}
\begin{adjustwidth}{0.0in}{0.14in}
{\fontsize{9pt}{10.8pt}\selectfont If the bowler throws the ball towards the striker’s end before entering the delivery stride, either umpire shall call and signal No ball. See clause (Batsmen stealing a run).\par}\par

\end{adjustwidth}


\vspace{\baselineskip}
\begin{adjustwidth}{0.0in}{0.36in}
{\fontsize{9pt}{10.8pt}\selectfont However, the procedure stated in clause of caution, informing, final warning, action against the bowler and reporting shall not apply.\par}\par

\end{adjustwidth}


\vspace{\baselineskip}
{\fontsize{11pt}{13.2pt}\selectfont \textbf{21.5 \tabto{0.47in} Fair delivery – the feet}\par}\par


\vspace{\baselineskip}
{\fontsize{9pt}{10.8pt}\selectfont For a delivery to be fair in respect of the feet, in the delivery stride\par}\par


\vspace{\baselineskip}
\begin{adjustwidth}{0.5in}{0.0in}
{\fontsize{9pt}{10.8pt}\selectfont 21.5.1 \tabto{0.49in} the bowler’s back foot must land within and not touching the return crease appertaining to his stated mode of delivery.\par}\par

\end{adjustwidth}


\vspace{\baselineskip}
{\fontsize{9pt}{10.8pt}\selectfont 21.5.2 \tabto{0.49in} {\fontsize{8pt}{9.6pt}\selectfont the bowler’s front foot must land with some part of the foot, whether grounded or raised\par}\par}\par


\vspace{\baselineskip}
\begin{itemize}
	\item {\fontsize{9pt}{10.8pt}\selectfont on the same side of the imaginary line joining the two middle stumps as the return crease described in clause and\par}\par


\vspace{\baselineskip}
	\item {\fontsize{9pt}{10.8pt}\selectfont behind the popping crease.\par}
\end{itemize}\par


\vspace{\baselineskip}
\begin{adjustwidth}{0.0in}{0.29in}
{\fontsize{9pt}{10.8pt}\selectfont If the bowler’s end umpire is satisfied that any of these three conditions have not been met, he/she shall call and signal No ball. See clause (Bowling of deliberate front foot No ball).\par}\par

\end{adjustwidth}


\vspace{\baselineskip}

\vspace{\baselineskip}

\vspace{\baselineskip}
\begin{Center}
{\fontsize{8pt}{9.6pt}\selectfont 29\par}
\end{Center}\par


\vspace{\baselineskip}
{\fontsize{11pt}{13.2pt}\selectfont \textbf{21.6 \tabto{0.47in} Bowler breaking wicket in delivering ball}\par}\par


\vspace{\baselineskip}
\begin{adjustwidth}{0.0in}{0.19in}
\begin{justify}
{\fontsize{9pt}{10.8pt}\selectfont Either umpire shall call and signal No ball if, other than in an attempt to run out the non-striker under clause  the bowler breaks the wicket at any time after the ball comes into play and before completion of the stride after the delivery stride. This shall include any clothing or other object that falls from his person and breaks the wicket.\par}
\end{justify}\par

\end{adjustwidth}


\vspace{\baselineskip}
{\fontsize{11pt}{13.2pt}\selectfont \textbf{21.7 \tabto{0.47in} Ball bouncing more than once, rolling along the ground or pitching off the pitch}\par}\par


\vspace{\baselineskip}
\begin{adjustwidth}{0.0in}{0.39in}
{\fontsize{9pt}{10.8pt}\selectfont The umpire shall call and signal No ball if a ball which he/she considers to have been delivered, without having previously touched bat or person of the striker,\par}\par

\end{adjustwidth}


\vspace{\baselineskip}
\begin{itemize}
	\item {\fontsize{9pt}{10.8pt}\selectfont bounces more than once\par}\par


\vspace{\baselineskip}
	\item {\fontsize{9pt}{10.8pt}\selectfont or rolls along the ground before it reaches the popping crease.\par}\par


\vspace{\baselineskip}
	\item {\fontsize{9pt}{10.8pt}\selectfont or pitches wholly or partially off the pitch as defined in clause (Area of pitch) before it reaches the line of the striker’s wicket.\par}
\end{itemize}\par


\vspace{\baselineskip}
{\fontsize{11pt}{13.2pt}\selectfont \textbf{21.8 \tabto{0.47in} Ball coming to rest in front of striker’s wicket}\par}\par


\vspace{\baselineskip}
\begin{adjustwidth}{0.0in}{0.04in}
{\fontsize{9pt}{10.8pt}\selectfont If a ball delivered by the bowler comes to rest in front of the line of the striker’s wicket, without having previously touched the bat or person of the striker, the umpire shall call and signal No ball and immediately call and signal Dead ball.\par}\par

\end{adjustwidth}


\vspace{\baselineskip}
{\fontsize{11pt}{13.2pt}\selectfont \textbf{21.9 \tabto{0.47in} Fielder intercepting a delivery}\par}\par


\vspace{\baselineskip}
\begin{adjustwidth}{0.0in}{0.14in}
{\fontsize{9pt}{10.8pt}\selectfont If, except in the circumstances of clause (Position of wicket-keeper) a ball delivered by the bowler, makes contact with any part of a fielder’s person before it either makes contact with the striker’s bat or person, or it passes the striker’s wicket, the umpire shall call and signal No ball and immediately call and signal Dead ball.\par}\par

\end{adjustwidth}


\vspace{\baselineskip}
{\fontsize{11pt}{13.2pt}\selectfont \textbf{21.10\  Ball bouncing over head height of striker}\par}\par


\vspace{\baselineskip}
{\fontsize{9pt}{10.8pt}\selectfont See clauses and 41.6.1.7.\par}\par


\vspace{\baselineskip}
{\fontsize{11pt}{13.2pt}\selectfont \textbf{21.11\  Call of No ball for infringement of other Playing Conditions}\par}\par


\vspace{\baselineskip}
{\fontsize{9pt}{10.8pt}\selectfont In addition to the instances above, No ball is to be called and signalled as required by the following clauses:\par}\par


\vspace{\baselineskip}
\begin{adjustwidth}{0.04in}{0.0in}
{\fontsize{9pt}{10.8pt}\selectfont Clause – Position of wicket-keeper\par}\par

\end{adjustwidth}


\vspace{\baselineskip}
\begin{adjustwidth}{0.04in}{0.0in}
{\fontsize{9pt}{10.8pt}\selectfont Clause – Limitation of on side fielders\par}\par

\end{adjustwidth}


\vspace{\baselineskip}
\begin{adjustwidth}{0.04in}{0.0in}
{\fontsize{9pt}{10.8pt}\selectfont Clause 28.5 – Fielders not to encroach on pitch\par}\par

\end{adjustwidth}


\vspace{\baselineskip}
\begin{adjustwidth}{0.04in}{0.0in}
{\fontsize{9pt}{10.8pt}\selectfont Clause – Bowling of dangerous and unfair short pitched deliveries\par}\par

\end{adjustwidth}


\vspace{\baselineskip}
\begin{adjustwidth}{0.04in}{0.0in}
{\fontsize{9pt}{10.8pt}\selectfont Clause – Bowling of dangerous and unfair non-pitching deliveries\par}\par

\end{adjustwidth}


\vspace{\baselineskip}
\begin{adjustwidth}{0.04in}{0.0in}
{\fontsize{9pt}{10.8pt}\selectfont Clause – Bowling of deliberate front foot No ball.\par}\par

\end{adjustwidth}


\vspace{\baselineskip}
{\fontsize{11pt}{13.2pt}\selectfont \textbf{21.12\  Revoking a call of No ball}\par}\par


\vspace{\baselineskip}
\begin{adjustwidth}{0.0in}{0.31in}
{\fontsize{9pt}{10.8pt}\selectfont An umpire shall revoke the call of No ball if Dead ball is called under any of clauses to (Umpire calling and signaling Dead ball).\par}\par

\end{adjustwidth}


\vspace{\baselineskip}
{\fontsize{11pt}{13.2pt}\selectfont \textbf{21.13\  No ball to over-ride Wide}\par}\par


\vspace{\baselineskip}
\begin{adjustwidth}{0.0in}{0.01in}
{\fontsize{9pt}{10.8pt}\selectfont A call of No ball shall over-ride the call of Wide ball at any time. See clauses 22.1(Judging a Wide) and (Call and signal of Wide ball).\par}\par

\end{adjustwidth}


\vspace{\baselineskip}
{\fontsize{11pt}{13.2pt}\selectfont \textbf{21.14\  Ball not dead}\par}\par


\vspace{\baselineskip}
{\fontsize{9pt}{10.8pt}\selectfont The ball does not become dead on the call of No ball.\par}\par


\vspace{\baselineskip}

\vspace{\baselineskip}

\vspace{\baselineskip}

\vspace{\baselineskip}

\vspace{\baselineskip}
\begin{Center}
{\fontsize{8pt}{9.6pt}\selectfont 30\par}
\end{Center}\par


\vspace{\baselineskip}
{\fontsize{11pt}{13.2pt}\selectfont \textbf{21.15\  Penalty for a No ball}\par}\par


\vspace{\baselineskip}
\begin{adjustwidth}{0.0in}{0.21in}
{\fontsize{9pt}{10.8pt}\selectfont A penalty of one run shall be awarded instantly on the call of No ball. Unless the call is revoked, the penalty shall stand even if a batsman is dismissed. It shall be in addition to any other runs scored, any boundary allowance and any other runs awarded for penalties.\par}\par

\end{adjustwidth}


\vspace{\baselineskip}
{\fontsize{11pt}{13.2pt}\selectfont \textbf{21.16\  Runs resulting from a No ball – how scored}\par}\par


\vspace{\baselineskip}
\begin{adjustwidth}{0.0in}{0.03in}
{\fontsize{9pt}{10.8pt}\selectfont The one run penalty shall be scored as a No ball extra and shall be debited against the bowler. If other Penalty runs have been awarded to either side these shall be scored as stated in clause (Penalty runs). Any runs completed by the batsmen or any boundary allowance shall be credited to the striker if the ball has been struck by the bat; otherwise they shall also be scored as Byes or Leg byes as appropriate.\par}\par

\end{adjustwidth}


\vspace{\baselineskip}
{\fontsize{11pt}{13.2pt}\selectfont \textbf{21.17\  No ball not to count}\par}\par


\vspace{\baselineskip}
{\fontsize{9pt}{10.8pt}\selectfont A No ball shall not count as one of the over. See clause (Validity of balls).\par}\par


\vspace{\baselineskip}
{\fontsize{11pt}{13.2pt}\selectfont \textbf{21.18\  Out from a No ball}\par}\par


\vspace{\baselineskip}
\begin{adjustwidth}{0.0in}{0.22in}
{\fontsize{9pt}{10.8pt}\selectfont When No ball has been called, neither batsman shall be out under any of the Playing Conditions except clause  (Hit the ball twice), clause (Obstructing the field) or clause (Run out).\par}\par

\end{adjustwidth}


\vspace{\baselineskip}
{\fontsize{11pt}{13.2pt}\selectfont \textbf{21.19\  Free Hit}\par}\par


\vspace{\baselineskip}
\begin{adjustwidth}{0.5in}{0.14in}
{\fontsize{9pt}{10.8pt}\selectfont 21.19.1 In addition to the above, the delivery following a No ball called (all modes of No ball) shall be a free hit for whichever batsman is facing it. If the delivery for the free hit is not a legitimate delivery (any kind of No ball or a Wide) then the next delivery will become a free hit for whichever batsman is facing it.\par}\par

\end{adjustwidth}


\vspace{\baselineskip}
\begin{adjustwidth}{0.5in}{0.15in}
{\fontsize{9pt}{10.8pt}\selectfont 21.19.2 For any free hit, the striker can be dismissed only under the circumstances that apply for a No ball, even if the delivery for the free hit is called Wide.\par}\par

\end{adjustwidth}


\vspace{\baselineskip}
\begin{adjustwidth}{0.5in}{0.24in}
{\fontsize{9pt}{10.8pt}\selectfont 21.19.3 Neither field changes nor the exchange of individuals between fielding positions are permitted for free hit deliveries unless:\par}\par

\end{adjustwidth}


\vspace{\baselineskip}
\begin{adjustwidth}{0.49in}{0.0in}
{\fontsize{9pt}{10.8pt}\selectfont 21.19.3.1 \tabto{1.17in} There is a change of striker (the provisions of clause shall apply), or\par}\par

\end{adjustwidth}


\vspace{\baselineskip}
\begin{adjustwidth}{1.18in}{0.47in}
{\fontsize{9pt}{10.8pt}\selectfont 21.19.3.2 \tabto{1.17in} The No ball was the result of a fielding restriction breach, in which case the field may be changed to the extent of correcting the breach.\par}\par

\end{adjustwidth}


\vspace{\baselineskip}
\begin{adjustwidth}{0.5in}{0.4in}
{\fontsize{9pt}{10.8pt}\selectfont 21.19.4 For clarity, the bowler can change his mode of delivery for the free hit delivery. In such circumstances Clause shall apply.\par}\par

\end{adjustwidth}


\vspace{\baselineskip}
\begin{adjustwidth}{0.5in}{0.0in}
{\fontsize{9pt}{10.8pt}\selectfont 21.19.5 The umpires will signal a free hit by (after the normal No ball signal) extending one arm straight upwards and moving it in a circular motion.\par}\par

\end{adjustwidth}


\vspace{\baselineskip}
{\fontsize{16pt}{19.2pt}\selectfont \textbf{22 WIDE BALL}\par}\par


\vspace{\baselineskip}
{\fontsize{11pt}{13.2pt}\selectfont \textbf{22.1 \tabto{0.47in} Judging a Wide}\par}\par


\vspace{\baselineskip}
\begin{adjustwidth}{0.5in}{0.51in}
{\fontsize{9pt}{10.8pt}\selectfont 22.1.1 \tabto{0.49in} If the bowler bowls a ball, not being a No ball, the umpire shall adjudge it a Wide if, according to the definition in clause \par}\par

\end{adjustwidth}


\vspace{\baselineskip}
\begin{adjustwidth}{1.18in}{0.07in}
{\fontsize{9pt}{10.8pt}\selectfont 22.1.1.1 \tabto{1.17in} the ball passes wide of where the striker is standing and which also would have passed wide of the striker standing in a normal guard position.\par}\par

\end{adjustwidth}


\vspace{\baselineskip}
\begin{adjustwidth}{0.49in}{0.0in}
{\fontsize{9pt}{10.8pt}\selectfont 22.1.1.2 \tabto{1.17in} the ball passes above the head height of the striker standing upright at the popping crease.\par}\par

\end{adjustwidth}


\vspace{\baselineskip}
\begin{adjustwidth}{0.5in}{0.22in}
{\fontsize{9pt}{10.8pt}\selectfont 22.1.2 \tabto{0.49in} The ball will be considered as passing wide of the striker unless it is sufficiently within reach for him to be able to hit it with the bat by means of a normal cricket stroke.\par}\par

\end{adjustwidth}


\vspace{\baselineskip}
\begin{adjustwidth}{0.5in}{0.21in}
{\fontsize{9pt}{10.8pt}\selectfont 22.1.3 \tabto{0.49in} Umpires are instructed to apply very strict and consistent interpretation in regard to this clause in order to prevent negative bowling wide of the wicket.\par}\par

\end{adjustwidth}


\vspace{\baselineskip}

\vspace{\baselineskip}

\vspace{\baselineskip}

\vspace{\baselineskip}

\vspace{\baselineskip}
\begin{Center}
{\fontsize{8pt}{9.6pt}\selectfont 31\par}
\end{Center}\par


\vspace{\baselineskip}
{\fontsize{11pt}{13.2pt}\selectfont \textbf{22.2 \tabto{0.47in} Call and signal of Wide ball}\par}\par


\vspace{\baselineskip}
\begin{adjustwidth}{0.0in}{0.17in}
{\fontsize{9pt}{10.8pt}\selectfont If the umpire adjudges a delivery to be a Wide he/she shall call and signal Wide ball as soon as the ball passes the striker’s wicket. It shall, however, be considered to have been a Wide from the instant that the bowler entered his delivery stride, even though it cannot be called Wide until it passes the striker’s wicket.\par}\par

\end{adjustwidth}


\vspace{\baselineskip}
{\fontsize{11pt}{13.2pt}\selectfont \textbf{22.3 \tabto{0.47in} Revoking a call of Wide ball}\par}\par


\vspace{\baselineskip}
\begin{adjustwidth}{0.5in}{0.19in}
{\fontsize{9pt}{10.8pt}\selectfont 22.3.1 \tabto{0.49in} The umpire shall revoke the call of Wide ball if there is then any contact between the ball and the striker’s bat or person before the ball comes into contact with any fielder.\par}\par

\end{adjustwidth}


\vspace{\baselineskip}
\begin{adjustwidth}{0.5in}{0.22in}
{\fontsize{9pt}{10.8pt}\selectfont 22.3.2 \tabto{0.49in} The umpire shall revoke the call of Wide ball if a delivery is called a No ball. See clause (No ball to over-ride Wide).\par}\par

\end{adjustwidth}


\vspace{\baselineskip}
{\fontsize{11pt}{13.2pt}\selectfont \textbf{22.4 \tabto{0.47in} Delivery not a Wide}\par}\par


\vspace{\baselineskip}
\begin{adjustwidth}{0.5in}{0.08in}
{\fontsize{9pt}{10.8pt}\selectfont 22.4.1 \tabto{0.49in} The umpire shall not adjudge a delivery as being a Wide, if the striker, by moving, either causes the ball to pass wide of him, as defined in clause or brings the ball sufficiently within reach to be able to hit it by means of a normal cricket stroke.\par}\par

\end{adjustwidth}


\vspace{\baselineskip}
\begin{adjustwidth}{0.5in}{0.19in}
{\fontsize{9pt}{10.8pt}\selectfont 22.4.2 \tabto{0.49in} The umpire shall not adjudge a delivery as being a Wide if the ball touches the striker’s bat or person, but only as the ball passes the striker.\par}\par

\end{adjustwidth}


\vspace{\baselineskip}
{\fontsize{11pt}{13.2pt}\selectfont \textbf{22.5 \tabto{0.47in} Ball not dead}\par}\par


\vspace{\baselineskip}
{\fontsize{9pt}{10.8pt}\selectfont The ball does not become dead on the call of Wide ball.\par}\par


\vspace{\baselineskip}
{\fontsize{11pt}{13.2pt}\selectfont \textbf{22.6 \tabto{0.47in} Penalty for a Wide}\par}\par


\vspace{\baselineskip}
\begin{adjustwidth}{0.0in}{0.15in}
{\fontsize{9pt}{10.8pt}\selectfont A penalty of one run shall be awarded instantly on the call of Wide ball. Unless the call is revoked, see clause  this penalty shall stand even if a batsman is dismissed, and shall be in addition to any other runs scored, any boundary allowance and any other runs awarded for penalties.\par}\par

\end{adjustwidth}


\vspace{\baselineskip}
{\fontsize{11pt}{13.2pt}\selectfont \textbf{22.7 \tabto{0.47in} Runs resulting from a Wide – how scored}\par}\par


\vspace{\baselineskip}
\begin{adjustwidth}{0.0in}{0.14in}
{\fontsize{9pt}{10.8pt}\selectfont All runs completed by the batsmen or a boundary allowance, together with the penalty for the Wide, shall be scored as Wide balls. Apart from any award of 5 Penalty runs, all runs resulting from a Wide shall be debited against the bowler.\par}\par

\end{adjustwidth}


\vspace{\baselineskip}
{\fontsize{11pt}{13.2pt}\selectfont \textbf{22.8 \tabto{0.47in} Wide not to count}\par}\par


\vspace{\baselineskip}
{\fontsize{9pt}{10.8pt}\selectfont A Wide shall not count as one of the over. See clause (Validity of balls).\par}\par


\vspace{\baselineskip}
{\fontsize{11pt}{13.2pt}\selectfont \textbf{22.9 \tabto{0.47in} Out from a Wide}\par}\par


\vspace{\baselineskip}
\begin{adjustwidth}{0.0in}{0.1in}
\begin{justify}
{\fontsize{9pt}{10.8pt}\selectfont When Wide ball has been called, neither batsman shall be out under any of the Playing Conditions except clause 35 (Hit wicket), clause (Obstructing the field), clause (Run out) or clause (Stumped).\par}
\end{justify}\par

\end{adjustwidth}


\vspace{\baselineskip}
{\fontsize{16pt}{19.2pt}\selectfont \textbf{23 BYE AND LEG BYE}\par}\par


\vspace{\baselineskip}
{\fontsize{11pt}{13.2pt}\selectfont \textbf{23.1 \tabto{0.47in} Byes}\par}\par


\vspace{\baselineskip}
\begin{adjustwidth}{0.0in}{0.1in}
\begin{justify}
{\fontsize{9pt}{10.8pt}\selectfont If the ball, delivered by the bowler, not being a Wide, passes the striker without touching his bat or person, any runs completed by the batsmen from that delivery, or a boundary allowance, shall be credited as Byes to the batting side. Additionally, if the delivery is a No ball, the one run penalty for such a delivery shall be incurred.\par}
\end{justify}\par

\end{adjustwidth}


\vspace{\baselineskip}
{\fontsize{11pt}{13.2pt}\selectfont \textbf{23.2 \tabto{0.47in} Leg byes}\par}\par


\vspace{\baselineskip}
\begin{adjustwidth}{0.5in}{0.12in}
{\fontsize{9pt}{10.8pt}\selectfont 23.2.1 \tabto{0.49in} If a ball delivered by the bowler first strikes the person of the striker, runs shall be scored only if the umpire is satisfied that the striker has\par}\par

\end{adjustwidth}


\vspace{\baselineskip}
\begin{adjustwidth}{0.5in}{0.0in}
{\fontsize{9pt}{10.8pt}\selectfont either attempted to play the ball with the bat\par}\par

\end{adjustwidth}


\vspace{\baselineskip}

\vspace{\baselineskip}

\vspace{\baselineskip}

\vspace{\baselineskip}

\vspace{\baselineskip}
\begin{Center}
{\fontsize{8pt}{9.6pt}\selectfont 32\par}
\end{Center}\par


\vspace{\baselineskip}
\begin{adjustwidth}{0.5in}{0.0in}
{\fontsize{9pt}{10.8pt}\selectfont or tried to avoid being hit by the ball.\par}\par

\end{adjustwidth}


\vspace{\baselineskip}
{\fontsize{9pt}{10.8pt}\selectfont 23.2.2 \tabto{0.49in} {\fontsize{8pt}{9.6pt}\selectfont If the umpire is satisfied that either of these conditions has been met runs shall be scored as follows.\par}\par}\par


\vspace{\baselineskip}
\begin{adjustwidth}{0.49in}{0.0in}
{\fontsize{9pt}{10.8pt}\selectfont 23.2.2.1 \tabto{1.17in} {\fontsize{8pt}{9.6pt}\selectfont If there is\par}\par}\par

\end{adjustwidth}


\vspace{\baselineskip}
\begin{adjustwidth}{1.12in}{1.89in}
{\fontsize{9pt}{10.8pt}\selectfont either no subsequent contact with the striker’s bat or person, or only inadvertent contact with the striker’s bat or person\par}\par

\end{adjustwidth}


\vspace{\baselineskip}
\begin{adjustwidth}{1.12in}{0.07in}
{\fontsize{9pt}{10.8pt}\selectfont any runs completed by the batsmen or a boundary allowance shall be credited to the striker in the case of subsequent contact with his bat but otherwise to the batting side as in clause \par}\par

\end{adjustwidth}


\vspace{\baselineskip}
\begin{adjustwidth}{0.49in}{0.22in}
\begin{FlushRight}
{\fontsize{9pt}{10.8pt}\selectfont 23.2.2.2\ \ \ \  If the striker wilfully makes a lawful second strike, clause (Ball lawfully struck more than once) and clause (Runs permitted from ball lawfully struck more than once) shall apply.\par}
\end{FlushRight}\par

\end{adjustwidth}


\vspace{\baselineskip}
{\fontsize{9pt}{10.8pt}\selectfont 23.2.3 \tabto{0.49in} {\fontsize{8pt}{9.6pt}\selectfont The runs in clause unless credited to the striker, shall be scored as Leg byes.\par}\par}\par


\vspace{\baselineskip}
\begin{adjustwidth}{0.5in}{0.0in}
{\fontsize{9pt}{10.8pt}\selectfont Additionally, if the delivery is a No ball, the one run penalty for the No ball shall be incurred.\par}\par

\end{adjustwidth}


\vspace{\baselineskip}
{\fontsize{11pt}{13.2pt}\selectfont \textbf{23.3 \tabto{0.47in} Leg byes not to be awarded}\par}\par


\vspace{\baselineskip}
\begin{adjustwidth}{0.0in}{0.08in}
{\fontsize{9pt}{10.8pt}\selectfont If in the circumstance of clause the umpire considers that neither of the conditions therein has been met, then Leg byes shall not be awarded.\par}\par

\end{adjustwidth}


\vspace{\baselineskip}
\begin{adjustwidth}{0.0in}{0.11in}
{\fontsize{9pt}{10.8pt}\selectfont If the ball does not become dead for any other reason, the umpire shall call and signal Dead ball as soon as the ball reaches the boundary or at the completion of the first run.\par}\par

\end{adjustwidth}


\vspace{\baselineskip}
{\fontsize{9pt}{10.8pt}\selectfont The umpire shall then:\par}\par


\vspace{\baselineskip}
\begin{itemize}
	\item {\fontsize{9pt}{10.8pt}\selectfont disallow all runs to the batting side;\par}\par


\vspace{\baselineskip}
	\item {\fontsize{9pt}{10.8pt}\selectfont return any not out batsman to his original end;\par}\par


\vspace{\baselineskip}
	\item {\fontsize{9pt}{10.8pt}\selectfont signal No ball to the scorers if applicable;\par}\par


\vspace{\baselineskip}
	\item {\fontsize{9pt}{10.8pt}\selectfont award any 5-run Penalty that is applicable except for Penalty runs under clause (Protective helmets belonging to the fielding side).\par}
\end{itemize}\par


\vspace{\baselineskip}
{\fontsize{16pt}{19.2pt}\selectfont \textbf{24 FIELDER’S ABSENCE; SUBSTITUTES}\par}\par


\vspace{\baselineskip}
{\fontsize{11pt}{13.2pt}\selectfont \textbf{24.1 \tabto{0.47in} Substitute fielders}\par}\par


\vspace{\baselineskip}
{\fontsize{9pt}{10.8pt}\selectfont 24.1.1 \tabto{0.49in} The umpires shall allow a substitute fielder\par}\par


\vspace{\baselineskip}
\begin{adjustwidth}{1.18in}{0.03in}
{\fontsize{9pt}{10.8pt}\selectfont 24.1.1.1 \tabto{1.17in} if they are satisfied that a fielder has been injured or become ill and that this occurred during the match, or\par}\par

\end{adjustwidth}


\vspace{\baselineskip}
\begin{adjustwidth}{0.49in}{0.0in}
{\fontsize{9pt}{10.8pt}\selectfont 24.1.1.2 \tabto{1.17in} for any other wholly acceptable reason.\par}\par

\end{adjustwidth}


\vspace{\baselineskip}
\begin{adjustwidth}{0.5in}{0.0in}
{\fontsize{9pt}{10.8pt}\selectfont In all other circumstances, a substitute is not allowed.\par}\par

\end{adjustwidth}


\vspace{\baselineskip}
\begin{adjustwidth}{0.5in}{0.4in}
{\fontsize{9pt}{10.8pt}\selectfont 24.1.2 \tabto{0.49in} A substitute shall not bowl or act as captain but may act as wicket-keeper only with the consent of the umpires. Note, however, clause \par}\par

\end{adjustwidth}


\vspace{\baselineskip}
\begin{adjustwidth}{0.5in}{0.36in}
{\fontsize{9pt}{10.8pt}\selectfont 24.1.3 \tabto{0.49in} A nominated player may bowl or field even though a substitute has previously acted for him, subject to clauses and \par}\par

\end{adjustwidth}


\vspace{\baselineskip}
\begin{adjustwidth}{0.5in}{0.06in}
{\fontsize{9pt}{10.8pt}\selectfont 24.1.4 \tabto{0.49in} Squad members of the fielding or batting team who are not playing in the match and who are not acting as substitute fielders shall be required to wear a team training bib whilst on the playing area (including the area between the boundary and the perimeter fencing).\par}\par

\end{adjustwidth}


\vspace{\baselineskip}

\vspace{\baselineskip}

\vspace{\baselineskip}

\vspace{\baselineskip}

\vspace{\baselineskip}

\vspace{\baselineskip}

\vspace{\baselineskip}

\vspace{\baselineskip}
\begin{Center}
{\fontsize{8pt}{9.6pt}\selectfont 33\par}
\end{Center}\par


\vspace{\baselineskip}
{\fontsize{11pt}{13.2pt}\selectfont \textbf{24.2 \tabto{0.47in} Fielder absent or leaving the field of play}\par}\par


\vspace{\baselineskip}
\begin{adjustwidth}{0.5in}{0.14in}
{\fontsize{9pt}{10.8pt}\selectfont 24.2.1 \tabto{0.49in} A player going briefly outside the boundary while carrying out any duties as a fielder is not absent from the field of play nor, for the purposes of this clause, is he to be regarded as having left the field of play.\par}\par

\end{adjustwidth}


\vspace{\baselineskip}
{\fontsize{9pt}{10.8pt}\selectfont 24.2.2 \tabto{0.49in} {\fontsize{8pt}{9.6pt}\selectfont If a fielder fails to take the field at the start of play or at any later time, or leaves the field during play,\par}\par}\par


\vspace{\baselineskip}
\begin{adjustwidth}{0.49in}{0.0in}
{\fontsize{9pt}{10.8pt}\selectfont 24.2.2.1 \tabto{1.17in} an umpire shall be informed of the reason for this absence.\par}\par

\end{adjustwidth}


\vspace{\baselineskip}
\begin{adjustwidth}{1.18in}{0.06in}
{\fontsize{9pt}{10.8pt}\selectfont 24.2.2.2 \tabto{1.17in} he shall not thereafter come on to the field of play during a session of play without the consent of the umpire. See clause The umpire shall give such consent as soon as it is practicable.\par}\par

\end{adjustwidth}


\vspace{\baselineskip}
\begin{adjustwidth}{0.5in}{0.29in}
{\fontsize{9pt}{10.8pt}\selectfont 24.2.3 \tabto{0.49in} If a player is absent from the field for longer than 8 minutes, the following restrictions shall apply to their future participation in the match:\par}\par

\end{adjustwidth}


\vspace{\baselineskip}
\begin{adjustwidth}{1.18in}{0.08in}
{\fontsize{9pt}{10.8pt}\selectfont 24.2.3.1 \tabto{1.17in} The player shall not be permitted to bowl in the match until he has either been able to field, or his team has subsequently been batting, for the total length of playing time for which the player was absent (hereafter referred to as Penalty time). A player’s unexpired Penalty time shall be limited to a maximum of 40 minutes. If any unexpired Penalty time remains at the end of an innings, it is carried forward to the next and subsequent innings of the match.\par}\par

\end{adjustwidth}


\vspace{\baselineskip}
\begin{adjustwidth}{1.18in}{0.0in}
{\fontsize{9pt}{10.8pt}\selectfont 24.2.3.2 \tabto{1.17in} The player shall not be permitted to bat in the match until his team’s batting innings has been in progress for the length of playing time that is equal to the unexpired Penalty time carried forward from the previous innings. However, once his side has lost five wickets in its batting innings, he may bat immediately. If any unexpired penalty time remains at the end of that batting innings, it is carried forward to the next and subsequent innings of the match.\par}\par

\end{adjustwidth}


\vspace{\baselineskip}
\begin{adjustwidth}{0.5in}{0.18in}
{\fontsize{9pt}{10.8pt}\selectfont 24.2.4 \tabto{0.49in} If the player leaves the field before having served all of his Penalty time, the balance is carried forward as unserved Penalty time.\par}\par

\end{adjustwidth}


\vspace{\baselineskip}
\begin{adjustwidth}{0.5in}{0.01in}
{\fontsize{9pt}{10.8pt}\selectfont 24.2.5 \tabto{0.49in} On any occasion of absence, the amount of playing time for which the player is off the field shall be added to any Penalty time that remains unserved, subject to a maximum cumulative Penalty time of 40 minutes, and that player shall not bowl until all of his Penalty time has been served.\par}\par

\end{adjustwidth}


\vspace{\baselineskip}
\begin{adjustwidth}{0.5in}{0.15in}
{\fontsize{9pt}{10.8pt}\selectfont 24.2.6 \tabto{0.49in} For the purposes of clauses and playing time shall comprise the time play is in progress excluding intervals between innings. For clarity, a player’s Penalty time will continue to expire after he is dismissed, for the remainder of his team’s batting innings.\par}\par

\end{adjustwidth}


\vspace{\baselineskip}
{\fontsize{9pt}{10.8pt}\selectfont 24.2.7 \tabto{0.49in} If there is an unscheduled break in play, the stoppage time shall count as Penalty time served, provided that,\par}\par


\vspace{\baselineskip}
\begin{adjustwidth}{1.18in}{0.35in}
{\fontsize{9pt}{10.8pt}\selectfont 24.2.7.1 \tabto{1.17in} the fielder who was on the field of play at the start of the break either takes the field on the resumption of play, or his side is now batting.\par}\par

\end{adjustwidth}


\vspace{\baselineskip}
\begin{adjustwidth}{1.18in}{0.04in}
{\fontsize{9pt}{10.8pt}\selectfont 24.2.7.2 \tabto{1.17in} the fielder who was already off the field at the start of the break notifies an umpire in person as soon as he is able to participate, and either takes the field on the resumption of play, or his side is now batting. Stoppage time before an umpire has been so notified shall not count towards unserved Penalty time.\par}\par

\end{adjustwidth}


\vspace{\baselineskip}
{\fontsize{9pt}{10.8pt}\selectfont 24.2.8 \tabto{0.49in} {\fontsize{8pt}{9.6pt}\selectfont Any unserved Penalty time shall be carried forward into the next innings of the match, as applicable.\par}\par}\par


\vspace{\baselineskip}
{\fontsize{11pt}{13.2pt}\selectfont \textbf{24.3 \tabto{0.47in} }{\fontsize{10pt}{12.0pt}\selectfont \textbf{Penalty time not incurred}\par}\par}\par


\vspace{\baselineskip}
{\fontsize{9pt}{10.8pt}\selectfont A nominated player’s absence will not incur Penalty time if,\par}\par


\vspace{\baselineskip}
\begin{adjustwidth}{0.5in}{0.03in}
{\fontsize{9pt}{10.8pt}\selectfont 24.3.1 \tabto{0.49in} he has suffered an external blow during the match and, as a result, has justifiably left the field or is unable to take the field.\par}\par

\end{adjustwidth}


\vspace{\baselineskip}
\begin{adjustwidth}{0.5in}{0.28in}
{\fontsize{9pt}{10.8pt}\selectfont 24.3.2 \tabto{0.49in} in the opinion of the umpires, the player has been absent or has left the field for other wholly acceptable reasons, which shall not include illness or internal injury.\par}\par

\end{adjustwidth}


\vspace{\baselineskip}
{\fontsize{9pt}{10.8pt}\selectfont 24.3.3 \tabto{0.49in} the player is absent from the field for a period of 8 minutes or less.\par}\par


\vspace{\baselineskip}

\vspace{\baselineskip}

\vspace{\baselineskip}

\vspace{\baselineskip}

\vspace{\baselineskip}

\vspace{\baselineskip}

\vspace{\baselineskip}
\begin{Center}
{\fontsize{8pt}{9.6pt}\selectfont 34\par}
\end{Center}\par


\vspace{\baselineskip}
{\fontsize{11pt}{13.2pt}\selectfont \textbf{24.4 \tabto{0.47in} Player returning without permission}\par}\par


\vspace{\baselineskip}
\begin{adjustwidth}{0.0in}{0.1in}
{\fontsize{9pt}{10.8pt}\selectfont If a player comes on to the field of play in contravention of clause and comes into contact with the ball while it is in play, the ball shall immediately become dead.\par}\par

\end{adjustwidth}


\vspace{\baselineskip}
\begin{itemize}
	\item {\fontsize{9pt}{10.8pt}\selectfont The umpire shall award 5 Penalty runs to the batting side.\par}\par


\vspace{\baselineskip}
	\item {\fontsize{9pt}{10.8pt}\selectfont Runs completed by the batsmen shall be scored together with the run in progress if they had already crossed at the instant of the offence.\par}\par


\vspace{\baselineskip}
	\item {\fontsize{9pt}{10.8pt}\selectfont The ball shall not count as one of the over.\par}\par


\vspace{\baselineskip}
	\item {\fontsize{9pt}{10.8pt}\selectfont The umpire shall inform the other umpire, the captain of the fielding side, the batsmen and, as soon as practicable, the captain of the batting side of the reason for this action.\par}
\end{itemize}\par


\vspace{\baselineskip}
{\fontsize{16pt}{19.2pt}\selectfont \textbf{25 BATSMAN’S INNINGS}\par}\par


\vspace{\baselineskip}
{\fontsize{11pt}{13.2pt}\selectfont \textbf{25.1 \tabto{0.47in} Eligibility to act as a batsman}\par}\par


\vspace{\baselineskip}
\begin{adjustwidth}{0.0in}{0.49in}
{\fontsize{9pt}{10.8pt}\selectfont Only a nominated player may bat and, subject to clause may do so even though a substitute fielder has previously acted for him.\par}\par

\end{adjustwidth}


\vspace{\baselineskip}
{\fontsize{11pt}{13.2pt}\selectfont \textbf{25.2 \tabto{0.47in} Commencement of a batsman’s innings}\par}\par


\vspace{\baselineskip}
\begin{adjustwidth}{0.0in}{0.03in}
{\fontsize{9pt}{10.8pt}\selectfont The innings of the first two batsmen, and that of any new batsman on the resumption of play after a call of Time, shall commence at the call of Play. At any other time, a batsman’s innings shall be considered to have commenced when that batsman first steps onto the field of play.\par}\par

\end{adjustwidth}


\vspace{\baselineskip}
{\fontsize{11pt}{13.2pt}\selectfont \textbf{25.3 \tabto{0.47in} Restriction on batsman commencing an innings}\par}\par


\vspace{\baselineskip}
\begin{adjustwidth}{0.5in}{0.01in}
{\fontsize{9pt}{10.8pt}\selectfont 25.3.1 \tabto{0.49in} If a member of the batting side has unserved Penalty time, (see clause that player shall not be permitted to bat until that Penalty time has been served. However, even if the unserved Penalty time has not expired, that player may bat after his side has lost 5 wickets.\par}\par

\end{adjustwidth}


\vspace{\baselineskip}
\begin{adjustwidth}{0.5in}{0.18in}
{\fontsize{9pt}{10.8pt}\selectfont 25.3.2 \tabto{0.49in} A member of the batting side’s Penalty time is served during Playing time. In the event of an unscheduled stoppage, the stoppage time after the batsman notifies an umpire in person that he is able to participate shall count as Penalty time served.\par}\par

\end{adjustwidth}


\vspace{\baselineskip}
{\fontsize{11pt}{13.2pt}\selectfont \textbf{25.4 \tabto{0.47in} Batsman retiring}\par}\par


\vspace{\baselineskip}
\begin{adjustwidth}{0.5in}{0.01in}
{\fontsize{9pt}{10.8pt}\selectfont 25.4.1 \tabto{0.49in} A batsman may retire at any time during his innings when the ball is dead. The umpires, before allowing play to proceed, shall be informed of the reason for a batsman retiring.\par}\par

\end{adjustwidth}


\vspace{\baselineskip}
\begin{adjustwidth}{0.5in}{0.11in}
{\fontsize{9pt}{10.8pt}\selectfont 25.4.2 \tabto{0.49in} If a batsman retires because of illness, injury or any other unavoidable cause, that batsman is entitled to resume his innings. If for any reason this does not happen, that batsman is to be recorded as ‘Retired - not out’.\par}\par

\end{adjustwidth}


\vspace{\baselineskip}
\begin{adjustwidth}{0.5in}{0.25in}
{\fontsize{9pt}{10.8pt}\selectfont 25.4.3 \tabto{0.49in} If a batsman retires for any reason other than as in clause the innings of that batsman may be resumed only with the consent of the opposing captain. If for any reason his innings is not resumed, that batsman is to be recorded as ‘Retired - out’.\par}\par

\end{adjustwidth}


\vspace{\baselineskip}
\begin{adjustwidth}{0.5in}{0.15in}
{\fontsize{9pt}{10.8pt}\selectfont 25.4.4 \tabto{0.49in} If after retiring a batsman resumes his innings, subject to the requirements of clauses and it shall be only at the fall of a wicket or the retirement of another batsman.\par}\par

\end{adjustwidth}


\vspace{\baselineskip}
{\fontsize{11pt}{13.2pt}\selectfont \textbf{25.5 \tabto{0.47in} Runners}\par}\par


\vspace{\baselineskip}
{\fontsize{9pt}{10.8pt}\selectfont Runners shall not be permitted.\par}\par


\vspace{\baselineskip}

\vspace{\baselineskip}

\vspace{\baselineskip}

\vspace{\baselineskip}

\vspace{\baselineskip}

\vspace{\baselineskip}

\vspace{\baselineskip}

\vspace{\baselineskip}

\vspace{\baselineskip}
\begin{Center}
{\fontsize{8pt}{9.6pt}\selectfont 35\par}
\end{Center}\par


\vspace{\baselineskip}
{\fontsize{16pt}{19.2pt}\selectfont \textbf{26 PRACTICE ON THE FIELD}\par}\par


\vspace{\baselineskip}
{\fontsize{11pt}{13.2pt}\selectfont \textbf{26.1 \tabto{0.47in} Practice on the pitch or the rest of the square}\par}\par


\vspace{\baselineskip}
{\fontsize{9pt}{10.8pt}\selectfont 26.1.1 \tabto{0.49in} {\fontsize{8pt}{9.6pt}\selectfont There shall not be any practice on the pitch at any time.\par}\par}\par


\vspace{\baselineskip}
{\fontsize{9pt}{10.8pt}\selectfont 26.1.2 \tabto{0.49in} {\fontsize{8pt}{9.6pt}\selectfont There shall not be any practice on the rest of the square at any time except with the approval of the umpires.\par}\par}\par


\vspace{\baselineskip}
\begin{adjustwidth}{1.18in}{0.14in}
\begin{justify}
{\fontsize{9pt}{10.8pt}\selectfont 26.1.2.1 \tabto{1.17in} If approved by the umpires, the use of the square for practice on any day of any match will be restricted to any netted practice area or bowling strips specifically prepared on the edge of the square for that purpose.\par}
\end{justify}\par

\end{adjustwidth}


\vspace{\baselineskip}
\begin{adjustwidth}{1.18in}{0.15in}
{\fontsize{9pt}{10.8pt}\selectfont 26.1.2.2 \tabto{1.17in} {\fontsize{8pt}{9.6pt}\selectfont Bowling practice on the bowling strips referred to above shall also be permitted during the interval (and change of innings if not the interval) unless the umpires consider that, in the prevailing conditions of ground and weather, it will be detrimental to the surface of the square.\par}\par}\par

\end{adjustwidth}


\vspace{\baselineskip}
{\fontsize{11pt}{13.2pt}\selectfont \textbf{26.2 \tabto{0.47in} Practice on the outfield}\par}\par


\vspace{\baselineskip}
{\fontsize{9pt}{10.8pt}\selectfont 26.2.1 \tabto{0.49in} {\fontsize{8pt}{9.6pt}\selectfont On any day of the match, all forms of practice are permitted on the outfield\par}\par}\par


\vspace{\baselineskip}
\begin{itemize}
	\item {\fontsize{9pt}{10.8pt}\selectfont before the start of play;\par}\par


\vspace{\baselineskip}
	\item {\fontsize{9pt}{10.8pt}\selectfont after the close of play; and\par}\par


\vspace{\baselineskip}
	\item {\fontsize{9pt}{10.8pt}\selectfont during the interval or between innings\par}
\end{itemize}\par


\vspace{\baselineskip}
\begin{adjustwidth}{0.5in}{0.1in}
{\fontsize{9pt}{10.8pt}\selectfont providing the umpires are satisfied that such practice will not cause significant deterioration in the condition of the outfield.\par}\par

\end{adjustwidth}


\vspace{\baselineskip}
\begin{adjustwidth}{0.5in}{0.07in}
{\fontsize{9pt}{10.8pt}\selectfont 26.2.2 \tabto{0.49in} Between the call of Play and the call of Time, practice shall be permitted on the outfield, providing that all of the following conditions are met:\par}\par

\end{adjustwidth}


\vspace{\baselineskip}
\begin{itemize}
	\item {\fontsize{9pt}{10.8pt}\selectfont only the fielders as defined in paragraph of Appendix A participate in such practice.\par}\par


\vspace{\baselineskip}
	\item {\fontsize{9pt}{10.8pt}\selectfont no ball other than the match ball is used for this practice.\par}\par


\vspace{\baselineskip}
	\item {\fontsize{9pt}{10.8pt}\selectfont no bowling practice takes place in the area between the square and the boundary in a direction parallel to the match pitch.\par}\par


\vspace{\baselineskip}
	\item {\fontsize{9pt}{10.8pt}\selectfont the umpires are satisfied that it will not contravene either of clauses (The match ball changing its condition) or (Time wasting by the fielding side).\par}
\end{itemize}\par


\vspace{\baselineskip}
\begin{adjustwidth}{0.5in}{0.18in}
{\fontsize{9pt}{10.8pt}\selectfont Bowling a ball, using a short run up to a player in the outfield is not to be regarded as bowling practice but shall be subject to the other conditions in this clause.\par}\par

\end{adjustwidth}


\vspace{\baselineskip}
{\fontsize{11pt}{13.2pt}\selectfont \textbf{26.3 \tabto{0.47in} Trial run-up}\par}\par


\vspace{\baselineskip}
\begin{adjustwidth}{0.0in}{0.04in}
{\fontsize{9pt}{10.8pt}\selectfont A bowler is permitted to have a trial run-up provided the umpire is satisfied that it will not contravene either of clauses (Time wasting by the fielding side) or (Fielder damaging the pitch).\par}\par

\end{adjustwidth}


\vspace{\baselineskip}
{\fontsize{11pt}{13.2pt}\selectfont \textbf{26.4 \tabto{0.47in} Penalties for contravention}\par}\par


\vspace{\baselineskip}
\begin{adjustwidth}{0.0in}{0.35in}
{\fontsize{9pt}{10.8pt}\selectfont All forms of practice are subject to the provisions of clauses (The match ball – changing its condition),  (Time wasting by the fielding side) and (Fielder damaging the pitch).\par}\par

\end{adjustwidth}


\vspace{\baselineskip}
{\fontsize{9pt}{10.8pt}\selectfont 26.4.1 \tabto{0.49in} If there is a contravention of any of the provisions of clause or the umpire shall\par}\par


\vspace{\baselineskip}
\begin{itemize}
	\item {\fontsize{9pt}{10.8pt}\selectfont warn the player that the practice is not permitted;\par}\par


\vspace{\baselineskip}
	\item {\fontsize{9pt}{10.8pt}\selectfont inform the other umpire and, as soon as practicable, both captains of the reason for this action.\par}
\end{itemize}\par


\vspace{\baselineskip}
\begin{adjustwidth}{1.18in}{0.15in}
{\fontsize{9pt}{10.8pt}\selectfont 26.4.1.1 \tabto{1.17in} If the contravention is by a batsman at the wicket, the umpire shall inform the other batsman and each incoming batsman that the warning has been issued. The warning shall apply to the team of that player throughout the match.\par}\par

\end{adjustwidth}


\vspace{\baselineskip}

\vspace{\baselineskip}

\vspace{\baselineskip}

\vspace{\baselineskip}
\begin{Center}
{\fontsize{8pt}{9.6pt}\selectfont 36\par}
\end{Center}\par


\vspace{\baselineskip}
{\fontsize{9pt}{10.8pt}\selectfont 26.4.2 \tabto{0.49in} {\fontsize{8pt}{9.6pt}\selectfont If during the match there is any further contravention by any player of that team, the umpire shall\par}\par}\par


\vspace{\baselineskip}
\begin{itemize}
	\item {\fontsize{9pt}{10.8pt}\selectfont award 5 Penalty runs to the opposing side;\par}\par


\vspace{\baselineskip}
	\item {\fontsize{9pt}{10.8pt}\selectfont inform the other umpire, the scorers and, as soon as practicable, both captains, and, if the contravention is during play, the batsmen at the wicket.\par}
\end{itemize}\par


\vspace{\baselineskip}
{\fontsize{16pt}{19.2pt}\selectfont \textbf{27 THE WICKET-KEEPER}\par}\par


\vspace{\baselineskip}
{\fontsize{11pt}{13.2pt}\selectfont \textbf{27.1 \tabto{0.47in} Protective equipment}\par}\par


\vspace{\baselineskip}
\begin{adjustwidth}{0.0in}{0.07in}
{\fontsize{9pt}{10.8pt}\selectfont The wicket-keeper is the only fielder permitted to wear gloves and external leg guards. If these are worn, they are to be regarded as part of his person for the purposes of clause (Fielding the ball). If by the wicket-keeper’s actions and positioning when the ball comes into play it is apparent to the umpires that he will not be able to carry out the normal duties of a wicket-keeper, he shall forfeit this right and also the right to be recognised as a wicket-keeper for the purposes of clauses (A fair catch), (Stumped), (Protective equipment), (Limitation of on-side fielders) and (Fielders not to encroach on pitch).\par}\par

\end{adjustwidth}


\vspace{\baselineskip}
{\fontsize{11pt}{13.2pt}\selectfont \textbf{27.2 \tabto{0.47in} Gloves}\par}\par


\vspace{\baselineskip}
\begin{adjustwidth}{0.5in}{0.04in}
{\fontsize{9pt}{10.8pt}\selectfont 27.2.1 \tabto{0.49in} If, as permitted under clause the wicket-keeper wears gloves, they shall have no webbing between the fingers except joining index finger and thumb, where webbing may be inserted as a means of support.\par}\par

\end{adjustwidth}


\vspace{\baselineskip}
\begin{adjustwidth}{0.5in}{0.39in}
{\fontsize{9pt}{10.8pt}\selectfont 27.2.2 \tabto{0.49in} If used, the webbing shall be a single piece of non-stretch material which, although it may have facing material attached, shall have no reinforcements or tucks.\par}\par

\end{adjustwidth}


\vspace{\baselineskip}
\begin{adjustwidth}{0.5in}{0.11in}
{\fontsize{9pt}{10.8pt}\selectfont 27.2.3 \tabto{0.49in} The top edge of the webbing shall not protrude beyond the straight line joining the top of the index finger to the top of the thumb and shall be taut when a hand wearing the glove has the thumb fully extended. See paragraph of Appendix B.\par}\par

\end{adjustwidth}


\vspace{\baselineskip}
{\fontsize{11pt}{13.2pt}\selectfont \textbf{27.3 \tabto{0.47in} Position of wicket-keeper}\par}\par


\vspace{\baselineskip}
\begin{adjustwidth}{0.5in}{0.01in}
{\fontsize{9pt}{10.8pt}\selectfont 27.3.1 \tabto{0.49in} The wicket-keeper shall remain wholly behind the wicket at the striker’s end from the moment the ball comes into play until a ball delivered by the bowler\par}\par

\end{adjustwidth}


\vspace{\baselineskip}
\begin{adjustwidth}{0.5in}{0.0in}
{\fontsize{9pt}{10.8pt}\selectfont touches the bat or person of the striker; or\par}\par

\end{adjustwidth}


\vspace{\baselineskip}
\begin{adjustwidth}{0.5in}{0.0in}
{\fontsize{9pt}{10.8pt}\selectfont passes the wicket at the striker’s end; or\par}\par

\end{adjustwidth}


\vspace{\baselineskip}
\begin{adjustwidth}{0.5in}{0.0in}
{\fontsize{9pt}{10.8pt}\selectfont the striker attempts a run.\par}\par

\end{adjustwidth}


\vspace{\baselineskip}
\begin{adjustwidth}{0.5in}{0.14in}
{\fontsize{9pt}{10.8pt}\selectfont 27.3.2 \tabto{0.49in} In the event of the wicket-keeper contravening this clause, the striker’s end umpire shall call and signal No ball as soon as applicable after the delivery of the ball.\par}\par

\end{adjustwidth}


\vspace{\baselineskip}
{\fontsize{11pt}{13.2pt}\selectfont \textbf{27.4 \tabto{0.47in} Movement by wicket-keeper}\par}\par


\vspace{\baselineskip}
\begin{adjustwidth}{0.5in}{0.21in}
{\fontsize{9pt}{10.8pt}\selectfont 27.4.1 \tabto{0.49in} After the ball comes into play and before it reaches the striker, it is unfair if the wicket-keeper significantly alters his position in relation to the striker’s wicket, except for the following:\par}\par

\end{adjustwidth}


\vspace{\baselineskip}
\begin{adjustwidth}{1.18in}{0.17in}
{\fontsize{9pt}{10.8pt}\selectfont 27.4.1.1 \tabto{1.17in} movement of a few paces forward for a slower delivery, unless in so doing it brings him within reach of the wicket.\par}\par

\end{adjustwidth}


\vspace{\baselineskip}
\begin{adjustwidth}{0.49in}{0.0in}
{\fontsize{9pt}{10.8pt}\selectfont 27.4.1.2 \tabto{1.17in} {\fontsize{8pt}{9.6pt}\selectfont lateral movement in response to the direction in which the ball has been delivered.\par}\par}\par

\end{adjustwidth}


\vspace{\baselineskip}
\begin{adjustwidth}{1.18in}{0.29in}
{\fontsize{9pt}{10.8pt}\selectfont 27.4.1.3 \tabto{1.17in} movement in response to the stroke that the striker is playing or that his actions suggest he intends to play. However the provisions of clause shall apply.\par}\par

\end{adjustwidth}


\vspace{\baselineskip}
{\fontsize{9pt}{10.8pt}\selectfont 27.4.2 \tabto{0.49in} In the event of unfair movement by the wicket-keeper, either umpire shall call and signal Dead ball.\par}\par


\vspace{\baselineskip}
{\fontsize{11pt}{13.2pt}\selectfont \textbf{27.5 \tabto{0.47in} Restriction on actions of wicket-keeper}\par}\par


\vspace{\baselineskip}
\begin{adjustwidth}{0.0in}{0.12in}
{\fontsize{9pt}{10.8pt}\selectfont If, in the opinion of either umpire, the wicket-keeper interferes with the striker’s right to play the ball and to guard his wicket, clause (Umpire calling and signalling Dead ball) shall apply.\par}\par

\end{adjustwidth}


\vspace{\baselineskip}

\vspace{\baselineskip}

\vspace{\baselineskip}

\vspace{\baselineskip}
\begin{Center}
{\fontsize{8pt}{9.6pt}\selectfont 37\par}
\end{Center}\par


\vspace{\baselineskip}

\vspace{\baselineskip}
\begin{adjustwidth}{0.0in}{0.64in}
{\fontsize{9pt}{10.8pt}\selectfont If, however, either umpire considers that the interference by the wicket-keeper was wilful, then clause  (Deliberate attempt to distract striker) shall also apply.\par}\par

\end{adjustwidth}


\vspace{\baselineskip}
{\fontsize{11pt}{13.2pt}\selectfont \textbf{27.6 \tabto{0.47in} Interference with wicket-keeper by striker}\par}\par


\vspace{\baselineskip}
\begin{adjustwidth}{0.0in}{0.06in}
{\fontsize{9pt}{10.8pt}\selectfont If, in playing at the ball or in the legitimate defence of his wicket, the striker interferes with the wicket-keeper, he shall not be out except as provided for in clause (Obstructing a ball from being caught).\par}\par

\end{adjustwidth}


\vspace{\baselineskip}
{\fontsize{16pt}{19.2pt}\selectfont \textbf{28 THE FIELDER}\par}\par


\vspace{\baselineskip}
{\fontsize{11pt}{13.2pt}\selectfont \textbf{28.1 \tabto{0.47in} Protective equipment}\par}\par


\vspace{\baselineskip}
\begin{adjustwidth}{0.0in}{0.57in}
{\fontsize{9pt}{10.8pt}\selectfont No fielder other than the wicket-keeper shall be permitted to wear gloves or external leg guards. In addition, protection for the hand or fingers may be worn only with the consent of the umpires.\par}\par

\end{adjustwidth}


\vspace{\baselineskip}
{\fontsize{11pt}{13.2pt}\selectfont \textbf{28.2 \tabto{0.47in} Fielding the ball}\par}\par


\vspace{\baselineskip}
{\fontsize{9pt}{10.8pt}\selectfont 28.2.1 \tabto{0.49in} A fielder may field the ball with any part of his person (see paragraph of Appendix A), except as in clause\par}\par


\vspace{\baselineskip}
\begin{adjustwidth}{0.5in}{0.0in}
{\fontsize{9pt}{10.8pt}\selectfont However, he will be deemed to have fielded the ball illegally if, while the ball is in play he wilfully\par}\par

\end{adjustwidth}


\vspace{\baselineskip}
\begin{adjustwidth}{0.49in}{0.0in}
{\fontsize{9pt}{10.8pt}\selectfont 28.2.1.1 \tabto{1.17in} uses anything other than part of his person to field the ball.\par}\par

\end{adjustwidth}


\vspace{\baselineskip}
\begin{adjustwidth}{0.49in}{0.0in}
{\fontsize{9pt}{10.8pt}\selectfont 28.2.1.2 \tabto{1.17in} {\fontsize{8pt}{9.6pt}\selectfont extends his clothing with his hands and uses this to field the ball.\par}\par}\par

\end{adjustwidth}


\vspace{\baselineskip}
\begin{adjustwidth}{1.18in}{0.12in}
{\fontsize{9pt}{10.8pt}\selectfont 28.2.1.3 \tabto{1.17in} discards a piece of clothing, equipment or any other object which subsequently makes contact with the ball.\par}\par

\end{adjustwidth}


\vspace{\baselineskip}
\begin{adjustwidth}{0.5in}{0.01in}
{\fontsize{9pt}{10.8pt}\selectfont 28.2.2 \tabto{0.49in} It is not illegal fielding if the ball in play makes contact with a piece of clothing, equipment or any other object which has accidentally fallen from the fielder’s person.\par}\par

\end{adjustwidth}


\vspace{\baselineskip}
{\fontsize{9pt}{10.8pt}\selectfont 28.2.3 \tabto{0.49in} {\fontsize{8pt}{9.6pt}\selectfont If a fielder illegally fields the ball, the ball shall immediately become dead and\par}\par}\par


\vspace{\baselineskip}
\begin{itemize}
	\item {\fontsize{9pt}{10.8pt}\selectfont the penalty for a No ball or a Wide shall stand.\par}\par


\vspace{\baselineskip}
	\item {\fontsize{9pt}{10.8pt}\selectfont any runs completed by the batsmen shall be credited to the batting side, together with the run in progress if the batsmen had already crossed at the instant of the offence.\par}\par


\vspace{\baselineskip}
	\item {\fontsize{9pt}{10.8pt}\selectfont the ball shall not count as one of the over.\par}
\end{itemize}\par


\vspace{\baselineskip}
\begin{adjustwidth}{0.5in}{0.0in}
{\fontsize{9pt}{10.8pt}\selectfont In addition the umpire shall:\par}\par

\end{adjustwidth}


\vspace{\baselineskip}
\begin{itemize}
	\item {\fontsize{9pt}{10.8pt}\selectfont award 5 Penalty runs to the batting side.\par}\par


\vspace{\baselineskip}
	\item {\fontsize{9pt}{10.8pt}\selectfont inform the other umpire and the captain of the fielding side of the reason for this action.\par}\par


\vspace{\baselineskip}
	\item {\fontsize{9pt}{10.8pt}\selectfont inform the batsmen and, as soon as practicable, the captain of the batting side of what has occurred.\par}
\end{itemize}\par


\vspace{\baselineskip}
{\fontsize{11pt}{13.2pt}\selectfont \textbf{28.3 \tabto{0.47in} Protective helmets belonging to the fielding side}\par}\par


\vspace{\baselineskip}
\begin{adjustwidth}{0.5in}{0.06in}
{\fontsize{9pt}{10.8pt}\selectfont 28.3.1 \tabto{0.49in} Protective helmets, when not in use by fielders, may not be placed on the ground, above the surface except behind the wicket-keeper and in line with both sets of stumps.\par}\par

\end{adjustwidth}


\vspace{\baselineskip}
{\fontsize{9pt}{10.8pt}\selectfont 28.3.2 \tabto{0.49in} If the ball while in play strikes a helmet, placed as described in clause \par}\par


\vspace{\baselineskip}
\begin{adjustwidth}{0.49in}{0.0in}
{\fontsize{9pt}{10.8pt}\selectfont 28.3.2.1 \tabto{1.17in} {\fontsize{8pt}{9.6pt}\selectfont the ball shall become dead\par}\par}\par

\end{adjustwidth}


\vspace{\baselineskip}
\begin{adjustwidth}{1.12in}{0.0in}
{\fontsize{9pt}{10.8pt}\selectfont and, subject to clause \par}\par

\end{adjustwidth}


\vspace{\baselineskip}
\begin{adjustwidth}{0.49in}{0.0in}
{\fontsize{9pt}{10.8pt}\selectfont 28.3.2.2 \tabto{1.17in} an award of 5 Penalty runs shall be made to the batting side;\par}\par

\end{adjustwidth}


\vspace{\baselineskip}
\begin{adjustwidth}{1.18in}{0.01in}
{\fontsize{9pt}{10.8pt}\selectfont 28.3.2.3 \tabto{1.17in} any runs completed by the batsmen before the ball strikes the protective helmet shall be scored, together with the run in progress if the batsmen had already crossed at the instant of the ball striking the protective helmet.\par}\par

\end{adjustwidth}


\vspace{\baselineskip}

\vspace{\baselineskip}

\vspace{\baselineskip}

\vspace{\baselineskip}

\vspace{\baselineskip}

\vspace{\baselineskip}
\begin{Center}
{\fontsize{8pt}{9.6pt}\selectfont 38\par}
\end{Center}\par


\vspace{\baselineskip}

\vspace{\baselineskip}
\begin{adjustwidth}{0.5in}{0.17in}
{\fontsize{9pt}{10.8pt}\selectfont 28.3.3 \tabto{0.49in} If the ball while in play strikes a helmet, placed as described in clause unless the circumstances of clause (Leg byes not to be awarded) or clause (Hit the ball twice), apply, the umpire shall:\par}\par

\end{adjustwidth}


\vspace{\baselineskip}
\begin{itemize}
	\item {\fontsize{9pt}{10.8pt}\selectfont permit the batsmen’s runs as in clause to be scored\par}\par


\vspace{\baselineskip}
	\item {\fontsize{9pt}{10.8pt}\selectfont signal No ball or Wide ball to the scorers if applicable\par}\par


\vspace{\baselineskip}
	\item {\fontsize{9pt}{10.8pt}\selectfont award 5 Penalty runs as in clause \par}\par


\vspace{\baselineskip}
	\item {\fontsize{9pt}{10.8pt}\selectfont award any other Penalty runs due to the batting side.\par}
\end{itemize}\par


\vspace{\baselineskip}
\begin{adjustwidth}{0.5in}{0.32in}
{\fontsize{9pt}{10.8pt}\selectfont 28.3.4 \tabto{0.49in} If the ball while in play strikes a helmet, placed as described in clause and the circumstances of clause (Leg byes not to be awarded) or clause (Hit the ball twice) apply, the umpire shall:\par}\par

\end{adjustwidth}


\vspace{\baselineskip}
\begin{itemize}
	\item {\fontsize{9pt}{10.8pt}\selectfont disallow all runs to the batting side\par}\par


\vspace{\baselineskip}
	\item {\fontsize{9pt}{10.8pt}\selectfont return any not out batsman to his original end\par}\par


\vspace{\baselineskip}
	\item {\fontsize{9pt}{10.8pt}\selectfont signal No ball or Wide ball to the scorers if applicable\par}\par


\vspace{\baselineskip}
	\item {\fontsize{9pt}{10.8pt}\selectfont award any 5-run Penalty that is applicable except for Penalty runs under clause \par}
\end{itemize}\par


\vspace{\baselineskip}
{\fontsize{11pt}{13.2pt}\selectfont \textbf{28.4 \tabto{0.47in} Limitation of on side fielders}\par}\par


\vspace{\baselineskip}
{\fontsize{9pt}{10.8pt}\selectfont 28.4.1 \tabto{0.49in} {\fontsize{8pt}{9.6pt}\selectfont At the instant of delivery, there may not be more than 5 fielders on the leg side.\par}\par}\par


\vspace{\baselineskip}
\begin{adjustwidth}{0.5in}{0.03in}
{\fontsize{9pt}{10.8pt}\selectfont 28.4.2 \tabto{0.49in} At the instant of the bowler’s delivery there shall not be more than two fielders, other than the wicket-keeper, behind the popping crease on the on side. A fielder will be considered to be behind the popping crease unless the whole of his person whether grounded or in the air is in front of this line.\par}\par

\end{adjustwidth}


\vspace{\baselineskip}
\begin{adjustwidth}{0.5in}{0.26in}
{\fontsize{9pt}{10.8pt}\selectfont 28.4.3 \tabto{0.49in} In the event of infringement of this clause by any fielder, the striker’s end umpire shall call and signal No ball.\par}\par

\end{adjustwidth}


\vspace{\baselineskip}
{\fontsize{11pt}{13.2pt}\selectfont \textbf{28.5 \tabto{0.47in} Fielders not to encroach on pitch}\par}\par


\vspace{\baselineskip}
\begin{adjustwidth}{0.0in}{0.03in}
{\fontsize{9pt}{10.8pt}\selectfont While the ball is in play and until the ball has made contact with the striker’s bat or person, or has passed the striker’s bat, no fielder, other than the bowler, may have any part of his person grounded on or extended over the pitch.\par}\par

\end{adjustwidth}


\vspace{\baselineskip}
\begin{adjustwidth}{0.0in}{0.14in}
\begin{justify}
{\fontsize{9pt}{10.8pt}\selectfont In the event of infringement of this clause by any fielder other than the wicket-keeper, the bowler’s end umpire shall call and signal No ball as soon as possible after delivery of the ball. Note, however, clause (Position of wicket-keeper).\par}
\end{justify}\par

\end{adjustwidth}


\vspace{\baselineskip}
{\fontsize{11pt}{13.2pt}\selectfont \textbf{28.6 \tabto{0.47in} Movement by any fielder other than the wicket-keeper}\par}\par


\vspace{\baselineskip}
\begin{adjustwidth}{0.5in}{0.04in}
{\fontsize{9pt}{10.8pt}\selectfont 28.6.1 \tabto{0.49in} Any movement by any fielder, excluding the wicket-keeper, after the ball comes into play and before the ball reaches the striker, is unfair except for the following:\par}\par

\end{adjustwidth}


\vspace{\baselineskip}
\begin{adjustwidth}{0.49in}{0.0in}
{\fontsize{9pt}{10.8pt}\selectfont 28.6.1.1 \tabto{1.17in} minor adjustments to stance or position in relation to the striker’s wicket.\par}\par

\end{adjustwidth}


\vspace{\baselineskip}
\begin{adjustwidth}{1.18in}{0.22in}
{\fontsize{9pt}{10.8pt}\selectfont 28.6.1.2 \tabto{1.17in} movement by any fielder, other than a close fielder, towards the striker or the striker’s wicket that does not significantly alter the position of the fielder.\par}\par

\end{adjustwidth}


\vspace{\baselineskip}
\begin{adjustwidth}{1.18in}{0.15in}
{\fontsize{9pt}{10.8pt}\selectfont 28.6.1.3 \tabto{1.17in} movement by any fielder in response to the stroke that the striker is playing or that his actions suggest he intends to play.\par}\par

\end{adjustwidth}


\vspace{\baselineskip}
{\fontsize{9pt}{10.8pt}\selectfont 28.6.2 \tabto{0.49in} {\fontsize{8pt}{9.6pt}\selectfont In all circumstances clause (Limitation of on side fielders) shall apply.\par}\par}\par


\vspace{\baselineskip}
{\fontsize{9pt}{10.8pt}\selectfont 28.6.3 \tabto{0.49in} {\fontsize{8pt}{9.6pt}\selectfont In the event of such unfair movement, either umpire shall call and signal Dead ball.\par}\par}\par


\vspace{\baselineskip}
\begin{adjustwidth}{0.5in}{0.53in}
{\fontsize{9pt}{10.8pt}\selectfont 28.6.4 \tabto{0.49in} Note also the provisions of clause (Deliberate attempt to distract striker). See also clause  (Movement by wicket-keeper).\par}\par

\end{adjustwidth}


\vspace{\baselineskip}

\vspace{\baselineskip}

\vspace{\baselineskip}

\vspace{\baselineskip}

\vspace{\baselineskip}

\vspace{\baselineskip}

\vspace{\baselineskip}

\vspace{\baselineskip}
\begin{Center}
{\fontsize{8pt}{9.6pt}\selectfont 39\par}
\end{Center}\par


\vspace{\baselineskip}
{\fontsize{11pt}{13.2pt}\selectfont \textbf{28.7 \tabto{0.47in} Restrictions on the placement of fielders}\par}\par


\vspace{\baselineskip}
\begin{adjustwidth}{0.5in}{0.0in}
\begin{justify}
{\fontsize{9pt}{10.8pt}\selectfont 28.7.1 \tabto{0.49in} In addition to the restrictions contained in clause above, further fielding restrictions shall apply to certain overs in each innings. The nature of such fielding restrictions and the overs during which they shall apply are set out in the following paragraphs.\par}
\end{justify}\par

\end{adjustwidth}


\vspace{\baselineskip}
\begin{adjustwidth}{0.5in}{0.19in}
{\fontsize{9pt}{10.8pt}\selectfont 28.7.2 \tabto{0.49in} Subject to below these additional fielding restrictions shall apply to the first 6 overs of each innings (Powerplay overs).\par}\par

\end{adjustwidth}


\vspace{\baselineskip}
\begin{adjustwidth}{0.5in}{0.03in}
{\fontsize{9pt}{10.8pt}\selectfont 28.7.3 \tabto{0.49in} Two semi-circles shall be drawn on the field of play. The semi-circles shall have as their centre the middle stump at either end of the pitch. The radius of each of the semi-circles shall be 30 yards (27.43 metres). The semi-circles shall be linked by two parallel straight lines drawn on the field (see paragraph 2 of Appendix C). These fielding restriction areas should be marked by continuous painted white lines or ‘dots’ at 5 yard (4.57 metres) intervals, each ‘dot’ to be covered by a white plastic or rubber (but not metal) disc measuring 7 inches (18 cm) in diameter.\par}\par

\end{adjustwidth}


\vspace{\baselineskip}
\begin{adjustwidth}{0.5in}{0.21in}
{\fontsize{9pt}{10.8pt}\selectfont 28.7.4 \tabto{0.49in} During the Powerplay overs only two fielders shall be permitted outside this fielding restriction area at the instant of delivery.\par}\par

\end{adjustwidth}


\vspace{\baselineskip}
\begin{adjustwidth}{0.5in}{0.19in}
{\fontsize{9pt}{10.8pt}\selectfont 28.7.5 \tabto{0.49in} During the non Powerplay overs, no more than 5 fielders shall be permitted outside the fielding restriction area referred to in clause 28.7.3 above.\par}\par

\end{adjustwidth}


\vspace{\baselineskip}
\begin{adjustwidth}{0.5in}{0.07in}
\begin{justify}
{\fontsize{9pt}{10.8pt}\selectfont 28.7.6 \tabto{0.49in} In circumstances when the number of overs of the batting team is reduced, the number of Powerplay overs shall be reduced in accordance with the table below. For the sake of clarity, it should be noted that the table shall apply to both the 1st and 2nd innings of the match.\par}
\end{justify}\par

\end{adjustwidth}


\vspace{\baselineskip}


%%%%%%%%%%%%%%%%%%%% Table No: 1 starts here %%%%%%%%%%%%%%%%%%%%


\begin{table}[H]
 			\centering
\begin{tabular}{p{1in}p{1in}p{1in}p{1in}p{1in}p{1in}}
%row no:1
\multicolumn{1}{p{1in}}{{\fontsize{9pt}{10.8pt}\selectfont Total overs in innings}} & 
\multicolumn{1}{p{1in}}{{\fontsize{9pt}{10.8pt}\selectfont Number of overs for which fielding restrictions in clauses 28.7.2 and}} & 

\hhline{~~}
%row no:2
\multicolumn{1}{p{1in}}{} & 
\multicolumn{1}{p{1in}}{{\fontsize{9pt}{10.8pt}\selectfont 28.7.4 above will apply}} & 

\hhline{~~}
%row no:3
\multicolumn{1}{p{1in}}{{\fontsize{9pt}{10.8pt}\selectfont 5-8}} & 
\multicolumn{1}{p{1in}}{{\fontsize{9pt}{10.8pt}\selectfont 2}} & 

\hhline{~~}
%row no:4
\multicolumn{1}{p{1in}}{{\fontsize{9pt}{10.8pt}\selectfont 9-11}} & 
\multicolumn{1}{p{1in}}{{\fontsize{9pt}{10.8pt}\selectfont 3}} & 

\hhline{~~}
%row no:5
\multicolumn{1}{p{1in}}{{\fontsize{9pt}{10.8pt}\selectfont 12-14}} & 
\multicolumn{1}{p{1in}}{{\fontsize{9pt}{10.8pt}\selectfont 4}} & 

\hhline{~~}
%row no:6
\multicolumn{1}{p{1in}}{{\fontsize{9pt}{10.8pt}\selectfont 15-18}} & 
\multicolumn{1}{p{1in}}{{\fontsize{9pt}{10.8pt}\selectfont 5}} & 

\hhline{~~}
%row no:7
\multicolumn{1}{p{1in}}{{\fontsize{9pt}{10.8pt}\selectfont 19-20}} & 
\multicolumn{1}{p{1in}}{{\fontsize{9pt}{10.8pt}\selectfont 6}} & 

\hhline{~~}

\end{tabular}
 \end{table}


%%%%%%%%%%%%%%%%%%%% Table No: 1 ends here %%%%%%%%%%%%%%%%%%%%


\vspace{\baselineskip}
\begin{adjustwidth}{0.5in}{0.01in}
{\fontsize{9pt}{10.8pt}\selectfont 28.7.7 \tabto{0.49in} If an innings is interrupted during an over and if on the resumption of play, due to the reduced number of overs of the batting team, the required number of Powerplay overs have already been bowled, the remaining deliveries in the over to be completed shall not be subject to the fielding restrictions.\par}\par

\end{adjustwidth}


\vspace{\baselineskip}
\begin{adjustwidth}{0.5in}{0.11in}
{\fontsize{9pt}{10.8pt}\selectfont 28.7.8 \tabto{0.49in} In the event of an infringement of any of the above fielding restrictions, the square leg umpire shall call and signal No ball.\par}\par

\end{adjustwidth}


\vspace{\baselineskip}
{\fontsize{16pt}{19.2pt}\selectfont \textbf{29 THE WICKET IS DOWN}\par}\par


\vspace{\baselineskip}
{\fontsize{11pt}{13.2pt}\selectfont \textbf{29.1 \tabto{0.47in} Wicket put down}\par}\par


\vspace{\baselineskip}
\begin{adjustwidth}{0.5in}{0.04in}
{\fontsize{9pt}{10.8pt}\selectfont 29.1.1 \tabto{0.49in} The wicket is put down if a bail is completely removed from the top of the stumps, or a stump is struck out of the ground,\par}\par

\end{adjustwidth}


\vspace{\baselineskip}
\begin{adjustwidth}{0.49in}{0.0in}
{\fontsize{9pt}{10.8pt}\selectfont 29.1.1.1 \tabto{1.17in} by the ball,\par}\par

\end{adjustwidth}


\vspace{\baselineskip}
\begin{adjustwidth}{0.49in}{0.0in}
{\fontsize{9pt}{10.8pt}\selectfont 29.1.1.2 \tabto{1.17in} {\fontsize{8pt}{9.6pt}\selectfont by the striker’s bat if held or by any part of the bat that he is holding,\par}\par}\par

\end{adjustwidth}


\vspace{\baselineskip}
\begin{adjustwidth}{1.18in}{0.32in}
{\fontsize{9pt}{10.8pt}\selectfont 29.1.1.3 \tabto{1.17in} for the purpose of this clause only, by the striker's bat not in hand, or by any part of the bat which has become detached,\par}\par

\end{adjustwidth}


\vspace{\baselineskip}
\begin{adjustwidth}{1.18in}{0.12in}
{\fontsize{9pt}{10.8pt}\selectfont 29.1.1.4 \tabto{1.17in} by the striker’s person or by any part of his clothing or equipment becoming detached from his person,\par}\par

\end{adjustwidth}


\vspace{\baselineskip}
\begin{adjustwidth}{1.18in}{0.07in}
{\fontsize{9pt}{10.8pt}\selectfont 29.1.1.5 \tabto{1.17in} by a fielder with his hand or arm, providing that the ball is held in the hand or hands so used, or in the hand of the arm so used.\par}\par

\end{adjustwidth}


\vspace{\baselineskip}

\vspace{\baselineskip}

\vspace{\baselineskip}

\vspace{\baselineskip}

\vspace{\baselineskip}
\begin{Center}
{\fontsize{8pt}{9.6pt}\selectfont 40\par}
\end{Center}\par


\vspace{\baselineskip}

\vspace{\baselineskip}
\begin{adjustwidth}{1.18in}{0.24in}
{\fontsize{9pt}{10.8pt}\selectfont 29.1.1.6 \tabto{1.17in} The wicket is also put down if a fielder strikes or pulls a stump out of the ground in the same manner.\par}\par

\end{adjustwidth}


\vspace{\baselineskip}
\begin{adjustwidth}{0.5in}{0.04in}
{\fontsize{9pt}{10.8pt}\selectfont 29.1.2 \tabto{0.49in} The disturbance of a bail, whether temporary or not, shall not constitute its complete removal from the top of the stumps, but if a bail in falling lodges between two of the stumps this shall be regarded as complete removal.\par}\par

\end{adjustwidth}


\vspace{\baselineskip}
{\fontsize{11pt}{13.2pt}\selectfont \textbf{29.2 \tabto{0.47in} One bail off}\par}\par


\vspace{\baselineskip}
\begin{adjustwidth}{0.0in}{0.29in}
{\fontsize{9pt}{10.8pt}\selectfont If one bail is off, it shall be sufficient for the purpose of putting the wicket down to remove the remaining bail or to strike or pull any of the three stumps out of the ground, in any of the ways stated in clause \par}\par

\end{adjustwidth}


\vspace{\baselineskip}
{\fontsize{11pt}{13.2pt}\selectfont \textbf{29.3 \tabto{0.47in} Remaking wicket}\par}\par


\vspace{\baselineskip}
\begin{adjustwidth}{0.0in}{0.24in}
{\fontsize{9pt}{10.8pt}\selectfont If a wicket is broken or put down while the ball is in play, it shall not be remade by an umpire until the ball is dead. See clause (Dead ball). Any fielder may, however, while the ball is in play,\par}\par

\end{adjustwidth}


\vspace{\baselineskip}
\begin{itemize}
	\item {\fontsize{9pt}{10.8pt}\selectfont replace a bail or bails on top of the stumps.\par}\par


\vspace{\baselineskip}
	\item {\fontsize{9pt}{10.8pt}\selectfont put back one or more stumps into the ground where the wicket originally stood.\par}
\end{itemize}\par


\vspace{\baselineskip}
{\fontsize{11pt}{13.2pt}\selectfont \textbf{29.4 \tabto{0.47in} Dispensing with bails}\par}\par


\vspace{\baselineskip}
\begin{adjustwidth}{0.0in}{0.18in}
{\fontsize{9pt}{10.8pt}\selectfont If the umpires have agreed to dispense with bails in accordance with clause (Dispensing with bails), it is for the umpire concerned to decide whether or not the wicket has been put down.\par}\par

\end{adjustwidth}


\vspace{\baselineskip}
\begin{adjustwidth}{0.5in}{0.1in}
{\fontsize{9pt}{10.8pt}\selectfont 29.4.1 \tabto{0.49in} After a decision to play without bails, the wicket has been put down if the umpire concerned is satisfied that the wicket has been struck by the ball, by the striker’s bat, person or items of his clothing or equipment as described in clauses or or by a fielder in the manner described in clause 29.1.1.5.\par}\par

\end{adjustwidth}


\vspace{\baselineskip}
\begin{adjustwidth}{0.5in}{0.07in}
\begin{justify}
{\fontsize{9pt}{10.8pt}\selectfont 29.4.2 \tabto{0.49in} If the wicket has already been broken or put down, clause shall apply to any stump or stumps still in the ground. Any fielder may replace a stump or stumps, in accordance with clause in order to have an opportunity of putting the wicket down.\par}
\end{justify}\par

\end{adjustwidth}


\vspace{\baselineskip}
{\fontsize{16pt}{19.2pt}\selectfont \textbf{30 BATSMAN OUT OF HIS GROUND}\par}\par


\vspace{\baselineskip}
{\fontsize{11pt}{13.2pt}\selectfont \textbf{30.1 \tabto{0.47in} When out of his ground}\par}\par


\vspace{\baselineskip}
\begin{adjustwidth}{0.5in}{0.24in}
{\fontsize{9pt}{10.8pt}\selectfont 30.1.1 \tabto{0.49in} A batsman shall be considered to be out of his ground unless some part of his person or bat is grounded behind the popping crease at that end.\par}\par

\end{adjustwidth}


\vspace{\baselineskip}
\begin{adjustwidth}{0.5in}{0.08in}
{\fontsize{9pt}{10.8pt}\selectfont 30.1.2 \tabto{0.49in} However, a batsman shall not be considered to be out of his ground if, in running or diving towards his ground and beyond, and having grounded some part of his person or bat beyond the popping crease, there is subsequent loss of contact\par}\par

\end{adjustwidth}


\vspace{\baselineskip}
\begin{adjustwidth}{0.5in}{0.0in}
{\fontsize{9pt}{10.8pt}\selectfont between the ground and any part of his person or bat, or\par}\par

\end{adjustwidth}


\vspace{\baselineskip}
\begin{adjustwidth}{0.5in}{0.0in}
{\fontsize{9pt}{10.8pt}\selectfont between the bat and person,\par}\par

\end{adjustwidth}


\vspace{\baselineskip}
\begin{adjustwidth}{0.5in}{0.0in}
{\fontsize{9pt}{10.8pt}\selectfont provided that the batsman has continued movement in the same direction.\par}\par

\end{adjustwidth}


\vspace{\baselineskip}
{\fontsize{11pt}{13.2pt}\selectfont \textbf{30.2 \tabto{0.47in} Which is a batsman’s ground}\par}\par


\vspace{\baselineskip}
\begin{adjustwidth}{0.5in}{0.15in}
{\fontsize{9pt}{10.8pt}\selectfont 30.2.1 \tabto{0.49in} If only one batsman is within a ground, it is his ground and will remain so even if he is later joined there by the other batsman.\par}\par

\end{adjustwidth}


\vspace{\baselineskip}
\begin{adjustwidth}{0.5in}{0.12in}
{\fontsize{9pt}{10.8pt}\selectfont 30.2.2 \tabto{0.49in} If both batsmen are in the same ground and one of them subsequently leaves it, the ground belongs to the batsman who remains in it.\par}\par

\end{adjustwidth}


\vspace{\baselineskip}
\begin{adjustwidth}{0.5in}{0.06in}
{\fontsize{9pt}{10.8pt}\selectfont 30.2.3 \tabto{0.49in} If there is no batsman in either ground, then each ground belongs to whichever batsman is nearer to it, or, if the batsmen are level, to whichever batsman was nearer to it immediately prior to their drawing level.\par}\par

\end{adjustwidth}


\vspace{\baselineskip}

\vspace{\baselineskip}

\vspace{\baselineskip}

\vspace{\baselineskip}

\vspace{\baselineskip}
\begin{Center}
{\fontsize{8pt}{9.6pt}\selectfont 41\par}
\end{Center}\par


\vspace{\baselineskip}

\vspace{\baselineskip}
\begin{adjustwidth}{0.5in}{0.11in}
{\fontsize{9pt}{10.8pt}\selectfont 30.2.4 \tabto{0.49in} If a ground belongs to one batsman then the other ground belongs to the other batsman, irrespective of his position.\par}\par

\end{adjustwidth}


\vspace{\baselineskip}
{\fontsize{11pt}{13.2pt}\selectfont \textbf{30.3 \tabto{0.47in} Position of non-striker}\par}\par


\vspace{\baselineskip}
\begin{adjustwidth}{0.0in}{0.22in}
{\fontsize{9pt}{10.8pt}\selectfont The non-striker, when standing at the bowler’s end, should be positioned on the opposite side of the wicket to that from which the ball is being delivered, unless a request to do otherwise is granted by the umpire.\par}\par

\end{adjustwidth}


\vspace{\baselineskip}
{\fontsize{16pt}{19.2pt}\selectfont \textbf{31 APPEALS}\par}\par


\vspace{\baselineskip}
{\fontsize{11pt}{13.2pt}\selectfont \textbf{31.1 \tabto{0.47in} }{\fontsize{10pt}{12.0pt}\selectfont \textbf{Umpire not to give batsman out without an appeal}\par}\par}\par


\vspace{\baselineskip}
\begin{adjustwidth}{0.0in}{0.12in}
{\fontsize{9pt}{10.8pt}\selectfont Neither umpire shall give a batsman out, even though he may be out under these Playing Conditions, unless appealed to by a fielder. This shall not debar a batsman who is out under these Playing Conditions from leaving the wicket without an appeal having been made. Note, however, the provisions of clause \par}\par

\end{adjustwidth}


\vspace{\baselineskip}
{\fontsize{11pt}{13.2pt}\selectfont \textbf{31.2 \tabto{0.47in} Batsman dismissed}\par}\par


\vspace{\baselineskip}
{\fontsize{9pt}{10.8pt}\selectfont A batsman is dismissed if he is\par}\par


\vspace{\baselineskip}
{\fontsize{9pt}{10.8pt}\selectfont either given out by an umpire, on appeal\par}\par


\vspace{\baselineskip}
{\fontsize{9pt}{10.8pt}\selectfont or out under these Playing Conditions and leaves the wicket as in clause \par}\par


\vspace{\baselineskip}
{\fontsize{11pt}{13.2pt}\selectfont \textbf{31.3 \tabto{0.47in} Timing of appeals}\par}\par


\vspace{\baselineskip}
\begin{adjustwidth}{0.0in}{0.19in}
{\fontsize{9pt}{10.8pt}\selectfont For an appeal to be valid, it must be made before the bowler begins his run-up or, if there is no run-up, his bowling action to deliver the next ball, and before Time has been called.\par}\par

\end{adjustwidth}


\vspace{\baselineskip}
\begin{adjustwidth}{0.0in}{0.25in}
{\fontsize{9pt}{10.8pt}\selectfont The call of Over does not invalidate an appeal made prior to the start of the following over, provided Time has not been called. See clauses (Call of Time) and (Start of an over).\par}\par

\end{adjustwidth}


\vspace{\baselineskip}
{\fontsize{11pt}{13.2pt}\selectfont \textbf{31.4 \tabto{0.47in} Appeal $``$How’s That?$"$ }\par}\par


\vspace{\baselineskip}
{\fontsize{9pt}{10.8pt}\selectfont An appeal $``$How’s That?$"$  covers all ways of being out.\par}\par


\vspace{\baselineskip}
{\fontsize{11pt}{13.2pt}\selectfont \textbf{31.5 \tabto{0.47in} Answering appeals}\par}\par


\vspace{\baselineskip}
\begin{adjustwidth}{0.0in}{0.24in}
{\fontsize{9pt}{10.8pt}\selectfont The striker’s end umpire shall answer all appeals arising out of any of clauses (Hit wicket), (Stumped) or  (Run out) when this occurs at the wicket-keeper’s end. The bowler’s end umpire shall answer all other appeals.\par}\par

\end{adjustwidth}


\vspace{\baselineskip}
{\fontsize{9pt}{10.8pt}\selectfont When an appeal is made, each umpire shall answer on any matter that falls within his jurisdiction.\par}\par


\vspace{\baselineskip}
\begin{adjustwidth}{0.0in}{0.29in}
{\fontsize{9pt}{10.8pt}\selectfont When a batsman has been given Not out, either umpire may answer an appeal, made in accordance with clause if it is on a further matter and is within his jurisdiction.\par}\par

\end{adjustwidth}


\vspace{\baselineskip}
{\fontsize{11pt}{13.2pt}\selectfont \textbf{31.6 \tabto{0.47in} Consultation by umpires}\par}\par


\vspace{\baselineskip}
\begin{adjustwidth}{0.0in}{0.07in}
{\fontsize{9pt}{10.8pt}\selectfont Each umpire shall answer appeals on matters within his own jurisdiction. If an umpire is doubtful about any point that the other umpire may have been in a better position to see, he/she shall consult the latter on this point of fact and shall then give the decision. If, after consultation, there is still doubt remaining, the decision shall be Not out.\par}\par

\end{adjustwidth}


\vspace{\baselineskip}
{\fontsize{11pt}{13.2pt}\selectfont \textbf{31.7 \tabto{0.47in} Batsman leaving the wicket under a misapprehension}\par}\par


\vspace{\baselineskip}
\begin{adjustwidth}{0.0in}{0.07in}
{\fontsize{9pt}{10.8pt}\selectfont An umpire shall intervene if satisfied that a batsman, not having been given out, has left the wicket under a misapprehension of being out. The umpire intervening shall call and signal Dead ball to prevent any further action by the fielding side and shall recall the batsman.\par}\par

\end{adjustwidth}


\vspace{\baselineskip}
\begin{adjustwidth}{0.0in}{0.08in}
{\fontsize{9pt}{10.8pt}\selectfont A batsman may be recalled at any time up to the instant when the ball comes into play for the next delivery, unless it is the final wicket of the innings, in which case it should be up to the instant when the umpires leave the field.\par}\par

\end{adjustwidth}


\vspace{\baselineskip}

\vspace{\baselineskip}

\vspace{\baselineskip}

\vspace{\baselineskip}

\vspace{\baselineskip}

\vspace{\baselineskip}

\vspace{\baselineskip}
\begin{Center}
{\fontsize{8pt}{9.6pt}\selectfont 42\par}
\end{Center}\par


\vspace{\baselineskip}
{\fontsize{11pt}{13.2pt}\selectfont \textbf{31.8 \tabto{0.47in} Withdrawal of an appeal}\par}\par


\vspace{\baselineskip}
\begin{adjustwidth}{0.0in}{0.14in}
{\fontsize{9pt}{10.8pt}\selectfont The captain of the fielding side may withdraw an appeal only after obtaining the consent of the umpire within whose jurisdiction the appeal falls. If such consent is given, the umpire concerned shall, if applicable, revoke the decision and recall the batsman.\par}\par

\end{adjustwidth}


\vspace{\baselineskip}
\begin{adjustwidth}{0.0in}{0.17in}
{\fontsize{9pt}{10.8pt}\selectfont The withdrawal of an appeal must be before the instant when the ball comes into play for the next delivery or, if the innings has been completed, the instant when the umpires leave the field.\par}\par

\end{adjustwidth}


\vspace{\baselineskip}
{\fontsize{16pt}{19.2pt}\selectfont \textbf{32 BOWLED}\par}\par


\vspace{\baselineskip}
{\fontsize{11pt}{13.2pt}\selectfont \textbf{32.1 \tabto{0.47in} Out Bowled}\par}\par


\vspace{\baselineskip}
\begin{adjustwidth}{0.5in}{0.01in}
\begin{justify}
{\fontsize{9pt}{10.8pt}\selectfont 32.1.1 \tabto{0.49in} The striker is out Bowled if his wicket is put down by a ball delivered by the bowler, not being a No ball, even if it first touches the striker’s bat or person.\par}
\end{justify}\par

\end{adjustwidth}


\vspace{\baselineskip}
\begin{adjustwidth}{0.5in}{0.0in}
\begin{justify}
{\fontsize{9pt}{10.8pt}\selectfont 32.1.2 \tabto{0.49in} However, the striker shall not be out Bowled if before striking the wicket the ball has been in contact with any other player or an umpire. The striker will, however, be subject to clauses (Obstructing the field), 38 (Run out) and 39 (Stumped).\par}
\end{justify}\par

\end{adjustwidth}


\vspace{\baselineskip}
{\fontsize{11pt}{13.2pt}\selectfont \textbf{32.2 \tabto{0.47in} Bowled to take precedence}\par}\par


\vspace{\baselineskip}
\begin{adjustwidth}{0.0in}{0.06in}
{\fontsize{9pt}{10.8pt}\selectfont The striker is out Bowled if his wicket is put down as in clause even though a decision against him for any other method of dismissal would be justified.\par}\par

\end{adjustwidth}


\vspace{\baselineskip}
{\fontsize{16pt}{19.2pt}\selectfont \textbf{33 CAUGHT}\par}\par


\vspace{\baselineskip}
{\fontsize{11pt}{13.2pt}\selectfont \textbf{33.1 \tabto{0.47in} Out Caught}\par}\par


\vspace{\baselineskip}
\begin{adjustwidth}{0.0in}{0.38in}
{\fontsize{9pt}{10.8pt}\selectfont The striker is out Caught if a ball delivered by the bowler, not being a No ball, touches his bat without having previously been in contact with any fielder, and is subsequently held by a fielder as a fair catch, as described in clause and before it touches the ground.\par}\par

\end{adjustwidth}


\vspace{\baselineskip}
{\fontsize{11pt}{13.2pt}\selectfont \textbf{33.2 \tabto{0.47in} A fair catch}\par}\par


\vspace{\baselineskip}
{\fontsize{9pt}{10.8pt}\selectfont 33.2.1 \tabto{0.49in} {\fontsize{8pt}{9.6pt}\selectfont A catch will be fair only if, in every case\par}\par}\par


\vspace{\baselineskip}
\begin{adjustwidth}{0.5in}{0.0in}
{\fontsize{9pt}{10.8pt}\selectfont either the ball, at any time\par}\par

\end{adjustwidth}


\vspace{\baselineskip}
\begin{adjustwidth}{0.5in}{0.0in}
{\fontsize{9pt}{10.8pt}\selectfont or any fielder in contact with the ball,\par}\par

\end{adjustwidth}


\vspace{\baselineskip}
\begin{adjustwidth}{0.5in}{0.35in}
{\fontsize{9pt}{10.8pt}\selectfont is not grounded beyond the boundary before the catch is completed. Note clauses (Ball grounded beyond the boundary) and 19.5 (Fielder grounded beyond the boundary).\par}\par

\end{adjustwidth}


\vspace{\baselineskip}
{\fontsize{9pt}{10.8pt}\selectfont 33.2.2 \tabto{0.49in} {\fontsize{8pt}{9.6pt}\selectfont Furthermore, a catch will be fair if any of the following conditions applies:\par}\par}\par


\vspace{\baselineskip}
\begin{adjustwidth}{1.18in}{0.06in}
{\fontsize{9pt}{10.8pt}\selectfont 33.2.2.1 \tabto{1.17in} the ball is held in the hand or hands of a fielder, even if the hand holding the ball is touching the ground, or is hugged to the body, or lodges in the external protective equipment worn by a fielder, or lodges accidentally in a fielder’s clothing.\par}\par

\end{adjustwidth}


\vspace{\baselineskip}
\begin{adjustwidth}{1.18in}{0.07in}
{\fontsize{9pt}{10.8pt}\selectfont 33.2.2.2 \tabto{1.17in} a fielder catches the ball after it has been lawfully struck more than once by the striker, but only if it has not been grounded since it was first struck. See clause (Hit the ball twice).\par}\par

\end{adjustwidth}


\vspace{\baselineskip}
\begin{adjustwidth}{1.18in}{0.11in}
{\fontsize{9pt}{10.8pt}\selectfont 33.2.2.3 \tabto{1.17in} a fielder catches the ball after it has touched the wicket, an umpire, another fielder or the other batsman.\par}\par

\end{adjustwidth}


\vspace{\baselineskip}
\begin{adjustwidth}{1.18in}{0.01in}
{\fontsize{9pt}{10.8pt}\selectfont 33.2.2.4 \tabto{1.17in} a fielder catches the ball after it has crossed the boundary in the air, provided that the conditions in clause are met.\par}\par

\end{adjustwidth}


\vspace{\baselineskip}
\begin{adjustwidth}{1.18in}{0.01in}
{\fontsize{9pt}{10.8pt}\selectfont 33.2.2.5 \tabto{1.17in} the ball is caught off an obstruction within the boundary that is not designated a boundary by the umpires.\par}\par

\end{adjustwidth}


\vspace{\baselineskip}

\vspace{\baselineskip}

\vspace{\baselineskip}

\vspace{\baselineskip}

\vspace{\baselineskip}
\begin{Center}
{\fontsize{8pt}{9.6pt}\selectfont 43\par}
\end{Center}\par


\vspace{\baselineskip}
{\fontsize{11pt}{13.2pt}\selectfont \textbf{33.3 \tabto{0.47in} Making a catch}\par}\par


\vspace{\baselineskip}
\begin{adjustwidth}{0.0in}{0.17in}
{\fontsize{9pt}{10.8pt}\selectfont The act of making a catch shall start from the time when the ball first comes into contact with a fielder’s person and shall end when a fielder obtains complete control over both the ball and his own movement.\par}\par

\end{adjustwidth}


\vspace{\baselineskip}
{\fontsize{11pt}{13.2pt}\selectfont \textbf{33.4 \tabto{0.47in} No runs to be scored}\par}\par


\vspace{\baselineskip}
\begin{adjustwidth}{0.0in}{0.31in}
\begin{justify}
{\fontsize{9pt}{10.8pt}\selectfont If the striker is dismissed Caught, runs from that delivery completed by the batsmen before the completion of the catch shall not be scored but any runs for penalties awarded to either side shall stand. Clause (Batsman returning to original end) shall apply from the instant of the completion of the catch.\par}
\end{justify}\par

\end{adjustwidth}


\vspace{\baselineskip}
{\fontsize{11pt}{13.2pt}\selectfont \textbf{33.5 \tabto{0.47in} Caught to take precedence}\par}\par


\vspace{\baselineskip}
\begin{adjustwidth}{0.0in}{0.11in}
{\fontsize{9pt}{10.8pt}\selectfont If the criteria of clause are met and the striker is not out Bowled, then he is out Caught, even though a decision against either batsman for another method of dismissal would be justified.\par}\par

\end{adjustwidth}


\vspace{\baselineskip}
{\fontsize{16pt}{19.2pt}\selectfont \textbf{34 HIT THE BALL TWICE}\par}\par


\vspace{\baselineskip}
{\fontsize{11pt}{13.2pt}\selectfont \textbf{34.1 \tabto{0.47in} Out Hit the ball twice}\par}\par


\vspace{\baselineskip}
\begin{adjustwidth}{0.5in}{0.0in}
{\fontsize{9pt}{10.8pt}\selectfont 34.1.1 \tabto{0.49in} The striker is out Hit the ball twice if, while the ball is in play, it strikes any part of his person or is struck by his bat and, before the ball has been touched by a fielder, the striker wilfully strikes it again with his bat or person, other than a hand not holding the bat, except for the sole purpose of guarding his wicket. See clause and clause (Obstructing the field).\par}\par

\end{adjustwidth}


\vspace{\baselineskip}
{\fontsize{9pt}{10.8pt}\selectfont 34.1.2 \tabto{0.49in} {\fontsize{8pt}{9.6pt}\selectfont For the purpose of this clause ‘struck’ or ‘strike’ shall include contact with the person of the striker.\par}\par}\par


\vspace{\baselineskip}
{\fontsize{11pt}{13.2pt}\selectfont \textbf{34.2 \tabto{0.47in} Not out Hit the ball twice}\par}\par


\vspace{\baselineskip}
{\fontsize{9pt}{10.8pt}\selectfont The striker will not be out under this clause if he\par}\par


\vspace{\baselineskip}
{\fontsize{9pt}{10.8pt}\selectfont 34.2.1 \tabto{0.49in} {\fontsize{8pt}{9.6pt}\selectfont strikes the ball a second or subsequent time in order to return the ball to any fielder.\par}\par}\par


\vspace{\baselineskip}
\begin{adjustwidth}{0.5in}{0.0in}
{\fontsize{9pt}{10.8pt}\selectfont Note, however, the provisions of clause (Returning the ball to a fielder).\par}\par

\end{adjustwidth}


\vspace{\baselineskip}
\begin{adjustwidth}{0.5in}{0.44in}
{\fontsize{9pt}{10.8pt}\selectfont 34.2.2 \tabto{0.49in} wilfully strikes the ball after it has touched a fielder. Note, however the provisions of clause (Out Obstructing the field).\par}\par

\end{adjustwidth}


\vspace{\baselineskip}
{\fontsize{11pt}{13.2pt}\selectfont \textbf{34.3 \tabto{0.47in} Ball lawfully struck more than once}\par}\par


\vspace{\baselineskip}
\begin{adjustwidth}{0.0in}{0.17in}
{\fontsize{9pt}{10.8pt}\selectfont The striker may, solely in order to guard his wicket and before the ball has been touched by a fielder, lawfully strike the ball a second or subsequent time with the bat, or with any part of his person other than a hand not holding the bat.\par}\par

\end{adjustwidth}


\vspace{\baselineskip}
\begin{adjustwidth}{0.0in}{0.1in}
{\fontsize{9pt}{10.8pt}\selectfont However, the striker may not prevent the ball from being caught by striking the ball more than once in defence of his wicket. See clause (Obstructing a ball from being caught).\par}\par

\end{adjustwidth}


\vspace{\baselineskip}
{\fontsize{11pt}{13.2pt}\selectfont \textbf{34.4 \tabto{0.47in} Runs permitted from ball lawfully struck more than once}\par}\par


\vspace{\baselineskip}
\begin{adjustwidth}{0.0in}{0.04in}
{\fontsize{9pt}{10.8pt}\selectfont When the ball is lawfully struck more than once, as permitted in clause if the ball does not become dead for any reason, the umpire shall call and signal Dead ball as soon as the ball reaches the boundary or at the completion of the first run. However, the umpire shall delay the call of Dead ball to allow the opportunity for a catch to be completed.\par}\par

\end{adjustwidth}


\vspace{\baselineskip}
{\fontsize{9pt}{10.8pt}\selectfont The umpire shall\par}\par


\vspace{\baselineskip}
\begin{itemize}
	\item {\fontsize{9pt}{10.8pt}\selectfont disallow all runs to the batting side\par}\par


\vspace{\baselineskip}
	\item {\fontsize{9pt}{10.8pt}\selectfont return any not out batsman to his original end\par}\par


\vspace{\baselineskip}
	\item {\fontsize{9pt}{10.8pt}\selectfont signal No ball to the scorers if applicable; and\par}
\end{itemize}\par


\vspace{\baselineskip}

\vspace{\baselineskip}

\vspace{\baselineskip}

\vspace{\baselineskip}

\vspace{\baselineskip}

\vspace{\baselineskip}
\begin{Center}
{\fontsize{8pt}{9.6pt}\selectfont 44\par}
\end{Center}\par


\vspace{\baselineskip}

\vspace{\baselineskip}
\begin{itemize}
	\item {\fontsize{9pt}{10.8pt}\selectfont award any 5-run Penalty that is applicable except for Penalty runs under clause (Protective helmets belonging to the fielding side).\par}
\end{itemize}\par


\vspace{\baselineskip}
{\fontsize{11pt}{13.2pt}\selectfont \textbf{34.5 \tabto{0.47in} Bowler does not get credit}\par}\par


\vspace{\baselineskip}
{\fontsize{9pt}{10.8pt}\selectfont The bowler does not get credit for the wicket.\par}\par


\vspace{\baselineskip}
{\fontsize{16pt}{19.2pt}\selectfont \textbf{35 HIT WICKET}\par}\par


\vspace{\baselineskip}
{\fontsize{11pt}{13.2pt}\selectfont \textbf{35.1 \tabto{0.47in} Out Hit wicket}\par}\par


\vspace{\baselineskip}
\begin{adjustwidth}{0.5in}{0.01in}
{\fontsize{9pt}{10.8pt}\selectfont 35.1.1 \tabto{0.49in} The striker is out Hit wicket if, after the bowler has entered the delivery stride and while the ball is in play, his wicket is put down by either the striker’s bat or person as described in clauses to (Wicket put down) in any of the following circumstances:\par}\par

\end{adjustwidth}


\vspace{\baselineskip}
\begin{adjustwidth}{0.49in}{0.0in}
{\fontsize{9pt}{10.8pt}\selectfont 35.1.1.1 \tabto{1.17in} {\fontsize{8pt}{9.6pt}\selectfont in the course of any action taken by him in preparing to receive or in receiving a delivery,\par}\par}\par

\end{adjustwidth}


\vspace{\baselineskip}
\begin{adjustwidth}{0.49in}{0.0in}
{\fontsize{9pt}{10.8pt}\selectfont 35.1.1.2 \tabto{1.17in} in setting off for the first run immediately after playing or playing at the ball,\par}\par

\end{adjustwidth}


\vspace{\baselineskip}
\begin{adjustwidth}{1.18in}{0.14in}
{\fontsize{9pt}{10.8pt}\selectfont 35.1.1.3 \tabto{1.17in} {\fontsize{8pt}{9.6pt}\selectfont if no attempt is made to play the ball, in setting off for the first run, providing that in the opinion of the umpire this is immediately after the striker has had the opportunity of playing the ball,\par}\par}\par

\end{adjustwidth}


\vspace{\baselineskip}
\begin{adjustwidth}{1.18in}{0.21in}
{\fontsize{9pt}{10.8pt}\selectfont 35.1.1.4 \tabto{1.17in} in lawfully making a second or further stroke for the purpose of guarding his wicket within the provisions of clause (Ball lawfully struck more than once).\par}\par

\end{adjustwidth}


\vspace{\baselineskip}
\begin{adjustwidth}{0.5in}{0.25in}
{\fontsize{9pt}{10.8pt}\selectfont 35.1.2 \tabto{0.49in} If the striker puts his wicket down in any of the ways described in clauses to before the bowler has entered the delivery stride, either umpire shall call and signal Dead ball.\par}\par

\end{adjustwidth}


\vspace{\baselineskip}
{\fontsize{11pt}{13.2pt}\selectfont \textbf{35.2 \tabto{0.47in} Not out Hit wicket}\par}\par


\vspace{\baselineskip}
\begin{adjustwidth}{0.0in}{0.14in}
{\fontsize{9pt}{10.8pt}\selectfont The striker is not out under this clause should his wicket be put down in any of the ways referred to in clause if any of the following applies:\par}\par

\end{adjustwidth}


\vspace{\baselineskip}
\begin{itemize}
	\item {\fontsize{9pt}{10.8pt}\selectfont it occurs after the striker has completed any action in receiving the delivery, other than in clauses to \par}\par


\vspace{\baselineskip}
	\item {\fontsize{9pt}{10.8pt}\selectfont it occurs when the striker is in the act of running, other than setting off immediately for the first run.\par}\par


\vspace{\baselineskip}
	\item {\fontsize{9pt}{10.8pt}\selectfont it occurs when the striker is trying to avoid being run out or stumped.\par}\par


\vspace{\baselineskip}
	\item {\fontsize{9pt}{10.8pt}\selectfont it occurs when the striker is trying to avoid a throw in at any time.\par}\par


\vspace{\baselineskip}
	\item {\fontsize{9pt}{10.8pt}\selectfont the bowler after entering the delivery stride does not deliver the ball. In this case either umpire shall immediately call and signal Dead ball. See clause (Umpire calling and signalling Dead ball).\par}\par


\vspace{\baselineskip}
	\item {\fontsize{9pt}{10.8pt}\selectfont the delivery is a No ball.\par}
\end{itemize}\par


\vspace{\baselineskip}
{\fontsize{16pt}{19.2pt}\selectfont \textbf{36 LEG BEFORE WICKET}\par}\par


\vspace{\baselineskip}
{\fontsize{11pt}{13.2pt}\selectfont \textbf{36.1 \tabto{0.47in} Out LBW}\par}\par


\vspace{\baselineskip}
{\fontsize{9pt}{10.8pt}\selectfont The striker is out LBW if all the circumstances set out in clauses to apply.\par}\par


\vspace{\baselineskip}
{\fontsize{9pt}{10.8pt}\selectfont 36.1.1 \tabto{0.49in} {\fontsize{8pt}{9.6pt}\selectfont The bowler delivers a ball, not being a No ball\par}\par}\par


\vspace{\baselineskip}
\begin{adjustwidth}{0.5in}{0.22in}
{\fontsize{9pt}{10.8pt}\selectfont 36.1.2 \tabto{0.49in} the ball, if it is not intercepted full-pitch, pitches in line between wicket and wicket or on the off side of the striker’s wicket\par}\par

\end{adjustwidth}


\vspace{\baselineskip}
\begin{adjustwidth}{0.5in}{0.06in}
{\fontsize{9pt}{10.8pt}\selectfont 36.1.3 \tabto{0.49in} the ball not having previously touched his bat, the striker intercepts the ball, either full-pitch or after pitching, with any part of his person\par}\par

\end{adjustwidth}


\vspace{\baselineskip}
{\fontsize{9pt}{10.8pt}\selectfont 36.1.4 \tabto{0.49in} the point of impact, even if above the level of the bails,\par}\par


\vspace{\baselineskip}
\begin{adjustwidth}{0.5in}{0.0in}
{\fontsize{9pt}{10.8pt}\selectfont either is between wicket and wicket\par}\par

\end{adjustwidth}


\vspace{\baselineskip}

\vspace{\baselineskip}

\vspace{\baselineskip}
\begin{Center}
{\fontsize{8pt}{9.6pt}\selectfont 45\par}
\end{Center}\par


\vspace{\baselineskip}
\begin{adjustwidth}{0.5in}{0.0in}
{\fontsize{9pt}{10.8pt}\selectfont or if the striker has made no genuine attempt to play the ball with the bat, is\par}\par

\end{adjustwidth}


\vspace{\baselineskip}
\begin{adjustwidth}{0.5in}{0.0in}
{\fontsize{9pt}{10.8pt}\selectfont between wicket and wicket or outside the line of the off stump.\par}\par

\end{adjustwidth}


\vspace{\baselineskip}
{\fontsize{9pt}{10.8pt}\selectfont 36.1.5 \tabto{0.49in} {\fontsize{8pt}{9.6pt}\selectfont but for the interception, the ball would have hit the wicket.\par}\par}\par


\vspace{\baselineskip}
{\fontsize{11pt}{13.2pt}\selectfont \textbf{36.2 \tabto{0.47in} Interception of the ball}\par}\par


\vspace{\baselineskip}
\begin{adjustwidth}{0.5in}{0.51in}
{\fontsize{9pt}{10.8pt}\selectfont 36.2.1 \tabto{0.49in} In assessing points of impact in clauses and only the first interception is to be considered.\par}\par

\end{adjustwidth}


\vspace{\baselineskip}
\begin{adjustwidth}{0.5in}{0.04in}
{\fontsize{9pt}{10.8pt}\selectfont 36.2.2 \tabto{0.49in} In assessing if the bowler’s end umpire is not satisfied that the ball intercepted the batsman’s person before it touched the bat, the batsman shall be given Not out.\par}\par

\end{adjustwidth}


\vspace{\baselineskip}
\begin{adjustwidth}{0.5in}{0.38in}
{\fontsize{9pt}{10.8pt}\selectfont 36.2.3 \tabto{0.49in} {\fontsize{8pt}{9.6pt}\selectfont In assessing clause it is to be assumed that the path of the ball before interception would have continued after interception, irrespective of whether the ball might have pitched subsequently or not.\par}\par}\par

\end{adjustwidth}


\vspace{\baselineskip}
{\fontsize{11pt}{13.2pt}\selectfont \textbf{36.3 \tabto{0.47in} Off side of wicket}\par}\par


\vspace{\baselineskip}
\begin{adjustwidth}{0.0in}{0.11in}
{\fontsize{9pt}{10.8pt}\selectfont The off side of the striker’s wicket shall be determined by the striker’s stance at the moment the ball comes into play for that delivery. See paragraph of Appendix A.\par}\par

\end{adjustwidth}


\vspace{\baselineskip}
{\fontsize{16pt}{19.2pt}\selectfont \textbf{37 OBSTRUCTING THE FIELD}\par}\par


\vspace{\baselineskip}
{\fontsize{11pt}{13.2pt}\selectfont \textbf{37.1 \tabto{0.47in} Out Obstructing the field}\par}\par


\vspace{\baselineskip}
\begin{adjustwidth}{0.5in}{0.12in}
\begin{justify}
{\fontsize{9pt}{10.8pt}\selectfont 37.1.1 \tabto{0.49in} Either batsman is out Obstructing the field if, except in the circumstances of clause and while the ball is in play, he wilfully attempts to obstruct or distract the fielding side by word or action. See also clause  (Hit the ball twice).\par}
\end{justify}\par

\end{adjustwidth}


\vspace{\baselineskip}
\begin{adjustwidth}{0.5in}{0.06in}
{\fontsize{9pt}{10.8pt}\selectfont 37.1.2 \tabto{0.49in} The striker is out Obstructing the field if, except in the circumstances of clause in the act of receiving a ball delivered by the bowler, he wilfully strikes the ball with a hand not holding the bat. This will apply whether it is the first strike or a second or subsequent strike. The act of receiving the ball shall extend both to playing at the ball and to striking the ball more than once in defence of his wicket.\par}\par

\end{adjustwidth}


\vspace{\baselineskip}
{\fontsize{9pt}{10.8pt}\selectfont 37.1.3 \tabto{0.49in} This clause will apply whether or not No ball is called.\par}\par


\vspace{\baselineskip}
\begin{adjustwidth}{0.5in}{0.01in}
{\fontsize{9pt}{10.8pt}\selectfont 37.1.4 \tabto{0.49in} For the avoidance of doubt, if an umpire feels that a batsman, in running between the wickets, has significantly changed his direction without probable cause and thereby obstructed a fielder’s attempt to effect a run out, the batsman should, on appeal, be given out, obstructing the field. It shall not be relevant whether a run out would have occurred or not.\par}\par

\end{adjustwidth}


\vspace{\baselineskip}
\begin{adjustwidth}{0.5in}{0.0in}
{\fontsize{9pt}{10.8pt}\selectfont If the change of direction involves the batsman crossing the pitch, clause shall also apply.\par}\par

\end{adjustwidth}


\vspace{\baselineskip}
\begin{adjustwidth}{0.5in}{0.0in}
{\fontsize{9pt}{10.8pt}\selectfont See also paragraph of Appendix D.\par}\par

\end{adjustwidth}


\vspace{\baselineskip}
{\fontsize{11pt}{13.2pt}\selectfont \textbf{37.2 \tabto{0.47in} Not out Obstructing the field}\par}\par


\vspace{\baselineskip}
{\fontsize{9pt}{10.8pt}\selectfont A batsman shall not be out Obstructing the field if\par}\par


\vspace{\baselineskip}
{\fontsize{9pt}{10.8pt}\selectfont obstruction or distraction is accidental, or\par}\par


\vspace{\baselineskip}
{\fontsize{9pt}{10.8pt}\selectfont obstruction is in order to avoid injury, or\par}\par


\vspace{\baselineskip}
\begin{adjustwidth}{0.0in}{0.08in}
{\fontsize{9pt}{10.8pt}\selectfont in the case of the striker, he makes a second or subsequent strike to guard his wicket lawfully as in clause (Ball lawfully struck more than once). However, see clause \par}\par

\end{adjustwidth}


\vspace{\baselineskip}
{\fontsize{11pt}{13.2pt}\selectfont \textbf{37.3 \tabto{0.47in} Obstructing a ball from being caught}\par}\par


\vspace{\baselineskip}
\begin{adjustwidth}{0.0in}{0.06in}
\begin{justify}
{\fontsize{9pt}{10.8pt}\selectfont The striker is out Obstructing the field should wilful obstruction or distraction by either batsman prevent a catch being completed. This shall apply even though the obstruction is caused by the striker in lawfully guarding his wicket under the provision of clause (Ball lawfully struck more than once).\par}
\end{justify}\par

\end{adjustwidth}


\vspace{\baselineskip}

\vspace{\baselineskip}

\vspace{\baselineskip}

\vspace{\baselineskip}

\vspace{\baselineskip}
\begin{Center}
{\fontsize{8pt}{9.6pt}\selectfont 46\par}
\end{Center}\par


\vspace{\baselineskip}
{\fontsize{11pt}{13.2pt}\selectfont \textbf{37.4 \tabto{0.47in} Returning the ball to a fielder}\par}\par


\vspace{\baselineskip}
\begin{adjustwidth}{0.0in}{0.08in}
{\fontsize{9pt}{10.8pt}\selectfont Either batsman is out Obstructing the field if, at any time while the ball is in play and, without the consent of a fielder, he uses the bat or any part of his person to return the ball to any fielder.\par}\par

\end{adjustwidth}


\vspace{\baselineskip}
{\fontsize{11pt}{13.2pt}\selectfont \textbf{37.5 \tabto{0.47in} Runs scored}\par}\par


\vspace{\baselineskip}
{\fontsize{9pt}{10.8pt}\selectfont When either batsman is dismissed Obstructing the field,\par}\par


\vspace{\baselineskip}
\begin{adjustwidth}{0.5in}{0.03in}
{\fontsize{9pt}{10.8pt}\selectfont 37.5.1 \tabto{0.49in} unless the obstruction prevents a catch from being made, any runs completed by the batsmen before the offence shall be scored, together with any runs awarded for penalties to either side. See clauses (Runs awarded for penalties) and (Runs scored when a batsman is dismissed).\par}\par

\end{adjustwidth}


\vspace{\baselineskip}
\begin{adjustwidth}{0.5in}{0.03in}
{\fontsize{9pt}{10.8pt}\selectfont 37.5.2 \tabto{0.49in} if the obstruction prevents a catch from being made, any runs completed by the batsmen shall not be scored but any penalties awarded to either side shall stand.\par}\par

\end{adjustwidth}


\vspace{\baselineskip}
{\fontsize{11pt}{13.2pt}\selectfont \textbf{37.6 \tabto{0.47in} Bowler does not get credit}\par}\par


\vspace{\baselineskip}
{\fontsize{9pt}{10.8pt}\selectfont The bowler does not get credit for the wicket.\par}\par


\vspace{\baselineskip}
{\fontsize{16pt}{19.2pt}\selectfont \textbf{38 RUN OUT}\par}\par


\vspace{\baselineskip}
{\fontsize{11pt}{13.2pt}\selectfont \textbf{38.1 \tabto{0.47in} Out Run out}\par}\par


\vspace{\baselineskip}
{\fontsize{9pt}{10.8pt}\selectfont Either batsman is out Run out, except as in clause if, at any time while the ball is in play,\par}\par


\vspace{\baselineskip}
{\fontsize{9pt}{10.8pt}\selectfont he is out of his ground\par}\par


\vspace{\baselineskip}
{\fontsize{9pt}{10.8pt}\selectfont and his wicket is fairly put down by the action of a fielder\par}\par


\vspace{\baselineskip}
\begin{adjustwidth}{0.0in}{0.29in}
{\fontsize{9pt}{10.8pt}\selectfont even though No ball has been called, except in the circumstances of clause and whether or not a run is being attempted.\par}\par

\end{adjustwidth}


\vspace{\baselineskip}
{\fontsize{11pt}{13.2pt}\selectfont \textbf{38.2 \tabto{0.47in} Batsman not out Run out}\par}\par


\vspace{\baselineskip}
{\fontsize{9pt}{10.8pt}\selectfont 38.2.1 \tabto{0.49in} A batsman is not out Run out in the circumstances of clauses or \par}\par


\vspace{\baselineskip}
\begin{adjustwidth}{1.18in}{0.19in}
{\fontsize{9pt}{10.8pt}\selectfont 38.2.1.1 \tabto{1.17in} He has been within his ground and has subsequently left it to avoid injury, when the wicket is put down.\par}\par

\end{adjustwidth}


\vspace{\baselineskip}
\begin{adjustwidth}{1.12in}{0.0in}
{\fontsize{9pt}{10.8pt}\selectfont Note also the provisions of clause (When out of his ground).\par}\par

\end{adjustwidth}


\vspace{\baselineskip}
\begin{adjustwidth}{1.18in}{0.17in}
{\fontsize{9pt}{10.8pt}\selectfont 38.2.1.2 \tabto{1.17in} The ball, delivered by the bowler, has not made contact with a fielder, before the wicket is put down.\par}\par

\end{adjustwidth}


\vspace{\baselineskip}
{\fontsize{9pt}{10.8pt}\selectfont 38.2.2 \tabto{0.49in} The striker is not out Run out in any of the circumstances in clauses and \par}\par


\vspace{\baselineskip}
\begin{adjustwidth}{0.49in}{0.0in}
{\fontsize{9pt}{10.8pt}\selectfont 38.2.2.1 \tabto{1.17in} {\fontsize{8pt}{9.6pt}\selectfont He is out Stumped. See clause (Out Stumped).\par}\par}\par

\end{adjustwidth}


\vspace{\baselineskip}
\begin{adjustwidth}{0.49in}{0.0in}
{\fontsize{9pt}{10.8pt}\selectfont 38.2.2.2 \tabto{1.17in} No ball has been called\par}\par

\end{adjustwidth}


\vspace{\baselineskip}
\begin{adjustwidth}{1.12in}{0.0in}
{\fontsize{9pt}{10.8pt}\selectfont and he is out of his ground not attempting a run\par}\par

\end{adjustwidth}


\vspace{\baselineskip}
\begin{adjustwidth}{1.12in}{0.0in}
{\fontsize{9pt}{10.8pt}\selectfont and the wicket is fairly put down by the wicket-keeper without the intervention of another fielder.\par}\par

\end{adjustwidth}


\vspace{\baselineskip}
{\fontsize{11pt}{13.2pt}\selectfont \textbf{38.3 \tabto{0.47in} Which batsman is out}\par}\par


\vspace{\baselineskip}
\begin{adjustwidth}{0.0in}{0.15in}
{\fontsize{9pt}{10.8pt}\selectfont The batsman out in the circumstances of clause is the one whose ground is at the end where the wicket is put down. See clause (Which is a batsman’s ground).\par}\par

\end{adjustwidth}


\vspace{\baselineskip}
{\fontsize{11pt}{13.2pt}\selectfont \textbf{38.4 \tabto{0.47in} Runs scored}\par}\par


\vspace{\baselineskip}
\begin{adjustwidth}{0.0in}{0.08in}
\begin{justify}
{\fontsize{9pt}{10.8pt}\selectfont If either batsman is dismissed Run out, the run in progress when the wicket is put down shall not be scored, but any runs completed by the batsmen shall stand, together with any runs for penalties awarded to either side. See clauses (Runs awarded for penalties) and (Runs scored when a batsman is dismissed).\par}
\end{justify}\par

\end{adjustwidth}


\vspace{\baselineskip}

\vspace{\baselineskip}

\vspace{\baselineskip}

\vspace{\baselineskip}
\begin{Center}
{\fontsize{8pt}{9.6pt}\selectfont 47\par}
\end{Center}\par


\vspace{\baselineskip}
{\fontsize{11pt}{13.2pt}\selectfont \textbf{38.5 \tabto{0.47in} Bowler does not get credit}\par}\par


\vspace{\baselineskip}
{\fontsize{9pt}{10.8pt}\selectfont The bowler does not get credit for the wicket.\par}\par


\vspace{\baselineskip}
{\fontsize{16pt}{19.2pt}\selectfont \textbf{39 STUMPED}\par}\par


\vspace{\baselineskip}
{\fontsize{11pt}{13.2pt}\selectfont \textbf{39.1 \tabto{0.47in} Out Stumped}\par}\par


\vspace{\baselineskip}
{\fontsize{9pt}{10.8pt}\selectfont 39.1.1 \tabto{0.49in} {\fontsize{8pt}{9.6pt}\selectfont The striker is out Stumped, except as in clause if\par}\par}\par


\vspace{\baselineskip}
\begin{adjustwidth}{0.5in}{0.0in}
{\fontsize{9pt}{10.8pt}\selectfont a ball which is delivered is not called No ball\par}\par

\end{adjustwidth}


\vspace{\baselineskip}
\begin{adjustwidth}{0.5in}{0.0in}
{\fontsize{9pt}{10.8pt}\selectfont and he is out of his ground, other than as in clause \par}\par

\end{adjustwidth}


\vspace{\baselineskip}
\begin{adjustwidth}{0.5in}{0.0in}
{\fontsize{9pt}{10.8pt}\selectfont and he has not attempted a run\par}\par

\end{adjustwidth}


\vspace{\baselineskip}
\begin{adjustwidth}{0.5in}{0.32in}
{\fontsize{9pt}{10.8pt}\selectfont when his wicket is fairly put down by the wicket-keeper without the intervention of another fielder. Note, however clause (Position of wicket-keeper).\par}\par

\end{adjustwidth}


\vspace{\baselineskip}
\begin{adjustwidth}{0.5in}{0.1in}
{\fontsize{9pt}{10.8pt}\selectfont 39.1.2 \tabto{0.49in} The striker is out Stumped if all the conditions of clause are satisfied, even though a decision of Run out would be justified.\par}\par

\end{adjustwidth}


\vspace{\baselineskip}
{\fontsize{11pt}{13.2pt}\selectfont \textbf{39.2 \tabto{0.47in} Ball rebounding from wicket-keeper’s person}\par}\par


\vspace{\baselineskip}
\begin{adjustwidth}{0.0in}{0.01in}
{\fontsize{9pt}{10.8pt}\selectfont If the wicket is put down by the ball, it shall be regarded as having been put down by the wicket-keeper if the ball rebounds on to the stumps from any part of the wicket-keeper’s person or equipment or has been kicked or thrown on to the stumps by the wicket-keeper.\par}\par

\end{adjustwidth}


\vspace{\baselineskip}
{\fontsize{11pt}{13.2pt}\selectfont \textbf{39.3 \tabto{0.47in} Not out Stumped}\par}\par


\vspace{\baselineskip}
{\fontsize{9pt}{10.8pt}\selectfont 39.3.1 \tabto{0.49in} The striker will not be out Stumped if he has left his ground in order to avoid injury.\par}\par


\vspace{\baselineskip}
\begin{adjustwidth}{0.5in}{0.01in}
{\fontsize{9pt}{10.8pt}\selectfont 39.3.2 \tabto{0.49in} If the striker is not out Stumped he may, except in the circumstances of (Batsman not out Run out), be out Run out if the conditions of clause (Out Run out) apply.\par}\par

\end{adjustwidth}


\vspace{\baselineskip}
{\fontsize{16pt}{19.2pt}\selectfont \textbf{40 TIMED OUT}\par}\par


\vspace{\baselineskip}
{\fontsize{11pt}{13.2pt}\selectfont \textbf{40.1 \tabto{0.47in} Out Timed out}\par}\par


\vspace{\baselineskip}
\begin{adjustwidth}{0.5in}{0.11in}
{\fontsize{9pt}{10.8pt}\selectfont 40.1.1 \tabto{0.49in} After the fall of a wicket or the retirement of a batsman, the incoming batsman must, unless Time has been called, be in position to take guard or for the other batsman to be ready to receive the next ball within 1 minute 30 seconds of the dismissal or retirement. If this requirement is not met, the incoming batsman will be out, Timed out.\par}\par

\end{adjustwidth}


\vspace{\baselineskip}
{\fontsize{9pt}{10.8pt}\selectfont 40.1.2 \tabto{0.49in} The incoming batsman is expected to be ready to make his way to the wicket immediately a wicket falls.\par}\par


\vspace{\baselineskip}
\begin{adjustwidth}{0.5in}{0.0in}
{\fontsize{9pt}{10.8pt}\selectfont Dugouts shall be provided.\par}\par

\end{adjustwidth}


\vspace{\baselineskip}
\begin{adjustwidth}{0.5in}{0.14in}
{\fontsize{9pt}{10.8pt}\selectfont 40.1.3 \tabto{0.49in} In the event of an extended delay in which no batsman comes to the wicket, the umpires shall adopt the procedure of clause (ICC Match Referee awarding a match). For the purposes of that clause the start of the action shall be taken as the expiry of the 1 minute 30 seconds referred to above.\par}\par

\end{adjustwidth}


\vspace{\baselineskip}
{\fontsize{11pt}{13.2pt}\selectfont \textbf{40.2 \tabto{0.47in} Bowler does not get credit}\par}\par


\vspace{\baselineskip}
{\fontsize{9pt}{10.8pt}\selectfont The bowler does not get credit for the wicket.\par}\par


\vspace{\baselineskip}
{\fontsize{16pt}{19.2pt}\selectfont \textbf{41 UNFAIR PLAY}\par}\par


\vspace{\baselineskip}
{\fontsize{11pt}{13.2pt}\selectfont \textbf{41.1 \tabto{0.47in} Fair and unfair play – responsibility of captains}\par}\par


\vspace{\baselineskip}
\begin{adjustwidth}{0.0in}{0.19in}
{\fontsize{9pt}{10.8pt}\selectfont The captains are responsible for ensuring that play is conducted within the Spirit of Cricket, as well as within these Playing Conditions.\par}\par

\end{adjustwidth}


\vspace{\baselineskip}

\vspace{\baselineskip}

\vspace{\baselineskip}

\vspace{\baselineskip}
\begin{Center}
{\fontsize{8pt}{9.6pt}\selectfont 48\par}
\end{Center}\par


\vspace{\baselineskip}
{\fontsize{11pt}{13.2pt}\selectfont \textbf{41.2 \tabto{0.47in} Fair and unfair play – responsibility of umpires}\par}\par


\vspace{\baselineskip}
\begin{adjustwidth}{0.0in}{0.01in}
{\fontsize{9pt}{10.8pt}\selectfont The umpires shall be the sole judges of fair and unfair play. If either umpire considers an action, not covered by these Playing Conditions, to be unfair he/she shall intervene without appeal and, if the ball is in play, call and signal Dead ball and implement the procedure as set out in clause 41.19. Otherwise umpires shall not interfere with the progress of play without appeal except as required to do so by these Playing Conditions.\par}\par

\end{adjustwidth}


\vspace{\baselineskip}
{\fontsize{11pt}{13.2pt}\selectfont \textbf{41.3 \tabto{0.47in} The match ball – changing its condition}\par}\par


\vspace{\baselineskip}
\begin{adjustwidth}{0.5in}{0.07in}
{\fontsize{9pt}{10.8pt}\selectfont 41.3.1 \tabto{0.49in} The umpires shall make frequent and irregular inspections of the ball. In addition, they shall immediately inspect the ball if they suspect anyone of attempting to change the condition of the ball, except as permitted in clause \par}\par

\end{adjustwidth}


\vspace{\baselineskip}
{\fontsize{9pt}{10.8pt}\selectfont 41.3.2 \tabto{0.49in} {\fontsize{8pt}{9.6pt}\selectfont It is an offence for any player to take any action which changes the condition of the ball.\par}\par}\par


\vspace{\baselineskip}
\begin{adjustwidth}{0.5in}{0.14in}
{\fontsize{9pt}{10.8pt}\selectfont Except in carrying out his normal duties, a batsman is not allowed to damage the ball other than, when the ball is in play, in striking it with the bat. See also clause (Damage to the ball).\par}\par

\end{adjustwidth}


\vspace{\baselineskip}
\begin{adjustwidth}{0.5in}{0.0in}
{\fontsize{9pt}{10.8pt}\selectfont A fielder may, however:\par}\par

\end{adjustwidth}


\vspace{\baselineskip}
\begin{adjustwidth}{1.18in}{0.5in}
{\fontsize{9pt}{10.8pt}\selectfont 41.3.2.1 \tabto{1.17in} polish the ball on his clothing provided that no artificial substance is used and that such polishing wastes no time.\par}\par

\end{adjustwidth}


\vspace{\baselineskip}
\begin{adjustwidth}{0.49in}{0.0in}
{\fontsize{9pt}{10.8pt}\selectfont 41.3.2.2 \tabto{1.17in} {\fontsize{8pt}{9.6pt}\selectfont remove mud from the ball under the supervision of an umpire.\par}\par}\par

\end{adjustwidth}


\vspace{\baselineskip}
\begin{adjustwidth}{0.49in}{0.0in}
{\fontsize{9pt}{10.8pt}\selectfont 41.3.2.3 \tabto{1.17in} dry a wet ball on a piece of cloth that has been approved by the umpires.\par}\par

\end{adjustwidth}


\vspace{\baselineskip}
\begin{adjustwidth}{0.5in}{0.38in}
{\fontsize{9pt}{10.8pt}\selectfont 41.3.3 \tabto{0.49in} The umpires shall consider the condition of the ball to have been unfairly changed if any action by any player does not comply with the conditions in clause \par}\par

\end{adjustwidth}


\vspace{\baselineskip}
\begin{adjustwidth}{0.5in}{0.14in}
{\fontsize{9pt}{10.8pt}\selectfont 41.3.4 \tabto{0.49in} If the umpires together agree that the condition of the ball has been unfairly changed by a member or members of either side, or that its condition is inconsistent with the use it has received, they shall consider that there has been a contravention of this clause and decide together whether they can identify the player(s) responsible for such conduct.\par}\par

\end{adjustwidth}


\vspace{\baselineskip}
{\fontsize{9pt}{10.8pt}\selectfont 41.3.5 \tabto{0.49in} If it is possible to identify the player(s) responsible for changing the condition of the ball, the umpires shall;\par}\par


\vspace{\baselineskip}
\begin{adjustwidth}{0.49in}{0.0in}
{\fontsize{9pt}{10.8pt}\selectfont 41.3.5.1 \tabto{1.17in} {\fontsize{8pt}{9.6pt}\selectfont Change the ball forthwith.\par}\par}\par

\end{adjustwidth}


\vspace{\baselineskip}
\begin{adjustwidth}{2.0in}{0.03in}
{\fontsize{9pt}{10.8pt}\selectfont 41.3.5.1.1 \tabto{1.99in} If the umpires together agree that the condition of the ball has been unfairly changed by a member or members of the fielding side, the batsman at the wicket shall choose the replacement ball from a selection of six other balls of various degrees of usage (including a new ball) and of the same brand as the ball in use prior to the contravention.\par}\par

\end{adjustwidth}


\vspace{\baselineskip}
\begin{adjustwidth}{2.0in}{0.11in}
{\fontsize{9pt}{10.8pt}\selectfont 41.3.5.1.2 \tabto{1.99in} If the umpires together agree that the condition of the ball has been unfairly changed by a member or members of the batting side, the umpires shall select and bring into use immediately, a ball which shall have wear comparable to that of the previous ball immediately prior to the contravention.\par}\par

\end{adjustwidth}


\vspace{\baselineskip}
\begin{adjustwidth}{0.49in}{0.0in}
{\fontsize{9pt}{10.8pt}\selectfont 41.3.5.2 \tabto{1.17in} Additionally, the bowler’s end umpire shall\par}\par

\end{adjustwidth}


\vspace{\baselineskip}
\begin{itemize}
	\item {\fontsize{9pt}{10.8pt}\selectfont award 5 Penalty runs to the opposing side.\par}\par


\vspace{\baselineskip}
	\item {\fontsize{9pt}{10.8pt}\selectfont if appropriate, inform the batsmen at the wicket and the captain of the fielding side that the ball has been changed and the reason for their action.\par}\par


\vspace{\baselineskip}
	\item {\fontsize{9pt}{10.8pt}\selectfont inform the captain of the batting side as soon as practicable of what has occurred.\par}
\end{itemize}\par


\vspace{\baselineskip}
\begin{adjustwidth}{1.12in}{0.04in}
{\fontsize{9pt}{10.8pt}\selectfont The umpires shall then report the matter to the ICC Match Referee who shall take such action as is considered appropriate against the player(s) concerned.\par}\par

\end{adjustwidth}


\vspace{\baselineskip}
\begin{adjustwidth}{0.5in}{0.26in}
{\fontsize{9pt}{10.8pt}\selectfont 41.3.6 \tabto{0.49in} If it is not possible to identify the player(s) responsible for changing the condition of the ball, the umpires shall;\par}\par

\end{adjustwidth}


\vspace{\baselineskip}

\vspace{\baselineskip}

\vspace{\baselineskip}

\vspace{\baselineskip}

\vspace{\baselineskip}
\begin{Center}
{\fontsize{8pt}{9.6pt}\selectfont 49\par}
\end{Center}\par


\vspace{\baselineskip}

\vspace{\baselineskip}
\begin{adjustwidth}{1.18in}{0.03in}
{\fontsize{9pt}{10.8pt}\selectfont 41.3.6.1 \tabto{1.17in} Change the ball forthwith. The umpires shall choose the replacement ball for one of similar wear and of the same brand as the ball in use prior to the contravention.\par}\par

\end{adjustwidth}


\vspace{\baselineskip}
\begin{adjustwidth}{0.49in}{0.0in}
{\fontsize{9pt}{10.8pt}\selectfont 41.3.6.2 \tabto{1.17in} {\fontsize{8pt}{9.6pt}\selectfont The bowler’s end umpire shall issue the captain with a first and final warning, and\par}\par}\par

\end{adjustwidth}


\vspace{\baselineskip}
\begin{adjustwidth}{1.18in}{0.04in}
{\fontsize{9pt}{10.8pt}\selectfont 41.3.6.3 \tabto{1.17in} Advise the captain that should there be any further instances of changing the condition of the ball by that team during the remainder of the series, clause above will be adopted, with the captain deemed to be the player responsible for the contravention.\par}\par

\end{adjustwidth}


\vspace{\baselineskip}
{\fontsize{11pt}{13.2pt}\selectfont \textbf{41.4 \tabto{0.47in} Deliberate attempt to distract striker}\par}\par


\vspace{\baselineskip}
\begin{adjustwidth}{0.5in}{0.33in}
{\fontsize{9pt}{10.8pt}\selectfont 41.4.1 \tabto{0.49in} It is unfair for any fielder deliberately to attempt to distract the striker while he is preparing to receive or receiving a delivery.\par}\par

\end{adjustwidth}


\vspace{\baselineskip}
\begin{adjustwidth}{0.5in}{0.15in}
{\fontsize{9pt}{10.8pt}\selectfont 41.4.2 \tabto{0.49in} If either umpire considers that any action by a fielder is such an attempt, he/she shall immediately call and signal Dead ball and inform the other umpire of the reason for the call. The bowler’s end umpire shall\par}\par

\end{adjustwidth}


\vspace{\baselineskip}
\begin{itemize}
	\item {\fontsize{9pt}{10.8pt}\selectfont award 5 Penalty runs to the batting side.\par}\par


\vspace{\baselineskip}
	\item {\fontsize{9pt}{10.8pt}\selectfont inform the captain of the fielding side, the batsmen and, as soon as practicable, the captain of the batting side of the reason for the action.\par}
\end{itemize}\par


\vspace{\baselineskip}
\begin{adjustwidth}{0.5in}{0.0in}
{\fontsize{9pt}{10.8pt}\selectfont Neither batsman shall be dismissed from that delivery and the ball shall not count as one of the over.\par}\par

\end{adjustwidth}


\vspace{\baselineskip}
\begin{adjustwidth}{0.5in}{0.57in}
{\fontsize{9pt}{10.8pt}\selectfont The umpires may then report the matter to the ICC Match Referee who shall take such action as is considered appropriate against the fielder concerned.\par}\par

\end{adjustwidth}


\vspace{\baselineskip}
{\fontsize{11pt}{13.2pt}\selectfont \textbf{41.5 \tabto{0.47in} Deliberate distraction, deception or obstruction of batsman}\par}\par


\vspace{\baselineskip}
\begin{adjustwidth}{0.5in}{0.08in}
{\fontsize{9pt}{10.8pt}\selectfont 41.5.1 \tabto{0.49in} In addition to clause it is unfair for any fielder wilfully to attempt, by word or action, to distract, deceive or obstruct either batsman after the striker has received the ball.\par}\par

\end{adjustwidth}


\vspace{\baselineskip}
{\fontsize{9pt}{10.8pt}\selectfont 41.5.2 \tabto{0.49in} It is for either one of the umpires to decide whether any distraction, deception or obstruction is wilful or not.\par}\par


\vspace{\baselineskip}
\begin{adjustwidth}{0.5in}{0.04in}
{\fontsize{9pt}{10.8pt}\selectfont 41.5.3 \tabto{0.49in} If either umpire considers that a fielder has caused or attempted to cause such a distraction, deception or obstruction, he/she shall immediately call and signal Dead ball and inform the other umpire of the reason for the call.\par}\par

\end{adjustwidth}


\vspace{\baselineskip}
{\fontsize{9pt}{10.8pt}\selectfont 41.5.4 \tabto{0.49in} {\fontsize{8pt}{9.6pt}\selectfont Neither batsman shall be dismissed from that delivery.\par}\par}\par


\vspace{\baselineskip}
\begin{adjustwidth}{0.5in}{0.33in}
{\fontsize{9pt}{10.8pt}\selectfont 41.5.5 \tabto{0.49in} If an obstruction involves physical contact, the umpires together shall decide whether or not an offence under clause (Players’ conduct) has been committed.\par}\par

\end{adjustwidth}


\vspace{\baselineskip}
\begin{adjustwidth}{1.18in}{0.38in}
{\fontsize{9pt}{10.8pt}\selectfont 41.5.5.1 \tabto{1.17in} {\fontsize{8pt}{9.6pt}\selectfont If an offence under clause (Players’ conduct) has been committed, they shall apply the relevant procedures in clause and shall also apply each of clauses to \par}\par}\par

\end{adjustwidth}


\vspace{\baselineskip}
\begin{adjustwidth}{1.18in}{0.18in}
{\fontsize{9pt}{10.8pt}\selectfont 41.5.5.2 \tabto{1.17in} If they consider that there has been no offence under clause (Players’ conduct), they shall apply each of clauses to \par}\par

\end{adjustwidth}


\vspace{\baselineskip}
{\fontsize{9pt}{10.8pt}\selectfont 41.5.6 \tabto{0.49in} {\fontsize{8pt}{9.6pt}\selectfont The bowler’s end umpire shall;\par}\par}\par


\vspace{\baselineskip}
\begin{itemize}
	\item {\fontsize{9pt}{10.8pt}\selectfont award 5 Penalty runs to the batting side.\par}\par


\vspace{\baselineskip}
	\item {\fontsize{9pt}{10.8pt}\selectfont inform the captain of the fielding side of the reason for this action and as soon as practicable inform the captain of the batting side.\par}
\end{itemize}\par


\vspace{\baselineskip}
{\fontsize{9pt}{10.8pt}\selectfont 41.5.7 \tabto{0.49in} The ball shall not count as one of the over.\par}\par


\vspace{\baselineskip}
\begin{adjustwidth}{0.5in}{0.07in}
{\fontsize{9pt}{10.8pt}\selectfont 41.5.8 \tabto{0.49in} Any runs completed by the batsmen before the offence shall be scored, together with any runs for penalties awarded to either side. Additionally, the run in progress shall be scored whether or not the batsmen had already crossed at the instant of the offence.\par}\par

\end{adjustwidth}


\vspace{\baselineskip}
{\fontsize{9pt}{10.8pt}\selectfont 41.5.9 \tabto{0.49in} {\fontsize{8pt}{9.6pt}\selectfont The batsmen at the wicket shall decide which of them is to face the next delivery.\par}\par}\par


\vspace{\baselineskip}

\vspace{\baselineskip}

\vspace{\baselineskip}

\vspace{\baselineskip}

\vspace{\baselineskip}

\vspace{\baselineskip}
\begin{Center}
{\fontsize{8pt}{9.6pt}\selectfont 50\par}
\end{Center}\par


\vspace{\baselineskip}

\vspace{\baselineskip}
\begin{adjustwidth}{0.5in}{0.57in}
{\fontsize{9pt}{10.8pt}\selectfont 41.5.10 The umpires may then report the matter to the ICC Match Referee who shall take such action as is considered appropriate against the fielder concerned.\par}\par

\end{adjustwidth}


\vspace{\baselineskip}
{\fontsize{11pt}{13.2pt}\selectfont \textbf{41.6 \tabto{0.47in} Bowling of dangerous and unfair short pitched deliveries}\par}\par


\vspace{\baselineskip}
\begin{adjustwidth}{0.5in}{0.29in}
{\fontsize{9pt}{10.8pt}\selectfont 41.6.1 \tabto{0.49in} Notwithstanding clause the bowling of short pitched deliveries is dangerous if the bowler’s end umpire considers that, taking into consideration the skill of the striker, by their speed, length, height and direction they are likely to inflict physical injury on him. The fact that the striker is wearing protective equipment shall be disregarded.\par}\par

\end{adjustwidth}


\vspace{\baselineskip}
\begin{adjustwidth}{0.5in}{0.15in}
{\fontsize{9pt}{10.8pt}\selectfont In the first instance the umpire decides that the bowling of short pitched deliveries has become dangerous under 41.6.1\par}\par

\end{adjustwidth}


\vspace{\baselineskip}
\begin{adjustwidth}{1.18in}{0.11in}
{\fontsize{9pt}{10.8pt}\selectfont 41.6.1.1 \tabto{1.17in} The umpire shall call and signal No ball, and when the ball is dead, caution the bowler and inform the other umpire, the captain of the fielding side and the batsmen of what has occurred. This caution shall apply to that bowler throughout the innings.\par}\par

\end{adjustwidth}


\vspace{\baselineskip}
\begin{adjustwidth}{1.18in}{0.18in}
{\fontsize{9pt}{10.8pt}\selectfont 41.6.1.2 \tabto{1.17in} If there is a second instance, the umpire shall repeat the above procedure and indicate to the bowler that this is a final warning, which shall apply to that bowler throughout the innings.\par}\par

\end{adjustwidth}


\vspace{\baselineskip}
\begin{adjustwidth}{0.49in}{0.0in}
{\fontsize{9pt}{10.8pt}\selectfont 41.6.1.3 \tabto{1.17in} {\fontsize{8pt}{9.6pt}\selectfont Should there be any further instance by the same bowler in that innings, the umpire shall\par}\par}\par

\end{adjustwidth}


\vspace{\baselineskip}
\begin{itemize}
	\item {\fontsize{9pt}{10.8pt}\selectfont call and signal No ball\par}\par


\vspace{\baselineskip}
	\item {\fontsize{9pt}{10.8pt}\selectfont when the ball is dead, direct the captain of the fielding side to suspend the bowler immediately from bowling\par}\par


\vspace{\baselineskip}
	\item {\fontsize{9pt}{10.8pt}\selectfont inform the other umpire for the reason for this action.\par}
\end{itemize}\par


\vspace{\baselineskip}
\begin{adjustwidth}{1.12in}{0.0in}
{\fontsize{9pt}{10.8pt}\selectfont The bowler thus suspended shall not be allowed to bowl again in that innings.\par}\par

\end{adjustwidth}


\vspace{\baselineskip}
\begin{adjustwidth}{1.12in}{0.1in}
{\fontsize{9pt}{10.8pt}\selectfont If applicable, the over shall be completed by another bowler, who shall neither have bowled any part of the previous over, nor be allowed to bowl any part of the next over.\par}\par

\end{adjustwidth}


\vspace{\baselineskip}
\begin{itemize}
	\item {\fontsize{9pt}{10.8pt}\selectfont The umpire shall report the occurrence to the batsmen and, as soon as practicable, to the captain of the batting side.\par}
\end{itemize}\par


\vspace{\baselineskip}
\begin{adjustwidth}{1.12in}{0.07in}
{\fontsize{9pt}{10.8pt}\selectfont The umpires may then report the matter to the ICC Match Referee who shall take such action as is considered appropriate against the bowler concerned.\par}\par

\end{adjustwidth}


\vspace{\baselineskip}
\begin{adjustwidth}{0.49in}{0.0in}
{\fontsize{9pt}{10.8pt}\selectfont 41.6.1.4 \tabto{1.17in} A bowler shall be limited to one fast short-pitched delivery per over.\par}\par

\end{adjustwidth}


\vspace{\baselineskip}
\begin{adjustwidth}{1.18in}{0.06in}
{\fontsize{9pt}{10.8pt}\selectfont 41.6.1.5 \tabto{1.17in} A fast short-pitched delivery is defined as a ball, which passes or would have passed above the shoulder height of the striker standing upright at the popping crease.\par}\par

\end{adjustwidth}


\vspace{\baselineskip}
\begin{adjustwidth}{1.18in}{0.19in}
{\fontsize{9pt}{10.8pt}\selectfont 41.6.1.6 \tabto{1.17in} The umpire at the bowler’s end shall advise the bowler and the batsman on strike when each fast short pitched delivery has been bowled.\par}\par

\end{adjustwidth}


\vspace{\baselineskip}
\begin{adjustwidth}{1.18in}{0.24in}
{\fontsize{9pt}{10.8pt}\selectfont 41.6.1.7 \tabto{1.17in} In addition, a ball that passes above head height of the batsman, standing upright at the popping crease, that prevents him from being able to hit it with his bat by means of a normal cricket stroke shall be called a Wide. See also clause \par}\par

\end{adjustwidth}


\vspace{\baselineskip}
\begin{adjustwidth}{1.97in}{0.07in}
{\fontsize{9pt}{10.8pt}\selectfont 41.6.1.7.1 \tabto{1.96in} For the avoidance of doubt any fast short pitched delivery that is called a Wide under this clause shall also count as one of the allowable short pitched deliveries in that over.\par}\par

\end{adjustwidth}


\vspace{\baselineskip}
\begin{adjustwidth}{1.18in}{0.07in}
{\fontsize{9pt}{10.8pt}\selectfont 41.6.1.8 \tabto{1.17in} In the event of a bowler bowling more than one fast short-pitched delivery in an over as defined in clause above, the umpire at the bowler’s end shall call and signal No ball on each occasion. A differential signal shall be used to signify a fast short pitched delivery. The umpire shall call and signal ‘No ball’ and then tap the head with the other hand.\par}\par

\end{adjustwidth}


\vspace{\baselineskip}
\begin{adjustwidth}{1.18in}{0.12in}
{\fontsize{9pt}{10.8pt}\selectfont 41.6.1.9 \tabto{1.17in} If a bowler delivers a second fast short pitched ball in an over, the umpire, after the call of No ball and when the ball is dead, shall caution the bowler, inform the other umpire, the captain of the fielding side and the batsmen at the wicket of what has occurred. This caution shall apply throughout the innings.\par}\par

\end{adjustwidth}


\vspace{\baselineskip}

\vspace{\baselineskip}

\vspace{\baselineskip}

\vspace{\baselineskip}

\vspace{\baselineskip}
\begin{Center}
{\fontsize{8pt}{9.6pt}\selectfont 51\par}
\end{Center}\par


\vspace{\baselineskip}

\vspace{\baselineskip}
\begin{adjustwidth}{1.18in}{0.06in}
\begin{justify}
{\fontsize{9pt}{10.8pt}\selectfont 41.6.1.10 \tabto{1.17in} If there is a second instance of the bowler being No balled in the innings for bowling more than one fast short pitched delivery in an over, the umpire shall advise the bowler that this is his final warning for the innings.\par}
\end{justify}\par

\end{adjustwidth}


\vspace{\baselineskip}
\begin{adjustwidth}{0.49in}{0.0in}
{\fontsize{9pt}{10.8pt}\selectfont 41.6.1.11 \tabto{1.17in} {\fontsize{8pt}{9.6pt}\selectfont Should there be any further instance by the same bowler in that innings, the umpire shall\par}\par}\par

\end{adjustwidth}


\vspace{\baselineskip}
\begin{itemize}
	\item {\fontsize{9pt}{10.8pt}\selectfont call and signal No ball\par}\par


\vspace{\baselineskip}
	\item {\fontsize{9pt}{10.8pt}\selectfont when the ball is dead, direct the captain of the fielding side to suspend the bowler immediately from bowling\par}\par


\vspace{\baselineskip}
	\item {\fontsize{9pt}{10.8pt}\selectfont inform the other umpire for the reason for this action.\par}
\end{itemize}\par


\vspace{\baselineskip}
\begin{adjustwidth}{1.12in}{0.0in}
{\fontsize{9pt}{10.8pt}\selectfont The bowler thus suspended shall not be allowed to bowl again in that innings.\par}\par

\end{adjustwidth}


\vspace{\baselineskip}
\begin{adjustwidth}{1.12in}{0.11in}
{\fontsize{9pt}{10.8pt}\selectfont If applicable, the over shall be completed by another bowler, who shall neither have bowled any part of the previous over, nor be allowed to bowl any part of the next over.\par}\par

\end{adjustwidth}


\vspace{\baselineskip}
\begin{itemize}
	\item {\fontsize{9pt}{10.8pt}\selectfont The umpire shall report the occurrence to the batsmen and, as soon as practicable, to the captain of the batting side.\par}
\end{itemize}\par


\vspace{\baselineskip}
\begin{adjustwidth}{1.12in}{0.07in}
{\fontsize{9pt}{10.8pt}\selectfont The umpires may then report the matter to the ICC Match Referee who shall take such action as is considered appropriate against the bowler concerned.\par}\par

\end{adjustwidth}


\vspace{\baselineskip}
{\fontsize{9pt}{10.8pt}\selectfont 41.6.2 \tabto{0.49in} Should the umpires initiate the caution and warning procedures set out in clauses and\par}\par


\vspace{\baselineskip}
\begin{adjustwidth}{0.5in}{0.0in}
{\fontsize{9pt}{10.8pt}\selectfont such cautions and warnings are not to be cumulative.\par}\par

\end{adjustwidth}


\vspace{\baselineskip}
{\fontsize{11pt}{13.2pt}\selectfont \textbf{41.7 \tabto{0.47in} Bowling of dangerous and unfair non-pitching deliveries}\par}\par


\vspace{\baselineskip}
\begin{adjustwidth}{0.5in}{0.04in}
{\fontsize{9pt}{10.8pt}\selectfont 41.7.1 \tabto{0.49in} Any delivery, which passes or would have passed, without pitching, above waist height of the striker standing upright at the popping crease, is to be deemed to be unfair, whether or not it is likely to inflict physical injury on the striker. If the bowler bowls such a delivery the umpire shall immediately call and signal No ball.\par}\par

\end{adjustwidth}


\vspace{\baselineskip}
\begin{adjustwidth}{0.5in}{0.03in}
{\fontsize{9pt}{10.8pt}\selectfont If, in the opinion of the umpire, such a delivery is considered likely to inflict physical injury on the batsman by its speed and direction, it shall be considered dangerous. When the ball is dead the umpire shall caution the bowler, indicating that this is a first and final warning. The umpire shall also inform the other umpire, the captain of the fielding side and the batsmen of what has occurred. This caution shall apply to that bowler throughout the innings.\par}\par

\end{adjustwidth}


\vspace{\baselineskip}
\begin{adjustwidth}{0.0in}{0.08in}
\begin{Center}
{\fontsize{9pt}{10.8pt}\selectfont 41.7.2\ \  Should there be any further instance (where a dangerous non-pitching delivery is bowled and is considered likely to inflict physical injury on the batsman) by the same bowler in that innings, the umpire shall\par}
\end{Center}\par

\end{adjustwidth}


\vspace{\baselineskip}
\begin{itemize}
	\item {\fontsize{9pt}{10.8pt}\selectfont call and signal No ball\par}\par


\vspace{\baselineskip}
	\item {\fontsize{9pt}{10.8pt}\selectfont when the ball is dead, direct the captain of the fielding side to suspend the bowler immediately from bowling\par}\par


\vspace{\baselineskip}
	\item {\fontsize{9pt}{10.8pt}\selectfont inform the other umpire for the reason for this action.\par}
\end{itemize}\par


\vspace{\baselineskip}
\begin{adjustwidth}{0.5in}{0.0in}
{\fontsize{9pt}{10.8pt}\selectfont The bowler thus suspended shall not be allowed to bowl again in that innings.\par}\par

\end{adjustwidth}


\vspace{\baselineskip}
\begin{adjustwidth}{0.5in}{0.12in}
{\fontsize{9pt}{10.8pt}\selectfont If applicable, the over shall be completed by another bowler, who shall neither have bowled any part of the previous over, nor be allowed to bowl any part of the next over.\par}\par

\end{adjustwidth}


\vspace{\baselineskip}
\begin{adjustwidth}{0.5in}{0.0in}
{\fontsize{9pt}{10.8pt}\selectfont Additionally the umpire shall\par}\par

\end{adjustwidth}


\vspace{\baselineskip}
\begin{adjustwidth}{0.5in}{0.0in}
{\fontsize{9pt}{10.8pt}\selectfont - report the occurrence to the batsmen and, as soon as practicable, to the captain of the batting side.\par}\par

\end{adjustwidth}


\vspace{\baselineskip}
\begin{adjustwidth}{0.5in}{0.57in}
{\fontsize{9pt}{10.8pt}\selectfont The umpires may then report the matter to the ICC Match Referee who shall take such action as is considered appropriate against the bowler concerned.\par}\par

\end{adjustwidth}


\vspace{\baselineskip}
\begin{adjustwidth}{0.5in}{0.14in}
{\fontsize{9pt}{10.8pt}\selectfont 41.7.3 \tabto{0.49in} The warning sequence in clauses and is independent of the warning and action sequence in clause \par}\par

\end{adjustwidth}


\vspace{\baselineskip}

\vspace{\baselineskip}

\vspace{\baselineskip}

\vspace{\baselineskip}

\vspace{\baselineskip}
\begin{Center}
{\fontsize{8pt}{9.6pt}\selectfont 52\par}
\end{Center}\par


\vspace{\baselineskip}

\vspace{\baselineskip}
\begin{adjustwidth}{0.5in}{0.25in}
{\fontsize{9pt}{10.8pt}\selectfont 41.7.4 \tabto{0.49in} If the umpire considers that a bowler deliberately bowled a high full-pitched delivery, deemed to be dangerous and unfair as defined in clause then the caution and warning in clause shall be dispensed with. The umpire shall\par}\par

\end{adjustwidth}


\vspace{\baselineskip}
\begin{itemize}
	\item {\fontsize{9pt}{10.8pt}\selectfont immediately call and signal No ball.\par}\par


\vspace{\baselineskip}
	\item {\fontsize{9pt}{10.8pt}\selectfont when the ball is dead, direct the captain of the fielding side to suspend the bowler immediately from bowling and inform the other umpire for the reason for this action.\par}
\end{itemize}\par


\vspace{\baselineskip}
\begin{adjustwidth}{0.5in}{0.0in}
{\fontsize{9pt}{10.8pt}\selectfont The bowler thus suspended shall not be allowed to bowl again in that innings.\par}\par

\end{adjustwidth}


\vspace{\baselineskip}
\begin{adjustwidth}{0.5in}{0.12in}
{\fontsize{9pt}{10.8pt}\selectfont If applicable, the over shall be completed by another bowler, who shall neither have bowled any part of the previous over, nor be allowed to bowl any part of the next over.\par}\par

\end{adjustwidth}


\vspace{\baselineskip}
\begin{adjustwidth}{0.5in}{0.0in}
{\fontsize{9pt}{10.8pt}\selectfont - report the occurrence to the batsmen and, as soon as practicable, to the captain of the batting side.\par}\par

\end{adjustwidth}


\vspace{\baselineskip}
\begin{adjustwidth}{0.5in}{0.07in}
{\fontsize{9pt}{10.8pt}\selectfont The umpires together shall report the occurrence to the ICC Match Referee who shall take such action as is considered appropriate against the bowler concerned.\par}\par

\end{adjustwidth}


\vspace{\baselineskip}
{\fontsize{11pt}{13.2pt}\selectfont \textbf{41.8 \tabto{0.47in} Bowling of deliberate front-foot No ball}\par}\par


\vspace{\baselineskip}
{\fontsize{9pt}{10.8pt}\selectfont If the umpire considers that the bowler has delivered a deliberate front-foot No ball, he/she shall\par}\par


\vspace{\baselineskip}
\begin{itemize}
	\item {\fontsize{9pt}{10.8pt}\selectfont immediately call and signal No ball.\par}\par


\vspace{\baselineskip}
	\item {\fontsize{9pt}{10.8pt}\selectfont when the ball is dead, direct the captain of the fielding side to suspend the bowler immediately from bowling\par}\par


\vspace{\baselineskip}
	\item {\fontsize{9pt}{10.8pt}\selectfont inform the other umpire for the reason for this action.\par}
\end{itemize}\par


\vspace{\baselineskip}
{\fontsize{9pt}{10.8pt}\selectfont The bowler thus suspended shall not be allowed to bowl again in that innings.\par}\par


\vspace{\baselineskip}
\begin{adjustwidth}{0.0in}{0.12in}
{\fontsize{9pt}{10.8pt}\selectfont If applicable, the over shall be completed by another bowler, who shall neither have bowled any part of the previous over, nor be allowed to bowl any part of the next over.\par}\par

\end{adjustwidth}


\vspace{\baselineskip}
{\fontsize{9pt}{10.8pt}\selectfont - report the occurrence to the batsmen and, as soon as practicable, to the captain of the batting side.\par}\par


\vspace{\baselineskip}
\begin{adjustwidth}{0.0in}{0.57in}
{\fontsize{9pt}{10.8pt}\selectfont The umpires together shall report the occurrence to the ICC Match Referee who shall take such action as is considered appropriate against the bowler concerned.\par}\par

\end{adjustwidth}


\vspace{\baselineskip}
{\fontsize{11pt}{13.2pt}\selectfont \textbf{41.9 \tabto{0.47in} Time wasting by the fielding side}\par}\par


\vspace{\baselineskip}
{\fontsize{9pt}{10.8pt}\selectfont 41.9.1 \tabto{0.49in} {\fontsize{8pt}{9.6pt}\selectfont It is unfair for any fielder to waste time.\par}\par}\par


\vspace{\baselineskip}
\begin{adjustwidth}{0.5in}{0.11in}
{\fontsize{9pt}{10.8pt}\selectfont 41.9.2 \tabto{0.49in} If either umpire considers that the progress of an over is unnecessarily slow, or time is being wasted in any other way, by the captain of the fielding side or by any other fielder, at the first instance the umpire concerned shall\par}\par

\end{adjustwidth}


\vspace{\baselineskip}
\begin{itemize}
	\item {\fontsize{9pt}{10.8pt}\selectfont if the ball is in play, call and signal Dead ball.\par}\par


\vspace{\baselineskip}
	\item {\fontsize{9pt}{10.8pt}\selectfont inform the other umpire of what has occurred.\par}
\end{itemize}\par


\vspace{\baselineskip}
\begin{adjustwidth}{0.5in}{0.0in}
{\fontsize{9pt}{10.8pt}\selectfont The bowler’s end umpire shall then\par}\par

\end{adjustwidth}


\vspace{\baselineskip}
\begin{itemize}
	\item {\fontsize{9pt}{10.8pt}\selectfont warn the captain of the fielding side, indicating that this is a first and final warning.\par}\par


\vspace{\baselineskip}
	\item {\fontsize{9pt}{10.8pt}\selectfont inform the batsmen of what has occurred.\par}
\end{itemize}\par


\vspace{\baselineskip}
\begin{adjustwidth}{0.5in}{0.26in}
{\fontsize{9pt}{10.8pt}\selectfont 41.9.3 \tabto{0.49in} If either umpire considers that there is any further waste of time in that innings by any fielder, the umpire concerned shall\par}\par

\end{adjustwidth}


\vspace{\baselineskip}
\begin{itemize}
	\item {\fontsize{9pt}{10.8pt}\selectfont if the ball is in play, call and signal Dead ball.\par}\par


\vspace{\baselineskip}
	\item {\fontsize{9pt}{10.8pt}\selectfont inform the other umpire of what has occurred.\par}
\end{itemize}\par


\vspace{\baselineskip}

\vspace{\baselineskip}

\vspace{\baselineskip}

\vspace{\baselineskip}

\vspace{\baselineskip}

\vspace{\baselineskip}
\begin{Center}
{\fontsize{8pt}{9.6pt}\selectfont 53\par}
\end{Center}\par


\vspace{\baselineskip}

\vspace{\baselineskip}
\begin{adjustwidth}{0.5in}{0.21in}
{\fontsize{9pt}{10.8pt}\selectfont The bowler’s end umpire shall then award 5 Penalty runs to the batting side and inform the captain of the fielding side of the reason for this action.\par}\par

\end{adjustwidth}


\vspace{\baselineskip}
\begin{adjustwidth}{0.5in}{0.22in}
{\fontsize{9pt}{10.8pt}\selectfont Additionally the umpire shall inform the batsmen and, as soon as is practicable, the captain of the batting side of what has occurred.\par}\par

\end{adjustwidth}


\vspace{\baselineskip}
\begin{adjustwidth}{0.5in}{0.03in}
{\fontsize{9pt}{10.8pt}\selectfont If the umpires believe that the act of time wasting was deliberate or repetitive, they may lodge a report under the ICC Code of Conduct. In such circumstances the Captain and/or any individual members of the fielding team responsible for the time wasting will be charged.\par}\par

\end{adjustwidth}


\vspace{\baselineskip}
{\fontsize{11pt}{13.2pt}\selectfont \textbf{41.10\  Batsman wasting time}\par}\par


\vspace{\baselineskip}
\begin{adjustwidth}{0.5in}{0.1in}
{\fontsize{9pt}{10.8pt}\selectfont 41.10.1 It is unfair for a batsman to waste time. In normal circumstances, the striker should always be ready to take strike when the bowler is ready to start his run-up.\par}\par

\end{adjustwidth}


\vspace{\baselineskip}
\begin{adjustwidth}{0.5in}{0.12in}
{\fontsize{9pt}{10.8pt}\selectfont 41.10.2 Should either batsman waste time by failing to meet this requirement, or in any other way, the following procedure shall be adopted. At the first instance, either before the bowler starts his run-up or when the ball becomes dead, as appropriate, the umpire shall\par}\par

\end{adjustwidth}


\vspace{\baselineskip}
\begin{itemize}
	\item {\fontsize{9pt}{10.8pt}\selectfont warn both batsmen and indicate that this is a first and final warning. This warning shall apply throughout the innings. The umpire shall so inform each incoming batsman.\par}\par


\vspace{\baselineskip}
	\item {\fontsize{9pt}{10.8pt}\selectfont inform the other umpire of what has occurred.\par}\par


\vspace{\baselineskip}
	\item {\fontsize{9pt}{10.8pt}\selectfont inform the captain of the fielding side and, as soon as practicable, the captain of the batting side of what has occurred.\par}
\end{itemize}\par


\vspace{\baselineskip}
\begin{adjustwidth}{0.5in}{0.12in}
{\fontsize{9pt}{10.8pt}\selectfont 41.10.3 If there is any further time wasting by any batsman in that innings, the umpire shall, at the appropriate time while the ball is dead\par}\par

\end{adjustwidth}


\vspace{\baselineskip}
\begin{itemize}
	\item {\fontsize{9pt}{10.8pt}\selectfont award 5 Penalty runs to the fielding side.\par}\par


\vspace{\baselineskip}
	\item {\fontsize{9pt}{10.8pt}\selectfont inform the other umpire of the reason for this action.\par}\par


\vspace{\baselineskip}
	\item {\fontsize{9pt}{10.8pt}\selectfont inform the other batsman, the captain of the fielding side and, as soon as practicable, the captain of the batting side of what has occurred.\par}
\end{itemize}\par


\vspace{\baselineskip}
\begin{adjustwidth}{0.5in}{0.06in}
{\fontsize{8pt}{9.6pt}\selectfont If the umpires believe that the act of time wasting was deemed to be deliberate or repetitive, they may lodge a report under the ICC Code of Conduct. In such circumstances the batsman concerned will be charged.\par}\par

\end{adjustwidth}


\vspace{\baselineskip}
{\fontsize{11pt}{13.2pt}\selectfont \textbf{41.11\  The protected area}\par}\par


\vspace{\baselineskip}
\begin{adjustwidth}{0.0in}{0.08in}
{\fontsize{9pt}{10.8pt}\selectfont The protected area is defined as that area of the pitch contained within a rectangle bounded at each end by imaginary lines parallel to the popping creases and 5 ft/1.52 m in front of each, and on the sides by imaginary lines, one each side of the imaginary line joining the centres of the two middle stumps, each parallel to it and 1 ft/30.48 cm from it.\par}\par

\end{adjustwidth}


\vspace{\baselineskip}
{\fontsize{11pt}{13.2pt}\selectfont \textbf{41.12\  Fielder damaging the pitch}\par}\par


\vspace{\baselineskip}
\begin{adjustwidth}{0.5in}{0.26in}
{\fontsize{9pt}{10.8pt}\selectfont 41.12.1 It is unfair to cause deliberate or avoidable damage to the pitch. A fielder will be deemed to be causing avoidable damage if either umpire considers that his presence on the pitch is without reasonable cause.\par}\par

\end{adjustwidth}


\vspace{\baselineskip}
\begin{adjustwidth}{0.5in}{0.21in}
{\fontsize{9pt}{10.8pt}\selectfont 41.12.2 If a fielder causes avoidable damage to the pitch, other than as in clause at the first instance the umpire seeing the contravention shall, when the ball is dead, inform the other umpire. The bowler’s end umpire shall then\par}\par

\end{adjustwidth}


\vspace{\baselineskip}
\begin{itemize}
	\item {\fontsize{9pt}{10.8pt}\selectfont caution the captain of the fielding side and indicate that this is a first and final warning. This warning shall apply throughout the innings.\par}\par


\vspace{\baselineskip}
	\item {\fontsize{9pt}{10.8pt}\selectfont inform the batsmen of what has occurred.\par}
\end{itemize}\par


\vspace{\baselineskip}

\vspace{\baselineskip}

\vspace{\baselineskip}

\vspace{\baselineskip}

\vspace{\baselineskip}

\vspace{\baselineskip}

\vspace{\baselineskip}
\begin{Center}
{\fontsize{8pt}{9.6pt}\selectfont 54\par}
\end{Center}\par


\vspace{\baselineskip}

\vspace{\baselineskip}
\begin{adjustwidth}{0.5in}{0.01in}
{\fontsize{9pt}{10.8pt}\selectfont 41.12.3 If, in that innings, there is any further instance of avoidable damage to the pitch, by any fielder, the umpire seeing the contravention shall, when the ball is dead, inform the other umpire. The bowler’s end umpire shall then\par}\par

\end{adjustwidth}


\vspace{\baselineskip}
\begin{itemize}
	\item {\fontsize{9pt}{10.8pt}\selectfont award 5 Penalty runs to the batting side. Additionally the umpire shall\par}\par


\vspace{\baselineskip}
	\item {\fontsize{9pt}{10.8pt}\selectfont inform the fielding captain of the reason for this action.\par}\par


\vspace{\baselineskip}
	\item {\fontsize{9pt}{10.8pt}\selectfont inform the batsmen and, as soon as practicable, the captain of the batting side of what has occurred.\par}
\end{itemize}\par


\vspace{\baselineskip}
\begin{adjustwidth}{0.5in}{0.07in}
{\fontsize{9pt}{10.8pt}\selectfont The umpires together shall report the occurrence to the ICC Match Referee who shall take such action as is considered appropriate against the fielder concerned.\par}\par

\end{adjustwidth}


\vspace{\baselineskip}
{\fontsize{11pt}{13.2pt}\selectfont \textbf{41.13\  Bowler running on protected area}\par}\par


\vspace{\baselineskip}
\begin{adjustwidth}{0.5in}{0.0in}
{\fontsize{9pt}{10.8pt}\selectfont 41.13.1 It is unfair for a bowler to enter the protected area in his follow-through without reasonable cause, whether or not the ball is delivered.\par}\par

\end{adjustwidth}


\vspace{\baselineskip}
{\fontsize{9pt}{10.8pt}\selectfont 41.13.2\  If a bowler contravenes this clause, at the first instance and when the ball is dead, the umpire shall\par}\par


\vspace{\baselineskip}
\begin{itemize}
	\item {\fontsize{9pt}{10.8pt}\selectfont caution the bowler and inform the other umpire of what has occurred. This caution shall apply to that bowler throughout the innings.\par}\par


\vspace{\baselineskip}
	\item {\fontsize{9pt}{10.8pt}\selectfont inform the captain of the fielding side and the batsmen of what has occurred.\par}
\end{itemize}\par


\vspace{\baselineskip}
\begin{adjustwidth}{0.0in}{0.46in}
\begin{FlushRight}
{\fontsize{9pt}{10.8pt}\selectfont 41.13.3\  If, in that innings, the same bowler again contravenes this clause, the umpire shall repeat the above procedure indicating that this is a final warning. This warning shall also apply throughout the innings.\par}
\end{FlushRight}\par

\end{adjustwidth}


\vspace{\baselineskip}
\begin{adjustwidth}{0.5in}{0.21in}
{\fontsize{9pt}{10.8pt}\selectfont 41.13.4 If, in that innings, the same bowler contravenes this clause a third time, when the ball is dead, the umpire shall,\par}\par

\end{adjustwidth}


\vspace{\baselineskip}
\begin{itemize}
	\item {\fontsize{9pt}{10.8pt}\selectfont direct the captain of the fielding side to suspend the bowler immediately from bowling. If applicable, the over shall be completed by another bowler, who shall neither have bowled any part of the previous over, nor be allowed to bowl any part of the next over. The bowler taken off shall not be allowed to bowl again in that innings.\par}\par


\vspace{\baselineskip}
	\item {\fontsize{9pt}{10.8pt}\selectfont inform the other umpire of the reason for this action.\par}\par


\vspace{\baselineskip}
	\item {\fontsize{9pt}{10.8pt}\selectfont inform the batsmen and, as soon as practicable, the captain of the batting side of what has occurred.\par}
\end{itemize}\par


\vspace{\baselineskip}
\begin{adjustwidth}{0.5in}{0.57in}
{\fontsize{9pt}{10.8pt}\selectfont The umpires may then report the matter to the ICC Match Referee who shall take such action as is considered appropriate against the bowler concerned.\par}\par

\end{adjustwidth}


\vspace{\baselineskip}
{\fontsize{11pt}{13.2pt}\selectfont \textbf{41.14\  Batsman damaging the pitch}\par}\par


\vspace{\baselineskip}
\begin{adjustwidth}{0.5in}{0.1in}
{\fontsize{9pt}{10.8pt}\selectfont 41.14.1 It is unfair to cause deliberate or avoidable damage to the pitch. If the striker enters the protected area in playing or playing at the ball, he must move from it immediately thereafter. A batsman will be deemed to be causing avoidable damage if either umpire considers that his presence on the pitch is without reasonable cause.\par}\par

\end{adjustwidth}


\vspace{\baselineskip}
\begin{adjustwidth}{0.5in}{0.03in}
{\fontsize{9pt}{10.8pt}\selectfont 41.14.2 If either batsman causes deliberate or avoidable damage to the pitch, other than as in clause at the first instance the umpire seeing the contravention shall, when the ball is dead, inform the other umpire of the occurrence. The bowler’s end umpire shall then\par}\par

\end{adjustwidth}


\vspace{\baselineskip}
\begin{itemize}
	\item {\fontsize{9pt}{10.8pt}\selectfont warn both batsmen that the practice is unfair and indicate that this is a first and final warning. This warning shall apply throughout the innings. The umpire shall so inform each incoming batsman.\par}\par


\vspace{\baselineskip}
	\item {\fontsize{9pt}{10.8pt}\selectfont inform the captain of the fielding side and, as soon as practicable, the captain of the batting side of what has occurred.\par}
\end{itemize}\par


\vspace{\baselineskip}
\begin{adjustwidth}{0.5in}{0.12in}
{\fontsize{9pt}{10.8pt}\selectfont 41.14.3 If there is any further instance of avoidable damage to the pitch by any batsman in that innings, the umpire seeing the contravention shall, when the ball is dead, inform the other umpire of the occurrence.\par}\par

\end{adjustwidth}


\vspace{\baselineskip}

\vspace{\baselineskip}

\vspace{\baselineskip}
\begin{Center}
{\fontsize{8pt}{9.6pt}\selectfont 55\par}
\end{Center}\par


\vspace{\baselineskip}
\begin{adjustwidth}{0.5in}{0.0in}
{\fontsize{9pt}{10.8pt}\selectfont The bowler’s end umpire shall\par}\par

\end{adjustwidth}


\vspace{\baselineskip}
\begin{itemize}
	\item {\fontsize{9pt}{10.8pt}\selectfont disallow all runs to the batting side\par}\par


\vspace{\baselineskip}
	\item {\fontsize{9pt}{10.8pt}\selectfont return any not out batsman to his original end\par}\par


\vspace{\baselineskip}
	\item {\fontsize{9pt}{10.8pt}\selectfont signal No ball or Wide to the scorers if applicable.\par}\par


\vspace{\baselineskip}
	\item {\fontsize{9pt}{10.8pt}\selectfont award 5 Penalty runs to the fielding side.\par}\par


\vspace{\baselineskip}
	\item {\fontsize{9pt}{10.8pt}\selectfont award any other 5-run Penalty that is applicable except for Penalty runs under clause (Protective helmets belonging to the fielding side).\par}\par


\vspace{\baselineskip}
	\item {\fontsize{9pt}{10.8pt}\selectfont Inform the captain of the fielding side and, as soon as practicable, the captain of the batting side of the reason for this action.\par}
\end{itemize}\par


\vspace{\baselineskip}
\begin{adjustwidth}{0.5in}{0.06in}
{\fontsize{9pt}{10.8pt}\selectfont The umpires together shall report the occurrence to the ICC Match Referee who shall take such action as is considered appropriate against the batsman concerned.\par}\par

\end{adjustwidth}


\vspace{\baselineskip}
{\fontsize{11pt}{13.2pt}\selectfont \textbf{41.15\  Striker in protected area}\par}\par


\vspace{\baselineskip}
\begin{adjustwidth}{0.5in}{0.31in}
{\fontsize{9pt}{10.8pt}\selectfont 41.15.1 The striker shall not adopt a stance in the protected area or so close to it that frequent encroachment is inevitable.\par}\par

\end{adjustwidth}


\vspace{\baselineskip}
\begin{adjustwidth}{0.5in}{0.31in}
{\fontsize{9pt}{10.8pt}\selectfont The striker may mark a guard on the pitch provided that no mark is unreasonably close to the protected area.\par}\par

\end{adjustwidth}


\vspace{\baselineskip}
\begin{adjustwidth}{0.5in}{0.06in}
{\fontsize{9pt}{10.8pt}\selectfont 41.15.2 If either umpire considers that the striker is in breach of any of the conditions in clause if the bowler has not entered the delivery stride, he/she shall immediately call Dead ball, otherwise, wait until the ball is dead; he/she shall then inform the other umpire of the occurrence.\par}\par

\end{adjustwidth}


\vspace{\baselineskip}
\begin{adjustwidth}{0.5in}{0.0in}
{\fontsize{9pt}{10.8pt}\selectfont The bowler’s end umpire shall then\par}\par

\end{adjustwidth}


\vspace{\baselineskip}
\begin{itemize}
	\item {\fontsize{9pt}{10.8pt}\selectfont warn the striker that the practice is unfair and indicate that this is a first and final warning. This warning shall apply throughout the innings. The umpire shall so inform the non-striker and each incoming batsman.\par}\par


\vspace{\baselineskip}
	\item {\fontsize{9pt}{10.8pt}\selectfont inform the captain of the fielding side and, as soon as practicable, the captain of the batting side of what has occurred.\par}
\end{itemize}\par


\vspace{\baselineskip}
\begin{adjustwidth}{0.5in}{0.08in}
{\fontsize{9pt}{10.8pt}\selectfont 41.15.3 If there is any further breach of any of the conditions in clause by any batsman in that innings, the umpire seeing the contravention shall, if the bowler has not entered his delivery stride, immediately call and signal Dead ball, otherwise, he/she shall wait until the ball is dead and then inform the other umpire of the occurrence.\par}\par

\end{adjustwidth}


\vspace{\baselineskip}
\begin{adjustwidth}{0.5in}{0.0in}
{\fontsize{9pt}{10.8pt}\selectfont The bowler’s end umpire shall\par}\par

\end{adjustwidth}


\vspace{\baselineskip}
\begin{itemize}
	\item {\fontsize{9pt}{10.8pt}\selectfont disallow all runs to the batting side\par}\par


\vspace{\baselineskip}
	\item {\fontsize{9pt}{10.8pt}\selectfont return any not out batsman to his original end\par}\par


\vspace{\baselineskip}
	\item {\fontsize{9pt}{10.8pt}\selectfont signal No ball or Wide to the scorers if applicable.\par}\par


\vspace{\baselineskip}
	\item {\fontsize{9pt}{10.8pt}\selectfont award 5 Penalty runs to the fielding side.\par}\par


\vspace{\baselineskip}
	\item {\fontsize{9pt}{10.8pt}\selectfont award any other 5-run Penalty that is applicable except for Penalty runs under clause (Protective helmets belonging to the fielding side).\par}\par


\vspace{\baselineskip}
	\item {\fontsize{9pt}{10.8pt}\selectfont inform the captain of the fielding side and, as soon as practicable, the captain of the batting side of the reason for this action.\par}
\end{itemize}\par


\vspace{\baselineskip}
\begin{adjustwidth}{0.5in}{0.06in}
{\fontsize{9pt}{10.8pt}\selectfont The umpires together shall report the occurrence to the ICC Match Referee who shall take such action as is considered appropriate against the batsman concerned.\par}\par

\end{adjustwidth}


\vspace{\baselineskip}

\vspace{\baselineskip}

\vspace{\baselineskip}

\vspace{\baselineskip}

\vspace{\baselineskip}

\vspace{\baselineskip}
\begin{Center}
{\fontsize{8pt}{9.6pt}\selectfont 56\par}
\end{Center}\par


\vspace{\baselineskip}
{\fontsize{11pt}{13.2pt}\selectfont \textbf{41.16\  Non-striker leaving his ground early}\par}\par


\vspace{\baselineskip}
\begin{adjustwidth}{0.0in}{0.12in}
{\fontsize{9pt}{10.8pt}\selectfont If the non-striker is out of his ground from the moment the ball comes into play to the instant when the bowler would normally have been expected to release the ball, the bowler is permitted to attempt to run him out. Whether the attempt is successful or not, the ball shall not count as one in the over.\par}\par

\end{adjustwidth}


\vspace{\baselineskip}
\begin{adjustwidth}{0.0in}{0.43in}
{\fontsize{9pt}{10.8pt}\selectfont If the bowler fails in an attempt to run out the non-striker, the umpire shall call and signal Dead ball as soon as possible.\par}\par

\end{adjustwidth}


\vspace{\baselineskip}
{\fontsize{11pt}{13.2pt}\selectfont \textbf{41.17\  Batsmen stealing a run}\par}\par


\vspace{\baselineskip}
{\fontsize{9pt}{10.8pt}\selectfont 41.17.1\  It is unfair for the batsmen to attempt to steal a run during the bowler’s run-up.\par}\par


\vspace{\baselineskip}
\begin{adjustwidth}{0.5in}{0.43in}
{\fontsize{9pt}{10.8pt}\selectfont Unless the bowler attempts to run out either batsman – see clauses and (Bowler throwing towards striker’s end before delivery) – the umpire shall\par}\par

\end{adjustwidth}


\vspace{\baselineskip}
\begin{itemize}
	\item {\fontsize{9pt}{10.8pt}\selectfont call and signal Dead ball as soon as the batsmen cross in such an attempt.\par}\par


\vspace{\baselineskip}
	\item {\fontsize{9pt}{10.8pt}\selectfont inform the other umpire of the reason for this action.\par}
\end{itemize}\par


\vspace{\baselineskip}
\begin{adjustwidth}{0.5in}{0.0in}
{\fontsize{9pt}{10.8pt}\selectfont The bowler’s end umpire shall then\par}\par

\end{adjustwidth}


\vspace{\baselineskip}
\begin{itemize}
	\item {\fontsize{9pt}{10.8pt}\selectfont return the batsmen to their original ends.\par}\par


\vspace{\baselineskip}
	\item {\fontsize{9pt}{10.8pt}\selectfont award 5 Penalty runs to the fielding side.\par}\par


\vspace{\baselineskip}
	\item {\fontsize{9pt}{10.8pt}\selectfont inform the batsmen, the captain of the fielding side and, as soon as practicable, the captain of the batting side, of the reason for this action.\par}
\end{itemize}\par


\vspace{\baselineskip}
\begin{adjustwidth}{0.5in}{0.57in}
{\fontsize{9pt}{10.8pt}\selectfont The umpires may then report the matter to the ICC Match Referee who shall take such action as is considered appropriate against the batsman concerned.\par}\par

\end{adjustwidth}


\vspace{\baselineskip}
{\fontsize{11pt}{13.2pt}\selectfont \textbf{41.18\  Penalty runs}\par}\par


\vspace{\baselineskip}
\begin{adjustwidth}{0.5in}{0.0in}
{\fontsize{9pt}{10.8pt}\selectfont 41.18.1 When Penalty runs are awarded to either side, when the ball is dead the umpire shall signal the Penalty runs to the scorers. See clause (Signals).\par}\par

\end{adjustwidth}


\vspace{\baselineskip}
\begin{adjustwidth}{0.5in}{0.25in}
{\fontsize{9pt}{10.8pt}\selectfont 41.18.2 Penalty runs shall be awarded in each case where these Playing Conditions require the award, even if a result has already been achieved. See clause (Winning hit or extras).\par}\par

\end{adjustwidth}


\vspace{\baselineskip}
\begin{adjustwidth}{0.5in}{0.03in}
{\fontsize{9pt}{10.8pt}\selectfont Note, however, that the restrictions on awarding Penalty runs, in clauses (Leg byes not to be awarded), (Runs scored from ball lawfully struck more than once) and (Protective helmets belonging to the fielding side), will apply.\par}\par

\end{adjustwidth}


\vspace{\baselineskip}
\begin{adjustwidth}{0.5in}{0.01in}
{\fontsize{9pt}{10.8pt}\selectfont 41.18.3 When 5 Penalty runs are awarded to the batting side under any of clauses (Player returning without permission), (Fielding the ball), or (Protective helmets belonging to the fielding side) or under  or then\par}\par

\end{adjustwidth}


\vspace{\baselineskip}
\begin{itemize}
	\item {\fontsize{9pt}{10.8pt}\selectfont they shall be scored as Penalty extras and shall be in addition to any other penalties.\par}\par


\vspace{\baselineskip}
	\item {\fontsize{9pt}{10.8pt}\selectfont they are awarded when the ball is dead and shall not be regarded as runs scored from either the immediately preceding delivery or the immediately following delivery, and shall be in addition to any runs from those deliveries.\par}\par


\vspace{\baselineskip}
	\item {\fontsize{9pt}{10.8pt}\selectfont the batsmen shall not change ends solely by reason of the 5 run penalty.\par}
\end{itemize}\par


\vspace{\baselineskip}
\begin{adjustwidth}{0.5in}{0.08in}
{\fontsize{9pt}{10.8pt}\selectfont 41.18.4 When 5 Penalty runs are awarded to the fielding side, under clause (Deliberate short runs), or under or they shall be added as Penalty extras to that side’s total of runs in its most recently completed innings. If the fielding side has not completed an innings, the 5 Penalty runs shall be added to the score in its next innings.\par}\par

\end{adjustwidth}


\vspace{\baselineskip}

\vspace{\baselineskip}

\vspace{\baselineskip}

\vspace{\baselineskip}

\vspace{\baselineskip}

\vspace{\baselineskip}

\vspace{\baselineskip}
\begin{Center}
{\fontsize{8pt}{9.6pt}\selectfont 57\par}
\end{Center}\par


\vspace{\baselineskip}
{\fontsize{11pt}{13.2pt}\selectfont \textbf{41.19\  Unfair actions}\par}\par


\vspace{\baselineskip}
\begin{adjustwidth}{0.5in}{0.06in}
\begin{justify}
{\fontsize{9pt}{10.8pt}\selectfont 41.19.1 If an umpire considers that any action by a player, not covered in these Playing Conditions, is unfair, he/she shall call and signal Dead ball, if appropriate, as soon as it becomes clear that the call will not disadvantage the non-offending side, and report the matter to the other umpire.\par}
\end{justify}\par

\end{adjustwidth}


\vspace{\baselineskip}
\begin{adjustwidth}{0.5in}{0.0in}
{\fontsize{9pt}{10.8pt}\selectfont The bowler’s end umpire shall\par}\par

\end{adjustwidth}


\vspace{\baselineskip}
\begin{adjustwidth}{0.49in}{0.0in}
{\fontsize{9pt}{10.8pt}\selectfont 41.19.1.1 \tabto{1.17in} If this is a first offence by that side\par}\par

\end{adjustwidth}


\vspace{\baselineskip}
\begin{itemize}
	\item {\fontsize{9pt}{10.8pt}\selectfont summon the offending player’s captain and issue a first and final warning which shall apply to all members of the team for the remainder of the match.\par}\par


\vspace{\baselineskip}
	\item {\fontsize{9pt}{10.8pt}\selectfont warn the offending player’s captain that any further such offence by any member of his team shall result in the award of 5 Penalty runs to the opposing team.\par}
\end{itemize}\par


\vspace{\baselineskip}
\begin{adjustwidth}{0.49in}{0.0in}
{\fontsize{9pt}{10.8pt}\selectfont 41.19.1.2 \tabto{1.17in} If this is a second or subsequent offence by that side\par}\par

\end{adjustwidth}


\vspace{\baselineskip}
\begin{adjustwidth}{1.12in}{0.0in}
{\fontsize{9pt}{10.8pt}\selectfont - award 5 Penalty runs to the opposing side\par}\par

\end{adjustwidth}


\vspace{\baselineskip}
\begin{adjustwidth}{1.18in}{0.01in}
{\fontsize{9pt}{10.8pt}\selectfont 41.19.1.3 \tabto{1.17in} The umpires may then report the matter to the ICC Match Referee who shall take such action as is considered appropriate against the player concerned.\par}\par

\end{adjustwidth}


\vspace{\baselineskip}

\vspace{\baselineskip}
{\fontsize{16pt}{19.2pt}\selectfont \textbf{42 PLAYERS’ CONDUCT}\par}\par


\vspace{\baselineskip}
{\fontsize{11pt}{13.2pt}\selectfont \textbf{42.1 \tabto{0.47in} Serious misconduct}\par}\par


\vspace{\baselineskip}
\begin{adjustwidth}{0.5in}{0.01in}
{\fontsize{9pt}{10.8pt}\selectfont 42.1.1 \tabto{0.49in} The umpires shall act upon any serious misconduct. The relevant offences and the corresponding actions by the umpires are identified in clause These offences correspond with Level 4 offences in the ICC Code of Conduct. Level 1 to Level 3 offences continue to be dealt with separately under the ICC Code of Conduct.\par}\par

\end{adjustwidth}


\vspace{\baselineskip}
\begin{adjustwidth}{0.5in}{0.15in}
\begin{justify}
{\fontsize{9pt}{10.8pt}\selectfont 42.1.2 \tabto{0.49in} If either umpire considers that a player has committed one of these offences at any time during the match, the umpire concerned shall call and signal Dead ball. This call may be delayed until the umpire is satisfied that it will not disadvantage the non-offending side.\par}
\end{justify}\par

\end{adjustwidth}


\vspace{\baselineskip}
\begin{adjustwidth}{0.5in}{0.07in}
{\fontsize{9pt}{10.8pt}\selectfont 42.1.3 \tabto{0.49in} The umpire concerned shall report the matter to the other umpire and together they shall decide whether an offence has been committed. The umpires may also consult with the third umpire and the match referee, who may review any audio or video replays to confirm whether an offence has been committed. If so, the umpires shall then apply the related sanctions.\par}\par

\end{adjustwidth}


\vspace{\baselineskip}
\begin{adjustwidth}{0.5in}{0.22in}
{\fontsize{9pt}{10.8pt}\selectfont 42.1.4 \tabto{0.49in} {\fontsize{8pt}{9.6pt}\selectfont If the offence is committed by a batsman, the umpires shall summon the offending player’s captain to the field. Solely for the purpose of this clause, the batsmen at the wicket may not deputise for their captain.\par}\par}\par

\end{adjustwidth}


\vspace{\baselineskip}
{\fontsize{11pt}{13.2pt}\selectfont \textbf{42.2 \tabto{0.47in} Level 4 offences and action by umpires}\par}\par


\vspace{\baselineskip}
{\fontsize{9pt}{10.8pt}\selectfont 42.2.1 \tabto{0.49in} {\fontsize{8pt}{9.6pt}\selectfont Any of the following actions by a player shall constitute a Level 4 offence:\par}\par}\par


\vspace{\baselineskip}
\begin{itemize}
	\item {\fontsize{9pt}{10.8pt}\selectfont threatening to assault an umpire\par}\par


\vspace{\baselineskip}
	\item {\fontsize{9pt}{10.8pt}\selectfont making inappropriate and deliberate physical contact with an umpire\par}\par


\vspace{\baselineskip}
	\item {\fontsize{9pt}{10.8pt}\selectfont physically assaulting a player or any other person\par}\par


\vspace{\baselineskip}
	\item {\fontsize{9pt}{10.8pt}\selectfont committing any other act of violence.\par}
\end{itemize}\par


\vspace{\baselineskip}
{\fontsize{9pt}{10.8pt}\selectfont 42.2.2 \tabto{0.49in} If such an offence is committed, to shall be implemented.\par}\par


\vspace{\baselineskip}
\begin{adjustwidth}{0.49in}{0.0in}
{\fontsize{9pt}{10.8pt}\selectfont 42.2.2.1 \tabto{1.17in} The umpire shall call Time.\par}\par

\end{adjustwidth}


\vspace{\baselineskip}
\begin{adjustwidth}{1.18in}{0.04in}
{\fontsize{9pt}{10.8pt}\selectfont 42.2.2.2 \tabto{1.17in} Together the umpires shall summon and inform the offending player’s captain that an offence at this Level has occurred.\par}\par

\end{adjustwidth}


\vspace{\baselineskip}

\vspace{\baselineskip}

\vspace{\baselineskip}

\vspace{\baselineskip}
\begin{Center}
{\fontsize{8pt}{9.6pt}\selectfont 58\par}
\end{Center}\par


\vspace{\baselineskip}

\vspace{\baselineskip}
\begin{adjustwidth}{1.18in}{0.07in}
{\fontsize{9pt}{10.8pt}\selectfont 42.2.2.3 \tabto{1.17in} The umpires shall instruct the captain to remove the offending player immediately from the field of play for the remainder of the match and shall apply the following:\par}\par

\end{adjustwidth}


\vspace{\baselineskip}
\begin{adjustwidth}{1.97in}{0.06in}
\begin{justify}
{\fontsize{9pt}{10.8pt}\selectfont 42.2.2.3.1 \tabto{1.96in} If the offending player is a fielder, no substitute shall be allowed for him. He is to be recorded as Retired – out at the commencement of any subsequent innings in which his team is the batting side.\par}
\end{justify}\par

\end{adjustwidth}


\vspace{\baselineskip}
\begin{adjustwidth}{1.97in}{0.47in}
\begin{justify}
{\fontsize{9pt}{10.8pt}\selectfont 42.2.2.3.2 \tabto{1.96in} If a bowler is suspended mid-over, then that over must be completed by a different bowler, who shall not have bowled the previous over nor shall be permitted to bowl the next over.\par}
\end{justify}\par

\end{adjustwidth}


\vspace{\baselineskip}
\begin{adjustwidth}{1.97in}{0.01in}
{\fontsize{9pt}{10.8pt}\selectfont 42.2.2.3.3 \tabto{1.96in} If the offending player is a batsman he is to be recorded as Retired – out in the current innings, unless he has been dismissed under any of clauses to and at the commencement of any subsequent innings in which his team is the batting side. If no further batsman is available to bat, the innings is completed.\par}\par

\end{adjustwidth}


\vspace{\baselineskip}
\begin{adjustwidth}{0.49in}{0.0in}
{\fontsize{9pt}{10.8pt}\selectfont 42.2.2.4 \tabto{1.17in} As soon as practicable, the umpire shall:\par}\par

\end{adjustwidth}


\vspace{\baselineskip}
\begin{itemize}
	\item {\fontsize{9pt}{10.8pt}\selectfont award 5 Penalty runs to the opposing team\par}\par


\vspace{\baselineskip}
	\item {\fontsize{9pt}{10.8pt}\selectfont signal the Level 4 penalty to the scorers\par}\par


\vspace{\baselineskip}
	\item {\fontsize{9pt}{10.8pt}\selectfont call Play.\par}
\end{itemize}\par


\vspace{\baselineskip}
\begin{adjustwidth}{1.18in}{0.36in}
{\fontsize{9pt}{10.8pt}\selectfont 42.2.2.5 \tabto{1.17in} The umpires shall then report the matter to the ICC Match Referee under the ICC Code of Conduct.\par}\par

\end{adjustwidth}


\vspace{\baselineskip}
{\fontsize{11pt}{13.2pt}\selectfont \textbf{42.3 \tabto{0.47in} Captain refusing to remove a player from the field}\par}\par


\vspace{\baselineskip}
\begin{adjustwidth}{0.5in}{0.29in}
{\fontsize{9pt}{10.8pt}\selectfont 42.3.1 \tabto{0.49in} If a captain refuses to carry out an instruction under the umpires shall invoke clause (ICC Match Referee awarding a match).\par}\par

\end{adjustwidth}


\vspace{\baselineskip}
\begin{adjustwidth}{0.5in}{0.11in}
\begin{justify}
{\fontsize{9pt}{10.8pt}\selectfont 42.3.2 \tabto{0.49in} If both captains refuse to carry out instructions under in respect of the same incident, the umpires shall instruct the players to leave the field. The match is not concluded as in clause and there shall be no result under clause \par}
\end{justify}\par

\end{adjustwidth}


\vspace{\baselineskip}
{\fontsize{11pt}{13.2pt}\selectfont \textbf{42.4 \tabto{0.47in} Additional points relating to Level 4 offences}\par}\par


\vspace{\baselineskip}
\begin{adjustwidth}{0.5in}{0.04in}
{\fontsize{9pt}{10.8pt}\selectfont 42.4.1 \tabto{0.49in} If a player, while acting as wicket-keeper, commits a Level 4 offence, clause shall not apply, meaning that only a nominated player may keep wicket, even if another fielder becomes injured or ill and is replaced by a substitute.\par}\par

\end{adjustwidth}


\vspace{\baselineskip}
\begin{adjustwidth}{0.5in}{0.21in}
{\fontsize{9pt}{10.8pt}\selectfont 42.4.2 \tabto{0.49in} A nominated player who has a substitute will also suffer the penalty for any Level 4 offence committed by the substitute. However, only the substitute will be reported under clause \par}\par

\end{adjustwidth}


\vspace{\baselineskip}

\vspace{\baselineskip}

\vspace{\baselineskip}

\vspace{\baselineskip}

\vspace{\baselineskip}

\vspace{\baselineskip}

\vspace{\baselineskip}

\vspace{\baselineskip}

\vspace{\baselineskip}

\vspace{\baselineskip}

\vspace{\baselineskip}

\vspace{\baselineskip}

\vspace{\baselineskip}

\vspace{\baselineskip}

\vspace{\baselineskip}

\vspace{\baselineskip}

\vspace{\baselineskip}

\vspace{\baselineskip}

\vspace{\baselineskip}

\vspace{\baselineskip}

\vspace{\baselineskip}

\vspace{\baselineskip}

\vspace{\baselineskip}

\vspace{\baselineskip}
\begin{Center}
{\fontsize{8pt}{9.6pt}\selectfont 59\par}
\end{Center}\par


\vspace{\baselineskip}
\begin{adjustwidth}{2.42in}{0.0in}
\begin{Center}
{\fontsize{9pt}{10.8pt}\selectfont \textbf{Appendices to ICC Twenty20 International Playing Conditions (incorporating the 2017 Code of the MCC Laws of Cricket) Effective 1 October 2017}\par}
\end{Center}\par

\end{adjustwidth}


\vspace{\baselineskip}
\begin{enumerate}
	\item {\fontsize{9pt}{10.8pt}\selectfont Definitions\par}\par


\vspace{\baselineskip}
	\item {\fontsize{9pt}{10.8pt}\selectfont Equipment\par}\par


\vspace{\baselineskip}
\begin{enumerate}
	\item {\fontsize{9pt}{10.8pt}\selectfont The bat\par}\par


\vspace{\baselineskip}
	\item {\fontsize{9pt}{10.8pt}\selectfont The wickets\par}\par


\vspace{\baselineskip}
	\item {\fontsize{9pt}{10.8pt}\selectfont Wicket-keeping gloves\par}\par


\vspace{\baselineskip}

\end{enumerate}
	\item {\fontsize{9pt}{10.8pt}\selectfont The venue\par}\par


\vspace{\baselineskip}
\begin{enumerate}
	\item {\fontsize{9pt}{10.8pt}\selectfont The pitch and the creases\par}\par


\vspace{\baselineskip}
	\item {\fontsize{9pt}{10.8pt}\selectfont Advertising on grounds, perimeter boards and sight-screens\par}\par


\vspace{\baselineskip}
	\item {\fontsize{9pt}{10.8pt}\selectfont Markings on outfield\par}\par


\vspace{\baselineskip}

\end{enumerate}
	\item {\fontsize{9pt}{10.8pt}\selectfont Decision Review System (DRS) and Third Umpire Protocol\par}\par


\vspace{\baselineskip}
	\item {\fontsize{9pt}{10.8pt}\selectfont Calculations\par}\par


\vspace{\baselineskip}
	\item {\fontsize{9pt}{10.8pt}\selectfont The Super Over\par}
\end{enumerate}\par


\vspace{\baselineskip}

\vspace{\baselineskip}

\vspace{\baselineskip}

\vspace{\baselineskip}

\vspace{\baselineskip}

\vspace{\baselineskip}

\vspace{\baselineskip}

\vspace{\baselineskip}

\vspace{\baselineskip}

\vspace{\baselineskip}

\vspace{\baselineskip}

\vspace{\baselineskip}

\vspace{\baselineskip}

\vspace{\baselineskip}

\vspace{\baselineskip}

\vspace{\baselineskip}

\vspace{\baselineskip}

\vspace{\baselineskip}

\vspace{\baselineskip}

\vspace{\baselineskip}

\vspace{\baselineskip}

\vspace{\baselineskip}

\vspace{\baselineskip}

\vspace{\baselineskip}

\vspace{\baselineskip}

\vspace{\baselineskip}

\vspace{\baselineskip}

\vspace{\baselineskip}

\vspace{\baselineskip}

\vspace{\baselineskip}

\vspace{\baselineskip}

\vspace{\baselineskip}

\vspace{\baselineskip}

\vspace{\baselineskip}

\vspace{\baselineskip}

\vspace{\baselineskip}

\vspace{\baselineskip}

\vspace{\baselineskip}
\begin{Center}
{\fontsize{8pt}{9.6pt}\selectfont 60\par}
\end{Center}\par


\vspace{\baselineskip}
\begin{Center}
{\fontsize{11pt}{13.2pt}\selectfont \textbf{Appendix A}\par}
\end{Center}\par


\vspace{\baselineskip}
\begin{Center}
{\fontsize{11pt}{13.2pt}\selectfont \textbf{Definitions}\par}
\end{Center}\par


\vspace{\baselineskip}
{\fontsize{16pt}{19.2pt}\selectfont \textbf{1 \tabto{0.47in} The match}\par}\par


\vspace{\baselineskip}
{\fontsize{9pt}{10.8pt}\selectfont 1.1 \tabto{0.39in} {\fontsize{8pt}{9.6pt}\selectfont \textbf{The game }is used in these Playing Conditions as a general term meaning the Game of Cricket.\par}\par}\par


\vspace{\baselineskip}
\begin{adjustwidth}{0.4in}{0.65in}
{\fontsize{9pt}{10.8pt}\selectfont 1.2 \tabto{0.39in} \textbf{A match }is a single Twenty20 International match between two teams, played under these Playing\textbf{ }Conditions.\par}\par

\end{adjustwidth}


\vspace{\baselineskip}
{\fontsize{9pt}{10.8pt}\selectfont 1.3 \tabto{0.39in} {\fontsize{8pt}{9.6pt}\selectfont \textbf{T20I }is an abbreviation for Twenty20 International.\par}\par}\par


\vspace{\baselineskip}
\begin{adjustwidth}{0.4in}{0.29in}
{\fontsize{9pt}{10.8pt}\selectfont 1.4 \tabto{0.39in} A \textbf{Super Over} is a procedure that may be adopted for determining the result of a tied match, as set out in Appendix F.\par}\par

\end{adjustwidth}


\vspace{\baselineskip}
{\fontsize{9pt}{10.8pt}\selectfont 1.5 \tabto{0.39in} {\fontsize{8pt}{9.6pt}\selectfont \textbf{The toss }is the toss for choice of innings.\par}\par}\par


\vspace{\baselineskip}
{\fontsize{9pt}{10.8pt}\selectfont 1.6 \tabto{0.39in} {\fontsize{8pt}{9.6pt}\selectfont \textbf{Before the toss }is at any time before the toss on the day of the match.\par}\par}\par


\vspace{\baselineskip}
{\fontsize{9pt}{10.8pt}\selectfont 1.7 \tabto{0.39in} \textbf{Before the match }is at any time before the toss, not restricted to the day of the match.\par}\par


\vspace{\baselineskip}
\begin{adjustwidth}{0.4in}{0.01in}
{\fontsize{9pt}{10.8pt}\selectfont 1.8 \tabto{0.39in} \textbf{During the match }is at any time after the toss until the conclusion of the match, whether play is in progress or\textbf{ }not.\par}\par

\end{adjustwidth}


\vspace{\baselineskip}
{\fontsize{9pt}{10.8pt}\selectfont 1.9 \tabto{0.39in} {\fontsize{8pt}{9.6pt}\selectfont \textbf{Playing time }is\ any time between the call of Play and the call of Time.  See clauses (Call of Play) and\par}\par}\par


\vspace{\baselineskip}
\begin{adjustwidth}{0.4in}{0.0in}
{\fontsize{9pt}{10.8pt}\selectfont (Call of Time).\par}\par

\end{adjustwidth}


\vspace{\baselineskip}
{\fontsize{9pt}{10.8pt}\selectfont 1.10 \tabto{0.39in} {\fontsize{8pt}{9.6pt}\selectfont \textbf{Conduct of the match }includes any action relevant to the match at any time.\par}\par}\par


\vspace{\baselineskip}
\begin{adjustwidth}{0.4in}{0.14in}
{\fontsize{9pt}{10.8pt}\selectfont 1.11 \tabto{0.39in} \textbf{Ground Authority }is the entity responsible for the selection and preparation of the pitch and other functions\textbf{ }relating to the hosting and management of the match, including any agents acting on their behalf (including but not limited to the curator or other ground staff).\par}\par

\end{adjustwidth}


\vspace{\baselineskip}
{\fontsize{9pt}{10.8pt}\selectfont 1.12 \tabto{0.39in} {\fontsize{8pt}{9.6pt}\selectfont \textbf{Home Board }is the ICC member responsible for the home team and the hosting of the match.\par}\par}\par


\vspace{\baselineskip}
{\fontsize{9pt}{10.8pt}\selectfont 1.13 \tabto{0.39in} {\fontsize{8pt}{9.6pt}\selectfont \textbf{Visiting Board }is the ICC member responsible for the visiting team.\par}\par}\par


\vspace{\baselineskip}
\begin{adjustwidth}{0.4in}{0.04in}
{\fontsize{9pt}{10.8pt}\selectfont 1.14 \tabto{0.39in} \textbf{The Spirit of Cricket }refers to the values of respect and fair play that underpin the game of cricket, as set out\textbf{ }in the Preamble to these Playing Conditions.\par}\par

\end{adjustwidth}


\vspace{\baselineskip}
\begin{adjustwidth}{0.4in}{0.5in}
{\fontsize{9pt}{10.8pt}\selectfont 1.15 \tabto{0.39in} \textbf{The ICC Code of Conduct }is the ICC Code of Conduct for Players and Player Support Personnel, as\textbf{ }amended from time to time.\par}\par

\end{adjustwidth}


\vspace{\baselineskip}
{\fontsize{16pt}{19.2pt}\selectfont \textbf{2 \tabto{0.29in} }{\fontsize{15pt}{18.0pt}\selectfont \textbf{Implements and equipment}\par}\par}\par


\vspace{\baselineskip}
{\fontsize{9pt}{10.8pt}\selectfont 2.1 \tabto{0.39in} \textbf{Implements used in the match }are the bat, the ball, the stumps and bails.\par}\par


\vspace{\baselineskip}
{\fontsize{9pt}{10.8pt}\selectfont 2.2 \tabto{0.39in} {\fontsize{8pt}{9.6pt}\selectfont \textbf{External protective equipment }is any visible item of apparel worn for protection against external blows.\par}\par}\par


\vspace{\baselineskip}
\begin{adjustwidth}{0.49in}{0.21in}
{\fontsize{9pt}{10.8pt}\selectfont For a batsman, items permitted are a protective helmet, external leg guards (batting pads), batting gloves and, if visible, forearm guards.\par}\par

\end{adjustwidth}


\vspace{\baselineskip}
\begin{adjustwidth}{0.49in}{0.14in}
{\fontsize{9pt}{10.8pt}\selectfont For a fielder, only a protective helmet is permitted, except in the case of a wicket-keeper, for whom wicket-keeping pads and gloves are also permitted.\par}\par

\end{adjustwidth}


\vspace{\baselineskip}
\begin{adjustwidth}{0.4in}{0.03in}
{\fontsize{9pt}{10.8pt}\selectfont 2.3 \tabto{0.39in} \textbf{A protective helmet }is headwear made of hard material and designed to protect the head or the face or both,\textbf{ }which shall (in line with the Clothing and Equipment Regulations) be certified to BS7928:2013. For the purposes of interpreting these Playing Conditions, such a description will include faceguards.\par}\par

\end{adjustwidth}


\vspace{\baselineskip}
\begin{adjustwidth}{0.4in}{0.21in}
{\fontsize{9pt}{10.8pt}\selectfont 2.4 \tabto{0.39in} \textbf{Equipment }– a batsman’s equipment is his/her bat\textbf{ }as defined above, together with any external protective\textbf{ }equipment he is wearing.\par}\par

\end{adjustwidth}


\vspace{\baselineskip}
\begin{adjustwidth}{0.49in}{0.0in}
{\fontsize{9pt}{10.8pt}\selectfont A fielder’s equipment is any external protective equipment that he is wearing.\par}\par

\end{adjustwidth}


\vspace{\baselineskip}

\vspace{\baselineskip}

\vspace{\baselineskip}

\vspace{\baselineskip}

\vspace{\baselineskip}
\begin{Center}
{\fontsize{8pt}{9.6pt}\selectfont 61\par}
\end{Center}\par


\vspace{\baselineskip}
{\fontsize{9pt}{10.8pt}\selectfont 2.5 \tabto{0.39in} {\fontsize{8pt}{9.6pt}\selectfont \textbf{The bat }–\textbf{ }the following are to be considered as part of the bat:\par}\par}\par


\vspace{\baselineskip}
\begin{itemize}
	\item {\fontsize{9pt}{10.8pt}\selectfont the whole of the bat itself.\par}\par


\vspace{\baselineskip}
	\item {\fontsize{9pt}{10.8pt}\selectfont the whole of a glove (or gloves) worn on the hand (or hands) holding the bat.\par}\par


\vspace{\baselineskip}
	\item {\fontsize{9pt}{10.8pt}\selectfont the hand (or hands) holding the bat, if the batsman is not wearing a glove on that hand or on those\par}
\end{itemize}\par


\vspace{\baselineskip}
\begin{adjustwidth}{0.49in}{0.0in}
{\fontsize{9pt}{10.8pt}\selectfont hands.\par}\par

\end{adjustwidth}


\vspace{\baselineskip}
\begin{adjustwidth}{0.4in}{0.01in}
{\fontsize{9pt}{10.8pt}\selectfont 2.6 \tabto{0.39in} \textbf{Held in batsman’s hand}. Contact between a batsman’s hand, or glove worn on his/her hand, and any part of\textbf{ }the bat shall constitute the bat being held in that hand.\par}\par

\end{adjustwidth}


\vspace{\baselineskip}
{\fontsize{16pt}{19.2pt}\selectfont \textbf{3 \tabto{0.29in} }{\fontsize{15pt}{18.0pt}\selectfont \textbf{The playing area}\par}\par}\par


\vspace{\baselineskip}
{\fontsize{9pt}{10.8pt}\selectfont 3.1 \tabto{0.39in} {\fontsize{8pt}{9.6pt}\selectfont \textbf{The field of play }is the area contained within the boundary.\par}\par}\par


\vspace{\baselineskip}
{\fontsize{9pt}{10.8pt}\selectfont 3.2 \tabto{0.39in} {\fontsize{8pt}{9.6pt}\selectfont \textbf{The square }is a specially prepared area of the field of play within which the match pitch is situated.\par}\par}\par


\vspace{\baselineskip}
{\fontsize{9pt}{10.8pt}\selectfont 3.3 \tabto{0.39in} {\fontsize{8pt}{9.6pt}\selectfont \textbf{The outfield }is that part of the field of play between the square and the boundary.\par}\par}\par


\vspace{\baselineskip}
{\fontsize{16pt}{19.2pt}\selectfont \textbf{4 \tabto{0.29in} }{\fontsize{15pt}{18.0pt}\selectfont \textbf{Positioning}\par}\par}\par


\vspace{\baselineskip}
\begin{adjustwidth}{0.4in}{0.19in}
{\fontsize{9pt}{10.8pt}\selectfont 4.1 \tabto{0.39in} \textbf{Behind the popping crease }at one end of the pitch is that area of the field of play, including any other\textbf{ }marking, objects and persons therein, that is on that side of the popping crease that does not include the creases at the opposite end of the pitch. \textbf{Behind}, in relation to any other marking, object or person, follows the same principle. See the diagram in paragraph \par}\par

\end{adjustwidth}


\vspace{\baselineskip}
\begin{adjustwidth}{0.4in}{0.15in}
{\fontsize{9pt}{10.8pt}\selectfont 4.2 \tabto{0.39in} \textbf{In front of the popping crease }at one end of the pitch is that area of the field of play, including any other\textbf{ }marking, objects and persons therein, that is on that side of the popping crease that includes the creases at the opposite end of the pitch. In front of, in relation to any other marking, object or person, follows the same principle. See the diagram in paragraph \par}\par

\end{adjustwidth}


\vspace{\baselineskip}
\begin{adjustwidth}{0.4in}{0.1in}
\begin{justify}
{\fontsize{9pt}{10.8pt}\selectfont 4.3 \tabto{0.39in} \textbf{The striker’s end }is the place where the striker stands to receive a delivery from the bowler only insofar as it\textbf{ }identifies, independently of where the striker may subsequently move, one end of the pitch.\par}
\end{justify}\par

\end{adjustwidth}


\vspace{\baselineskip}
\begin{adjustwidth}{0.4in}{0.14in}
{\fontsize{9pt}{10.8pt}\selectfont 4.4 \tabto{0.39in} \textbf{The bowler’s end }is the end from which the bowler delivers the ball. It is the other end of the pitch from the\textbf{ }striker’s end and identifies that end of the pitch that is not the striker’s end as described in paragraph \par}\par

\end{adjustwidth}


\vspace{\baselineskip}
{\fontsize{9pt}{10.8pt}\selectfont 4.5 \tabto{0.39in} \textbf{The wicket-keeper’s end }is the same as the striker’s end as described in paragraph\textbf{ }\par}\par


\vspace{\baselineskip}
{\fontsize{9pt}{10.8pt}\selectfont 4.6 \tabto{0.39in} {\fontsize{8pt}{9.6pt}\selectfont \textbf{In front of the line of the striker’s wicket }is in the area of the field of play in front of the imaginary line joining\par}\par}\par


\vspace{\baselineskip}
\begin{adjustwidth}{0.4in}{0.42in}
{\fontsize{9pt}{10.8pt}\selectfont the fronts of the stumps at the striker’s end; this line to be considered extended in both directions to the boundary. See paragraph \par}\par

\end{adjustwidth}


\vspace{\baselineskip}
\begin{adjustwidth}{0.4in}{0.1in}
\begin{justify}
{\fontsize{9pt}{10.8pt}\selectfont 4.7 \tabto{0.39in} \textbf{Behind the wicket }is in the area of the field of play behind the imaginary line joining the backs of the stumps\textbf{ }at the appropriate end; this line to be considered extended in both directions to the boundary. See paragraph \par}
\end{justify}\par

\end{adjustwidth}


\vspace{\baselineskip}
\begin{adjustwidth}{0.4in}{0.25in}
{\fontsize{9pt}{10.8pt}\selectfont 4.8 \tabto{0.39in} \textbf{Behind the wicket-keeper }is behind the wicket at the striker’s end, as defined above, but in line with both\textbf{ }sets of stumps and further from the stumps than the wicket-keeper.\par}\par

\end{adjustwidth}


\vspace{\baselineskip}
{\fontsize{9pt}{10.8pt}\selectfont 4.9 \tabto{0.39in} {\fontsize{8pt}{9.6pt}\selectfont \textbf{Off side/on (leg) side }–\textbf{ }see diagram in paragraph \par}\par}\par


\vspace{\baselineskip}
{\fontsize{9pt}{10.8pt}\selectfont 4.10 \tabto{0.39in} {\fontsize{8pt}{9.6pt}\selectfont \textbf{Inside edge }is the edge on the same side as the nearer wicket.\par}\par}\par


\vspace{\baselineskip}
{\fontsize{16pt}{19.2pt}\selectfont \textbf{5 \tabto{0.29in} }{\fontsize{15pt}{18.0pt}\selectfont \textbf{Umpires and decision-making}\par}\par}\par


\vspace{\baselineskip}
\begin{adjustwidth}{0.4in}{0.18in}
{\fontsize{9pt}{10.8pt}\selectfont 5.1 \tabto{0.39in} \textbf{Umpire }–\textbf{ }where the description\textbf{ the umpire }is used on its own, it always means ‘the bowler’s end umpire’\textbf{ }though this full description is sometimes used for emphasis or clarity. Similarly \textbf{the umpires} always means both umpires and the third umpire. \textbf{An umpire} and \textbf{umpires} are generalised terms. Otherwise, a fuller description indicates which one of the umpires is specifically intended. Each umpire will be bowler’s end umpire and striker’s end umpire in alternate overs.\par}\par

\end{adjustwidth}


\vspace{\baselineskip}

\vspace{\baselineskip}

\vspace{\baselineskip}

\vspace{\baselineskip}

\vspace{\baselineskip}
\begin{Center}
{\fontsize{8pt}{9.6pt}\selectfont 62\par}
\end{Center}\par


\vspace{\baselineskip}

\vspace{\baselineskip}
\begin{adjustwidth}{0.4in}{0.19in}
{\fontsize{9pt}{10.8pt}\selectfont 5.2 \tabto{0.39in} \textbf{Bowler’s end umpire }is the umpire who is standing at the bowler’s end (see paragraph\textbf{ for }the current\textbf{ }delivery.\par}\par

\end{adjustwidth}


\vspace{\baselineskip}
\begin{adjustwidth}{0.4in}{0.03in}
\begin{justify}
{\fontsize{9pt}{10.8pt}\selectfont 5.3 \tabto{0.39in} \textbf{Striker’s end umpire }is the umpire who is standing at the striker’s end (see paragraph\textbf{ to }one side of the\textbf{ }pitch or the other, depending on his/her choice, for the current delivery.\par}
\end{justify}\par

\end{adjustwidth}


\vspace{\baselineskip}
{\fontsize{9pt}{10.8pt}\selectfont 5.4 \tabto{0.39in} \textbf{On-field umpires }shall mean, collectively, the bowler’s end umpire and the striker’s end umpire.\par}\par


\vspace{\baselineskip}
\begin{adjustwidth}{0.4in}{0.04in}
\begin{justify}
{\fontsize{9pt}{10.8pt}\selectfont 5.5 \tabto{0.39in} \textbf{Third umpire }is the umpire who may use television evidence and other available technology in order review a\textbf{ }decision of the on-field umpires, either by way of an Umpire Review or a Player Review under the protocol set out in Appendix D.\par}
\end{justify}\par

\end{adjustwidth}


\vspace{\baselineskip}
\begin{adjustwidth}{0.4in}{0.29in}
{\fontsize{9pt}{10.8pt}\selectfont 5.6 \tabto{0.39in} \textbf{Umpires together agree }applies to decisions which the umpires are to make jointly, independently of the\textbf{ }players.\par}\par

\end{adjustwidth}


\vspace{\baselineskip}
\begin{adjustwidth}{0.4in}{0.1in}
\begin{justify}
{\fontsize{9pt}{10.8pt}\selectfont 5.7 \tabto{0.39in} \textbf{Decision Review System }or\textbf{ DRS }is the process covered by the Decision Review System and Third Umpire\textbf{ }Protocol set out in Appendix D, under which the third umpire may be consulted in relation to a decision of the on-field umpires, either by way of an Umpire Review or a Player Review.\par}
\end{justify}\par

\end{adjustwidth}


\vspace{\baselineskip}
\begin{adjustwidth}{0.4in}{0.06in}
{\fontsize{9pt}{10.8pt}\selectfont 5.8 \tabto{0.39in} \textbf{Player Review }is the process set out in Appendix D by which a player may request a review of any decision\textbf{ }taken by the on-field umpires concerning whether or not a batsman is dismissed (with the exception of ‘Timed out’).\par}\par

\end{adjustwidth}


\vspace{\baselineskip}
\begin{adjustwidth}{0.4in}{0.07in}
{\fontsize{9pt}{10.8pt}\selectfont 5.9 \tabto{0.39in} \textbf{Umpire Review }is the process set out in Appendix D by which an on-field umpire has the discretion to refer a\textbf{ }decision to the third umpire or, under certain circumstances, to consult with the third umpire before making a decision.\par}\par

\end{adjustwidth}


\vspace{\baselineskip}
\begin{adjustwidth}{0.4in}{0.17in}
{\fontsize{9pt}{10.8pt}\selectfont 5.10 \tabto{0.39in} \textbf{Soft Signal }is the visual communication by the bowler’s end umpire to the third umpire (accompanied by\textbf{ }additional information via two-way radio where necessary) of his/her initial on-field decision prior to initiating an Umpire Review.\par}\par

\end{adjustwidth}


\vspace{\baselineskip}
\begin{adjustwidth}{0.4in}{0.17in}
{\fontsize{9pt}{10.8pt}\selectfont 5.11 \tabto{0.39in} \textbf{Umpire’s Call }is the concept within the DRS under which the on-field decision of the bowler’s end umpire\textbf{ }shall stand, which shall apply under the specific circumstances set out in paragraphs and of Appendix D, where the ball-tracking technology indicates a marginal decision in respect of either the Impact Zone or the Wicket Zone.\par}\par

\end{adjustwidth}


\vspace{\baselineskip}
\begin{adjustwidth}{0.4in}{0.17in}
\begin{justify}
{\fontsize{9pt}{10.8pt}\selectfont 5.12 \tabto{0.39in} The \textbf{Pitching Zone} as used in the DRS is a two dimensional area on the pitch between both sets of stumps with its boundaries consisting of the base of both sets of stumps and a line between the outside of the outer stumps at each end.\par}
\end{justify}\par

\end{adjustwidth}


\vspace{\baselineskip}
\begin{adjustwidth}{0.4in}{0.04in}
{\fontsize{9pt}{10.8pt}\selectfont 5.13 \tabto{0.39in} The \textbf{Impact Zone} as used in the DRS is a three dimensional space extending between both sets of stumps to an indefinite height vertically and with its boundaries consisting of the base of the stumps and the outside of the outer stumps at each end.\par}\par

\end{adjustwidth}


\vspace{\baselineskip}
\begin{adjustwidth}{0.4in}{0.11in}
{\fontsize{9pt}{10.8pt}\selectfont 5.14 \tabto{0.39in} The \textbf{Wicket Zone} as used in the DRS is a two dimensional area with its boundaries consisting of the outside of the outer stumps, the base of the stumps, and the lower edge of the bails.\par}\par

\end{adjustwidth}


\vspace{\baselineskip}
{\fontsize{9pt}{10.8pt}\selectfont 5.15 \tabto{0.39in} A \textbf{Fair Catch} is a catch that has been taken cleanly by the fielder in accordance with clause \par}\par


\vspace{\baselineskip}
\begin{adjustwidth}{0.4in}{0.01in}
{\fontsize{9pt}{10.8pt}\selectfont 5.16 \tabto{0.39in} A \textbf{Bump Ball} is where the ball has made contact with the ground shortly after making contact with the striker’s bat.\par}\par

\end{adjustwidth}


\vspace{\baselineskip}
{\fontsize{9pt}{10.8pt}\selectfont 5.17 \tabto{0.39in} {\fontsize{8pt}{9.6pt}\selectfont The \textbf{Elite Panel} is the group of umpires contracted to the ICC to officiate in international cricket.\par}\par}\par


\vspace{\baselineskip}
\begin{adjustwidth}{0.4in}{0.28in}
{\fontsize{9pt}{10.8pt}\selectfont 5.18 \tabto{0.39in} The \textbf{International Panel} is the group of umpires nominated by the ICC’s full members in accordance with clause of the Playing Conditions.\par}\par

\end{adjustwidth}


\vspace{\baselineskip}
{\fontsize{16pt}{19.2pt}\selectfont \textbf{6 \tabto{0.29in} }{\fontsize{15pt}{18.0pt}\selectfont \textbf{Batsmen}\par}\par}\par


\vspace{\baselineskip}
{\fontsize{9pt}{10.8pt}\selectfont 6.1 \tabto{0.39in} {\fontsize{8pt}{9.6pt}\selectfont \textbf{Batting side }is the side currently batting, whether or not play is in progress.\par}\par}\par


\vspace{\baselineskip}
\begin{adjustwidth}{0.4in}{0.49in}
{\fontsize{9pt}{10.8pt}\selectfont 6.2 \tabto{0.39in} \textbf{Member of the batting side }is one of the players nominated by the captain of the batting side, or any\textbf{ }authorised replacement for such nominated player.\par}\par

\end{adjustwidth}


\vspace{\baselineskip}
\begin{adjustwidth}{0.4in}{0.04in}
{\fontsize{9pt}{10.8pt}\selectfont 6.3 \tabto{0.39in} \textbf{A batsman’s ground }–\textbf{ }at each end of the pitch, the whole area of the field of play behind the popping crease\textbf{ }is the ground at that end for a batsman.\par}\par

\end{adjustwidth}


\vspace{\baselineskip}

\vspace{\baselineskip}

\vspace{\baselineskip}
\begin{Center}
{\fontsize{8pt}{9.6pt}\selectfont 63\par}
\end{Center}\par


\vspace{\baselineskip}
{\fontsize{9pt}{10.8pt}\selectfont 6.4 \tabto{0.39in} {\fontsize{8pt}{9.6pt}\selectfont \textbf{Original end }is the end where a batsman was when the ball came into play for that delivery.\par}\par}\par


\vspace{\baselineskip}
{\fontsize{9pt}{10.8pt}\selectfont 6.5 \tabto{0.39in} {\fontsize{8pt}{9.6pt}\selectfont \textbf{Wicket he has }left is the wicket at the end where a batsman was at the start of the run in progress.\par}\par}\par


\vspace{\baselineskip}
{\fontsize{9pt}{10.8pt}\selectfont 6.6 \tabto{0.39in} {\fontsize{8pt}{9.6pt}\selectfont \textbf{Guard position }is the position and posture adopted by the striker to receive a ball delivered by the bowler\par}\par}\par


\vspace{\baselineskip}
{\fontsize{16pt}{19.2pt}\selectfont \textbf{7 \tabto{0.29in} }{\fontsize{15pt}{18.0pt}\selectfont \textbf{Fielders}\par}\par}\par


\vspace{\baselineskip}
{\fontsize{9pt}{10.8pt}\selectfont 7.1 \tabto{0.39in} {\fontsize{8pt}{9.6pt}\selectfont \textbf{Fielding side }is the side currently fielding, whether or not play is in progress.\par}\par}\par


\vspace{\baselineskip}
\begin{adjustwidth}{0.4in}{0.44in}
{\fontsize{9pt}{10.8pt}\selectfont 7.2 \tabto{0.39in} \textbf{Member of the fielding side }is one of the players nominated by the captain of the fielding side, or any\textbf{ }authorised replacement or substitute for such nominated player.\par}\par

\end{adjustwidth}


\vspace{\baselineskip}
\begin{adjustwidth}{0.4in}{0.12in}
{\fontsize{9pt}{10.8pt}\selectfont 7.3 \tabto{0.39in} \textbf{Fielder }is one of the 11 or fewer players who together represent the fielding side on the field of play. This\textbf{ }definition includes not only both the bowler and the wicket-keeper but also nominated players who are legitimately on the field of play, together with players legitimately acting as substitutes for absent nominated players. It excludes any nominated player who is absent from the field of play, or who has been absent from the field of play and who has not yet obtained the umpire’s permission to return.\par}\par

\end{adjustwidth}


\vspace{\baselineskip}
\begin{adjustwidth}{0.49in}{0.07in}
{\fontsize{9pt}{10.8pt}\selectfont A player going briefly outside the boundary in the course of discharging his/her duties as a fielder is not absent from the field of play nor, for the purposes of clause (Fielder absent or leaving the field of play), is he to be regarded as having left the field of play.\par}\par

\end{adjustwidth}


\vspace{\baselineskip}
{\fontsize{16pt}{19.2pt}\selectfont \textbf{8 \tabto{0.29in} }{\fontsize{15pt}{18.0pt}\selectfont \textbf{Substitutes}\par}\par}\par


\vspace{\baselineskip}
\begin{adjustwidth}{0.4in}{0.04in}
{\fontsize{9pt}{10.8pt}\selectfont 8.1 \tabto{0.39in} A \textbf{Substitute} is a player who takes the place of a fielder on the field of play, but does not replace the player for whom he substitutes on that side’s list of nominated players. A substitute’s activities are limited to fielding.\par}\par

\end{adjustwidth}


\vspace{\baselineskip}
{\fontsize{16pt}{19.2pt}\selectfont \textbf{9 \tabto{0.29in} }{\fontsize{15pt}{18.0pt}\selectfont \textbf{Bowlers}\par}\par}\par


\vspace{\baselineskip}
\begin{adjustwidth}{0.4in}{0.04in}
{\fontsize{9pt}{10.8pt}\selectfont 9.1 \tabto{0.39in} \textbf{Over the wicket / round the wicket }–\textbf{ }If, as the bowler runs up between the wicket and the return crease, the\textbf{ }wicket is on the same side as his bowling arm, he is bowling over the wicket. If the return crease is on the same side as his bowling arm, he is bowling round the wicket.\par}\par

\end{adjustwidth}


\vspace{\baselineskip}
{\fontsize{9pt}{10.8pt}\selectfont 9.2 \tabto{0.39in} {\fontsize{8pt}{9.6pt}\selectfont \textbf{Delivery swing }is the motion of the\textbf{ }bowler’s arm during which he\textbf{ }normally releases the ball for a delivery.\par}\par}\par


\vspace{\baselineskip}
\begin{adjustwidth}{0.4in}{0.11in}
{\fontsize{9pt}{10.8pt}\selectfont 9.3 \tabto{0.39in} \textbf{Delivery stride }is the stride during which the delivery swing is made, whether the ball is released or not. It\textbf{ }starts when the bowler’s back foot lands for that stride and ends when the front foot lands in the same stride. The stride after the delivery stride is completed when the next foot lands, i.e. when the back foot of the delivery stride lands again.\par}\par

\end{adjustwidth}


\vspace{\baselineskip}
{\fontsize{9pt}{10.8pt}\selectfont 9.4 \tabto{0.39in} {\fontsize{8pt}{9.6pt}\selectfont The \textbf{Illegal Bowling Regulations} are the ICC’s regulations governing Illegal Bowling Actions.\par}\par}\par


\vspace{\baselineskip}
\begin{adjustwidth}{0.4in}{0.19in}
{\fontsize{9pt}{10.8pt}\selectfont 9.5 \tabto{0.39in} An \textbf{Illegal Bowling Action} is a bowling action where a bowler’s Elbow Extension exceeds 15 degrees, measured from the point at which the bowling arm reaches the horizontal until the point at which the ball is released (any Elbow Hyperextension shall be discounted for the purposes of determining an Illegal Bowling Action).\par}\par

\end{adjustwidth}


\vspace{\baselineskip}
\begin{adjustwidth}{0.4in}{0.12in}
{\fontsize{9pt}{10.8pt}\selectfont 9.6 \tabto{0.39in} \textbf{Elbow Extension }means the motion that occurs when a bowler's arm moves from a flexed (bent) position at\textbf{ }the elbow, to a more extended (straight) position (full Elbow Extension occurs when the arm is straight).\par}\par

\end{adjustwidth}


\vspace{\baselineskip}
{\fontsize{9pt}{10.8pt}\selectfont 9.7 \tabto{0.39in} {\fontsize{8pt}{9.6pt}\selectfont \textbf{Elbow Hyperextension }is the motion that occurs when a bowler's elbow extends beyond the straight position.\par}\par}\par


\vspace{\baselineskip}
\begin{adjustwidth}{0.4in}{0.03in}
\begin{justify}
{\fontsize{9pt}{10.8pt}\selectfont 9.8 \tabto{0.39in} The \textbf{ICC Bowling Action Report Form} is the form provided for by Article 3 of the Illegal Bowling Regulations, by which an umpire and/or the ICC Match Referee may submit a report relating to a suspected Illegal Bowling Action.\par}
\end{justify}\par

\end{adjustwidth}


\vspace{\baselineskip}
{\fontsize{16pt}{19.2pt}\selectfont \textbf{10 The ball}\par}\par


\vspace{\baselineskip}
\begin{adjustwidth}{0.4in}{0.4in}
{\fontsize{9pt}{10.8pt}\selectfont 10.1 \tabto{0.39in} \textbf{The ball is struck/strikes the ball }unless specifically defined otherwise, mean ‘the ball is struck by the\textbf{ }bat’/‘strikes the ball with the bat’.\par}\par

\end{adjustwidth}


\vspace{\baselineskip}

\vspace{\baselineskip}

\vspace{\baselineskip}

\vspace{\baselineskip}
\begin{Center}
{\fontsize{8pt}{9.6pt}\selectfont 64\par}
\end{Center}\par


\vspace{\baselineskip}

\vspace{\baselineskip}
\begin{adjustwidth}{0.4in}{0.25in}
{\fontsize{9pt}{10.8pt}\selectfont 10.2 \tabto{0.39in} \textbf{Rebounds directly/strikes directly }and similar phrases mean ‘without contact with any fielder’ but do not\textbf{ }exclude contact with the ground.\par}\par

\end{adjustwidth}


\vspace{\baselineskip}
\begin{adjustwidth}{0.4in}{0.12in}
{\fontsize{9pt}{10.8pt}\selectfont 10.3 \tabto{0.39in} \textbf{Full-pitch }describes a ball delivered by the bowler that reaches or passes the striker without having touched\textbf{ }the ground. Sometimes described as non-pitching.\par}\par

\end{adjustwidth}


\vspace{\baselineskip}
{\fontsize{16pt}{19.2pt}\selectfont \textbf{11 Runs}\par}\par


\vspace{\baselineskip}
\begin{adjustwidth}{0.4in}{0.12in}
{\fontsize{9pt}{10.8pt}\selectfont 11.1 \tabto{0.39in} \textbf{A run to be disallowed }is one that in these Playing Conditions should not have been taken. It is not only to\textbf{ }be cancelled but the batsmen are to be returned to their original ends.\par}\par

\end{adjustwidth}


\vspace{\baselineskip}
\begin{adjustwidth}{0.4in}{0.12in}
{\fontsize{9pt}{10.8pt}\selectfont 11.2 \tabto{0.39in} \textbf{A run not to be scored }is one that is not illegal, but is not recognised as a properly executed run. It is not a\textbf{ }run that has been made, so the question of cancellation does not arise. The loss of the run so attempted is not a disallowance and the batsmen will not be returned to their original ends on that account.\par}\par

\end{adjustwidth}


\vspace{\baselineskip}
{\fontsize{16pt}{19.2pt}\selectfont \textbf{12 The person}\par}\par


\vspace{\baselineskip}
\begin{adjustwidth}{0.4in}{0.07in}
{\fontsize{9pt}{10.8pt}\selectfont 12.1 \tabto{0.39in} \textbf{Person}; A player’s person is his/her physical person (flesh and blood) together with any clothing or legitimate\textbf{ }external protective equipment that he is wearing except, in the case of a batsman, his/her bat.\par}\par

\end{adjustwidth}


\vspace{\baselineskip}
\begin{adjustwidth}{0.5in}{0.0in}
{\fontsize{9pt}{10.8pt}\selectfont A hand, whether gloved or not, that is not holding the bat is part of the batsman’s person.\par}\par

\end{adjustwidth}


\vspace{\baselineskip}
\begin{adjustwidth}{0.5in}{0.0in}
{\fontsize{9pt}{10.8pt}\selectfont No item of clothing or equipment is part of the player’s person unless it is attached to him.\par}\par

\end{adjustwidth}


\vspace{\baselineskip}
\begin{adjustwidth}{0.5in}{0.0in}
{\fontsize{9pt}{10.8pt}\selectfont For a batsman, a glove being held but not worn is part of his/her person.\par}\par

\end{adjustwidth}


\vspace{\baselineskip}
\begin{adjustwidth}{0.5in}{0.0in}
{\fontsize{9pt}{10.8pt}\selectfont For a fielder, an item of clothing or equipment he is holding in his/her hand or hands is not part of his person.\par}\par

\end{adjustwidth}


\vspace{\baselineskip}
\begin{adjustwidth}{0.4in}{0.1in}
{\fontsize{9pt}{10.8pt}\selectfont 12.2 \tabto{0.39in} \textbf{Clothing }–\textbf{ }anything that a player is wearing, including such items as spectacles or jewellery, that is not\textbf{ }classed as external protective equipment is classed as clothing, even though he may be wearing some items of apparel, which are not visible, for protection. A bat being carried by a batsman does not come within this definition of clothing.\par}\par

\end{adjustwidth}


\vspace{\baselineskip}
\begin{adjustwidth}{0.4in}{0.25in}
{\fontsize{9pt}{10.8pt}\selectfont 12.3 \tabto{0.39in} \textbf{Hand }for batsman or wicket-keeper shall include both the hand itself and the whole of a glove worn on the\textbf{ }hand.\par}\par

\end{adjustwidth}


\vspace{\baselineskip}

\vspace{\baselineskip}

\vspace{\baselineskip}

\vspace{\baselineskip}

\vspace{\baselineskip}

\vspace{\baselineskip}

\vspace{\baselineskip}

\vspace{\baselineskip}

\vspace{\baselineskip}

\vspace{\baselineskip}

\vspace{\baselineskip}

\vspace{\baselineskip}

\vspace{\baselineskip}

\vspace{\baselineskip}

\vspace{\baselineskip}

\vspace{\baselineskip}

\vspace{\baselineskip}

\vspace{\baselineskip}

\vspace{\baselineskip}

\vspace{\baselineskip}

\vspace{\baselineskip}

\vspace{\baselineskip}

\vspace{\baselineskip}

\vspace{\baselineskip}

\vspace{\baselineskip}

\vspace{\baselineskip}

\vspace{\baselineskip}

\vspace{\baselineskip}

\vspace{\baselineskip}

\vspace{\baselineskip}

\vspace{\baselineskip}
\begin{Center}
{\fontsize{8pt}{9.6pt}\selectfont 65\par}
\end{Center}\par


\vspace{\baselineskip}

\vspace{\baselineskip}
{\fontsize{15pt}{18.0pt}\selectfont \textbf{13 Off side / on side; in front of / behind the popping crease.}\par}\par



%%%%%%%%%%%%%%%%%%%% Figure/Image No: 1 starts here %%%%%%%%%%%%%%%%%%%%

\begin{figure}[H]
\advance\leftskip 1.04in		\includegraphics[width=4.42in,height=4.42in]{./media/image1.jpeg}
\end{figure}


%%%%%%%%%%%%%%%%%%%% Figure/Image No: 1 Ends here %%%%%%%%%%%%%%%%%%%%

\par


\vspace{\baselineskip}

\vspace{\baselineskip}

\vspace{\baselineskip}

\vspace{\baselineskip}

\vspace{\baselineskip}

\vspace{\baselineskip}

\vspace{\baselineskip}

\vspace{\baselineskip}

\vspace{\baselineskip}

\vspace{\baselineskip}

\vspace{\baselineskip}

\vspace{\baselineskip}

\vspace{\baselineskip}

\vspace{\baselineskip}

\vspace{\baselineskip}

\vspace{\baselineskip}

\vspace{\baselineskip}

\vspace{\baselineskip}

\vspace{\baselineskip}

\vspace{\baselineskip}

\vspace{\baselineskip}

\vspace{\baselineskip}

\vspace{\baselineskip}

\vspace{\baselineskip}

\vspace{\baselineskip}

\vspace{\baselineskip}

\vspace{\baselineskip}

\vspace{\baselineskip}

\vspace{\baselineskip}

\vspace{\baselineskip}

\vspace{\baselineskip}

\vspace{\baselineskip}

\vspace{\baselineskip}

\vspace{\baselineskip}

\vspace{\baselineskip}

\vspace{\baselineskip}

\vspace{\baselineskip}

\vspace{\baselineskip}

\vspace{\baselineskip}

\vspace{\baselineskip}

\vspace{\baselineskip}

\vspace{\baselineskip}

\vspace{\baselineskip}

\vspace{\baselineskip}

\vspace{\baselineskip}

\vspace{\baselineskip}

\vspace{\baselineskip}

\vspace{\baselineskip}

\vspace{\baselineskip}

\vspace{\baselineskip}

\vspace{\baselineskip}

\vspace{\baselineskip}

\vspace{\baselineskip}

\vspace{\baselineskip}

\vspace{\baselineskip}

\vspace{\baselineskip}

\vspace{\baselineskip}

\vspace{\baselineskip}

\vspace{\baselineskip}

\vspace{\baselineskip}

\vspace{\baselineskip}

\vspace{\baselineskip}

\vspace{\baselineskip}

\vspace{\baselineskip}

\vspace{\baselineskip}

\vspace{\baselineskip}
\begin{Center}
{\fontsize{8pt}{9.6pt}\selectfont 66\par}
\end{Center}\par


\vspace{\baselineskip}
\begin{Center}
{\fontsize{11pt}{13.2pt}\selectfont \textbf{Appendix B}\par}
\end{Center}\par


\vspace{\baselineskip}
\begin{Center}
{\fontsize{11pt}{13.2pt}\selectfont \textbf{Equipment}\par}
\end{Center}\par


\vspace{\baselineskip}
{\fontsize{16pt}{19.2pt}\selectfont \textbf{1 \tabto{0.47in} The Bat}\par}\par


\vspace{\baselineskip}
{\fontsize{11pt}{13.2pt}\selectfont \textbf{1.1 \tabto{0.47in} General guidance}\par}\par


\vspace{\baselineskip}
\begin{adjustwidth}{0.5in}{0.46in}
{\fontsize{9pt}{10.8pt}\selectfont 1.1.1 \tabto{0.49in} \textbf{Measurements }- All provisions in paragraphs to below are subject to the measurements and restrictions stated in the Playing Conditions and this Appendix.\par}\par

\end{adjustwidth}


\vspace{\baselineskip}
{\fontsize{9pt}{10.8pt}\selectfont 1.1.2 \tabto{0.49in} {\fontsize{8pt}{9.6pt}\selectfont \textbf{Adhesives }–\textbf{ }Throughout, adhesives are permitted only where essential and only in minimal quantity.\par}\par}\par


\vspace{\baselineskip}
{\fontsize{11pt}{13.2pt}\selectfont \textbf{1.2 \tabto{0.47in} Specifications for the Handle}\par}\par


\vspace{\baselineskip}
{\fontsize{9pt}{10.8pt}\selectfont 1.2.1 \tabto{0.49in} One end of the handle is inserted into a recess in the blade as a means of joining the handle and the blade.\par}\par


\vspace{\baselineskip}
\begin{adjustwidth}{0.5in}{0.04in}
{\fontsize{9pt}{10.8pt}\selectfont This lower portion is used purely for joining the blade and the handle together. It is not part of the blade but, solely in interpreting paragraphs and below, references to the blade shall be considered to extend also to this lower portion of the handle where relevant.\par}\par

\end{adjustwidth}


\vspace{\baselineskip}
{\fontsize{9pt}{10.8pt}\selectfont 1.2.2 \tabto{0.49in} The handle may be glued where necessary and bound with twine along the upper portion.\par}\par


\vspace{\baselineskip}
\begin{adjustwidth}{0.5in}{0.04in}
{\fontsize{9pt}{10.8pt}\selectfont Providing clause is not contravened, the upper portion may be covered with materials solely to provide a surface suitable for gripping. Such covering is an addition and is not part of the bat, except in relation to clause The bottom of this grip should not extend below the point defined in paragraph below.\par}\par

\end{adjustwidth}


\vspace{\baselineskip}
\begin{adjustwidth}{0.5in}{0.12in}
{\fontsize{9pt}{10.8pt}\selectfont Twine binding and the covering grip may extend beyond the junction of the upper and lower portions of the handle, to cover part of the shoulders of the bat as defined in paragraph \par}\par

\end{adjustwidth}


\vspace{\baselineskip}
\begin{adjustwidth}{0.5in}{0.07in}
{\fontsize{9pt}{10.8pt}\selectfont No material may be placed on or inserted into the lower portion of the handle other than as permitted above together with the minimal adhesives or adhesive tape used solely for fixing these items, or for fixing the handle to the blade.\par}\par

\end{adjustwidth}


\vspace{\baselineskip}
\begin{adjustwidth}{0.5in}{0.08in}
{\fontsize{9pt}{10.8pt}\selectfont 1.2.3 \tabto{0.49in} \textbf{Materials in handle }–\textbf{ }As a proportion of the total volume of the handle, materials other than cane, wood or\textbf{ }twine are restricted to one-tenth. Such materials must not project more than 3.25 in/8.26 cm into the lower portion of the handle\par}\par

\end{adjustwidth}


\vspace{\baselineskip}
\begin{adjustwidth}{0.5in}{0.12in}
{\fontsize{9pt}{10.8pt}\selectfont 1.2.4 \tabto{0.49in} \textbf{Binding and covering of handle }–\textbf{ }The permitted continuation beyond the junction of the upper and lower\textbf{ }portions of the handle is restricted to a maximum, measured along the length of the handle, of\par}\par

\end{adjustwidth}


\vspace{\baselineskip}
\begin{adjustwidth}{0.5in}{0.0in}
{\fontsize{9pt}{10.8pt}\selectfont 2.5 in/6.35 cm in for the twine binding\par}\par

\end{adjustwidth}


\vspace{\baselineskip}
\begin{adjustwidth}{0.5in}{0.0in}
{\fontsize{9pt}{10.8pt}\selectfont 2.75 in/6.99 cm for the covering grip.\par}\par

\end{adjustwidth}


\vspace{\baselineskip}
{\fontsize{11pt}{13.2pt}\selectfont \textbf{1.3 \tabto{0.47in} Specifications for the Blade}\par}\par


\vspace{\baselineskip}
{\fontsize{9pt}{10.8pt}\selectfont 1.3.1 \tabto{0.49in} The blade has a face, a back, a toe, sides and shoulders\par}\par


\vspace{\baselineskip}
\begin{adjustwidth}{1.18in}{0.12in}
{\fontsize{9pt}{10.8pt}\selectfont 1.3.1.1 \tabto{1.17in} The face of the blade is its main striking surface and shall be flat or have a slight convex curve resulting from traditional pressing techniques. The back is the opposite surface.\par}\par

\end{adjustwidth}


\vspace{\baselineskip}
\begin{adjustwidth}{0.49in}{0.0in}
{\fontsize{9pt}{10.8pt}\selectfont 1.3.1.2 \tabto{1.17in} {\fontsize{8pt}{9.6pt}\selectfont The shoulders, sides and toe are the remaining surfaces, separating the face and the back.\par}\par}\par

\end{adjustwidth}


\vspace{\baselineskip}
\begin{adjustwidth}{1.18in}{0.1in}
{\fontsize{9pt}{10.8pt}\selectfont 1.3.1.3 \tabto{1.17in} The shoulders, one on each side of the handle, are along that portion of the blade between the first entry point of the handle and the point at which the blade first reaches its full width.\par}\par

\end{adjustwidth}


\vspace{\baselineskip}
\begin{adjustwidth}{0.49in}{0.0in}
{\fontsize{9pt}{10.8pt}\selectfont 1.3.1.4 \tabto{1.17in} The toe is the surface opposite to the shoulders taken as a pair.\par}\par

\end{adjustwidth}


\vspace{\baselineskip}
\begin{adjustwidth}{1.18in}{0.1in}
{\fontsize{9pt}{10.8pt}\selectfont 1.3.1.5 \tabto{1.17in} The sides, one each side of the blade, are along the rest of the blade, between the toe and the shoulders.\par}\par

\end{adjustwidth}


\vspace{\baselineskip}

\vspace{\baselineskip}

\vspace{\baselineskip}

\vspace{\baselineskip}

\vspace{\baselineskip}

\vspace{\baselineskip}
\begin{Center}
{\fontsize{8pt}{9.6pt}\selectfont 67\par}
\end{Center}\par


\vspace{\baselineskip}

\vspace{\baselineskip}
\begin{adjustwidth}{0.5in}{0.1in}
{\fontsize{9pt}{10.8pt}\selectfont 1.3.2 \tabto{0.49in} No material may be placed on or inserted into the blade other than as permitted in paragraph  paragraph and clause together with the minimal adhesives or adhesive tape used solely for fixing these items, or for fixing the handle to the blade.\par}\par

\end{adjustwidth}


\vspace{\baselineskip}
{\fontsize{9pt}{10.8pt}\selectfont 1.3.3 \tabto{0.49in} \textbf{Covering the blade}. Bats shall have no covering on the blade except as permitted in clause \par}\par


\vspace{\baselineskip}
\begin{adjustwidth}{0.49in}{0.1in}
{\fontsize{9pt}{10.8pt}\selectfont Any materials referred to above, in clause and paragraph below, are to be considered as part of the bat, which must still pass through the gauge as defined in paragraph \par}\par

\end{adjustwidth}


\vspace{\baselineskip}
{\fontsize{11pt}{13.2pt}\selectfont \textbf{1.4 \tabto{0.47in} Protection and repair}\par}\par


\vspace{\baselineskip}
\begin{adjustwidth}{0.5in}{0.01in}
{\fontsize{9pt}{10.8pt}\selectfont 1.4.1 \tabto{0.49in} The surface of the blade may be treated with non-solid materials to improve resistance to moisture penetration and/or mask natural blemishes in the appearance of the wood. Save for the purpose of giving a homogeneous appearance by masking natural blemishes, such treatment shall not materially alter the colour of the blade.\par}\par

\end{adjustwidth}


\vspace{\baselineskip}
\begin{adjustwidth}{0.5in}{0.08in}
{\fontsize{9pt}{10.8pt}\selectfont 1.4.2 \tabto{0.49in} Materials can be used for protection and repair as stated in clause and are additional to the blade. Note however clause \par}\par

\end{adjustwidth}


\vspace{\baselineskip}
\begin{adjustwidth}{0.5in}{0.11in}
{\fontsize{9pt}{10.8pt}\selectfont Any such material shall not extend over any part of the back of the blade except in the case of clause  and then only when it is applied as a continuous wrapping covering the damaged area.\par}\par

\end{adjustwidth}


\vspace{\baselineskip}
\begin{adjustwidth}{0.5in}{0.07in}
{\fontsize{9pt}{10.8pt}\selectfont The repair material shall not extend along the length of the blade more than 0.79 in/2.0 cm in each direction beyond the limits of the damaged area. Where used as a continuous binding, any overlapping shall not breach the maximum of 0.04 in/0.1 cm in total thickness.\par}\par

\end{adjustwidth}


\vspace{\baselineskip}
\begin{adjustwidth}{0.5in}{0.08in}
{\fontsize{9pt}{10.8pt}\selectfont The use of non-solid material which when dry forms a hard layer more than 0.004 in/0.01 cm in thickness is not permitted.\par}\par

\end{adjustwidth}


\vspace{\baselineskip}
\begin{adjustwidth}{0.5in}{0.07in}
{\fontsize{9pt}{10.8pt}\selectfont 1.4.3 \tabto{0.49in} Permitted coverings, repair material and toe guards, not exceeding their specified thicknesses, may be additional to the dimensions above, but the bat must still pass through the gauge as described in paragraph 1.6.\par}\par

\end{adjustwidth}


\vspace{\baselineskip}
{\fontsize{11pt}{13.2pt}\selectfont \textbf{1.5 \tabto{0.47in} Commercial identifications}\par}\par


\vspace{\baselineskip}
\begin{adjustwidth}{0.49in}{0.15in}
{\fontsize{9pt}{10.8pt}\selectfont Such identifications shall comply with the restrictions set out in the Clothing and Equipment Regulations in relation to the size and position of marks and logos.\par}\par

\end{adjustwidth}


\vspace{\baselineskip}
{\fontsize{11pt}{13.2pt}\selectfont \textbf{1.6 \tabto{0.47in} Bat Gauge}\par}\par


\vspace{\baselineskip}
\begin{adjustwidth}{0.49in}{0.08in}
{\fontsize{9pt}{10.8pt}\selectfont All bats must meet the specifications defined in clause They must also, with or without protective coverings permitted in clause be able to pass through a bat gauge, the dimensions and shape of which are shown in the following diagram:\par}\par

\end{adjustwidth}



%%%%%%%%%%%%%%%%%%%% Figure/Image No: 2 starts here %%%%%%%%%%%%%%%%%%%%

\begin{figure}[H]
\advance\leftskip 0.62in		\includegraphics[width=5.25in,height=2.74in]{./media/image2.jpeg}
\end{figure}


%%%%%%%%%%%%%%%%%%%% Figure/Image No: 2 Ends here %%%%%%%%%%%%%%%%%%%%

\par


\vspace{\baselineskip}

\vspace{\baselineskip}

\vspace{\baselineskip}

\vspace{\baselineskip}

\vspace{\baselineskip}

\vspace{\baselineskip}

\vspace{\baselineskip}

\vspace{\baselineskip}

\vspace{\baselineskip}

\vspace{\baselineskip}

\vspace{\baselineskip}

\vspace{\baselineskip}

\vspace{\baselineskip}

\vspace{\baselineskip}

\vspace{\baselineskip}

\vspace{\baselineskip}

\vspace{\baselineskip}

\vspace{\baselineskip}

\vspace{\baselineskip}

\vspace{\baselineskip}

\vspace{\baselineskip}

\vspace{\baselineskip}

\vspace{\baselineskip}

\vspace{\baselineskip}
\begin{Center}
{\fontsize{8pt}{9.6pt}\selectfont 68\par}
\end{Center}\par


\vspace{\baselineskip}
{\fontsize{16pt}{19.2pt}\selectfont \textbf{2 \tabto{0.29in} }{\fontsize{15pt}{18.0pt}\selectfont \textbf{The wickets}\par}\par}\par



%%%%%%%%%%%%%%%%%%%% Figure/Image No: 3 starts here %%%%%%%%%%%%%%%%%%%%

\begin{figure}[H]
\advance\leftskip 2.01in		\includegraphics[width=2.49in,height=3.45in]{./media/image3.jpeg}
\end{figure}


%%%%%%%%%%%%%%%%%%%% Figure/Image No: 3 Ends here %%%%%%%%%%%%%%%%%%%%

\par


\vspace{\baselineskip}

\vspace{\baselineskip}

\vspace{\baselineskip}

\vspace{\baselineskip}

\vspace{\baselineskip}

\vspace{\baselineskip}

\vspace{\baselineskip}

\vspace{\baselineskip}

\vspace{\baselineskip}

\vspace{\baselineskip}

\vspace{\baselineskip}

\vspace{\baselineskip}

\vspace{\baselineskip}

\vspace{\baselineskip}

\vspace{\baselineskip}

\vspace{\baselineskip}

\vspace{\baselineskip}

\vspace{\baselineskip}

\vspace{\baselineskip}

\vspace{\baselineskip}

\vspace{\baselineskip}

\vspace{\baselineskip}

\vspace{\baselineskip}

\vspace{\baselineskip}

\vspace{\baselineskip}
{\fontsize{11pt}{13.2pt}\selectfont \textbf{2.1 \tabto{0.47in} Bails}\par}\par


\vspace{\baselineskip}
\begin{adjustwidth}{0.49in}{0.0in}
{\fontsize{9pt}{10.8pt}\selectfont Overall 4.31 in / 10.95 cm\par}\par

\end{adjustwidth}


\vspace{\baselineskip}
\begin{adjustwidth}{0.49in}{0.0in}
{\fontsize{9pt}{10.8pt}\selectfont a = 1.38 in / 3.50 cm\par}\par

\end{adjustwidth}


\vspace{\baselineskip}
\begin{adjustwidth}{0.49in}{0.0in}
{\fontsize{9pt}{10.8pt}\selectfont b = 2.13 in / 5.40 cm\par}\par

\end{adjustwidth}


\vspace{\baselineskip}
\begin{adjustwidth}{0.49in}{0.0in}
{\fontsize{9pt}{10.8pt}\selectfont c = 0.81 in / 2.06 cm\par}\par

\end{adjustwidth}


\vspace{\baselineskip}
{\fontsize{11pt}{13.2pt}\selectfont \textbf{2.2 \tabto{0.47in} Stumps}\par}\par


\vspace{\baselineskip}
\begin{adjustwidth}{0.49in}{0.0in}
{\fontsize{9pt}{10.8pt}\selectfont Height (d) = 28 in / 71.1 cm\par}\par

\end{adjustwidth}


\vspace{\baselineskip}
\begin{adjustwidth}{0.49in}{0.0in}
{\fontsize{9pt}{10.8pt}\selectfont Diameter (e) - maximum = 1.5 in / 3.81 cm; minimum = 1.38 in / 3.50 cm\par}\par

\end{adjustwidth}


\vspace{\baselineskip}
{\fontsize{11pt}{13.2pt}\selectfont \textbf{2.3 \tabto{0.47in} Overall}\par}\par


\vspace{\baselineskip}
\begin{adjustwidth}{0.49in}{0.0in}
{\fontsize{9pt}{10.8pt}\selectfont Width (f) of wicket 9 in / 22.86 cm\par}\par

\end{adjustwidth}


\vspace{\baselineskip}
{\fontsize{16pt}{19.2pt}\selectfont \textbf{3 \tabto{0.29in} }{\fontsize{15pt}{18.0pt}\selectfont \textbf{Wicket-keeping gloves}\par}\par}\par


\vspace{\baselineskip}
{\fontsize{9pt}{10.8pt}\selectfont 3.1 \tabto{0.39in} The images below illustrate the requirements of clause in relation to:\par}\par


\vspace{\baselineskip}
\begin{itemize}
	\item {\fontsize{9pt}{10.8pt}\selectfont no webbing between the fingers;\par}\par


\vspace{\baselineskip}
	\item {\fontsize{9pt}{10.8pt}\selectfont a single piece of non-stretch material between finger and thumb as a means of support; and\par}\par


\vspace{\baselineskip}
	\item {\fontsize{9pt}{10.8pt}\selectfont when a hand wearing the glove has the thumb fully extended, the top edge being taut and not protruding beyond the straight line joining the top of the index finger to the top of the thumb.\par}
\end{itemize}\par


\vspace{\baselineskip}

\vspace{\baselineskip}

\vspace{\baselineskip}

\vspace{\baselineskip}

\vspace{\baselineskip}

\vspace{\baselineskip}

\vspace{\baselineskip}

\vspace{\baselineskip}

\vspace{\baselineskip}

\vspace{\baselineskip}
\begin{Center}
{\fontsize{8pt}{9.6pt}\selectfont 69\par}
\end{Center}\par


\vspace{\baselineskip}


%%%%%%%%%%%%%%%%%%%% Figure/Image No: 4 starts here %%%%%%%%%%%%%%%%%%%%

\begin{figure}[H]
\advance\leftskip 1.0in		\includegraphics[width=5.83in,height=1.96in]{./media/image4.jpeg}
\end{figure}


%%%%%%%%%%%%%%%%%%%% Figure/Image No: 4 Ends here %%%%%%%%%%%%%%%%%%%%

\par


\vspace{\baselineskip}

\vspace{\baselineskip}

\vspace{\baselineskip}

\vspace{\baselineskip}

\vspace{\baselineskip}

\vspace{\baselineskip}

\vspace{\baselineskip}

\vspace{\baselineskip}

\vspace{\baselineskip}

\vspace{\baselineskip}

\vspace{\baselineskip}

\vspace{\baselineskip}

\vspace{\baselineskip}

\vspace{\baselineskip}

\vspace{\baselineskip}
\begin{adjustwidth}{0.4in}{0.12in}
{\fontsize{9pt}{10.8pt}\selectfont 3.2 \tabto{0.39in} Note also the requirement for wicket-keeping gloves to comply with the Clothing and Equipment Regulations in relation to the size and position of marks and logos.\par}\par

\end{adjustwidth}


\vspace{\baselineskip}

\vspace{\baselineskip}

\vspace{\baselineskip}

\vspace{\baselineskip}

\vspace{\baselineskip}

\vspace{\baselineskip}

\vspace{\baselineskip}

\vspace{\baselineskip}

\vspace{\baselineskip}

\vspace{\baselineskip}

\vspace{\baselineskip}

\vspace{\baselineskip}

\vspace{\baselineskip}

\vspace{\baselineskip}

\vspace{\baselineskip}

\vspace{\baselineskip}

\vspace{\baselineskip}

\vspace{\baselineskip}

\vspace{\baselineskip}

\vspace{\baselineskip}

\vspace{\baselineskip}

\vspace{\baselineskip}

\vspace{\baselineskip}

\vspace{\baselineskip}

\vspace{\baselineskip}

\vspace{\baselineskip}

\vspace{\baselineskip}

\vspace{\baselineskip}

\vspace{\baselineskip}

\vspace{\baselineskip}

\vspace{\baselineskip}

\vspace{\baselineskip}

\vspace{\baselineskip}

\vspace{\baselineskip}

\vspace{\baselineskip}

\vspace{\baselineskip}

\vspace{\baselineskip}

\vspace{\baselineskip}

\vspace{\baselineskip}

\vspace{\baselineskip}

\vspace{\baselineskip}

\vspace{\baselineskip}

\vspace{\baselineskip}

\vspace{\baselineskip}

\vspace{\baselineskip}

\vspace{\baselineskip}

\vspace{\baselineskip}

\vspace{\baselineskip}
\begin{Center}
{\fontsize{9pt}{10.8pt}\selectfont 70\par}
\end{Center}\par


\vspace{\baselineskip}
\begin{Center}
{\fontsize{11pt}{13.2pt}\selectfont \textbf{Appendix C}\par}
\end{Center}\par


\vspace{\baselineskip}
\begin{adjustwidth}{2.86in}{0.0in}
{\fontsize{11pt}{13.2pt}\selectfont \textbf{The venue}\par}\par

\end{adjustwidth}


\vspace{\baselineskip}
{\fontsize{16pt}{19.2pt}\selectfont \textbf{1 \tabto{0.47in} The pitch and the creases}\par}\par



%%%%%%%%%%%%%%%%%%%% Figure/Image No: 5 starts here %%%%%%%%%%%%%%%%%%%%

\begin{figure}[H]
\advance\leftskip 0.0in		\includegraphics[width=6.45in,height=3.63in]{./media/image5.jpeg}
\end{figure}


%%%%%%%%%%%%%%%%%%%% Figure/Image No: 5 Ends here %%%%%%%%%%%%%%%%%%%%

\par


\vspace{\baselineskip}

\vspace{\baselineskip}

\vspace{\baselineskip}

\vspace{\baselineskip}

\vspace{\baselineskip}

\vspace{\baselineskip}

\vspace{\baselineskip}

\vspace{\baselineskip}

\vspace{\baselineskip}

\vspace{\baselineskip}

\vspace{\baselineskip}

\vspace{\baselineskip}

\vspace{\baselineskip}

\vspace{\baselineskip}

\vspace{\baselineskip}

\vspace{\baselineskip}

\vspace{\baselineskip}

\vspace{\baselineskip}

\vspace{\baselineskip}

\vspace{\baselineskip}

\vspace{\baselineskip}

\vspace{\baselineskip}

\vspace{\baselineskip}

\vspace{\baselineskip}

\vspace{\baselineskip}

\vspace{\baselineskip}

\vspace{\baselineskip}
{\fontsize{16pt}{19.2pt}\selectfont \textbf{2 \tabto{0.29in} }{\fontsize{15pt}{18.0pt}\selectfont \textbf{Restriction on the placement of fielders}\par}\par}\par



%%%%%%%%%%%%%%%%%%%% Figure/Image No: 6 starts here %%%%%%%%%%%%%%%%%%%%

\begin{figure}[H]
\advance\leftskip 0.0in		\includegraphics[width=2.68in,height=3.61in]{./media/image6.jpeg}
\end{figure}


%%%%%%%%%%%%%%%%%%%% Figure/Image No: 6 Ends here %%%%%%%%%%%%%%%%%%%%

\par


\vspace{\baselineskip}

\vspace{\baselineskip}

\vspace{\baselineskip}

\vspace{\baselineskip}

\vspace{\baselineskip}

\vspace{\baselineskip}

\vspace{\baselineskip}

\vspace{\baselineskip}

\vspace{\baselineskip}

\vspace{\baselineskip}

\vspace{\baselineskip}

\vspace{\baselineskip}

\vspace{\baselineskip}

\vspace{\baselineskip}

\vspace{\baselineskip}

\vspace{\baselineskip}

\vspace{\baselineskip}

\vspace{\baselineskip}

\vspace{\baselineskip}

\vspace{\baselineskip}

\vspace{\baselineskip}

\vspace{\baselineskip}

\vspace{\baselineskip}

\vspace{\baselineskip}

\vspace{\baselineskip}

\vspace{\baselineskip}

\vspace{\baselineskip}

\vspace{\baselineskip}

\vspace{\baselineskip}

\vspace{\baselineskip}

\vspace{\baselineskip}
\begin{Center}
{\fontsize{8pt}{9.6pt}\selectfont 71\par}
\end{Center}\par


\vspace{\baselineskip}

\vspace{\baselineskip}
\begin{enumerate}[label*={\fontsize{16pt}{16pt}\selectfont \textbf{\arabic*.}}]
	\item {\fontsize{16pt}{19.2pt}\selectfont \textbf{Advertising on grounds, perimeter boards and sight-screens}\par}
\end{enumerate}\par


\vspace{\baselineskip}
{\fontsize{11pt}{13.2pt}\selectfont \textbf{3.1 \tabto{0.47in} }{\fontsize{10pt}{12.0pt}\selectfont \textbf{Advertising on grounds}\par}\par}\par


\vspace{\baselineskip}
{\fontsize{9pt}{10.8pt}\selectfont 3.1.1 \tabto{0.49in} {\fontsize{8pt}{9.6pt}\selectfont The logos on outfields are to be positioned as follows:\par}\par}\par


\vspace{\baselineskip}
\begin{enumerate}
	\item {\fontsize{9pt}{10.8pt}\selectfont Behind the stumps – a minimum of 25.15 yards (23 meters) from the stumps.\par}\par


\vspace{\baselineskip}
	\item {\fontsize{9pt}{10.8pt}\selectfont Midwicket/cover area – no advertising to be positioned within 30 yards (27.50 meters) of the centre of the pitch being used for the match.\par}
\end{enumerate}\par


\vspace{\baselineskip}
\begin{adjustwidth}{0.5in}{0.38in}
{\fontsize{9pt}{10.8pt}\selectfont 3.1.2 \tabto{0.49in} Note: Advertising closer to the stumps as set out above which is required to meet 3D requirements for broadcasters may be permitted, subject to prior ICC approval having been obtained.\par}\par

\end{adjustwidth}


\vspace{\baselineskip}
{\fontsize{11pt}{13.2pt}\selectfont \textbf{3.2 \tabto{0.47in} Perimeter boards}\par}\par


\vspace{\baselineskip}
\begin{adjustwidth}{0.5in}{0.12in}
{\fontsize{9pt}{10.8pt}\selectfont 3.2.1 \tabto{0.49in} Advertising on perimeter boards placed in front of the sight-screens is permitted save that the predominant colour of such advertising shall be of a contrasting colour to that of the ball.\par}\par

\end{adjustwidth}


\vspace{\baselineskip}
\begin{adjustwidth}{0.5in}{0.04in}
{\fontsize{9pt}{10.8pt}\selectfont 3.2.2 \tabto{0.49in} Advertising on perimeter boards behind the stumps at both ends shall not contain moving, flashing or flickering images and operators should ensure that the images are only changed or moved at a time that will not be distracting to the players or the umpires.\par}\par

\end{adjustwidth}


\vspace{\baselineskip}
\begin{adjustwidth}{0.5in}{0.11in}
{\fontsize{9pt}{10.8pt}\selectfont 3.2.3 \tabto{0.49in} The brightness of any electronic images shall be set at a level so that it is not a distraction to the players or umpires.\par}\par

\end{adjustwidth}


\vspace{\baselineskip}
{\fontsize{11pt}{13.2pt}\selectfont \textbf{3.3 \tabto{0.47in} Sight-screens}\par}\par


\vspace{\baselineskip}
{\fontsize{9pt}{10.8pt}\selectfont 3.3.1 \tabto{0.49in} Sight-screens shall be provided at both ends of all grounds.\par}\par


\vspace{\baselineskip}
\begin{adjustwidth}{0.5in}{0.56in}
{\fontsize{9pt}{10.8pt}\selectfont 3.3.2 \tabto{0.49in} Advertising shall be permitted on the sight-screen behind the striker, providing it is removed for the subsequent over from that end.\par}\par

\end{adjustwidth}


\vspace{\baselineskip}
\begin{adjustwidth}{0.5in}{0.21in}
{\fontsize{9pt}{10.8pt}\selectfont 3.3.3 \tabto{0.49in} Such advertising shall not contain flashing or flickering images and particular care should be taken by the operators that the advertising is not changed at a time which is distracting to the umpire.\par}\par

\end{adjustwidth}


\vspace{\baselineskip}
{\fontsize{16pt}{19.2pt}\selectfont \textbf{4 \tabto{0.29in} }{\fontsize{15pt}{18.0pt}\selectfont \textbf{Markings on outfield}\par}\par}\par


\vspace{\baselineskip}
\begin{adjustwidth}{0.49in}{0.03in}
{\fontsize{8pt}{9.6pt}\selectfont With the permission of the Ground Authority, a bowler may use paint to make a small marking on the outfield for the purposes of identifying their run-up. Paint used for this purpose shall be any colour other than white.\par}\par

\end{adjustwidth}


\vspace{\baselineskip}

\vspace{\baselineskip}

\vspace{\baselineskip}

\vspace{\baselineskip}

\vspace{\baselineskip}

\vspace{\baselineskip}

\vspace{\baselineskip}

\vspace{\baselineskip}

\vspace{\baselineskip}

\vspace{\baselineskip}

\vspace{\baselineskip}

\vspace{\baselineskip}

\vspace{\baselineskip}

\vspace{\baselineskip}

\vspace{\baselineskip}

\vspace{\baselineskip}

\vspace{\baselineskip}

\vspace{\baselineskip}

\vspace{\baselineskip}

\vspace{\baselineskip}

\vspace{\baselineskip}

\vspace{\baselineskip}

\vspace{\baselineskip}

\vspace{\baselineskip}

\vspace{\baselineskip}
\begin{Center}
{\fontsize{8pt}{9.6pt}\selectfont 72\par}
\end{Center}\par


\vspace{\baselineskip}
\begin{adjustwidth}{0.0in}{0.08in}
\begin{Center}
{\fontsize{11pt}{13.2pt}\selectfont \textbf{Appendix D}\par}
\end{Center}\par

\end{adjustwidth}


\vspace{\baselineskip}
\begin{adjustwidth}{1.04in}{0.0in}
{\fontsize{11pt}{13.2pt}\selectfont \textbf{Decision Review System (DRS) and Third Umpire Protocol}\par}\par

\end{adjustwidth}


\vspace{\baselineskip}
{\fontsize{16pt}{19.2pt}\selectfont \textbf{1 \tabto{0.47in} General}\par}\par


\vspace{\baselineskip}
{\fontsize{9pt}{10.8pt}\selectfont 1.1 \tabto{0.39in} {\fontsize{8pt}{9.6pt}\selectfont Minimum requirements for use of DRS and appointment of third umpire\par}\par}\par


\vspace{\baselineskip}
\begin{adjustwidth}{0.5in}{0.19in}
{\fontsize{9pt}{10.8pt}\selectfont 1.1.1 \tabto{0.49in} Save with the express written consent of the ICC General Manager - Cricket, the Home Board shall ensure the live television broadcast of all T20I matches played in its country.\par}\par

\end{adjustwidth}


\vspace{\baselineskip}
\begin{adjustwidth}{0.5in}{0.32in}
{\fontsize{9pt}{10.8pt}\selectfont 1.1.2 \tabto{0.49in} Where matches are broadcast, the camera specification set out below shall be mandatory as a minimum requirement.\par}\par

\end{adjustwidth}



%%%%%%%%%%%%%%%%%%%% Figure/Image No: 7 starts here %%%%%%%%%%%%%%%%%%%%

\begin{figure}[H]
\advance\leftskip 0.95in		\includegraphics[width=4.83in,height=2.92in]{./media/image7.jpeg}
\end{figure}


%%%%%%%%%%%%%%%%%%%% Figure/Image No: 7 Ends here %%%%%%%%%%%%%%%%%%%%

\par


\vspace{\baselineskip}

\vspace{\baselineskip}

\vspace{\baselineskip}

\vspace{\baselineskip}

\vspace{\baselineskip}

\vspace{\baselineskip}

\vspace{\baselineskip}

\vspace{\baselineskip}

\vspace{\baselineskip}

\vspace{\baselineskip}

\vspace{\baselineskip}

\vspace{\baselineskip}

\vspace{\baselineskip}

\vspace{\baselineskip}

\vspace{\baselineskip}

\vspace{\baselineskip}

\vspace{\baselineskip}

\vspace{\baselineskip}

\vspace{\baselineskip}

\vspace{\baselineskip}

\vspace{\baselineskip}

\vspace{\baselineskip}

\vspace{\baselineskip}
{\fontsize{9pt}{10.8pt}\selectfont 1.1.3 \tabto{0.49in} Where the camera specification set out above is provided, a third umpire shall be appointed to the match.\par}\par


\vspace{\baselineskip}
\begin{adjustwidth}{0.5in}{0.18in}
{\fontsize{9pt}{10.8pt}\selectfont 1.1.4 \tabto{0.49in} The provisions of paragraphs 1.1.1, 1.1.2, and 1.1.3 above shall not apply for matches between a Full Member country and Associate Member countries (whose matches have been granted T20I status) and for matches between such Associate Member countries.\par}\par

\end{adjustwidth}


\vspace{\baselineskip}
\begin{adjustwidth}{0.5in}{0.28in}
{\fontsize{9pt}{10.8pt}\selectfont 1.1.5 \tabto{0.49in} If the minimum requirements for DRS to be used are satisfied, both participating Boards may agree to employ the DRS for a T20I match. Otherwise, the third umpire shall be appointed and empowered to use broadcast replays to make decisions that are referred to him/her in accordance with paragraph (Umpire Reviews).\par}\par

\end{adjustwidth}


\vspace{\baselineskip}
\begin{adjustwidth}{0.5in}{0.12in}
{\fontsize{9pt}{10.8pt}\selectfont 1.1.6 \tabto{0.49in} The table below summarises the minimum requirements for DRS to be used, and the regulations around the appointment of the third umpire:\par}\par

\end{adjustwidth}

\par 
 \begin{tikzpicture}

\path (6.58in,-0.12in) node [shape=rectangle,draw,fill={rgb:red,0;green,0;blue,0},minimum height=0.01in,minimum width=0.01in,]{};

\end{tikzpicture}

\vspace{\baselineskip}


%%%%%%%%%%%%%%%%%%%% Table No: 2 starts here %%%%%%%%%%%%%%%%%%%%


\begin{table}[H]
 			\centering
\begin{tabular}{p{0.33in}p{0.33in}p{0.33in}p{0.33in}p{0.33in}p{0.33in}p{0.33in}p{0.33in}p{0.33in}p{0.33in}p{0.33in}p{0.33in}p{0.33in}p{0.33in}p{0.33in}p{0.33in}p{0.33in}p{0.33in}p{0.33in}p{0.33in}}
%row no:1
\multicolumn{1}{p{0.33in}}{\cellcolor[HTML]{D9D9D9}} & 
\multicolumn{1}{p{0.33in}}{\cellcolor[HTML]{D9D9D9}{\fontsize{9pt}{10.8pt}\selectfont \textbf{Third Umpire (non-DRS)}}} & 
\multicolumn{1}{p{0.33in}}{\cellcolor[HTML]{D9D9D9}{\fontsize{9pt}{10.8pt}\selectfont \textbf{DRS}}} & 
\multicolumn{1}{p{0.33in}}{\cellcolor[HTML]{D9D9D9}} & 
\multicolumn{1}{p{0.33in}}{} & 

\hhline{~~~~~}
%row no:2
\multicolumn{1}{p{0.33in}}{\cellcolor[HTML]{D9D9D9}} & 
\multicolumn{1}{p{0.33in}}{\cellcolor[HTML]{D9D9D9}} & 
\multicolumn{1}{p{0.33in}}{\cellcolor[HTML]{D9D9D9}} & 
\multicolumn{1}{p{0.33in}}{\cellcolor[HTML]{D9D9D9}} & 
\multicolumn{1}{p{0.33in}}{} & 

\hhline{~~~~~}
%row no:3
\multicolumn{1}{p{0.33in}}{{\fontsize{9pt}{10.8pt}\selectfont \textbf{Minimum}}} & 
\multicolumn{1}{p{0.33in}}{{\fontsize{9pt}{10.8pt}\selectfont Cameras Specification detailed in paragraph }} & 
\multicolumn{2}{p{0.61in}}{{\fontsize{9pt}{10.8pt}\selectfont Cameras}} & 
\multicolumn{1}{p{0.33in}}{} & 

\hhline{~~~~~}
%row no:4
\multicolumn{1}{p{0.33in}}{{\fontsize{9pt}{10.8pt}\selectfont \textbf{Requirement}}} & 
\multicolumn{1}{p{0.33in}}{} & 
\multicolumn{1}{p{0.33in}}{\multirow{1}{*}{\begin{tabular}{p{0.33in}}{\fontsize{9pt}{10.8pt}\selectfont -}\\\end{tabular}}} & 
\multicolumn{1}{p{0.33in}}{\multirow{1}{*}{\begin{tabular}{p{0.33in}}{\fontsize{9pt}{10.8pt}\selectfont Specification detailed in}\\\end{tabular}}} & 
\multicolumn{1}{p{0.33in}}{} & 

\hhline{~~~~~}
%row no:5
\multicolumn{1}{p{0.33in}}{} & 
\multicolumn{1}{p{0.33in}}{} & 
\multicolumn{1}{p{0.33in}}{\multirow{1}{*}{\begin{tabular}{p{0.33in}}\end{tabular}}} & 
\multicolumn{1}{p{0.33in}}{\multirow{1}{*}{\begin{tabular}{p{0.33in}}\end{tabular}}} & 
\multicolumn{1}{p{0.33in}}{} & 

\hhline{~~~~~}
%row no:6
\multicolumn{1}{p{0.33in}}{} & 
\multicolumn{1}{p{0.33in}}{} & 
\multicolumn{1}{p{0.33in}}{} & 
\multicolumn{1}{p{0.33in}}{{\fontsize{9pt}{10.8pt}\selectfont paragraph }} & 
\multicolumn{1}{p{0.33in}}{} & 

\hhline{~~~~~}
%row no:7
\multicolumn{1}{p{0.33in}}{} & 
\multicolumn{1}{p{0.33in}}{} & 
\multicolumn{1}{p{0.33in}}{} & 
\multicolumn{1}{p{0.33in}}{{\fontsize{9pt}{10.8pt}\selectfont Technology}} & 
\multicolumn{1}{p{0.33in}}{} & 

\hhline{~~~~~}
%row no:8
\multicolumn{1}{p{0.33in}}{} & 
\multicolumn{1}{p{0.33in}}{} & 
\multicolumn{1}{p{0.33in}}{{\fontsize{9pt}{10.8pt}\selectfont -}} & 
\multicolumn{1}{p{0.33in}}{{\fontsize{9pt}{10.8pt}\selectfont Approved ball-tracking}} & 
\multicolumn{1}{p{0.33in}}{} & 

\hhline{~~~~~}
%row no:9
\multicolumn{1}{p{0.33in}}{} & 
\multicolumn{1}{p{0.33in}}{} & 
\multicolumn{1}{p{0.33in}}{} & 
\multicolumn{1}{p{0.33in}}{{\fontsize{9pt}{10.8pt}\selectfont technology.}} & 
\multicolumn{1}{p{0.33in}}{} & 

\hhline{~~~~~}
%row no:10
\multicolumn{1}{p{0.33in}}{} & 
\multicolumn{1}{p{0.33in}}{} & 
\multicolumn{1}{p{0.33in}}{{\fontsize{9pt}{10.8pt}\selectfont -}} & 
\multicolumn{1}{p{0.33in}}{{\fontsize{9pt}{10.8pt}\selectfont Approved sound-based}} & 
\multicolumn{1}{p{0.33in}}{} & 

\hhline{~~~~~}
%row no:11
\multicolumn{1}{p{0.33in}}{} & 
\multicolumn{1}{p{0.33in}}{} & 
\multicolumn{1}{p{0.33in}}{} & 
\multicolumn{1}{p{0.33in}}{{\fontsize{9pt}{10.8pt}\selectfont edge detection}} & 
\multicolumn{1}{p{0.33in}}{} & 

\hhline{~~~~~}

\end{tabular}
 \end{table}


%%%%%%%%%%%%%%%%%%%% Table No: 2 ends here %%%%%%%%%%%%%%%%%%%%

\par 
 \begin{tikzpicture}

\path (6.58in,0.01in) node [shape=rectangle,draw,fill={rgb:red,0;green,0;blue,0},minimum height=0.01in,minimum width=0.01in,]{};

\end{tikzpicture}

\vspace{\baselineskip}

\vspace{\baselineskip}

\vspace{\baselineskip}

\vspace{\baselineskip}
\begin{adjustwidth}{0.0in}{0.08in}
\begin{Center}
{\fontsize{8pt}{9.6pt}\selectfont 73\par}
\end{Center}\par

\end{adjustwidth}


\vspace{\baselineskip}


%%%%%%%%%%%%%%%%%%%% Table No: 3 starts here %%%%%%%%%%%%%%%%%%%%


{
\scriptsize
\setlength\extrarowheight{3pt}
\begin{longtable}{p{0.19in}p{0.19in}p{0.19in}p{0.19in}p{0.19in}p{0.19in}p{0.19in}p{0.19in}p{0.19in}p{0.19in}p{0.19in}p{0.19in}p{0.19in}p{0.19in}p{0.19in}p{0.19in}p{0.19in}p{0.19in}p{0.19in}p{0.19in}p{0.19in}p{0.19in}p{0.19in}p{0.19in}p{0.19in}p{0.19in}p{0.19in}p{0.19in}p{0.19in}p{0.19in}p{0.19in}p{0.19in}p{0.19in}p{0.19in}p{0.19in}}

\endfirsthead
\multicolumn{35}{c}{\textit{continued from previous page}}\hline
\endhead\hline
\multicolumn{35}{r}{\textit{continued on next page}} \\
\endfoot
\hline 
\endlastfoot%row no:1
\multicolumn{1}{p{0.19in}}{} & 
\multicolumn{1}{p{0.19in}}{} & 
\multicolumn{1}{p{0.19in}}{} & 
\multicolumn{1}{p{0.19in}}{{\fontsize{9pt}{10.8pt}\selectfont technology.}} & 
\multicolumn{1}{p{0.19in}}{} & 

\hhline{~~~~~}
%row no:2
\multicolumn{1}{p{0.19in}}{} & 
\multicolumn{1}{p{0.19in}}{} & 
\multicolumn{1}{p{0.19in}}{} & 
\multicolumn{1}{p{0.19in}}{} & 
\multicolumn{1}{p{0.19in}}{} & 

\hhline{~~~~~}
%row no:3
\multicolumn{1}{p{0.19in}}{{\fontsize{9pt}{10.8pt}\selectfont \textbf{Third}}} & 
\multicolumn{1}{p{0.19in}}{{\fontsize{9pt}{10.8pt}\selectfont Appointed by Home Board.}} & 
\multicolumn{1}{p{0.19in}}{} & 
\multicolumn{1}{p{0.19in}}{{\fontsize{9pt}{10.8pt}\selectfont Appointed by the Home Board.}} & 
\multicolumn{1}{p{0.19in}}{} & 

\hhline{~~~~~}
%row no:4
\multicolumn{1}{p{0.19in}}{{\fontsize{9pt}{10.8pt}\selectfont \textbf{Umpire}}} & 
\multicolumn{1}{p{0.19in}}{\multirow{1}{*}{\begin{tabular}{p{0.19in}}{\fontsize{9pt}{10.8pt}\selectfont From the home country.}\\\end{tabular}}} & 
\multicolumn{1}{p{0.19in}}{} & 
\multicolumn{1}{p{0.19in}}{\multirow{1}{*}{\begin{tabular}{p{0.19in}}{\fontsize{9pt}{10.8pt}\selectfont From ICC Elite Panel or}\\\end{tabular}}} & 
\multicolumn{1}{p{0.19in}}{} & 

\hhline{~~~~~}
%row no:5
\multicolumn{1}{p{0.19in}}{{\fontsize{9pt}{10.8pt}\selectfont \textbf{Appointment}}} & 
\multicolumn{1}{p{0.19in}}{\multirow{1}{*}{\begin{tabular}{p{0.19in}}\end{tabular}}} & 
\multicolumn{1}{p{0.19in}}{} & 
\multicolumn{1}{p{0.19in}}{\multirow{1}{*}{\begin{tabular}{p{0.19in}}\end{tabular}}} & 
\multicolumn{1}{p{0.19in}}{} & 

\hhline{~~~~~}
%row no:6
\multicolumn{1}{p{0.19in}}{} & 
\multicolumn{1}{p{0.19in}}{\multirow{1}{*}{\begin{tabular}{p{0.19in}}{\fontsize{9pt}{10.8pt}\selectfont From ICC Elite Panel or International Panel of}\\\end{tabular}}} & 
\multicolumn{1}{p{0.19in}}{} & 
\multicolumn{1}{p{0.19in}}{{\fontsize{9pt}{10.8pt}\selectfont International Panel of umpires.}} & 
\multicolumn{1}{p{0.19in}}{} & 

\hhline{~~~~~}
%row no:7
\multicolumn{1}{p{0.19in}}{} & 
\multicolumn{1}{p{0.19in}}{\multirow{1}{*}{\begin{tabular}{p{0.19in}}\end{tabular}}} & 
\multicolumn{1}{p{0.19in}}{} & 
\multicolumn{1}{p{0.19in}}{} & 
\multicolumn{1}{p{0.19in}}{} & 

\hhline{~~~~~}
%row no:8
\multicolumn{1}{p{0.19in}}{} & 
\multicolumn{1}{p{0.19in}}{{\fontsize{9pt}{10.8pt}\selectfont umpires.}} & 
\multicolumn{1}{p{0.19in}}{} & 
\multicolumn{1}{p{0.19in}}{} & 
\multicolumn{1}{p{0.19in}}{} & 

\hhline{~~~~~}
%row no:9
\multicolumn{1}{p{0.19in}}{} & 
\multicolumn{1}{p{0.19in}}{} & 
\multicolumn{1}{p{0.19in}}{} & 
\multicolumn{1}{p{0.19in}}{} & 
\multicolumn{1}{p{0.19in}}{} & 

\hhline{~~~~~}
%row no:10
\multicolumn{1}{p{0.19in}}{{\fontsize{9pt}{10.8pt}\selectfont \textbf{Third}}} & 
\multicolumn{1}{p{0.19in}}{{\fontsize{9pt}{10.8pt}\selectfont Umpire Reviews only}} & 
\multicolumn{1}{p{0.19in}}{} & 
\multicolumn{1}{p{0.19in}}{{\fontsize{9pt}{10.8pt}\selectfont Umpire Reviews and Player}} & 
\multicolumn{1}{p{0.19in}}{} & 

\hhline{~~~~~}
%row no:11
\multicolumn{1}{p{0.19in}}{{\fontsize{9pt}{10.8pt}\selectfont \textbf{Umpire}}} & 
\multicolumn{1}{p{0.19in}}{} & 
\multicolumn{1}{p{0.19in}}{} & 
\multicolumn{1}{p{0.19in}}{{\fontsize{9pt}{10.8pt}\selectfont Reviews}} & 
\multicolumn{1}{p{0.19in}}{} & 

\hhline{~~~~~}
%row no:12
\multicolumn{1}{p{0.19in}}{{\fontsize{9pt}{10.8pt}\selectfont \textbf{Jurisdiction}}} & 
\multicolumn{1}{p{0.19in}}{} & 
\multicolumn{1}{p{0.19in}}{} & 
\multicolumn{1}{p{0.19in}}{} & 
\multicolumn{1}{p{0.19in}}{} & 

\hhline{~~~~~}
%row no:13
\multicolumn{1}{p{0.19in}}{} & 
\multicolumn{1}{p{0.19in}}{} & 
\multicolumn{1}{p{0.19in}}{} & 
\multicolumn{1}{p{0.19in}}{} & 
\multicolumn{1}{p{0.19in}}{} & 

\hhline{~~~~~}
%row no:14
\multicolumn{1}{p{0.19in}}{{\fontsize{9pt}{10.8pt}\selectfont \textbf{Replays that}}} & 
\multicolumn{1}{p{0.19in}}{{\fontsize{9pt}{10.8pt}\selectfont The third umpire shall only have access to replays of}} & 
\multicolumn{1}{p{0.19in}}{} & 
\multicolumn{1}{p{0.19in}}{{\fontsize{9pt}{10.8pt}\selectfont Any replay, stump microphone}} & 
\multicolumn{1}{p{0.19in}}{} & 

\hhline{~~~~~}
%row no:15
\multicolumn{1}{p{0.19in}}{{\fontsize{9pt}{10.8pt}\selectfont \textbf{can be used}}} & 
\multicolumn{1}{p{0.19in}}{{\fontsize{9pt}{10.8pt}\selectfont any camera images. Other technology which may be}} & 
\multicolumn{1}{p{0.19in}}{} & 
\multicolumn{1}{p{0.19in}}{{\fontsize{9pt}{10.8pt}\selectfont audio or technology detailed in}} & 
\multicolumn{1}{p{0.19in}}{} & 

\hhline{~~~~~}
%row no:16
\multicolumn{1}{p{0.19in}}{} & 
\multicolumn{1}{p{0.19in}}{{\fontsize{9pt}{10.8pt}\selectfont in use by the broadcaster for broadcast purposes (for}} & 
\multicolumn{1}{p{0.19in}}{} & 
\multicolumn{1}{p{0.19in}}{{\fontsize{9pt}{10.8pt}\selectfont paragraph below.}} & 
\multicolumn{1}{p{0.19in}}{} & 

\hhline{~~~~~}
%row no:17
\multicolumn{1}{p{0.19in}}{} & 
\multicolumn{1}{p{0.19in}}{{\fontsize{9pt}{10.8pt}\selectfont example, ball-tracking technology, sound-based edge}} & 
\multicolumn{1}{p{0.19in}}{} & 
\multicolumn{1}{p{0.19in}}{} & 
\multicolumn{1}{p{0.19in}}{} & 

\hhline{~~~~~}
%row no:18
\multicolumn{1}{p{0.19in}}{} & 
\multicolumn{1}{p{0.19in}}{{\fontsize{9pt}{10.8pt}\selectfont detection technology, and heat-based edge detection}} & 
\multicolumn{1}{p{0.19in}}{} & 
\multicolumn{1}{p{0.19in}}{} & 
\multicolumn{1}{p{0.19in}}{} & 

\hhline{~~~~~}
%row no:19
\multicolumn{1}{p{0.19in}}{} & 
\multicolumn{1}{p{0.19in}}{{\fontsize{9pt}{10.8pt}\selectfont technology) shall not be used during Umpire Reviews.}} & 
\multicolumn{1}{p{0.19in}}{} & 
\multicolumn{1}{p{0.19in}}{} & 
\multicolumn{1}{p{0.19in}}{} & 

\hhline{~~~~~}
%row no:20
\multicolumn{1}{p{0.19in}}{} & 
\multicolumn{1}{p{0.19in}}{} & 
\multicolumn{1}{p{0.19in}}{} & 
\multicolumn{1}{p{0.19in}}{} & 
\multicolumn{1}{p{0.19in}}{} & 

\hhline{~~~~~}
%row no:21
\multicolumn{1}{p{0.19in}}{{\fontsize{9pt}{10.8pt}\selectfont \textbf{ICC}}} & 
\multicolumn{1}{p{0.19in}}{{\fontsize{9pt}{10.8pt}\selectfont Not required.}} & 
\multicolumn{1}{p{0.19in}}{} & 
\multicolumn{1}{p{0.19in}}{{\fontsize{9pt}{10.8pt}\selectfont The ICC shall appoint an}} & 
\multicolumn{1}{p{0.19in}}{} & 

\hhline{~~~~~}
%row no:22
\multicolumn{1}{p{0.19in}}{{\fontsize{9pt}{10.8pt}\selectfont \textbf{Technical}}} & 
\multicolumn{1}{p{0.19in}}{} & 
\multicolumn{1}{p{0.19in}}{} & 
\multicolumn{1}{p{0.19in}}{{\fontsize{9pt}{10.8pt}\selectfont independent technology expert}} & 
\multicolumn{1}{p{0.19in}}{} & 

\hhline{~~~~~}
%row no:23
\multicolumn{1}{p{0.19in}}{{\fontsize{9pt}{10.8pt}\selectfont \textbf{Officer}}} & 
\multicolumn{1}{p{0.19in}}{} & 
\multicolumn{1}{p{0.19in}}{} & 
\multicolumn{1}{p{0.19in}}{{\fontsize{9pt}{10.8pt}\selectfont (ICC Technical Officer) to be}} & 
\multicolumn{1}{p{0.19in}}{} & 

\hhline{~~~~~}
%row no:24
\multicolumn{1}{p{0.19in}}{} & 
\multicolumn{1}{p{0.19in}}{} & 
\multicolumn{1}{p{0.19in}}{} & 
\multicolumn{1}{p{0.19in}}{{\fontsize{9pt}{10.8pt}\selectfont present at every series in which}} & 
\multicolumn{1}{p{0.19in}}{} & 

\hhline{~~~~~}
%row no:25
\multicolumn{1}{p{0.19in}}{} & 
\multicolumn{1}{p{0.19in}}{} & 
\multicolumn{1}{p{0.19in}}{} & 
\multicolumn{1}{p{0.19in}}{{\fontsize{9pt}{10.8pt}\selectfont the DRS is used to assist the third}} & 
\multicolumn{1}{p{0.19in}}{} & 

\hhline{~~~~~}
%row no:26
\multicolumn{1}{p{0.19in}}{} & 
\multicolumn{1}{p{0.19in}}{} & 
\multicolumn{1}{p{0.19in}}{} & 
\multicolumn{1}{p{0.19in}}{{\fontsize{9pt}{10.8pt}\selectfont umpire and to protect the integrity}} & 
\multicolumn{1}{p{0.19in}}{} & 

\hhline{~~~~~}
%row no:27
\multicolumn{1}{p{0.19in}}{} & 
\multicolumn{1}{p{0.19in}}{} & 
\multicolumn{1}{p{0.19in}}{} & 
\multicolumn{1}{p{0.19in}}{{\fontsize{9pt}{10.8pt}\selectfont of the DRS process.}} & 
\multicolumn{1}{p{0.19in}}{} & 

\hhline{~~~~~}
%row no:28
\multicolumn{1}{p{0.19in}}{} & 
\multicolumn{1}{p{0.19in}}{} & 
\multicolumn{1}{p{0.19in}}{} & 
\multicolumn{1}{p{0.19in}}{} & 
\multicolumn{1}{p{0.19in}}{} & 

\hhline{~~~~~}

\end{longtable}}

%%%%%%%%%%%%%%%%%%%% Table No: 3 ends here %%%%%%%%%%%%%%%%%%%%

\par 
 \begin{tikzpicture}

\path (6.58in,0.01in) node [shape=rectangle,draw,fill={rgb:red,0;green,0;blue,0},minimum height=0.01in,minimum width=0.01in,]{};

\end{tikzpicture}

\vspace{\baselineskip}
\begin{adjustwidth}{0.5in}{0.24in}
{\fontsize{9pt}{10.8pt}\selectfont 1.1.7 \tabto{0.49in} The Home Board shall ensure that a separate room is provided for the third umpire and that he/she has access to the television equipment and technology (where DRS is used) so as to be in the best position to facilitate the referral and/or consultation processes referred to in paragraphs (Umpire Review) and  (Player Review) below.\par}\par

\end{adjustwidth}


\vspace{\baselineskip}
{\fontsize{16pt}{19.2pt}\selectfont \textbf{2 \tabto{0.29in} }{\fontsize{15pt}{18.0pt}\selectfont \textbf{Umpire Review}\par}\par}\par


\vspace{\baselineskip}
\begin{adjustwidth}{0.49in}{0.19in}
\begin{justify}
{\fontsize{9pt}{10.8pt}\selectfont In the circumstances detailed in paragraphs and below, the on-field umpire shall have the discretion to refer the decision to the third umpire or, in the case of paragraphs and to consult with the third umpire before making the decision.\par}
\end{justify}\par

\end{adjustwidth}


\vspace{\baselineskip}
\begin{adjustwidth}{0.49in}{0.12in}
{\fontsize{9pt}{10.8pt}\selectfont Save for requesting the umpire to review his/her decision under paragraph (Player Review) below, players may not appeal to the on-field umpires to use the Umpire Review. Breach of this provision may constitute dissent and the player may be subject to disciplinary action under the ICC Code of Conduct for Players and Player Support Personnel.\par}\par

\end{adjustwidth}


\vspace{\baselineskip}
{\fontsize{9pt}{10.8pt}\selectfont 2.1 \tabto{0.39in} {\fontsize{8pt}{9.6pt}\selectfont Run Out, Stumped, Bowled and Hit Wicket Decisions\par}\par}\par


\vspace{\baselineskip}
\begin{adjustwidth}{0.5in}{0.15in}
{\fontsize{9pt}{10.8pt}\selectfont 2.1.1 \tabto{0.49in} The relevant on-field umpire shall be entitled to refer an appeal for run-out, stumped, bowled or hit wicket to the third umpire.\par}\par

\end{adjustwidth}


\vspace{\baselineskip}
\begin{adjustwidth}{0.5in}{0.25in}
{\fontsize{9pt}{10.8pt}\selectfont 2.1.2 \tabto{0.49in} An on-field umpire wishing to refer a decision to the third umpire shall signal to the third umpire by making the shape of a TV screen with his/her hands.\par}\par

\end{adjustwidth}


\vspace{\baselineskip}
\begin{adjustwidth}{0.5in}{0.14in}
{\fontsize{9pt}{10.8pt}\selectfont 2.1.3 \tabto{0.49in} In the case of a referral of a bowled, hit wicket or stumped decision, the third umpire shall first check the fairness of the delivery (all modes of No ball except for the bowler using an Illegal Bowling Action, subject to the proviso that the third umpire may review whether the bowler has used a prohibited Specific Variation under Article 6.2 of the Illegal Bowling Regulations). If the delivery was not a fair delivery the third umpire shall indicate that the batsman is Not out and advise the on-field umpire to signal No ball. See also paragraph below.\par}\par

\end{adjustwidth}


\vspace{\baselineskip}

\vspace{\baselineskip}

\vspace{\baselineskip}

\vspace{\baselineskip}

\vspace{\baselineskip}

\vspace{\baselineskip}

\vspace{\baselineskip}
\begin{adjustwidth}{0.0in}{0.08in}
\begin{Center}
{\fontsize{8pt}{9.6pt}\selectfont 74\par}
\end{Center}\par

\end{adjustwidth}


\vspace{\baselineskip}

\vspace{\baselineskip}
\begin{adjustwidth}{0.5in}{0.03in}
{\fontsize{9pt}{10.8pt}\selectfont 2.1.4 \tabto{0.49in} Additionally, if the third umpire finds the batsman is Out by another mode of dismissal (excluding LBW), or Not out by any mode of dismissal (excluding LBW), he/she shall notify the on-field umpire so that the correct decision is made.\par}\par

\end{adjustwidth}


\vspace{\baselineskip}
\begin{adjustwidth}{0.5in}{0.1in}
{\fontsize{9pt}{10.8pt}\selectfont 2.1.5 \tabto{0.49in} If the third umpire decides that the batsman is Out, a red light shall be displayed; if the third umpire decides that the batsman is Not out, a green light shall be displayed. Should the third umpire be temporarily unable to respond, a white light (where available) shall remain illuminated throughout the period of interruption to signify to the on-field umpires that Umpire Reviews are temporarily unavailable, in which case the decision shall be taken by the on-field umpire. As an alternative to the red/green light system, the replay screen (where available) may be used for the purpose of conveying the third umpire’s decision, in line with the ICC Big Screen Policy.\par}\par

\end{adjustwidth}


\vspace{\baselineskip}
{\fontsize{11pt}{13.2pt}\selectfont \textbf{2.2 \tabto{0.47in} Caught Decisions, Obstructing the Field}\par}\par


\vspace{\baselineskip}
\begin{adjustwidth}{0.5in}{0.14in}
{\fontsize{9pt}{10.8pt}\selectfont 2.2.1 \tabto{0.49in} {\fontsize{8pt}{9.6pt}\selectfont Where the bowler’s end umpire is unable to decide upon a Fair Catch or a Bump Ball, or if, on appeal from the fielding side, the batsman obstructed the field, he/she shall first consult with the striker’s end umpire.\par}\par}\par

\end{adjustwidth}


\vspace{\baselineskip}
\begin{adjustwidth}{0.5in}{0.03in}
{\fontsize{9pt}{10.8pt}\selectfont 2.2.2 \tabto{0.49in} Should both on-field umpires require assistance from the third umpire to make a decision, the bowler’s end umpire shall firstly take a decision on-field after consulting with the striker’s end umpire, before consulting by two-way radio with the third umpire. Such consultation shall be initiated by the bowler’s end umpire to the third umpire by making the shape of a TV screen with his/her hands, followed by a Soft Signal of Out or Not out made with the hands close to the chest at chest height. If the third umpire advises that the replay evidence is inconclusive, the on-field decision communicated at the start of the consultation process shall stand.\par}\par

\end{adjustwidth}


\vspace{\baselineskip}
\begin{adjustwidth}{0.5in}{0.01in}
{\fontsize{9pt}{10.8pt}\selectfont 2.2.3 \tabto{0.49in} The third umpire shall determine whether the batsman has been caught, whether the delivery was a Bump Ball, or if the batsman obstructed the field. However, in reviewing the television replay(s), the third umpire shall first check the fairness of the delivery for all decisions involving a catch (all modes of No ball except for the bowler using an Illegal Bowling Action, subject to the proviso that the third umpire may review whether the bowler has used a prohibited Specific Variation under Article 6.2 of the Illegal Bowling Regulations) and whether the batsman has hit the ball. If the delivery was not a fair delivery or if it is clear to the third umpire that the batsman did not hit the ball he/she shall indicate to the bowler’s end umpire that the batsman is Not out caught, and in the case of an unfair delivery, advise the bowler’s end umpire to signal No ball. See also paragraph below. Additionally, if it is clear to the third umpire that the batsman is Out by another mode of dismissal (excluding LBW), or Not out by any mode of dismissal (excluding LBW), he/she shall notify the bowler’s end umpire so that the correct decision can be made.\par}\par

\end{adjustwidth}


\vspace{\baselineskip}
{\fontsize{9pt}{10.8pt}\selectfont 2.2.4 \tabto{0.49in} {\fontsize{8pt}{9.6pt}\selectfont The third umpire shall communicate his/her decision as set out in paragraph \par}\par}\par


\vspace{\baselineskip}
{\fontsize{11pt}{13.2pt}\selectfont \textbf{2.3 \tabto{0.47in} Boundary Decisions}\par}\par


\vspace{\baselineskip}
{\fontsize{9pt}{10.8pt}\selectfont 2.3.1 \tabto{0.49in} {\fontsize{8pt}{9.6pt}\selectfont The bowler’s end umpire shall be entitled to refer to the third umpire for a decision on:\par}\par}\par


\vspace{\baselineskip}
\begin{adjustwidth}{0.49in}{0.0in}
{\fontsize{9pt}{10.8pt}\selectfont 2.3.1.1 \tabto{1.17in} {\fontsize{8pt}{9.6pt}\selectfont whether a four or six has been scored;\par}\par}\par

\end{adjustwidth}


\vspace{\baselineskip}
\begin{adjustwidth}{1.18in}{0.24in}
{\fontsize{9pt}{10.8pt}\selectfont 2.3.1.2 \tabto{1.17in} whether a fielder had any part of his/her person in contact with the ball when he touched the boundary; or\par}\par

\end{adjustwidth}


\vspace{\baselineskip}
\begin{adjustwidth}{1.18in}{0.1in}
{\fontsize{9pt}{10.8pt}\selectfont 2.3.1.3 \tabto{1.17in} whether the fielder had any part of his/her person in contact with the ball when he had any part of his person grounded beyond the boundary.\par}\par

\end{adjustwidth}


\vspace{\baselineskip}
{\fontsize{9pt}{10.8pt}\selectfont 2.3.2 \tabto{0.49in} A decision shall be made immediately and cannot be changed thereafter.\par}\par


\vspace{\baselineskip}
\begin{adjustwidth}{0.5in}{0.39in}
{\fontsize{9pt}{10.8pt}\selectfont 2.3.3 \tabto{0.49in} If the television evidence is inconclusive as to whether or not a boundary has been scored, the default presumption shall be in favour of no boundary being awarded.\par}\par

\end{adjustwidth}


\vspace{\baselineskip}
\begin{adjustwidth}{0.5in}{0.03in}
\begin{justify}
{\fontsize{9pt}{10.8pt}\selectfont 2.3.4 \tabto{0.49in} Where the bowler’s end umpire wishes to use the assistance of the third umpire in this circumstance, he/she shall communicate with the third umpire by use of a two-way radio and the third umpire shall convey his/her decision to the bowler’s end umpire by the same method.\par}
\end{justify}\par

\end{adjustwidth}


\vspace{\baselineskip}
\begin{adjustwidth}{0.5in}{0.24in}
{\fontsize{9pt}{10.8pt}\selectfont 2.3.5 \tabto{0.49in} {\fontsize{8pt}{9.6pt}\selectfont The third umpire may initiate contact with the on-field umpire by two-way radio if TV coverage shows a boundary line infringement or incident that appears not to have been acted upon by the on-field umpires.\par}\par}\par

\end{adjustwidth}


\vspace{\baselineskip}

\vspace{\baselineskip}

\vspace{\baselineskip}

\vspace{\baselineskip}
\begin{Center}
{\fontsize{8pt}{9.6pt}\selectfont 75\par}
\end{Center}\par


\vspace{\baselineskip}
{\fontsize{11pt}{13.2pt}\selectfont \textbf{2.4 \tabto{0.47in} Batsmen Running to the Same End}\par}\par


\vspace{\baselineskip}
\begin{adjustwidth}{0.5in}{0.1in}
{\fontsize{9pt}{10.8pt}\selectfont 2.4.1 \tabto{0.49in} Where both batsmen have run to the same end and the on-field umpires are uncertain over which batsman made his/her ground first, the on-field umpires may consult with the third umpire.\par}\par

\end{adjustwidth}


\vspace{\baselineskip}
{\fontsize{9pt}{10.8pt}\selectfont 2.4.2 \tabto{0.49in} {\fontsize{8pt}{9.6pt}\selectfont The procedure set out in paragraph shall apply.\par}\par}\par


\vspace{\baselineskip}
{\fontsize{11pt}{13.2pt}\selectfont \textbf{2.5 \tabto{0.47in} No Balls}\par}\par


\vspace{\baselineskip}
\begin{adjustwidth}{0.5in}{0.04in}
{\fontsize{9pt}{10.8pt}\selectfont 2.5.1 \tabto{0.49in} If the bowler’s end umpire is uncertain as to the fairness of the delivery following a dismissal, either affecting the validity of the dismissal or which batsman is dismissed, he/she shall be entitled to request the batsman to delay leaving the field and to check the fairness of the delivery with the third umpire. Communication with the third umpire shall be by two-way radio.\par}\par

\end{adjustwidth}


\vspace{\baselineskip}
\begin{adjustwidth}{0.5in}{0.03in}
{\fontsize{9pt}{10.8pt}\selectfont 2.5.2 \tabto{0.49in} The third umpire shall check all modes of No ball except for the bowler using an Illegal Bowling Action (subject to the proviso that the third umpire may review whether the bowler has used a prohibited Specific Variation under Article 6.2 of the Illegal Bowling Regulations). The third umpire shall apply clause when deciding whether a No ball should have been called (and must therefore be satisfied that none of the three conditions in clause have been met before calling a No ball).\par}\par

\end{adjustwidth}


\vspace{\baselineskip}
\begin{adjustwidth}{0.5in}{0.01in}
{\fontsize{9pt}{10.8pt}\selectfont 2.5.3 \tabto{0.49in} If the delivery was not a fair delivery, the bowler’s end umpire shall indicate that the batsman is Not out and signal No ball (except in the case of a dismissal for obstructing the field, which may still be effected despite a No ball being called, in which case the bowler’s end umpire shall indicate that the relevant batsman is Out and additionally call a No ball).\par}\par

\end{adjustwidth}


\vspace{\baselineskip}
\begin{adjustwidth}{0.5in}{0.01in}
{\fontsize{9pt}{10.8pt}\selectfont 2.5.4 \tabto{0.49in} If a No ball is called following the check by the third umpire, the batting side shall benefit from the reversal of the dismissal and the one run for the No ball, but shall not benefit from any runs that may subsequently have accrued from the delivery had the on-field umpire originally called a No ball. Where the batsmen crossed while the ball was in the air before being caught, the batsmen shall remain at the same ends as if the striker had been dismissed, but no runs shall be credited to the striker even if one (or more) runs were completed prior to the catch being taken.\par}\par

\end{adjustwidth}


\vspace{\baselineskip}
{\fontsize{11pt}{13.2pt}\selectfont \textbf{2.6 \tabto{0.47in} Cameras On or Over the Field of Play}\par}\par


\vspace{\baselineskip}
\begin{adjustwidth}{0.5in}{0.07in}
\begin{justify}
{\fontsize{9pt}{10.8pt}\selectfont 2.6.1 \tabto{0.49in} The on-field umpires shall be entitled to refer to the third umpire for a decision as to whether the ball has at any time during the normal course of play come into contact with any part of the camera, its apparatus or its cables above the playing area, as contemplated in clause \par}
\end{justify}\par

\end{adjustwidth}


\vspace{\baselineskip}
\begin{adjustwidth}{0.5in}{0.07in}
{\fontsize{9pt}{10.8pt}\selectfont 2.6.2 \tabto{0.49in} Where an on-field umpire wishes to use the assistance of the third umpire in this circumstance, he/she shall communicate with the third umpire by use of a two-way radio and the third umpire shall convey his/her decision to the bowler’s end umpire by the same method.\par}\par

\end{adjustwidth}


\vspace{\baselineskip}
\begin{adjustwidth}{0.5in}{0.03in}
{\fontsize{9pt}{10.8pt}\selectfont 2.6.3 \tabto{0.49in} A decision shall be made immediately and cannot be changed thereafter. If the television evidence is inconclusive as to whether or not the ball has come into contact with any part of the camera, its apparatus or its cables above the playing area, the default presumption shall be in favour of no contact having been made.\par}\par

\end{adjustwidth}


\vspace{\baselineskip}
\begin{adjustwidth}{0.5in}{0.01in}
\begin{justify}
{\fontsize{9pt}{10.8pt}\selectfont 2.6.4 \tabto{0.49in} The third umpire may initiate contact with the on-field umpire by two-way radio if TV coverage shows the ball to have been in contact with any part of the camera or its cables above the playing area as envisaged under this paragraph.\par}
\end{justify}\par

\end{adjustwidth}


\vspace{\baselineskip}
{\fontsize{16pt}{19.2pt}\selectfont \textbf{3 \tabto{0.29in} }{\fontsize{15pt}{18.0pt}\selectfont \textbf{Player Review}\par}\par}\par


\vspace{\baselineskip}
\begin{adjustwidth}{0.49in}{0.0in}
{\fontsize{9pt}{10.8pt}\selectfont The following paragraphs shall operate in addition to and in conjunction with paragraph (Umpire Review).\par}\par

\end{adjustwidth}


\vspace{\baselineskip}
{\fontsize{11pt}{13.2pt}\selectfont \textbf{3.1 \tabto{0.47in} Circumstances in which a Player Review may be requested}\par}\par


\vspace{\baselineskip}
\begin{adjustwidth}{0.5in}{0.22in}
{\fontsize{9pt}{10.8pt}\selectfont 3.1.1 \tabto{0.49in} A player may request a review of any decision taken by the on-field umpires concerning whether or not a batsman is dismissed, with the exception of ‘Timed Out’ (Player Review).\par}\par

\end{adjustwidth}


\vspace{\baselineskip}
\begin{adjustwidth}{0.0in}{0.56in}
\begin{FlushRight}
{\fontsize{9pt}{10.8pt}\selectfont 3.1.2\ \ \  No other decisions made by the umpires are eligible for a Player Review with the exception of Fair Catch/Bump Ball (even after the third umpire has been consulted and the decision communicated).\par}
\end{FlushRight}\par

\end{adjustwidth}


\vspace{\baselineskip}

\vspace{\baselineskip}

\vspace{\baselineskip}
\begin{Center}
{\fontsize{8pt}{9.6pt}\selectfont 76\par}
\end{Center}\par


\vspace{\baselineskip}

\vspace{\baselineskip}
\begin{adjustwidth}{0.5in}{0.32in}
{\fontsize{9pt}{10.8pt}\selectfont 3.1.3 \tabto{0.49in} Only the batsman involved in a dismissal may request a Player Review of an Out decision and only the captain (or acting captain) of the fielding team may request a Player Review of a Not out decision.\par}\par

\end{adjustwidth}


\vspace{\baselineskip}
\begin{adjustwidth}{0.5in}{0.1in}
\begin{justify}
{\fontsize{9pt}{10.8pt}\selectfont 3.1.4 \tabto{0.49in} A decision concerning whether or not a batsman is dismissed that could have been the subject of a Umpire Review under paragraph is eligible for a Player Review as soon as it is clear that the on-field umpire has chosen not to initiate the Umpire Review.\par}
\end{justify}\par

\end{adjustwidth}


\vspace{\baselineskip}
{\fontsize{11pt}{13.2pt}\selectfont \textbf{3.2 \tabto{0.47in} The manner of requesting the Player Review}\par}\par


\vspace{\baselineskip}
{\fontsize{9pt}{10.8pt}\selectfont 3.2.1 \tabto{0.49in} {\fontsize{8pt}{9.6pt}\selectfont The request shall be made by the player making a ‘T’ sign with both forearms at head height.\par}\par}\par


\vspace{\baselineskip}
\begin{adjustwidth}{0.5in}{0.06in}
{\fontsize{9pt}{10.8pt}\selectfont 3.2.2 \tabto{0.49in} The total time elapsed between the ball becoming dead and the review request being made shall be no more than 15 seconds. The only exception permitted shall be when an Umpire Review for Fair Catch or Bump Ball (as permitted in paragraph above) is required to answer an appeal for a caught decision, in which case either team is able to request a Player Review of that caught decision within 15 seconds of the decision being communicated. The bowler’s end umpire shall provide the relevant player with a prompt after 10 seconds if the request has not been made at that time and the player shall request the review immediately thereafter. If the on-field umpires believe that a request has not been made within the 15 second time limit, they shall decline the request for a Player Review.\par}\par

\end{adjustwidth}


\vspace{\baselineskip}
\begin{adjustwidth}{0.5in}{0.01in}
{\fontsize{9pt}{10.8pt}\selectfont 3.2.3 \tabto{0.49in} The captain may consult with the bowler and other fielders, and the two batsmen may consult with each other prior to deciding whether to request a Player Review. Under no circumstances is any player permitted to query an umpire about any aspect of a decision before deciding on whether or not to request a Player Review. If the on-field umpires believe that the captain or either batsman has received direct or indirect input emanating other than from the players on the field, then they may at their discretion decline the request for a Player Review. In particular, signals from the dressing room must not be given.\par}\par

\end{adjustwidth}


\vspace{\baselineskip}
\begin{adjustwidth}{0.5in}{0.01in}
{\fontsize{9pt}{10.8pt}\selectfont 3.2.4 \tabto{0.49in} No replays, either at normal speed or slow motion, shall be shown on a big screen to spectators until the 15 second time limit allowed for requesting a Player Review has elapsed. The only exception to this provision is where a Player Review of a caught decision is requested after the Umpire Review of a Fair Catch or Bump Ball has concluded, as detailed in paragraph above (due to the fact that replays may have been shown on the big screen during that Umpire Review process).\par}\par

\end{adjustwidth}


\vspace{\baselineskip}
\begin{adjustwidth}{0.5in}{0.07in}
{\fontsize{9pt}{10.8pt}\selectfont 3.2.5 \tabto{0.49in} Where either on-field umpire initiates an Umpire Review, this does not preclude a player seeking a Player Review of a separate incident from the same delivery. The request for a Player Review may be made after the Umpire Review, provided the request is still within the 15 second time limit described in paragraph  above. (See paragraphs and below for the process for addressing both an Umpire and Player Review).\par}\par

\end{adjustwidth}


\vspace{\baselineskip}
{\fontsize{9pt}{10.8pt}\selectfont 3.2.6 \tabto{0.49in} {\fontsize{8pt}{9.6pt}\selectfont A request for a Player Review cannot be withdrawn once it has been made.\par}\par}\par


\vspace{\baselineskip}
{\fontsize{11pt}{13.2pt}\selectfont \textbf{3.3 \tabto{0.47in} The process of consultation}\par}\par


\vspace{\baselineskip}
\begin{adjustwidth}{0.5in}{0.08in}
{\fontsize{9pt}{10.8pt}\selectfont 3.3.1 \tabto{0.49in} On receipt of an eligible and timely request for a Player Review, the relevant on-field umpire shall make the sign of a shape of a TV screen with his/her hands in the normal way.\par}\par

\end{adjustwidth}


\vspace{\baselineskip}
{\fontsize{9pt}{10.8pt}\selectfont 3.3.2 \tabto{0.49in} {\fontsize{8pt}{9.6pt}\selectfont The relevant on-field umpire shall initiate communication with the third umpire by confirming;\par}\par}\par


\vspace{\baselineskip}
\begin{adjustwidth}{0.49in}{0.0in}
{\fontsize{9pt}{10.8pt}\selectfont 3.3.2.1 \tabto{1.17in} {\fontsize{8pt}{9.6pt}\selectfont That a Player Review has been requested,\par}\par}\par

\end{adjustwidth}


\vspace{\baselineskip}
\begin{adjustwidth}{0.49in}{0.0in}
{\fontsize{9pt}{10.8pt}\selectfont 3.3.2.2 \tabto{1.17in} The mode of dismissal for which the relevant on-field umpire adjudicated the appeal,\par}\par

\end{adjustwidth}


\vspace{\baselineskip}
\begin{adjustwidth}{0.49in}{0.0in}
{\fontsize{9pt}{10.8pt}\selectfont 3.3.2.3 \tabto{1.17in} The decision that has been made (Out or Not out), and;\par}\par

\end{adjustwidth}


\vspace{\baselineskip}
\begin{adjustwidth}{1.18in}{0.12in}
\begin{justify}
{\fontsize{9pt}{10.8pt}\selectfont 3.3.2.4 \tabto{1.17in} For LBW appeals, where relevant, if the bowler’s end umpire believed that the striker made no genuine attempt to play the ball with the bat (the default presumption of the third umpire in the absence of any information on this point from the bowler’s end umpire shall be that a genuine attempt to play the ball with the bat was made).\par}
\end{justify}\par

\end{adjustwidth}


\vspace{\baselineskip}
\begin{adjustwidth}{0.5in}{0.03in}
{\fontsize{9pt}{10.8pt}\selectfont 3.3.3 \tabto{0.49in} A two-way consultation process shall begin to investigate whether there is anything that the third umpire can see or hear which would indicate that the on-field umpire should change his/her original decision.\par}\par

\end{adjustwidth}


\vspace{\baselineskip}

\vspace{\baselineskip}

\vspace{\baselineskip}

\vspace{\baselineskip}

\vspace{\baselineskip}

\vspace{\baselineskip}
\begin{Center}
{\fontsize{8pt}{9.6pt}\selectfont 77\par}
\end{Center}\par


\vspace{\baselineskip}

\vspace{\baselineskip}
\begin{adjustwidth}{0.5in}{0.15in}
{\fontsize{9pt}{10.8pt}\selectfont 3.3.4 \tabto{0.49in} The third umpire shall not withhold any factual information which may help in the decision making process. In particular, in reviewing a dismissal, if the third umpire believes that the batsman may instead be Out by any other mode of dismissal, he/she shall advise the on-field umpire accordingly. The process of consultation described in this paragraph in respect of such other mode of dismissal shall then be conducted as if the batsman has been given Not out.\par}\par

\end{adjustwidth}


\vspace{\baselineskip}
\begin{adjustwidth}{0.5in}{0.26in}
{\fontsize{9pt}{10.8pt}\selectfont 3.3.5 \tabto{0.49in} The third umpire shall initially check all modes of No ball except for the bowler using an Illegal Bowling Action (subject to the proviso that the third umpire may review whether the bowler has used a prohibited Specific Variation under Article 6.2 of the Illegal Bowling Regulations), where appropriate advising the on-field umpire accordingly.\par}\par

\end{adjustwidth}


\vspace{\baselineskip}
\begin{adjustwidth}{0.5in}{0.21in}
{\fontsize{9pt}{10.8pt}\selectfont 3.3.6 \tabto{0.49in} If despite the available technology, the third umpire is unable to decide with a high degree of confidence whether the original on-field decision should be changed, then he/she shall report that the replays are ‘inconclusive’, and that the on-field decision shall stand. The third umpire shall not give answers conveying likelihoods or probabilities.\par}\par

\end{adjustwidth}


\vspace{\baselineskip}
\begin{adjustwidth}{0.5in}{0.19in}
{\fontsize{9pt}{10.8pt}\selectfont 3.3.7 \tabto{0.49in} In circumstances where the television technology (all or parts thereof) is not available to the third umpire or fails for whatever reason, the third umpire shall advise the on-field umpire of this fact but still provide any relevant factual information that may be ascertained from the available television replays and other technology.\par}\par

\end{adjustwidth}


\vspace{\baselineskip}
\begin{adjustwidth}{0.5in}{0.17in}
{\fontsize{9pt}{10.8pt}\selectfont 3.3.8 \tabto{0.49in} The on-field umpire shall then make his/her decision based on the information provided by the third umpire, any other factual information offered by the third umpire and his/her recollection and opinion of the original incident.\par}\par

\end{adjustwidth}


\vspace{\baselineskip}
\begin{adjustwidth}{0.5in}{0.31in}
{\fontsize{9pt}{10.8pt}\selectfont 3.3.9 \tabto{0.49in} The on-field umpire shall reverse his/her decision if the nature of the supplementary information received from the third umpire leads him/her to conclude that his/her original decision was incorrect.\par}\par

\end{adjustwidth}


\vspace{\baselineskip}
{\fontsize{11pt}{13.2pt}\selectfont \textbf{3.4 \tabto{0.47in} Review of LBW Decisions}\par}\par


\vspace{\baselineskip}
\begin{adjustwidth}{0.5in}{0.11in}
{\fontsize{9pt}{10.8pt}\selectfont 3.4.1 \tabto{0.49in} In assessing whether a batsman is Out LBW in accordance with clause the third umpire shall first judge whether the delivery is fair (as set out in clause and second, whether or not the ball has touched the bat before being intercepted by any part of the striker’s person (as set out in clause .\par}\par

\end{adjustwidth}


\vspace{\baselineskip}
\begin{adjustwidth}{0.5in}{0.51in}
{\fontsize{9pt}{10.8pt}\selectfont 3.4.2 \tabto{0.49in} If the batsman is still eligible to be Out, the ball-tracking technology shall then present three pieces of information to the third umpire relating to the path of the ball:\par}\par

\end{adjustwidth}


\vspace{\baselineskip}
\begin{adjustwidth}{0.49in}{0.0in}
{\fontsize{9pt}{10.8pt}\selectfont 3.4.2.1 \tabto{1.17in} The point of pitching (where applicable) (PITCHING)\par}\par

\end{adjustwidth}


\vspace{\baselineskip}
\begin{adjustwidth}{0.49in}{0.0in}
{\fontsize{9pt}{10.8pt}\selectfont 3.4.2.2 \tabto{1.17in} {\fontsize{8pt}{9.6pt}\selectfont The position of the ball at the point of first interception (IMPACT)\par}\par}\par

\end{adjustwidth}


\vspace{\baselineskip}
\begin{adjustwidth}{0.49in}{0.0in}
{\fontsize{9pt}{10.8pt}\selectfont 3.4.2.3 \tabto{1.17in} {\fontsize{8pt}{9.6pt}\selectfont Whether the ball would have hit the wicket (WICKET)\par}\par}\par

\end{adjustwidth}


\vspace{\baselineskip}
\begin{adjustwidth}{0.5in}{0.14in}
\begin{justify}
{\fontsize{9pt}{10.8pt}\selectfont 3.4.3 \tabto{0.49in} This Decision Review System (DRS) and Third Umpire Protocol includes a category of Umpire’s Call, which shall be the conclusion reported where the technology indicates a marginal decision in respect of either the point of first interception or whether the ball would have hit the stumps.\par}
\end{justify}\par

\end{adjustwidth}


\vspace{\baselineskip}
{\fontsize{9pt}{10.8pt}\selectfont \textbf{3.4.4 \tabto{0.49in} }{\fontsize{8pt}{9.6pt}\selectfont \textbf{PITCHING}\par}\par}\par


\vspace{\baselineskip}
\begin{adjustwidth}{1.18in}{0.26in}
{\fontsize{9pt}{10.8pt}\selectfont 3.4.4.1 \tabto{1.17in} The interpretation of $``$pitches in line between wicket and wicket$"$  in clause shall refer to the position of the centre of the ball at the point of pitching, in relation to the Pitching Zone.\par}\par

\end{adjustwidth}


\vspace{\baselineskip}
\begin{adjustwidth}{1.18in}{0.14in}
{\fontsize{9pt}{10.8pt}\selectfont 3.4.4.2 \tabto{1.17in} The Pitching Zone is defined as a two dimensional area on the pitch between both sets of stumps with its boundaries consisting of the base of both sets of stumps and a line between the outside of the outer stumps at each end.\par}\par

\end{adjustwidth}


\vspace{\baselineskip}
\begin{adjustwidth}{1.18in}{0.35in}
{\fontsize{9pt}{10.8pt}\selectfont 3.4.4.3 \tabto{1.17in} Where applicable, the ball-tracking technology shall report that the ball pitched in one of the following three areas in relation to the Pitching Zone:\par}\par

\end{adjustwidth}


\vspace{\baselineskip}


%%%%%%%%%%%%%%%%%%%% Table No: 4 starts here %%%%%%%%%%%%%%%%%%%%


\begin{table}[H]
 			\centering
\begin{tabular}{p{1in}p{1in}p{1in}p{1in}}
%row no:1
\multicolumn{1}{p{1in}}{{\fontsize{9pt}{10.8pt}\selectfont \textbf{In Line}}} & 
\multicolumn{1}{p{1in}}{{\fontsize{9pt}{10.8pt}\selectfont The centre of the ball was inside the Pitching Zone}} & 

\hhline{~~}
%row no:2
\multicolumn{1}{p{1in}}{} & 
\multicolumn{1}{p{1in}}{} & 

\hhline{~~}
%row no:3
\multicolumn{1}{p{1in}}{{\fontsize{9pt}{10.8pt}\selectfont \textbf{Outside Off}}} & 
\multicolumn{1}{p{1in}}{{\fontsize{9pt}{10.8pt}\selectfont The centre of the ball was outside, and to the off side of, the Pitching Zone}} & 

\hhline{~~}
%row no:4
\multicolumn{1}{p{1in}}{} & 
\multicolumn{1}{p{1in}}{} & 

\hhline{~~}

\end{tabular}
 \end{table}


%%%%%%%%%%%%%%%%%%%% Table No: 4 ends here %%%%%%%%%%%%%%%%%%%%


\vspace{\baselineskip}

\vspace{\baselineskip}

\vspace{\baselineskip}

\vspace{\baselineskip}

\vspace{\baselineskip}
\begin{adjustwidth}{0.0in}{0.08in}
\begin{Center}
{\fontsize{8pt}{9.6pt}\selectfont 78\par}
\end{Center}\par

\end{adjustwidth}


\vspace{\baselineskip}


%%%%%%%%%%%%%%%%%%%% Table No: 5 starts here %%%%%%%%%%%%%%%%%%%%


\begin{table}[H]
 			\centering
\begin{tabular}{p{1in}p{1in}}
%row no:1
\multicolumn{1}{p{1in}}{{\fontsize{9pt}{10.8pt}\selectfont \textbf{Outside Leg}}} & 
\multicolumn{1}{p{1in}}{{\fontsize{9pt}{10.8pt}\selectfont The centre of the ball was outside, and to the leg side of, the Pitching Zone}} \\
\hhline{~~}
%row no:2
\multicolumn{1}{p{1in}}{} & 
\multicolumn{1}{p{1in}}{} \\
\hhline{~~}

\end{tabular}
 \end{table}


%%%%%%%%%%%%%%%%%%%% Table No: 5 ends here %%%%%%%%%%%%%%%%%%%%


\vspace{\baselineskip}
\begin{adjustwidth}{1.18in}{0.25in}
{\fontsize{9pt}{10.8pt}\selectfont 3.4.4.4 \tabto{1.17in} Subject to the satisfaction of the other elements of clause the batsman can be Out if the ball-tracking technology reports that the ball pitched Outside Off or In Line, but the batsman shall be Not out if the ball pitched Outside Leg.\par}\par

\end{adjustwidth}


\vspace{\baselineskip}
{\fontsize{9pt}{10.8pt}\selectfont \textbf{3.4.5 \tabto{0.49in} }{\fontsize{8pt}{9.6pt}\selectfont \textbf{IMPACT}\par}\par}\par


\vspace{\baselineskip}
\begin{adjustwidth}{1.18in}{0.12in}
{\fontsize{9pt}{10.8pt}\selectfont 3.4.5.1 \tabto{1.17in} The interpretation of $``$the (first) point of impact, even if in above the level of the bails, is between wicket and wicket$"$  in clause shall refer to position of the ball at the point of first interception, in relation to the Impact Zone.\par}\par

\end{adjustwidth}


\vspace{\baselineskip}
\begin{adjustwidth}{1.18in}{0.11in}
{\fontsize{9pt}{10.8pt}\selectfont 3.4.5.2 \tabto{1.17in} The Impact Zone is defined as a three dimensional space extending between both wickets to an indefinite height and with its boundaries consisting of a line between the outside of the outer stumps at each end.\par}\par

\end{adjustwidth}


\vspace{\baselineskip}
\begin{adjustwidth}{1.18in}{0.39in}
{\fontsize{9pt}{10.8pt}\selectfont 3.4.5.3 \tabto{1.17in} The ball-tracking technology shall report that the point of first interception was in one of the following categories in relation to the Impact Zone:\par}\par

\end{adjustwidth}


\vspace{\baselineskip}


%%%%%%%%%%%%%%%%%%%% Table No: 6 starts here %%%%%%%%%%%%%%%%%%%%


\begin{table}[H]
 			\centering
\begin{tabular}{p{0.65in}p{0.65in}p{0.65in}p{0.65in}p{0.65in}p{0.65in}p{0.65in}p{0.65in}p{0.65in}p{0.65in}}
%row no:1
\multicolumn{1}{p{0.65in}}{{\fontsize{9pt}{10.8pt}\selectfont \textbf{In Line}}} & 
\multicolumn{1}{p{0.65in}}{{\fontsize{9pt}{10.8pt}\selectfont The centre of the ball was inside the Impact Zone}} & 

\hhline{~~}
%row no:2
\multicolumn{1}{p{0.65in}}{} & 
\multicolumn{1}{p{0.65in}}{} & 

\hhline{~~}
%row no:3
\multicolumn{1}{p{0.65in}}{{\fontsize{9pt}{10.8pt}\selectfont \textbf{Umpire’s Call}}} & 
\multicolumn{1}{p{0.65in}}{{\fontsize{9pt}{10.8pt}\selectfont Some part of the ball was inside the Impact Zone, but the centre of the}} & 

\hhline{~~}
%row no:4
\multicolumn{1}{p{0.65in}}{} & 
\multicolumn{1}{p{0.65in}}{{\fontsize{9pt}{10.8pt}\selectfont ball was outside the Impact Zone, with the further sub-category of}} & 

\hhline{~~}
%row no:5
\multicolumn{1}{p{0.65in}}{} & 
\multicolumn{1}{p{0.65in}}{{\fontsize{9pt}{10.8pt}\selectfont ‘Umpire’s Call (off side)’ where the centre of the ball was to the off side of}} & 

\hhline{~~}
%row no:6
\multicolumn{1}{p{0.65in}}{} & 
\multicolumn{1}{p{0.65in}}{{\fontsize{9pt}{10.8pt}\selectfont the Impact Zone and the bowler’s end umpire communicates to the third}} & 

\hhline{~~}
%row no:7
\multicolumn{1}{p{0.65in}}{} & 
\multicolumn{1}{p{0.65in}}{{\fontsize{9pt}{10.8pt}\selectfont umpire that no genuine attempt to play the ball was made by the}} & 

\hhline{~~}
%row no:8
\multicolumn{1}{p{0.65in}}{} & 
\multicolumn{1}{p{0.65in}}{{\fontsize{9pt}{10.8pt}\selectfont batsman.}} & 

\hhline{~~}
%row no:9
\multicolumn{1}{p{0.65in}}{} & 
\multicolumn{1}{p{0.65in}}{} & 

\hhline{~~}
%row no:10
\multicolumn{1}{p{0.65in}}{{\fontsize{9pt}{10.8pt}\selectfont \textbf{Outside}}} & 
\multicolumn{1}{p{0.65in}}{{\fontsize{9pt}{10.8pt}\selectfont No part of the ball was inside the Impact Zone, with the further sub-}} & 

\hhline{~~}
%row no:11
\multicolumn{1}{p{0.65in}}{} & 
\multicolumn{1}{p{0.65in}}{{\fontsize{9pt}{10.8pt}\selectfont categories of ‘Outside (off)’ and ‘Outside (leg)’ to indicate the location of}} & 

\hhline{~~}
%row no:12
\multicolumn{1}{p{0.65in}}{} & 
\multicolumn{1}{p{0.65in}}{{\fontsize{9pt}{10.8pt}\selectfont the point of first interception in relation to the Impact Zone when the}} & 

\hhline{~~}
%row no:13
\multicolumn{1}{p{0.65in}}{} & 
\multicolumn{1}{p{0.65in}}{{\fontsize{9pt}{10.8pt}\selectfont bowler’s end umpire communicates to the third umpire that no genuine}} & 

\hhline{~~}
%row no:14
\multicolumn{1}{p{0.65in}}{} & 
\multicolumn{1}{p{0.65in}}{{\fontsize{9pt}{10.8pt}\selectfont attempt to play the ball was made by the batsman.}} & 

\hhline{~~}
%row no:15
\multicolumn{1}{p{0.65in}}{} & 
\multicolumn{1}{p{0.65in}}{} & 

\hhline{~~}

\end{tabular}
 \end{table}


%%%%%%%%%%%%%%%%%%%% Table No: 6 ends here %%%%%%%%%%%%%%%%%%%%


\vspace{\baselineskip}
\begin{adjustwidth}{1.18in}{0.25in}
{\fontsize{9pt}{10.8pt}\selectfont 3.4.5.4 \tabto{1.17in} Where a Not out decision is being reviewed, and it is judged that the batsman has made a genuine attempt to play the ball, the ball-tracking technology must report that the point of first interception was In Line for the batsman to be eligible to be given Out, otherwise the batsman shall remain Not out.\par}\par

\end{adjustwidth}


\vspace{\baselineskip}
\begin{adjustwidth}{1.18in}{0.11in}
{\fontsize{9pt}{10.8pt}\selectfont 3.4.5.5 \tabto{1.17in} Where a Not out decision is being reviewed, and it is judged that the batsman has made no genuine attempt to play the ball, the ball-tracking technology must report that the point of impact was In Line, or Umpire’s Call (off side), or Outside (off) for the batsman to be eligible to be given Out, otherwise the batsman shall remain Not out.\par}\par

\end{adjustwidth}


\vspace{\baselineskip}
\begin{adjustwidth}{0.49in}{0.0in}
{\fontsize{9pt}{10.8pt}\selectfont 3.4.5.6 \tabto{1.17in} Where an Out decision is being reviewed, and it is judged that the batsman has made a genuine\par}\par

\end{adjustwidth}


\vspace{\baselineskip}
\begin{adjustwidth}{1.18in}{0.12in}
{\fontsize{9pt}{10.8pt}\selectfont attempt to play the ball, the ball-tracking technology must report that the point of first interception was Outside for the decision to be reversed to Not out, otherwise the batsman shall remain eligible to be given Out.\par}\par

\end{adjustwidth}


\vspace{\baselineskip}
\begin{adjustwidth}{1.18in}{0.12in}
{\fontsize{9pt}{10.8pt}\selectfont 3.4.5.7 \tabto{1.17in} Where an Out decision is being reviewed, and it is judged that the batsman has made no genuine attempt to play the ball, the ball-tracking technology must report that the point of first interception was Outside (leg) for the decision to be reversed to Not out, otherwise the batsman shall remain eligible to be given Out.\par}\par

\end{adjustwidth}


\vspace{\baselineskip}
{\fontsize{9pt}{10.8pt}\selectfont \textbf{3.4.6 \tabto{0.49in} WICKET}\par}\par


\vspace{\baselineskip}
\begin{adjustwidth}{1.18in}{0.22in}
{\fontsize{9pt}{10.8pt}\selectfont 3.4.6.1 \tabto{1.17in} The interpretation of whether $``$the ball would have hit the wicket$"$  in clause shall refer to the position of the ball as it either hits or passes the wicket, in relation to the Wicket Zone.\par}\par

\end{adjustwidth}


\vspace{\baselineskip}
\begin{adjustwidth}{1.18in}{0.15in}
{\fontsize{9pt}{10.8pt}\selectfont 3.4.6.2 \tabto{1.17in} The Wicket Zone is defined as a two dimensional area whose boundaries are the outside of the outer stumps, the base of the stumps and the bottom of the bails.\par}\par

\end{adjustwidth}


\vspace{\baselineskip}

\vspace{\baselineskip}

\vspace{\baselineskip}
\begin{adjustwidth}{0.0in}{0.08in}
\begin{Center}
{\fontsize{8pt}{9.6pt}\selectfont 79\par}
\end{Center}\par

\end{adjustwidth}


\vspace{\baselineskip}

\vspace{\baselineskip}
\begin{adjustwidth}{1.18in}{0.58in}
{\fontsize{9pt}{10.8pt}\selectfont 3.4.6.3 \tabto{1.17in} The ball-tracking technology shall report whether the ball would have hit the wicket with reference to the following three categories:\par}\par

\end{adjustwidth}


\vspace{\baselineskip}


%%%%%%%%%%%%%%%%%%%% Table No: 7 starts here %%%%%%%%%%%%%%%%%%%%


\begin{table}[H]
 			\centering
\begin{tabular}{p{1in}p{1in}p{1in}p{1in}p{1in}p{1in}}
%row no:1
\multicolumn{1}{p{1in}}{{\fontsize{9pt}{10.8pt}\selectfont \textbf{Hitting}}} & 
\multicolumn{1}{p{1in}}{{\fontsize{9pt}{10.8pt}\selectfont The ball was hitting the wicket, and the centre of the ball was inside the}} & 

\hhline{~~}
%row no:2
\multicolumn{1}{p{1in}}{} & 
\multicolumn{1}{p{1in}}{{\fontsize{9pt}{10.8pt}\selectfont Wicket Zone}} & 

\hhline{~~}
%row no:3
\multicolumn{1}{p{1in}}{} & 
\multicolumn{1}{p{1in}}{} & 

\hhline{~~}
%row no:4
\multicolumn{1}{p{1in}}{{\fontsize{9pt}{10.8pt}\selectfont \textbf{Umpire’s Call}}} & 
\multicolumn{1}{p{1in}}{{\fontsize{9pt}{10.8pt}\selectfont The ball was hitting the wicket, but the centre of the ball was not inside}} & 

\hhline{~~}
%row no:5
\multicolumn{1}{p{1in}}{} & 
\multicolumn{1}{p{1in}}{{\fontsize{9pt}{10.8pt}\selectfont the Wicket Zone}} & 

\hhline{~~}
%row no:6
\multicolumn{1}{p{1in}}{} & 
\multicolumn{1}{p{1in}}{} & 

\hhline{~~}
%row no:7
\multicolumn{1}{p{1in}}{{\fontsize{9pt}{10.8pt}\selectfont \textbf{Missing}}} & 
\multicolumn{1}{p{1in}}{{\fontsize{9pt}{10.8pt}\selectfont The ball was missing the wicket}} & 

\hhline{~~}
%row no:8
\multicolumn{1}{p{1in}}{} & 
\multicolumn{1}{p{1in}}{} & 

\hhline{~~}

\end{tabular}
 \end{table}


%%%%%%%%%%%%%%%%%%%% Table No: 7 ends here %%%%%%%%%%%%%%%%%%%%


\vspace{\baselineskip}
\begin{adjustwidth}{1.18in}{0.29in}
{\fontsize{9pt}{10.8pt}\selectfont 3.4.6.4 \tabto{1.17in} Where a Not out decision is being reviewed, the ball-tracking technology must report that the ball was Hitting for the batsman to be eligible to be given Out, otherwise the batsman shall remain Not out.\par}\par

\end{adjustwidth}


\vspace{\baselineskip}
\begin{adjustwidth}{1.18in}{0.0in}
{\fontsize{9pt}{10.8pt}\selectfont However, where the evidence shows that the ball was Hitting, the point of first interception was\par}\par

\end{adjustwidth}


\vspace{\baselineskip}
\begin{adjustwidth}{1.18in}{0.0in}
{\fontsize{9pt}{10.8pt}\selectfont In Line, and the ball pitched In Line or Outside Off, but that:\par}\par

\end{adjustwidth}


\vspace{\baselineskip}
\begin{itemize}
	\item {\fontsize{9pt}{10.8pt}\selectfont The point of first interception was 300cm or more from the stumps; or\par}\par


\vspace{\baselineskip}
	\item {\fontsize{9pt}{10.8pt}\selectfont The point of first interception was more than 250cm but less than 300cm from the stumps and the distance between the point of pitching and the point of first interception was less than 40cm,\par}
\end{itemize}\par


\vspace{\baselineskip}
\begin{adjustwidth}{1.18in}{0.0in}
{\fontsize{9pt}{10.8pt}\selectfont the on-field decision shall stand (that is, Not out).\par}\par

\end{adjustwidth}


\vspace{\baselineskip}
\begin{adjustwidth}{1.18in}{0.19in}
{\fontsize{9pt}{10.8pt}\selectfont 3.4.6.5 \tabto{1.17in} Where an Out decision is being reviewed, the ball-tracking technology must report that the ball was Missing for the on-field decision to be reversed to Not out, otherwise the batsman shall remain eligible to be given Out.\par}\par

\end{adjustwidth}


\vspace{\baselineskip}
\begin{adjustwidth}{0.5in}{0.18in}
{\fontsize{9pt}{10.8pt}\selectfont 3.4.7 \tabto{0.49in} When the ball strikes the batsman on the full, and the evidence provided by the ball-tracking technology indicates that the ball would have pitched before striking or passing the wicket, there will be no information available from that delivery that will allow the ball-tracking technology to accurately predict the height of the ball after pitching.\par}\par

\end{adjustwidth}


\vspace{\baselineskip}
\begin{adjustwidth}{0.5in}{0.21in}
{\fontsize{9pt}{10.8pt}\selectfont 3.4.8 \tabto{0.49in} With regard to determining whether the ball would have hit the wicket under these circumstances, the ball-tracking technology shall project the line of the ball in accordance with clause (it is to be assumed that the path of the ball before interception would have continued after interception, irrespective of whether the ball might have pitched subsequently or not), and display the simulated path of the ball from directly above the wicket.\par}\par

\end{adjustwidth}


\vspace{\baselineskip}
\begin{adjustwidth}{0.5in}{0.11in}
{\fontsize{9pt}{10.8pt}\selectfont 3.4.9 \tabto{0.49in} The third umpire shall advise the bowler’s end umpire only on the point of first interception and whether the ball would have hit the stumps (in line with the process set out in paragraph above), but shall make no comment on the predicted height of the ball after pitching, which shall remain a judgment of the bowler’s end umpire.\par}\par

\end{adjustwidth}


\vspace{\baselineskip}
{\fontsize{11pt}{13.2pt}\selectfont \textbf{3.5 \tabto{0.47in} The process for communicating the final decision}\par}\par


\vspace{\baselineskip}
\begin{adjustwidth}{0.5in}{0.08in}
{\fontsize{9pt}{10.8pt}\selectfont 3.5.1 \tabto{0.49in} For Player Reviews concerning potential dismissals, the relevant on-field umpire shall indicate Out by raising his/her finger above his/her head in a normal yet prominent manner or indicate Not out by the call of ‘not out’ and by crossing his/her hands in a horizontal position side to side in front and above his/her waist three times. Where the decision is a reversal of the on-field umpire’s previous decision, he/she shall make the ‘revoke last signal’ indication immediately prior to the above.\par}\par

\end{adjustwidth}


\vspace{\baselineskip}
\begin{adjustwidth}{0.5in}{0.32in}
{\fontsize{9pt}{10.8pt}\selectfont 3.5.2 \tabto{0.49in} If the mode of dismissal is not obvious or not the same as that on which the original decision was based, then the umpire shall advise the scorers via the third umpire.\par}\par

\end{adjustwidth}


\vspace{\baselineskip}
{\fontsize{11pt}{13.2pt}\selectfont \textbf{3.6 \tabto{0.47in} Number of Player Review requests permitted}\par}\par


\vspace{\baselineskip}
\begin{adjustwidth}{0.5in}{0.47in}
{\fontsize{9pt}{10.8pt}\selectfont 3.6.1 \tabto{0.49in} In each innings, each team shall be allowed to make a maximum of one Player Review request that is categorised as ‘Unsuccessful’ (as set out in paragraph below).\par}\par

\end{adjustwidth}


\vspace{\baselineskip}

\vspace{\baselineskip}

\vspace{\baselineskip}

\vspace{\baselineskip}
\begin{adjustwidth}{0.0in}{0.08in}
\begin{Center}
{\fontsize{8pt}{9.6pt}\selectfont 80\par}
\end{Center}\par

\end{adjustwidth}


\vspace{\baselineskip}

\vspace{\baselineskip}
\begin{adjustwidth}{0.5in}{0.15in}
{\fontsize{9pt}{10.8pt}\selectfont 3.6.2 \tabto{0.49in} Where a request for a Player Review results in the original on-field decision being reversed, then the Player Review shall be categorised as ‘Successful’ and shall not count towards the innings limit.\par}\par

\end{adjustwidth}


\vspace{\baselineskip}
\begin{adjustwidth}{0.5in}{0.32in}
{\fontsize{9pt}{10.8pt}\selectfont 3.6.3 \tabto{0.49in} Where a request for a Player Review results in the original on-field decision remaining unchanged (other than in the circumstances set out in paragraphs or the Player Review shall be categorised as ’Unsuccessful’.\par}\par

\end{adjustwidth}


\vspace{\baselineskip}
\begin{adjustwidth}{0.5in}{0.11in}
{\fontsize{9pt}{10.8pt}\selectfont 3.6.4 \tabto{0.49in} Where a request for a Player Review of an LBW decision results in the on-field decision remaining unchanged solely on the basis of an Umpire’s Call, the Player Review shall be categorised as ‘Unchanged – Umpire’s Call’. A Player Review categorised as ‘Unchanged – Umpire’s Call’ shall not count towards the innings limit set out in paragraph \par}\par

\end{adjustwidth}


\vspace{\baselineskip}
{\fontsize{9pt}{10.8pt}\selectfont 3.6.5 \tabto{0.49in} Where, following a request for a Player Review, the original on-field decision of Out is unchanged, but for a\par}\par


\vspace{\baselineskip}
\begin{adjustwidth}{0.5in}{0.69in}
{\fontsize{9pt}{10.8pt}\selectfont different mode of dismissal from the original on-field decision, then the Player Review shall still be categorised as ’Unsuccessful’.\par}\par

\end{adjustwidth}


\vspace{\baselineskip}
\begin{adjustwidth}{0.5in}{0.26in}
{\fontsize{9pt}{10.8pt}\selectfont 3.6.6 \tabto{0.49in} Where, following a request for a Player Review, the original on-field decision of Not out is unchanged on account of the delivery being a No ball (for any reason), thereby not requiring any further evaluation, the Player Review shall not be counted as ‘Unsuccessful’ and accordingly shall not count towards the innings limit set out in paragraph \par}\par

\end{adjustwidth}


\vspace{\baselineskip}
\begin{adjustwidth}{0.5in}{0.1in}
{\fontsize{9pt}{10.8pt}\selectfont 3.6.7 \tabto{0.49in} Where a Player Review and an Umpire Review are requested from the same delivery and the decision of the third umpire from the Umpire Review renders the Player Review unnecessary (see paragraphs and the Player Review request shall be disregarded and accordingly shall not count towards the innings limit set out in paragraph \par}\par

\end{adjustwidth}


\vspace{\baselineskip}
\begin{adjustwidth}{0.5in}{0.12in}
{\fontsize{9pt}{10.8pt}\selectfont 3.6.8 \tabto{0.49in} A Player Review categorised as ‘Unsuccessful’ may be reinstated by the ICC Match Referee at his/her sole discretion (if appropriate after consultation with the ICC Technical Official and/or the television broadcast director) if the Player Review could not properly be concluded due to a failure of the technology. Any such decision shall be final and shall be taken as soon as possible, being communicated to both teams once all the relevant facts have been ascertained by the ICC Match Referee. A Player Review categorised as ‘Unsuccessful’ shall not be reinstated if, despite any technical failures, the correct decision could still have been made using the other available technology. Similarly, a Player Review categorised as ‘Unsuccessful’ shall not be reinstated where the technology worked as intended, but the evidence gleaned from its use was inconclusive.\par}\par

\end{adjustwidth}


\vspace{\baselineskip}
\begin{adjustwidth}{0.5in}{0.11in}
{\fontsize{9pt}{10.8pt}\selectfont 3.6.9 \tabto{0.49in} {\fontsize{8pt}{9.6pt}\selectfont The third umpire shall be responsible for counting the number Player Reviews categorised as ‘Unsuccessful’ and shall advise the on-field umpires once either team has exhausted their allowance for the innings.\par}\par}\par

\end{adjustwidth}


\vspace{\baselineskip}
\begin{adjustwidth}{0.5in}{0.14in}
{\fontsize{9pt}{10.8pt}\selectfont 3.6.10 \tabto{0.49in} The scoreboard shall display, for the innings in progress, the number of Player Reviews remaining available to each team.\par}\par

\end{adjustwidth}


\vspace{\baselineskip}


%%%%%%%%%%%%%%%%%%%% Table No: 8 starts here %%%%%%%%%%%%%%%%%%%%


{
\scriptsize
\setlength\extrarowheight{3pt}
\begin{longtable}{p{0.43in}p{0.43in}p{0.43in}p{0.43in}p{0.43in}p{0.43in}p{0.43in}p{0.43in}p{0.43in}p{0.43in}p{0.43in}p{0.43in}p{0.43in}p{0.43in}p{0.43in}}

\endfirsthead
\multicolumn{15}{c}{\textit{continued from previous page}}\hline
\endhead\hline
\multicolumn{15}{r}{\textit{continued on next page}} \\
\endfoot
\hline 
\endlastfoot%row no:1
\multicolumn{1}{p{0.43in}}{\cellcolor[HTML]{D9D9D9}{\fontsize{9pt}{10.8pt}\selectfont \textbf{Category of Player Review}}} & 
\multicolumn{1}{p{0.43in}}{\cellcolor[HTML]{D9D9D9}{\fontsize{9pt}{10.8pt}\selectfont \textbf{Outcome of Player Review}}} & 
\multicolumn{1}{p{0.43in}}{\cellcolor[HTML]{D9D9D9}{\fontsize{9pt}{10.8pt}\selectfont \textbf{Consequence of Player Review}}} & 

\hhline{~~~}
%row no:2
\multicolumn{1}{p{0.43in}}{\cellcolor[HTML]{D9D9D9}} & 
\multicolumn{1}{p{0.43in}}{\cellcolor[HTML]{D9D9D9}} & 
\multicolumn{1}{p{0.43in}}{\cellcolor[HTML]{D9D9D9}} & 

\hhline{~~~}
%row no:3
\multicolumn{1}{p{0.43in}}{{\fontsize{9pt}{10.8pt}\selectfont Successful (paragraph }} & 
\multicolumn{1}{p{0.43in}}{{\fontsize{9pt}{10.8pt}\selectfont On-field decision reversed}} & 
\multicolumn{1}{p{0.43in}}{{\fontsize{9pt}{10.8pt}\selectfont Does not count towards innings}} & 

\hhline{~~~}
%row no:4
\multicolumn{1}{p{0.43in}}{} & 
\multicolumn{1}{p{0.43in}}{} & 
\multicolumn{1}{p{0.43in}}{{\fontsize{9pt}{10.8pt}\selectfont limit set out in paragraph }} & 

\hhline{~~~}
%row no:5
\multicolumn{1}{p{0.43in}}{} & 
\multicolumn{1}{p{0.43in}}{} & 
\multicolumn{1}{p{0.43in}}{} & 

\hhline{~~~}
%row no:6
\multicolumn{1}{p{0.43in}}{{\fontsize{9pt}{10.8pt}\selectfont Unsuccessful (paragraphs }} & 
\multicolumn{1}{p{0.43in}}{{\fontsize{9pt}{10.8pt}\selectfont On-field decision unchanged}} & 
\multicolumn{1}{p{0.43in}}{{\fontsize{9pt}{10.8pt}\selectfont Counts towards innings limit set}} & 

\hhline{~~~}
%row no:7
\multicolumn{1}{p{0.43in}}{{\fontsize{9pt}{10.8pt}\selectfont and }} & 
\multicolumn{1}{p{0.43in}}{} & 
\multicolumn{1}{p{0.43in}}{{\fontsize{9pt}{10.8pt}\selectfont out in paragraph }} & 

\hhline{~~~}
%row no:8
\multicolumn{1}{p{0.43in}}{} & 
\multicolumn{1}{p{0.43in}}{} & 
\multicolumn{1}{p{0.43in}}{} & 

\hhline{~~~}
%row no:9
\multicolumn{1}{p{0.43in}}{{\fontsize{9pt}{10.8pt}\selectfont Unchanged – Umpire’s Call}} & 
\multicolumn{1}{p{0.43in}}{{\fontsize{9pt}{10.8pt}\selectfont On-field decision unchanged}} & 
\multicolumn{1}{p{0.43in}}{{\fontsize{9pt}{10.8pt}\selectfont Does not count towards innings}} & 

\hhline{~~~}
%row no:10
\multicolumn{1}{p{0.43in}}{{\fontsize{9pt}{10.8pt}\selectfont (paragraph }} & 
\multicolumn{1}{p{0.43in}}{} & 
\multicolumn{1}{p{0.43in}}{{\fontsize{9pt}{10.8pt}\selectfont limit set out in paragraph }} & 

\hhline{~~~}
%row no:11
\multicolumn{1}{p{0.43in}}{} & 
\multicolumn{1}{p{0.43in}}{} & 
\multicolumn{1}{p{0.43in}}{} & 

\hhline{~~~}
%row no:12
\multicolumn{1}{p{0.43in}}{{\fontsize{9pt}{10.8pt}\selectfont No ball – no evaluation required}} & 
\multicolumn{1}{p{0.43in}}{{\fontsize{9pt}{10.8pt}\selectfont On-field decision unchanged}} & 
\multicolumn{1}{p{0.43in}}{{\fontsize{9pt}{10.8pt}\selectfont Does not count towards innings}} & 

\hhline{~~~}
%row no:13
\multicolumn{1}{p{0.43in}}{{\fontsize{9pt}{10.8pt}\selectfont (paragraph }} & 
\multicolumn{1}{p{0.43in}}{} & 
\multicolumn{1}{p{0.43in}}{{\fontsize{9pt}{10.8pt}\selectfont limit set out in paragraph }} & 

\hhline{~~~}
%row no:14
\multicolumn{1}{p{0.43in}}{} & 
\multicolumn{1}{p{0.43in}}{} & 
\multicolumn{1}{p{0.43in}}{} & 

\hhline{~~~}
%row no:15
\multicolumn{1}{p{0.43in}}{{\fontsize{9pt}{10.8pt}\selectfont Failure of technology (paragraph}} & 
\multicolumn{1}{p{0.43in}}{{\fontsize{9pt}{10.8pt}\selectfont On-field decision unchanged}} & 
\multicolumn{1}{p{0.43in}}{{\fontsize{9pt}{10.8pt}\selectfont Does not count towards innings}} & 

\hhline{~~~}
%row no:16
\multicolumn{1}{p{0.43in}}{} & 
\multicolumn{1}{p{0.43in}}{} & 
\multicolumn{1}{p{0.43in}}{{\fontsize{9pt}{10.8pt}\selectfont limit set out in paragraph }} & 

\hhline{~~~}
%row no:17
\multicolumn{1}{p{0.43in}}{} & 
\multicolumn{1}{p{0.43in}}{} & 
\multicolumn{1}{p{0.43in}}{} & 

\hhline{~~~}

\end{longtable}}

%%%%%%%%%%%%%%%%%%%% Table No: 8 ends here %%%%%%%%%%%%%%%%%%%%

\par 
 \begin{tikzpicture}

\path (6.58in,1.23in) node [shape=rectangle,draw,fill={rgb:red,0;green,0;blue,0},minimum height=0.01in,minimum width=0.01in,]{};

\end{tikzpicture}

\vspace{\baselineskip}

\vspace{\baselineskip}

\vspace{\baselineskip}

\vspace{\baselineskip}

\vspace{\baselineskip}

\vspace{\baselineskip}

\vspace{\baselineskip}
\begin{adjustwidth}{0.0in}{0.08in}
\begin{Center}
{\fontsize{8pt}{9.6pt}\selectfont 81\par}
\end{Center}\par

\end{adjustwidth}


\vspace{\baselineskip}
{\fontsize{11pt}{13.2pt}\selectfont \textbf{3.7 \tabto{0.47in} Dead ball}\par}\par


\vspace{\baselineskip}
\begin{adjustwidth}{0.5in}{0.19in}
{\fontsize{9pt}{10.8pt}\selectfont 3.7.1 \tabto{0.49in} If following a Player Review request, an original decision of Out is changed to Not out, then the ball is still deemed to have become dead when the original decision was made (as per clause . The batting side, while benefiting from the reversal of the dismissal, shall not benefit from any runs that may subsequently have accrued from the delivery had the on-field umpire originally made a Not out decision, other than any No ball penalty that could arise under paragraph above.\par}\par

\end{adjustwidth}


\vspace{\baselineskip}
\begin{adjustwidth}{0.5in}{0.04in}
{\fontsize{9pt}{10.8pt}\selectfont 3.7.2 \tabto{0.49in} If an original decision of Not out is changed to Out, the ball shall retrospectively be deemed to have become dead from the moment of the dismissal event. All subsequent events, including any runs scored, shall be ignored.\par}\par

\end{adjustwidth}


\vspace{\baselineskip}
{\fontsize{11pt}{13.2pt}\selectfont \textbf{3.8 \tabto{0.47in} Use of technology}\par}\par


\vspace{\baselineskip}
{\fontsize{9pt}{10.8pt}\selectfont 3.8.1 \tabto{0.49in} {\fontsize{8pt}{9.6pt}\selectfont The following technology may be used by the third umpire during a Player Review:\par}\par}\par


\vspace{\baselineskip}
\begin{adjustwidth}{0.49in}{0.0in}
{\fontsize{9pt}{10.8pt}\selectfont 3.8.1.1 \tabto{1.17in} Replays, at any speed, from any available broadcast camera\par}\par

\end{adjustwidth}


\vspace{\baselineskip}
\begin{adjustwidth}{0.49in}{0.0in}
{\fontsize{9pt}{10.8pt}\selectfont 3.8.1.2 \tabto{1.17in} {\fontsize{8pt}{9.6pt}\selectfont Sound from the stump microphones with the replays at normal speed and slow motion\par}\par}\par

\end{adjustwidth}


\vspace{\baselineskip}
\begin{adjustwidth}{0.49in}{0.0in}
{\fontsize{9pt}{10.8pt}\selectfont 3.8.1.3 \tabto{1.17in} {\fontsize{8pt}{9.6pt}\selectfont Approved ball-tracking technology:\par}\par}\par

\end{adjustwidth}


\vspace{\baselineskip}
\begin{itemize}
	\item {\fontsize{9pt}{10.8pt}\selectfont HawkEye (HawkEye Innovations), or;\par}\par


\vspace{\baselineskip}
	\item {\fontsize{9pt}{10.8pt}\selectfont VirtualEye (ARL)\par}
\end{itemize}\par


\vspace{\baselineskip}
\begin{adjustwidth}{0.49in}{0.0in}
{\fontsize{9pt}{10.8pt}\selectfont 3.8.1.4 \tabto{1.17in} {\fontsize{8pt}{9.6pt}\selectfont Approved sound-based edge detection technology:\par}\par}\par

\end{adjustwidth}


\vspace{\baselineskip}
\begin{itemize}
	\item {\fontsize{9pt}{10.8pt}\selectfont Real-Time Snickometer (BBG Sports), or;\par}\par


\vspace{\baselineskip}
	\item {\fontsize{9pt}{10.8pt}\selectfont UltraEdge (HawkEye Innovations)\par}
\end{itemize}\par


\vspace{\baselineskip}
\begin{adjustwidth}{0.49in}{0.0in}
{\fontsize{9pt}{10.8pt}\selectfont 3.8.1.5 \tabto{1.17in} {\fontsize{8pt}{9.6pt}\selectfont Approved heat-based edge detection technology:\par}\par}\par

\end{adjustwidth}


\vspace{\baselineskip}
\begin{itemize}
	\item {\fontsize{9pt}{10.8pt}\selectfont Hot Spot cameras (BBG Sports)\par}
\end{itemize}\par


\vspace{\baselineskip}
\begin{adjustwidth}{0.49in}{0.0in}
{\fontsize{9pt}{10.8pt}\selectfont 3.8.1.6 \tabto{1.17in} LED Wickets (using the lights to indicate if the wicket is broken, as set out in paragraph \par}\par

\end{adjustwidth}


\vspace{\baselineskip}
\begin{itemize}
	\item {\fontsize{9pt}{10.8pt}\selectfont Zing Bails and Stumps\par}
\end{itemize}\par


\vspace{\baselineskip}
\begin{adjustwidth}{0.5in}{0.39in}
{\fontsize{9pt}{10.8pt}\selectfont 3.8.2 \tabto{0.49in} In addition, other forms of technology may be used subject to the ICC being satisfied that the required standards of accuracy and time efficiency can be met.\par}\par

\end{adjustwidth}


\vspace{\baselineskip}
\begin{adjustwidth}{0.5in}{0.1in}
{\fontsize{9pt}{10.8pt}\selectfont 3.8.3 \tabto{0.49in} {\fontsize{8pt}{9.6pt}\selectfont Where practical usage or further testing indicates that any of the above forms of technology cannot reliably provide accurate and timely information, then it may be removed prior to or during a match. The final decision regarding the technology to be used in a given match shall be taken by the ICC Match Referee in consultation with the ICC Technical Official, ICC management and the competing teams’ governing bodies.\par}\par}\par

\end{adjustwidth}


\vspace{\baselineskip}
{\fontsize{11pt}{13.2pt}\selectfont \textbf{3.9 \tabto{0.47in} Combining Umpire Review with Player Review}\par}\par


\vspace{\baselineskip}
\begin{adjustwidth}{0.5in}{0.1in}
{\fontsize{9pt}{10.8pt}\selectfont 3.9.1 \tabto{0.49in} If an Umpire Review (under paragraph and a request for a Player Review (under paragraph are made following the same delivery but relating to separate modes of dismissal, the following process shall apply.\par}\par

\end{adjustwidth}


\vspace{\baselineskip}
{\fontsize{9pt}{10.8pt}\selectfont 3.9.2 \tabto{0.49in} {\fontsize{8pt}{9.6pt}\selectfont The Umpire Review shall be carried out prior to the Player Review if all of the following conditions apply:\par}\par}\par


\vspace{\baselineskip}
\begin{adjustwidth}{0.49in}{0.0in}
{\fontsize{9pt}{10.8pt}\selectfont 3.9.2.1 \tabto{1.17in} The Player Review has been requested by the fielding side\par}\par

\end{adjustwidth}


\vspace{\baselineskip}
\begin{adjustwidth}{0.49in}{0.0in}
{\fontsize{9pt}{10.8pt}\selectfont 3.9.2.2 \tabto{1.17in} {\fontsize{8pt}{9.6pt}\selectfont The Umpire Review and the Player Review both relate to the dismissal of the same batsman\par}\par}\par

\end{adjustwidth}


\vspace{\baselineskip}
\begin{adjustwidth}{1.18in}{0.17in}
{\fontsize{9pt}{10.8pt}\selectfont 3.9.2.3 \tabto{1.17in} If the batsman is out, the number of runs scored from the delivery would be the same for both modes of dismissal\par}\par

\end{adjustwidth}


\vspace{\baselineskip}
\begin{adjustwidth}{1.18in}{0.26in}
{\fontsize{9pt}{10.8pt}\selectfont 3.9.2.4 \tabto{1.17in} If the batsman is out, the batsman on strike for the next delivery would be the same for both modes of dismissal.\par}\par

\end{adjustwidth}


\vspace{\baselineskip}
\begin{adjustwidth}{0.5in}{0.25in}
{\fontsize{9pt}{10.8pt}\selectfont 3.9.3 \tabto{0.49in} {\fontsize{8pt}{9.6pt}\selectfont If the Umpire Review leads the third umpire to make a decision of Out, then this shall be displayed in the usual manner and the Player Review shall not be undertaken. If the Umpire Review results in a Not out\par}\par}\par

\end{adjustwidth}


\vspace{\baselineskip}

\vspace{\baselineskip}

\vspace{\baselineskip}
\begin{Center}
{\fontsize{8pt}{9.6pt}\selectfont 82\par}
\end{Center}\par


\vspace{\baselineskip}

\vspace{\baselineskip}
\begin{adjustwidth}{0.5in}{0.12in}
{\fontsize{9pt}{10.8pt}\selectfont decision, then the third umpire shall make no public decision but shall proceed to address the request for a Player Review.\par}\par

\end{adjustwidth}


\vspace{\baselineskip}
\begin{adjustwidth}{0.5in}{0.11in}
{\fontsize{9pt}{10.8pt}\selectfont 3.9.4 \tabto{0.49in} For illustration, following an LBW appeal which is given Not out by the bowler’s end umpire, the striker sets off for a run, is sent back and there is an appeal for his/her run out. The players request that the LBW decision is reviewed and the umpires request that the run out be reviewed. The four criteria above are satisfied, so the run out referral is determined first. Should the appeal for run out be Out, then there is no requirement for the LBW review to take place.\par}\par

\end{adjustwidth}


\vspace{\baselineskip}
\begin{adjustwidth}{0.5in}{0.04in}
{\fontsize{9pt}{10.8pt}\selectfont 3.9.5 \tabto{0.49in} In all other circumstances, the incidents shall be addressed in chronological order. If the conclusion from the first incident is that a batsman is dismissed, then the ball would be deemed to have become dead at that point, rendering investigation of the second incident unnecessary.\par}\par

\end{adjustwidth}


\vspace{\baselineskip}
{\fontsize{16pt}{19.2pt}\selectfont \textbf{4 \tabto{0.47in} Interpretation of Playing Conditions}\par}\par


\vspace{\baselineskip}
\begin{adjustwidth}{0.49in}{0.1in}
{\fontsize{9pt}{10.8pt}\selectfont 4.1 \tabto{0.47in} When using a replay to determine the moment at which the wicket has been put down (as per clause  the third umpire shall deem this to be the first frame in which one of the bails is shown (or can be deduced) to have lost all contact with the top of the stumps and subsequent frames show the bail permanently removed from the top of the stumps.\par}\par

\end{adjustwidth}


\vspace{\baselineskip}
\begin{adjustwidth}{0.49in}{0.25in}
{\fontsize{9pt}{10.8pt}\selectfont 4.2 \tabto{0.47in} Where LED Wickets are used (as provided for in paragraph the moment at which the wicket has been put down (as per clause shall be deemed to be the first frame in which the LED lights are illuminated and subsequent frames show the bail permanently removed from the top of the stumps.\par}\par

\end{adjustwidth}


\vspace{\baselineskip}

\vspace{\baselineskip}

\vspace{\baselineskip}

\vspace{\baselineskip}

\vspace{\baselineskip}

\vspace{\baselineskip}

\vspace{\baselineskip}

\vspace{\baselineskip}

\vspace{\baselineskip}

\vspace{\baselineskip}

\vspace{\baselineskip}

\vspace{\baselineskip}

\vspace{\baselineskip}

\vspace{\baselineskip}

\vspace{\baselineskip}

\vspace{\baselineskip}

\vspace{\baselineskip}

\vspace{\baselineskip}

\vspace{\baselineskip}

\vspace{\baselineskip}

\vspace{\baselineskip}

\vspace{\baselineskip}

\vspace{\baselineskip}

\vspace{\baselineskip}

\vspace{\baselineskip}

\vspace{\baselineskip}

\vspace{\baselineskip}

\vspace{\baselineskip}

\vspace{\baselineskip}

\vspace{\baselineskip}

\vspace{\baselineskip}

\vspace{\baselineskip}

\vspace{\baselineskip}

\vspace{\baselineskip}

\vspace{\baselineskip}

\vspace{\baselineskip}

\vspace{\baselineskip}

\vspace{\baselineskip}

\vspace{\baselineskip}

\vspace{\baselineskip}

\vspace{\baselineskip}

\vspace{\baselineskip}

\vspace{\baselineskip}
\begin{Center}
{\fontsize{8pt}{9.6pt}\selectfont 83\par}
\end{Center}\par


\vspace{\baselineskip}
\begin{adjustwidth}{0.0in}{0.01in}
\begin{Center}
{\fontsize{11pt}{13.2pt}\selectfont \textbf{Appendix E}\par}
\end{Center}\par

\end{adjustwidth}


\vspace{\baselineskip}
\begin{adjustwidth}{0.0in}{0.01in}
\begin{Center}
{\fontsize{11pt}{13.2pt}\selectfont \textbf{Calculations}\par}
\end{Center}\par

\end{adjustwidth}


\vspace{\baselineskip}
{\fontsize{9pt}{10.8pt}\selectfont \textbf{Table 1: Calculation sheet for use when a delay or interruptions occur in the First Innings}\par}\par


\vspace{\baselineskip}


%%%%%%%%%%%%%%%%%%%% Table No: 9 starts here %%%%%%%%%%%%%%%%%%%%


{
\scriptsize
\setlength\extrarowheight{3pt}
\begin{longtable}{p{0.27in}p{0.27in}p{0.27in}p{0.27in}p{0.27in}p{0.27in}p{0.27in}p{0.27in}p{0.27in}p{0.27in}p{0.27in}p{0.27in}p{0.27in}p{0.27in}p{0.27in}p{0.27in}p{0.27in}p{0.27in}p{0.27in}p{0.27in}p{0.27in}p{0.27in}p{0.27in}p{0.27in}}

\endfirsthead
\multicolumn{24}{c}{\textit{continued from previous page}}\hline
\endhead\hline
\multicolumn{24}{r}{\textit{continued on next page}} \\
\endfoot
\hline 
\endlastfoot%row no:1
\multicolumn{3}{p{0.82in}}{{\fontsize{9pt}{10.8pt}\selectfont Time}} & 
\multicolumn{1}{p{0.27in}}{} & 

\hhline{~~~~}
%row no:2
\multicolumn{1}{p{0.27in}}{\cellcolor[HTML]{000000}} & 
\multicolumn{1}{p{0.27in}}{} & 
\multicolumn{1}{p{0.27in}}{} & 
\multicolumn{1}{p{0.27in}}{} & 

\hhline{~~~~}
%row no:3
\multicolumn{3}{p{0.82in}}{{\fontsize{9pt}{10.8pt}\selectfont Net playing time available at start of the match}} & 
\multicolumn{1}{p{0.27in}}{{\fontsize{9pt}{10.8pt}\selectfont 170 minutes (A)}} & 

\hhline{~~~~}
%row no:4
\multicolumn{3}{p{0.82in}}{{\fontsize{9pt}{10.8pt}\selectfont Time innings in progress}} & 
\multicolumn{1}{p{0.27in}}{{\fontsize{9pt}{10.8pt}\selectfont \_\_\_\_\_\_\_\_\_\_\_ (B)}} & 

\hhline{~~~~}
%row no:5
\multicolumn{3}{p{0.82in}}{{\fontsize{9pt}{10.8pt}\selectfont Playing time lost}} & 
\multicolumn{1}{p{0.27in}}{{\fontsize{9pt}{10.8pt}\selectfont \_\_\_\_\_\_\_\_\_\_\_ (C)}} & 

\hhline{~~~~}
%row no:6
\multicolumn{3}{p{0.82in}}{{\fontsize{9pt}{10.8pt}\selectfont Extra time available}} & 
\multicolumn{1}{p{0.27in}}{{\fontsize{9pt}{10.8pt}\selectfont \_\_\_\_\_\_\_\_\_\_\_ (D)}} & 

\hhline{~~~~}
%row no:7
\multicolumn{3}{p{0.82in}}{{\fontsize{9pt}{10.8pt}\selectfont Time made up from reduced interval}} & 
\multicolumn{1}{p{0.27in}}{{\fontsize{9pt}{10.8pt}\selectfont \_\_\_\_\_\_\_\_\_\_\_ (E)}} & 

\hhline{~~~~}
%row no:8
\multicolumn{3}{p{0.82in}}{{\fontsize{9pt}{10.8pt}\selectfont Effective playing time lost [C – (D + E)]}} & 
\multicolumn{1}{p{0.27in}}{{\fontsize{9pt}{10.8pt}\selectfont \_\_\_\_\_\_\_\_\_\_\_ (F)}} & 

\hhline{~~~~}
%row no:9
\multicolumn{3}{p{0.82in}}{{\fontsize{9pt}{10.8pt}\selectfont Remaining playing time available (A - F)}} & 
\multicolumn{1}{p{0.27in}}{{\fontsize{9pt}{10.8pt}\selectfont \_\_\_\_\_\_\_\_\_\_\_ (G)}} & 

\hhline{~~~~}
%row no:10
\multicolumn{3}{p{0.82in}}{{\fontsize{9pt}{10.8pt}\selectfont G divided by 4.25 (to 2 decimal places)}} & 
\multicolumn{1}{p{0.27in}}{{\fontsize{9pt}{10.8pt}\selectfont \_\_\_\_\_\_\_\_\_\_\_ (H)}} & 

\hhline{~~~~}
%row no:11
\multicolumn{3}{p{0.82in}}{{\fontsize{9pt}{10.8pt}\selectfont Max overs per team [H/2] (round up fractions)}} & 
\multicolumn{1}{p{0.27in}}{{\fontsize{9pt}{10.8pt}\selectfont \_\_\_\_\_\_\_\_\_\_\_ (I)}} & 

\hhline{~~~~}
%row no:12
\multicolumn{3}{p{0.82in}}{{\fontsize{9pt}{10.8pt}\selectfont Maximum overs per bowler [I / 5]}} & 
\multicolumn{1}{p{0.27in}}{{\fontsize{9pt}{10.8pt}\selectfont \_\_\_\_\_\_\_\_\_\_\_}} & 

\hhline{~~~~}
%row no:13
\multicolumn{3}{p{0.82in}}{{\fontsize{9pt}{10.8pt}\selectfont Number of Powerplay overs}} & 
\multicolumn{1}{p{0.27in}}{{\fontsize{9pt}{10.8pt}\selectfont \_\_\_\_\_\_\_\_\_\_\_}} & 

\hhline{~~~~}
%row no:14
\multicolumn{3}{p{0.82in}}{{\fontsize{9pt}{10.8pt}\selectfont Rescheduled Playing Hours}} & 
\multicolumn{1}{p{0.27in}}{} & 

\hhline{~~~~}
%row no:15
\multicolumn{2}{p{0.36in}}{\cellcolor[HTML]{000000}} & 
\multicolumn{1}{p{0.27in}}{} & 
\multicolumn{1}{p{0.27in}}{} & 

\hhline{~~~~}
%row no:16
\multicolumn{3}{p{0.82in}}{{\fontsize{9pt}{10.8pt}\selectfont First session to commence or recommence}} & 
\multicolumn{1}{p{0.27in}}{{\fontsize{9pt}{10.8pt}\selectfont \_\_\_\_\_\_\_\_\_\_\_ (J)}} & 

\hhline{~~~~}
%row no:17
\multicolumn{3}{p{0.82in}}{{\fontsize{9pt}{10.8pt}\selectfont Length of innings [I x 4.25] (round up fractions)}} & 
\multicolumn{1}{p{0.27in}}{{\fontsize{9pt}{10.8pt}\selectfont \_\_\_\_\_\_\_\_\_\_\_ (K)}} & 

\hhline{~~~~}
%row no:18
\multicolumn{3}{p{0.82in}}{{\fontsize{9pt}{10.8pt}\selectfont Rescheduled first innings cessation time [J + (K – B)]}} & 
\multicolumn{1}{p{0.27in}}{{\fontsize{9pt}{10.8pt}\selectfont \_\_\_\_\_\_\_\_\_\_\_ (L)}} & 

\hhline{~~~~}
%row no:19
\multicolumn{3}{p{0.82in}}{{\fontsize{9pt}{10.8pt}\selectfont Length of interval}} & 
\multicolumn{1}{p{0.27in}}{{\fontsize{9pt}{10.8pt}\selectfont \_\_\_\_\_\_\_\_\_\_\_ (M)}} & 

\hhline{~~~~}
%row no:20
\multicolumn{3}{p{0.82in}}{{\fontsize{9pt}{10.8pt}\selectfont Second innings commencement time [L + M]}} & 
\multicolumn{1}{p{0.27in}}{{\fontsize{9pt}{10.8pt}\selectfont \_\_\_\_\_\_\_\_\_\_\_ (N)}} & 

\hhline{~~~~}
%row no:21
\multicolumn{3}{p{0.82in}}{{\fontsize{9pt}{10.8pt}\selectfont Rescheduled second innings cessation time [N + K]}} & 
\multicolumn{1}{p{0.27in}}{{\fontsize{9pt}{10.8pt}\selectfont \_\_\_\_\_\_\_\_\_\_\_ $\ast$ (O)}} & 

\hhline{~~~~}

\end{longtable}}

%%%%%%%%%%%%%%%%%%%% Table No: 9 ends here %%%%%%%%%%%%%%%%%%%%


\vspace{\baselineskip}
\begin{itemize}
	\item {\fontsize{9pt}{10.8pt}\selectfont Ensure that the match is not finishing earlier than the original or rescheduled cessation time by applying clause 13.7.2. If so, add at least one over to each team and recalculate (I) to (O) above to prevent this from happening.\par}
\end{itemize}\par


\vspace{\baselineskip}

\vspace{\baselineskip}

\vspace{\baselineskip}

\vspace{\baselineskip}

\vspace{\baselineskip}

\vspace{\baselineskip}

\vspace{\baselineskip}

\vspace{\baselineskip}

\vspace{\baselineskip}

\vspace{\baselineskip}

\vspace{\baselineskip}

\vspace{\baselineskip}

\vspace{\baselineskip}

\vspace{\baselineskip}
\begin{Center}
{\fontsize{8pt}{9.6pt}\selectfont 84\par}
\end{Center}\par


\vspace{\baselineskip}

\vspace{\baselineskip}
{\fontsize{9pt}{10.8pt}\selectfont \textbf{Table 2: Calculation sheet to check whether an interruption during the First Innings should terminate the innings}\par}\par


\vspace{\baselineskip}


%%%%%%%%%%%%%%%%%%%% Table No: 10 starts here %%%%%%%%%%%%%%%%%%%%


{
\scriptsize
\setlength\extrarowheight{3pt}
\begin{longtable}{p{0.22in}p{0.22in}p{0.22in}p{0.22in}p{0.22in}p{0.22in}p{0.22in}p{0.22in}p{0.22in}p{0.22in}p{0.22in}p{0.22in}p{0.22in}p{0.22in}p{0.22in}p{0.22in}p{0.22in}p{0.22in}p{0.22in}p{0.22in}p{0.22in}p{0.22in}p{0.22in}p{0.22in}p{0.22in}p{0.22in}p{0.22in}p{0.22in}p{0.22in}p{0.22in}}

\endfirsthead
\multicolumn{30}{c}{\textit{continued from previous page}}\hline
\endhead\hline
\multicolumn{30}{r}{\textit{continued on next page}} \\
\endfoot
\hline 
\endlastfoot%row no:1
\multicolumn{4}{p{0.97in}}{{\fontsize{9pt}{10.8pt}\selectfont Proposed re-start time}} & 
\multicolumn{1}{p{0.22in}}{{\fontsize{9pt}{10.8pt}\selectfont \_\_\_\_\_\_\_\_\_\_\_ (P)}} & 

\hhline{~~~~~}
%row no:2
\multicolumn{4}{p{0.97in}}{{\fontsize{9pt}{10.8pt}\selectfont Rescheduled cut-off time allowing for full use of any extra time provision}} & 
\multicolumn{1}{p{0.22in}}{{\fontsize{9pt}{10.8pt}\selectfont \_\_\_\_\_\_\_\_\_\_\_ (Q)}} & 

\hhline{~~~~~}
%row no:3
\multicolumn{4}{p{0.97in}}{{\fontsize{9pt}{10.8pt}\selectfont Minutes between P and Q}} & 
\multicolumn{1}{p{0.22in}}{{\fontsize{9pt}{10.8pt}\selectfont \_\_\_\_\_\_\_\_\_\_\_ (R)}} & 

\hhline{~~~~~}
%row no:4
\multicolumn{4}{p{0.97in}}{{\fontsize{9pt}{10.8pt}\selectfont Potential overs to be bowled [R / 4.25] (round up fractions)}} & 
\multicolumn{1}{p{0.22in}}{{\fontsize{9pt}{10.8pt}\selectfont \_\_\_\_\_\_\_\_\_\_\_ (S)}} & 

\hhline{~~~~~}
%row no:5
\multicolumn{4}{p{0.97in}}{{\fontsize{9pt}{10.8pt}\selectfont Number of complete overs faced to date in first innings}} & 
\multicolumn{1}{p{0.22in}}{{\fontsize{9pt}{10.8pt}\selectfont \_\_\_\_\_\_\_\_\_\_\_ (T)}} & 

\hhline{~~~~~}
%row no:6
\multicolumn{4}{p{0.97in}}{{\fontsize{9pt}{10.8pt}\selectfont If S is greater than T then revert to Table 1}} & 
\multicolumn{1}{p{0.22in}}{} & 

\hhline{~~~~~}
%row no:7
\multicolumn{4}{p{0.97in}}{{\fontsize{9pt}{10.8pt}\selectfont If S is less than or equal to T then the first innings is terminated - go to Table 3}} & 
\multicolumn{1}{p{0.22in}}{} & 

\hhline{~~~~~}
%row no:8
\multicolumn{4}{p{0.97in}}{{\fontsize{9pt}{10.8pt}\selectfont \textbf{Table 3: Calculation sheet for the start of the Second Innings}}} & 
\multicolumn{1}{p{0.22in}}{} & 

\hhline{~~~~~}
%row no:9
\multicolumn{2}{p{0.32in}}{{\fontsize{9pt}{10.8pt}\selectfont Maximum overs to be bowled:}} & 
\multicolumn{2}{p{0.61in}}{} & 
\multicolumn{1}{p{0.22in}}{} & 

\hhline{~~~~~}
%row no:10
\multicolumn{4}{p{0.97in}}{{\fontsize{9pt}{10.8pt}\selectfont (If first innings was terminated, S from Table 2)}} & 
\multicolumn{1}{p{0.22in}}{{\fontsize{9pt}{10.8pt}\selectfont \_\_\_\_\_\_\_\_\_\_\_ (A)}} & 

\hhline{~~~~~}
%row no:11
\multicolumn{4}{p{0.97in}}{{\fontsize{9pt}{10.8pt}\selectfont Scheduled length of innings: [A x 4.25] (round up fractions)}} & 
\multicolumn{1}{p{0.22in}}{{\fontsize{9pt}{10.8pt}\selectfont \_\_\_\_\_\_\_\_\_\_\_ (B)}} & 

\hhline{~~~~~}
%row no:12
\multicolumn{4}{p{0.97in}}{{\fontsize{9pt}{10.8pt}\selectfont Start time}} & 
\multicolumn{1}{p{0.22in}}{{\fontsize{9pt}{10.8pt}\selectfont \_\_\_\_\_\_\_\_\_\_\_ (C)}} & 

\hhline{~~~~~}
%row no:13
\multicolumn{4}{p{0.97in}}{{\fontsize{9pt}{10.8pt}\selectfont Scheduled cessation time [C + B]}} & 
\multicolumn{1}{p{0.22in}}{{\fontsize{9pt}{10.8pt}\selectfont \_\_\_\_\_\_\_\_\_\_\_ (D)}} & 

\hhline{~~~~~}
%row no:14
\multicolumn{3}{p{0.47in}}{{\fontsize{9pt}{10.8pt}\selectfont Overs per bowler and Fielding Restrictions}} & 
\multicolumn{1}{p{0.22in}}{} & 
\multicolumn{1}{p{0.22in}}{} & 

\hhline{~~~~~}
%row no:15
\multicolumn{4}{p{0.97in}}{{\fontsize{9pt}{10.8pt}\selectfont Maximum overs per bowler [A / 5]}} & 
\multicolumn{1}{p{0.22in}}{{\fontsize{9pt}{10.8pt}\selectfont \_\_\_\_\_\_\_\_\_\_\_ overs}} & 

\hhline{~~~~~}
%row no:16
\multicolumn{4}{p{0.97in}}{{\fontsize{9pt}{10.8pt}\selectfont Number of Powerplay overs}} & 
\multicolumn{1}{p{0.22in}}{{\fontsize{9pt}{10.8pt}\selectfont \_\_\_\_\_\_\_\_\_\_\_ overs}} & 

\hhline{~~~~~}
%row no:17
\multicolumn{5}{p{1.28in}}{{\fontsize{9pt}{10.8pt}\selectfont \textbf{Table 4: Calculation sheet for use when interruption occurs after the start of the Second Innings}}} & 

\hhline{~~~~~}
%row no:18
\multicolumn{4}{p{0.97in}}{{\fontsize{9pt}{10.8pt}\selectfont Time}} & 
\multicolumn{1}{p{0.22in}}{} & 

\hhline{~~~~~}
%row no:19
\multicolumn{1}{p{0.22in}}{\cellcolor[HTML]{000000}} & 
\multicolumn{3}{p{0.91in}}{} & 
\multicolumn{1}{p{0.22in}}{} & 

\hhline{~~~~~}
%row no:20
\multicolumn{4}{p{0.97in}}{{\fontsize{9pt}{10.8pt}\selectfont Time at start of innings}} & 
\multicolumn{1}{p{0.22in}}{{\fontsize{9pt}{10.8pt}\selectfont \_\_\_\_\_\_\_\_\_\_\_ (A)}} & 

\hhline{~~~~~}
%row no:21
\multicolumn{4}{p{0.97in}}{{\fontsize{9pt}{10.8pt}\selectfont Time at start of interruption}} & 
\multicolumn{1}{p{0.22in}}{{\fontsize{9pt}{10.8pt}\selectfont \_\_\_\_\_\_\_\_\_\_\_ (B)}} & 

\hhline{~~~~~}
%row no:22
\multicolumn{4}{p{0.97in}}{{\fontsize{9pt}{10.8pt}\selectfont Time innings in progress}} & 
\multicolumn{1}{p{0.22in}}{{\fontsize{9pt}{10.8pt}\selectfont \_\_\_\_\_\_\_\_\_\_\_ (C)}} & 

\hhline{~~~~~}
%row no:23
\multicolumn{4}{p{0.97in}}{{\fontsize{9pt}{10.8pt}\selectfont Restart time}} & 
\multicolumn{1}{p{0.22in}}{{\fontsize{9pt}{10.8pt}\selectfont \_\_\_\_\_\_\_\_\_\_\_ (D)}} & 

\hhline{~~~~~}
%row no:24
\multicolumn{4}{p{0.97in}}{{\fontsize{9pt}{10.8pt}\selectfont Length of interruption [D – B]}} & 
\multicolumn{1}{p{0.22in}}{{\fontsize{9pt}{10.8pt}\selectfont \_\_\_\_\_\_\_\_\_\_\_ (E)}} & 

\hhline{~~~~~}
%row no:25
\multicolumn{4}{p{0.97in}}{{\fontsize{9pt}{10.8pt}\selectfont Additional time available:}} & 
\multicolumn{1}{p{0.22in}}{{\fontsize{9pt}{10.8pt}\selectfont \_\_\_\_\_\_\_\_\_\_\_ (F)}} & 

\hhline{~~~~~}
%row no:26
\multicolumn{5}{p{1.28in}}{{\fontsize{9pt}{10.8pt}\selectfont (Any unused provision for ‘Extra Time’ or for earlier than scheduled start of second innings)}} & 

\hhline{~~~~~}
%row no:27
\multicolumn{4}{p{0.97in}}{{\fontsize{9pt}{10.8pt}\selectfont Total playing time lost [E – F]}} & 
\multicolumn{1}{p{0.22in}}{{\fontsize{9pt}{10.8pt}\selectfont \_\_\_\_\_\_\_\_\_\_\_ (G)}} & 

\hhline{~~~~~}

\end{longtable}}

%%%%%%%%%%%%%%%%%%%% Table No: 10 ends here %%%%%%%%%%%%%%%%%%%%


\vspace{\baselineskip}

\vspace{\baselineskip}

\vspace{\baselineskip}

\vspace{\baselineskip}
\begin{Center}
{\fontsize{8pt}{9.6pt}\selectfont 85\par}
\end{Center}\par


\vspace{\baselineskip}


%%%%%%%%%%%%%%%%%%%% Table No: 11 starts here %%%%%%%%%%%%%%%%%%%%


\begin{table}[H]
 			\centering
\begin{tabular}{p{0.54in}p{0.54in}p{0.54in}p{0.54in}p{0.54in}p{0.54in}p{0.54in}p{0.54in}p{0.54in}p{0.54in}p{0.54in}p{0.54in}}
%row no:1
\multicolumn{1}{p{0.54in}}{{\fontsize{9pt}{10.8pt}\selectfont Overs}} & 
\multicolumn{2}{p{1.85in}}{} & 
\multicolumn{1}{p{0.54in}}{} & 

\hhline{~~~~}
%row no:2
\multicolumn{3}{p{2.03in}}{{\fontsize{9pt}{10.8pt}\selectfont Maximum overs at start of innings}} & 
\multicolumn{1}{p{0.54in}}{{\fontsize{9pt}{10.8pt}\selectfont \_\_\_\_\_\_\_\_\_\_\_ (H)}} & 

\hhline{~~~~}
%row no:3
\multicolumn{3}{p{2.03in}}{{\fontsize{9pt}{10.8pt}\selectfont Overs lost [G / 4.25] (rounded down)}} & 
\multicolumn{1}{p{0.54in}}{{\fontsize{9pt}{10.8pt}\selectfont \_\_\_\_\_\_\_\_\_\_\_ (I)}} & 

\hhline{~~~~}
%row no:4
\multicolumn{3}{p{2.03in}}{{\fontsize{9pt}{10.8pt}\selectfont Adjusted maximum length of innings [H – I]}} & 
\multicolumn{1}{p{0.54in}}{{\fontsize{9pt}{10.8pt}\selectfont \_\_\_\_\_\_\_\_\_\_\_ (J)}} & 

\hhline{~~~~}
%row no:5
\multicolumn{3}{p{2.03in}}{{\fontsize{9pt}{10.8pt}\selectfont Rescheduled length of innings [J x 4.25 rounded up]}} & 
\multicolumn{1}{p{0.54in}}{{\fontsize{9pt}{10.8pt}\selectfont \_\_\_\_\_\_\_\_\_\_\_ (K)}} & 

\hhline{~~~~}
%row no:6
\multicolumn{3}{p{2.03in}}{{\fontsize{9pt}{10.8pt}\selectfont Amended cessation time of innings [D + (K – C)]}} & 
\multicolumn{1}{p{0.54in}}{{\fontsize{9pt}{10.8pt}\selectfont \_\_\_\_\_\_\_\_\_\_\_ (L)}} & 

\hhline{~~~~}
%row no:7
\multicolumn{2}{p{1.17in}}{{\fontsize{9pt}{10.8pt}\selectfont Overs per bowler and Fielding Restrictions}} & 
\multicolumn{1}{p{0.54in}}{} & 
\multicolumn{1}{p{0.54in}}{} & 

\hhline{~~~~}
%row no:8
\multicolumn{3}{p{2.03in}}{{\fontsize{9pt}{10.8pt}\selectfont Maximum overs per bowler [J / 5]}} & 
\multicolumn{1}{p{0.54in}}{{\fontsize{9pt}{10.8pt}\selectfont \_\_\_\_\_\_\_\_\_\_\_ overs}} & 

\hhline{~~~~}
%row no:9
\multicolumn{3}{p{2.03in}}{{\fontsize{9pt}{10.8pt}\selectfont Number of Powerplay overs}} & 
\multicolumn{1}{p{0.54in}}{{\fontsize{9pt}{10.8pt}\selectfont \_\_\_\_\_\_\_\_\_\_\_ overs}} & 

\hhline{~~~~}

\end{tabular}
 \end{table}


%%%%%%%%%%%%%%%%%%%% Table No: 11 ends here %%%%%%%%%%%%%%%%%%%%


\vspace{\baselineskip}

\vspace{\baselineskip}

\vspace{\baselineskip}

\vspace{\baselineskip}

\vspace{\baselineskip}

\vspace{\baselineskip}

\vspace{\baselineskip}

\vspace{\baselineskip}

\vspace{\baselineskip}

\vspace{\baselineskip}

\vspace{\baselineskip}

\vspace{\baselineskip}

\vspace{\baselineskip}

\vspace{\baselineskip}

\vspace{\baselineskip}

\vspace{\baselineskip}

\vspace{\baselineskip}

\vspace{\baselineskip}

\vspace{\baselineskip}

\vspace{\baselineskip}

\vspace{\baselineskip}

\vspace{\baselineskip}

\vspace{\baselineskip}

\vspace{\baselineskip}

\vspace{\baselineskip}

\vspace{\baselineskip}

\vspace{\baselineskip}

\vspace{\baselineskip}

\vspace{\baselineskip}

\vspace{\baselineskip}

\vspace{\baselineskip}

\vspace{\baselineskip}

\vspace{\baselineskip}

\vspace{\baselineskip}

\vspace{\baselineskip}

\vspace{\baselineskip}

\vspace{\baselineskip}

\vspace{\baselineskip}

\vspace{\baselineskip}

\vspace{\baselineskip}

\vspace{\baselineskip}

\vspace{\baselineskip}

\vspace{\baselineskip}

\vspace{\baselineskip}

\vspace{\baselineskip}

\vspace{\baselineskip}

\vspace{\baselineskip}

\vspace{\baselineskip}

\vspace{\baselineskip}
\begin{Center}
{\fontsize{8pt}{9.6pt}\selectfont 86\par}
\end{Center}\par


\vspace{\baselineskip}
\begin{Center}
{\fontsize{11pt}{13.2pt}\selectfont \textbf{Appendix F}\par}
\end{Center}\par


\vspace{\baselineskip}
\begin{adjustwidth}{0.0in}{0.01in}
\begin{Center}
{\fontsize{11pt}{13.2pt}\selectfont \textbf{Procedure for the Super Over}\par}
\end{Center}\par

\end{adjustwidth}


\vspace{\baselineskip}
{\fontsize{9pt}{10.8pt}\selectfont The following procedure shall apply should the provision for a Super Over be adopted in any match.\par}\par


\vspace{\baselineskip}
\begin{enumerate}[label*={\fontsize{9pt}{9pt}\selectfont \arabic*.}]
	\item {\fontsize{9pt}{10.8pt}\selectfont Subject to weather conditions the Super Over will take place on the scheduled day of the match at a time to be determined by the ICC Match Referee. In normal circumstances it shall commence 10 minutes after the conclusion of the match.\par}\par


\vspace{\baselineskip}
	\item {\fontsize{9pt}{10.8pt}\selectfont The amount of extra time allocated to the Super Over is the greater of (a) the extra time allocated to the original match less the amount of extra time actually utilised and (b) the gap between the actual end of the match and the time the original match would have been scheduled to finish had the whole of the extra time provision been utilised. Should play be delayed prior to or during the Super Over once the playing time lost exceeds the extra time allocated, the Super Over shall be abandoned. See paragraph 16 below.\par}\par


\vspace{\baselineskip}
	\item {\fontsize{9pt}{10.8pt}\selectfont The Super Over shall take place on the pitch allocated for the match (the designated pitch) unless otherwise determined by the umpires in consultation with the Ground Authority and the ICC Match Referee.\par}\par


\vspace{\baselineskip}
	\item {\fontsize{9pt}{10.8pt}\selectfont The umpires shall stand at the same end as that in which they finished the match.\par}\par


\vspace{\baselineskip}
	\item {\fontsize{9pt}{10.8pt}\selectfont In both innings of the Super Over, the fielding side shall choose from which end to bowl.\par}\par


\vspace{\baselineskip}
	\item {\fontsize{9pt}{10.8pt}\selectfont Only nominated players in the match may participate in the Super Over. Should any player (including the batsmen and bowler) be unable to continue to participate in the Super Over due to injury, illness or other wholly acceptable reasons, the relevant Playing Conditions as they apply in the match shall also apply in the Super Over.\par}\par


\vspace{\baselineskip}
	\item {\fontsize{9pt}{10.8pt}\selectfont Any penalty time being served in the match shall be carried forward to the Super Over.\par}\par


\vspace{\baselineskip}
	\item {\fontsize{9pt}{10.8pt}\selectfont Each team’s over is played with the same fielding restrictions as apply for the last over in a match played under the ICC Twenty20 International Playing Conditions.\par}\par


\vspace{\baselineskip}
	\item {\fontsize{9pt}{10.8pt}\selectfont The team batting second in the match shall bat first in the Super Over.\par}\par


\vspace{\baselineskip}
	\item {\fontsize{9pt}{10.8pt}\selectfont The captain of the fielding team (or his/her nominee) shall select the ball with which the fielding team shall bowl their over in the Super Over from the box of spare balls provided by the umpires (which shall include the balls used in the match, but no new balls). The team fielding first in the Super Over shall have first choice of ball. The team fielding second may choose to use the same ball as chosen by the team bowling first. If the ball needs to be changed, the Playing Conditions shall apply.\par}\par


\vspace{\baselineskip}
	\item {\fontsize{9pt}{10.8pt}\selectfont The loss of two wickets in the over ends the team’s one over innings.\par}\par


\vspace{\baselineskip}
	\item {\fontsize{9pt}{10.8pt}\selectfont Each team shall be allowed to make one unsuccessful Player Review in each innings of the Super Over. This entitlement shall apply irrespective of the number of unsuccessful Player Review requests made during the match itself.\par}\par


\vspace{\baselineskip}
	\item {\fontsize{9pt}{10.8pt}\selectfont In the event of the teams having the same score after the Super Over has been completed, if the original match was a tie under the Duckworth/Lewis/Stern method, paragraph 15 below shall apply. Otherwise, the team whose batsmen hit the most number of boundaries combined from its two innings in both the match and the Super Over shall be the winner.\par}\par


\vspace{\baselineskip}
	\item {\fontsize{9pt}{10.8pt}\selectfont If the number of boundaries hit by both teams is equal, the team whose batsmen scored more boundaries during its innings in the main match (ignoring the Super Over) shall be the winner.\par}\par


\vspace{\baselineskip}
	\item {\fontsize{9pt}{10.8pt}\selectfont If still equal, a count-back from the final ball of the Super Over shall be conducted. The team with the higher scoring delivery shall be the winner. If a team loses two wickets during its over, then any unbowled deliveries will be counted as dot balls. Note that for this purpose, the runs scored from a delivery is defined as the total team runs scored since the completion of the previous legitimate ball, i.e including any runs resulting from Wides, No balls or penalty runs.\par}
\end{enumerate}\par


\vspace{\baselineskip}

\vspace{\baselineskip}

\vspace{\baselineskip}

\vspace{\baselineskip}

\vspace{\baselineskip}
\begin{Center}
{\fontsize{8pt}{9.6pt}\selectfont 87\par}
\end{Center}\par


\vspace{\baselineskip}
\begin{adjustwidth}{0.5in}{0.0in}
{\fontsize{9pt}{10.8pt}\selectfont Example:\par}\par

\end{adjustwidth}


\vspace{\baselineskip}


%%%%%%%%%%%%%%%%%%%% Table No: 12 starts here %%%%%%%%%%%%%%%%%%%%


\begin{table}[H]
 			\centering
\begin{tabular}{p{0.54in}p{0.54in}p{0.54in}p{0.54in}p{0.54in}p{0.54in}p{0.54in}p{0.54in}p{0.54in}p{0.54in}p{0.54in}p{0.54in}}
%row no:1
\multicolumn{1}{p{0.54in}}{{\fontsize{9pt}{10.8pt}\selectfont Runs scored from:}} & 
\multicolumn{1}{p{0.54in}}{{\fontsize{9pt}{10.8pt}\selectfont Team 1}} & 
\multicolumn{1}{p{0.54in}}{{\fontsize{9pt}{10.8pt}\selectfont Team 2}} & 

\hhline{~~~}
%row no:2
\multicolumn{1}{p{0.54in}}{} & 
\multicolumn{1}{p{0.54in}}{} & 
\multicolumn{1}{p{0.54in}}{} & 

\hhline{~~~}
%row no:3
\multicolumn{1}{p{0.54in}}{{\fontsize{9pt}{10.8pt}\selectfont Ball 6}} & 
\multicolumn{1}{p{0.54in}}{{\fontsize{9pt}{10.8pt}\selectfont 1}} & 
\multicolumn{1}{p{0.54in}}{{\fontsize{9pt}{10.8pt}\selectfont 1}} & 

\hhline{~~~}
%row no:4
\multicolumn{1}{p{0.54in}}{} & 
\multicolumn{1}{p{0.54in}}{} & 
\multicolumn{1}{p{0.54in}}{} & 

\hhline{~~~}
%row no:5
\multicolumn{1}{p{0.54in}}{{\fontsize{9pt}{10.8pt}\selectfont Ball 5}} & 
\multicolumn{1}{p{0.54in}}{{\fontsize{9pt}{10.8pt}\selectfont 4}} & 
\multicolumn{1}{p{0.54in}}{{\fontsize{9pt}{10.8pt}\selectfont 4}} & 

\hhline{~~~}
%row no:6
\multicolumn{1}{p{0.54in}}{} & 
\multicolumn{1}{p{0.54in}}{} & 
\multicolumn{1}{p{0.54in}}{} & 

\hhline{~~~}
%row no:7
\multicolumn{1}{p{0.54in}}{{\fontsize{9pt}{10.8pt}\selectfont Ball 4}} & 
\multicolumn{1}{p{0.54in}}{{\fontsize{9pt}{10.8pt}\selectfont 2}} & 
\multicolumn{1}{p{0.54in}}{{\fontsize{9pt}{10.8pt}\selectfont 1}} & 

\hhline{~~~}
%row no:8
\multicolumn{1}{p{0.54in}}{} & 
\multicolumn{1}{p{0.54in}}{} & 
\multicolumn{1}{p{0.54in}}{} & 

\hhline{~~~}
%row no:9
\multicolumn{1}{p{0.54in}}{{\fontsize{9pt}{10.8pt}\selectfont Ball 3}} & 
\multicolumn{1}{p{0.54in}}{{\fontsize{9pt}{10.8pt}\selectfont 6}} & 
\multicolumn{1}{p{0.54in}}{{\fontsize{9pt}{10.8pt}\selectfont 2}} & 

\hhline{~~~}
%row no:10
\multicolumn{1}{p{0.54in}}{} & 
\multicolumn{1}{p{0.54in}}{} & 
\multicolumn{1}{p{0.54in}}{} & 

\hhline{~~~}
%row no:11
\multicolumn{1}{p{0.54in}}{{\fontsize{9pt}{10.8pt}\selectfont Ball 2}} & 
\multicolumn{1}{p{0.54in}}{{\fontsize{9pt}{10.8pt}\selectfont 0}} & 
\multicolumn{1}{p{0.54in}}{{\fontsize{9pt}{10.8pt}\selectfont 1}} & 

\hhline{~~~}
%row no:12
\multicolumn{1}{p{0.54in}}{} & 
\multicolumn{1}{p{0.54in}}{} & 
\multicolumn{1}{p{0.54in}}{} & 

\hhline{~~~}
%row no:13
\multicolumn{1}{p{0.54in}}{{\fontsize{9pt}{10.8pt}\selectfont Ball 1}} & 
\multicolumn{1}{p{0.54in}}{{\fontsize{9pt}{10.8pt}\selectfont 2}} & 
\multicolumn{1}{p{0.54in}}{{\fontsize{9pt}{10.8pt}\selectfont 6}} & 

\hhline{~~~}
%row no:14
\multicolumn{1}{p{0.54in}}{} & 
\multicolumn{1}{p{0.54in}}{} & 
\multicolumn{1}{p{0.54in}}{} & 

\hhline{~~~}

\end{tabular}
 \end{table}


%%%%%%%%%%%%%%%%%%%% Table No: 12 ends here %%%%%%%%%%%%%%%%%%%%


\vspace{\baselineskip}
\begin{adjustwidth}{0.25in}{0.17in}
{\fontsize{9pt}{10.8pt}\selectfont In this example both teams scored an equal number of runs from the 6th and 5th ball of their innings. However team 1 scored 2 runs from its 4th ball while team 2 scored a single so team 1 is the winner.\par}\par

\end{adjustwidth}


\vspace{\baselineskip}
\begin{adjustwidth}{0.25in}{0.0in}
{\fontsize{9pt}{10.8pt}\selectfont 16.\  Paragraph 2 examples:\par}\par

\end{adjustwidth}


\vspace{\baselineskip}
\begin{adjustwidth}{0.25in}{0.26in}
{\fontsize{9pt}{10.8pt}\selectfont Scheduled finish 5.00, 30 minutes extra time available, so scheduled finish time if the whole of the extra time provision is utilised is 5.30.\par}\par

\end{adjustwidth}


\vspace{\baselineskip}
\begin{enumerate}
	\item {\fontsize{9pt}{10.8pt}\selectfont No extra time is utilised in the original match which overruns ten minutes and finishes at 5.10. The Super Over is scheduled to start at 5.20 with 30 minutes extra time available. It starts on time but is interrupted at 5.25. Play must resume by 5.55 otherwise the Super Over is abandoned.\par}\par


\vspace{\baselineskip}
	\item {\fontsize{9pt}{10.8pt}\selectfont 20 minutes of extra time was utilised, with the match scheduled to finish at 5.20, but it actually finishes at 5.10. Therefore the extra time allocated to the Super Over is the greater of a) 10 minutes (30 minutes extra time less 20 already utilised) and b) 20 minutes (the gap from the actual finish time of 5.10 and the scheduled finish had the full extra time been utilised of 5.30). The Super Over was due to start at 5.20, but is delayed by rain. It must therefore start by 5.40 or the Super Over is abandoned.\par}\par


\vspace{\baselineskip}
	\item {\fontsize{9pt}{10.8pt}\selectfont The match finishes at 5.40 (having started 30 minutes late and overrun by 10 minutes). There is no extra time allocated to the Super Over which should start at 5.50. Any delay or interruption after 5.50 means the Super Over is abandoned.\par}
\end{enumerate}\par


\vspace{\baselineskip}

\vspace{\baselineskip}

\vspace{\baselineskip}

\vspace{\baselineskip}

\vspace{\baselineskip}

\vspace{\baselineskip}

\vspace{\baselineskip}

\vspace{\baselineskip}

\vspace{\baselineskip}

\vspace{\baselineskip}

\vspace{\baselineskip}

\vspace{\baselineskip}

\vspace{\baselineskip}

\vspace{\baselineskip}

\vspace{\baselineskip}

\vspace{\baselineskip}

\vspace{\baselineskip}

\vspace{\baselineskip}

\vspace{\baselineskip}

\vspace{\baselineskip}

\vspace{\baselineskip}

\vspace{\baselineskip}

\vspace{\baselineskip}

\vspace{\baselineskip}

\vspace{\baselineskip}

\vspace{\baselineskip}

\vspace{\baselineskip}

\vspace{\baselineskip}
\begin{Center}
{\fontsize{8pt}{9.6pt}\selectfont 88\par}
\end{Center}\par

\end{document}